\documentclass[reqno]{amsart} \input{common.tex}


\title{Universal enveloping algebra}

\abstract{Jottings from student meeting}

\begin{document}

\maketitle

\section{Some general discussion of universal enveloping algebra}\label{sec:cngsxurern}
Say you have a Lie algebra $\mathfrak{g}$.  This means: a vector space, equipped with a bilinear map
\begin{equation*}
  [\cdot, \cdot] : \mathfrak{g} \times \mathfrak{g} \rightarrow \mathfrak{g}
\end{equation*}
that is alternating and satisfies the Jacobi identity.

\begin{example}\label{example:cngsxuq83x}
  For any vector space $V$, the space $\End(V)$ of endomorphisms of $V$ is a Lie algebra, with the bracket given by the commutator:
  \begin{equation*}
    [f, g] = f g - g f.
  \end{equation*}
\end{example}

Let $V$ be a (linear) representation of $\mathfrak{g}$.  This means: a vector space, equipped with a morphism of Lie algebras
\begin{equation*}
  \rho : \mathfrak{g} \rightarrow \End(V).
\end{equation*}
(Recall that ``morphism'' means ``something linear that respects the bracket''.)

Suppose we have some elements $x, y, z \in \mathfrak{g}$ that satisfy $[x,y] = z$.

Then, for any representation $\rho : \mathfrak{g} \rightarrow \End(V)$, we have
\begin{equation*}
  \rho(z) = \rho([x,y]) = [\rho(x), \rho(y)] = \rho(x) \rho(y) - \rho(y) \rho(x).
\end{equation*}

Let's suppose we have a list of representations $\rho_1, \rho_2, \rho_3, \dotsc$ on vector spaces $V_1, V_2, V_3, \dotsc$.  Then, for each representation in this list, we get an identity like the above:
\begin{equation*}
  \rho_1(z) = \rho_1(x) \rho_1(y) - \rho_1(y) \rho_1(x),
\end{equation*}
\begin{equation*}
  \rho_2(z) = \rho_2(x) \rho_2(y) - \rho_2(y) \rho_2(x),
\end{equation*}
\begin{equation*}
  \rho_3(z) = \rho_3(x) \rho_3(y) - \rho_3(y) \rho_3(x),
\end{equation*}
and so on.

Hmm. 🤔

The reason we can make sense of the right hand sides of these identities is because $\End(V)$ is not just a Lie algebra, but also an associative algebra, meaning, we can multiply.

Wouldn't it be cool if there were some ``universal'' associative algebra $A$, equipped with some ``universal'' morphism of Lie algebras $r : \mathfrak{g} \rightarrow A$, such that each of our cool representations $\rho_1, \rho_2, \dotsc$ arose (uniquely?) as a composition
\begin{equation*}
  \mathfrak{g} \xrightarrow{r} A \xrightarrow{?} \End(V_i)?
\end{equation*}
If that were the case, then we could just write down one ``universal'' identity
\begin{equation*}
  r(z) = r(x) r(y) - r(y) r(x)
\end{equation*}
and then compose that identity with the maps $?$ to get the identities for all of our representations.

How would we find $A$?  Just apply the left adjoint of the forgetful functor from associative algebras to Lie algebras.  Concretely, start with $\mathfrak{g}$, cook up its tensor algebra (which is the universal associative algebra generated by $\mathfrak{g}$), then mod out by the ideal generated by all the relations that need to be satisfied in order for the natural map from $\mathfrak{g}$ to be a morphism not just of vector spaces, but also of Lie algebras.  This just means that for all $x, y \in \mathfrak{g}$, we need to have $r([x,y]) = r(x) r(y) - r(y) r(x)$.  So we should mod out by the ideal generated by all such differences.

In other words:
\begin{itemize}
\item Let $\tilde{A}$ denote the tensor algebra on $\mathfrak{g}$ (i.e., all non-commutative polynomials in elements of $\mathfrak{g}$).
\item Let $\tilde{r} : \mathfrak{g} \rightarrow \tilde{A}$ be the inclusion.
\item Take
  \begin{equation*}
    A := \frac{\tilde{A}}{\langle \tilde{r}([x,y]) - (\tilde{r}(x)\tilde{r}(y) - \tilde{r}(y)\tilde{r}(x)) : x, y \in \mathfrak{g} \rangle}.
  \end{equation*}
\end{itemize}
That's the universal enveloping algebra.

We tend to simply write $x$ instead of $r(x)$, identifying $x$ with its image in $A$ (because the map $\mathfrak{g} \rightarrow A$ is injective, by PBW).  Then, our identities can be written even more simply, e.g., as
\begin{equation*}
  z = x y - y x.
\end{equation*}

Serre \cite{MR1808366} and Serre \cite{MR2179691} probably discuss such things.

\subsection{Something about Casimirs on compact Lie groups}\label{sec:cngsxurggj}
Suppose $G$ is a compact Lie group, and let $\mathfrak{g}$ be its Lie algebra.  We may then equip $\mathfrak{g}$ with a $G$-invariant inner product, using the compactness of $G$.  Let $\mathcal{B}(\mathfrak{g})$ be an orthonormal basis of $\mathfrak{g}$ with respect to this inner product.  We may then form the Casimir
\begin{equation}\label{eq:cngsxtalx1}
  \Omega := \sum_{x \in \mathcal{B}(\mathfrak{g})} x^2.
\end{equation}
What we mean by this is: it's something that can act in any representation by the indicated formula, i.e., for any representation $\rho : \mathfrak{g} \rightarrow \End(V)$ (i.e., linear map that respects the Lie bracket), we set
\begin{equation*}
  \rho(\Omega) := \sum_{x \in \mathcal{B}(\mathfrak{g})} \rho(x)^2.
\end{equation*}
(The universal enveloping algebra, as described above, is the algebraic gadget that allows us to make sense of the definition \eqref{eq:cngsxtalx1} without having to specify any specific representation.)

\begin{fact}\label{fact:cngsxurkz3}
  $\Omega$ is independent of the choice of orthonormal basis.
\end{fact}
\begin{proof}
  Welp, let $x_1, \dotsc, x_n$ and $y_1, \dotsc, y_n$ be orthonormal bases.  Then we can write $y_i = \sum_j a_{i j} x_j$, where the matrix $a_{i j}$ is orthonormal.  Then
  \begin{equation*}
    \sum_{i} y_i^2 = \sum_i \left( \sum_j a_{i j} x_j \right)^2.
  \end{equation*}
  Expanding the square, we obtain
  \begin{equation*}
    \sum_{j_1, j_2} x_{j_1} x_{j_2} \sum_i a_{i j_1} a_{i j_2}.
  \end{equation*}
  We now use the defining property of an orthogonal matrix, namely, that it's equal to its own transpose and that its row (or columns) are orthonormal, to see that this last sum collapses to $\sum_j x_j^2$.
\end{proof}

\begin{fact}\label{fact:cngsxurjbd}
  For any $g \in G$, we have
  \begin{equation*}
    g \Omega g^{-1} = \Omega.
  \end{equation*}
  (We again view this identity as holding in some universal space of operators, or equivalently, as holding for any representation that we act on.)
\end{fact}
\begin{proof}
  Welp, formally,
  \begin{equation*}
    g \Omega g^{-1} = \sum_i g x_i^2 g^{-1} = \sum_i (g x_i g^{-1})^2.
  \end{equation*}
  We now use that $G$ acts on the Lie algebra via orthogonal transformations to see that $\{g x_i g^{-1}\}$ is another orthonormal basis, and appeal to Fact \ref{fact:cngsxurkz3}.
\end{proof}

\begin{example}\label{example:cngsxuwa7f}
  If $G = \SO(3)$, then $\mathfrak{g}$ admits a basis $\{x, y, z\}$ such that $[x,y] = z$, $[y,z] = x$, and $[z,x] = y$, corresponding to infinitesimal rotations about the various axes.  This basis is orthonormal with respect to an invariant inner product, so we have $\Omega = x^2 + y^2 + z^2$.
\end{example}

\begin{fact}
  For any irreducible representation $\pi$ of $G$, we have that $\Omega$ acts on $\pi$ via a scalar.
\end{fact}
\begin{proof}
  Recall that by Schur's lemma, any endomorphism of $\pi$ that commutes with the action of $G$ is a scalar.  It follows from Fact~\ref{fact:cngsxurjbd} that $\Omega$ commutes with the action of $G$:
  \begin{equation*}
    g \Omega v = \Omega g v \quad \text{for all } v \in \pi.
  \end{equation*}
\end{proof}

\bibliography{refs}{} \bibliographystyle{plain}
\end{document}
