\documentclass[reqno]{amsart} \input{common.tex}

\begin{document}

\title{Aarhus Automorphic Forms Summer School}

\begin{abstract}
  Notes in progress on lectures taking place there.
\end{abstract}

\section{Gebhard Boeckle's lectures}
\subsection{Galois representations and congruences}

We first discuss profinite groups.  Let $G$ be a topological group.

\begin{theorem}
  The following are equivalent:
  \begin{enumerate}[(a)]
  \item\label{enumerate:cq6r3e8dsb} $G$ is compact, Hausdorff, and totally disconnected.
  \item\label{enumerate:cq6r3e8eqf} $G$ is compact, and admits a neighborhood basis of the identity by open normal subgroups.
  \item\label{enumerate:cq6r3e8f6g} There is a directed poset $I$ and an inverse system $(G_i)$ of finite (discrete) groups such that $G = \varprojlim_I G_i$.
  \end{enumerate}
\end{theorem}
We say that $G$ is \emph{profinite} if the above conditions hold.  The topology on $\varprojlim G_i$ is that obtained by regarding it as a closed subgroup of the product $\prod G_i$.

Constructions:
\begin{enumerate}[(a)]
\item\label{enumerate:cq6r3e8b0y} If $G$ is discrete, then we equip it with the profinite topology $G^{\mathrm{pf}} := \varprojlim G / N$, where $N$ runs over the finite index subgroups.
\item\label{enumerate:cq6r3e8cou} If $G = \varprojlim G_i$ is profinite, then
  \begin{enumerate}[(i)]
  \item\label{enumerate:cq6r3fehbe} The abelianization is given by
    \begin{equation*}
      G^{\mathrm{ab}} = G / \overline{[G, G]} = \varprojlim G_i^{\mathrm{ab}},
    \end{equation*}
    and in particular, is profinite.
  \item\label{enumerate:cq6r3feivc} For $H$ finite, write $H_p$ for its maximal $p$-group quotient.  Then
    \begin{equation*}
      G_p = \varprojlim(G_i)_p
    \end{equation*}
    is a pro-$p$-group (and in particular, profinite).
  \item\label{enumerate:cq6r3fej61} If $N \leq G$ is closed and normal, then $G /N$ is profinite.
  \end{enumerate}
\end{enumerate}

\begin{example}
  \begin{enumerate}[(a)]
  \item\label{enumerate:cq6r3fjkt4} Let $F$ be a field.  Set $G_F := \Aut_F(F^{\mathrm{sep}}) = \Gal(F^{\mathrm{sep}} / F)$ profinite.  Define the poset
    \begin{equation*}
      \mathcal{I}_F := \left\{ L \subseteq F^{\mathrm{sep}} : L \supseteq F \text{ finite Galois}, \subseteq \right\}.      
    \end{equation*}
    Then
    \begin{equation*}
      G_F \xrightarrow{\cong} \varprojlim_{L \in \mathcal{I}_F} \Gal(L / F).
    \end{equation*}
  \item\label{enumerate:cq6r3fm1jw} Let $F' \subseteq F^{\mathrm{sep}}$ be a normal extension of $F$.  Then $G_{F'} \leq G_F$ is closed and normal.  We may thus write
    \begin{equation*}
      \Gal(F ' / F) \cong G_F / G_{F'} = \lim_{
        \substack{
          L \in \mathcal{I}_F,  \\
          L \subseteq F'          
        }
      }
      \Gal(L /F).
    \end{equation*}
  \item\label{enumerate:cq6r3fm4tn} Let $\mathbb{N}$ denote the natural numbers, ordered by divisibility.  Then
    \begin{equation*}
      \hat{\mathbb{Z}} = \varprojlim \mathbb{Z} / n = \prod_{p} \mathbb{Z}_p,
    \end{equation*}
    where the last step is the Chinese remainder theorem.  We sometimes need a slight modification:
    \begin{equation*}
      \hat{\mathbb{Z}}^{(p)} = \varprojlim_{p \nmid n} \mathbb{Z} / n = \prod_{\ell \text{ prime}, \\ \ell \neq p } \mathbb{Z}_{\ell}.
    \end{equation*}

    Let's fix some notation:
    \begin{enumerate}
    \item\label{enumerate:cq6r3f3t8d} Let $K$ be a number field, $\mathcal{O}_K$ its ring of integers.  Let $\mathrm{P l}_K = \mathrm{P l}_K^\infty \sqcup \mathrm{P l}_K^{\mathrm{fin}}$ denote the set of places $v$ of $K$.  Let $v$ be a finite place.  We may then attach to it a maximal ideal $\mathfrak{q}_v$ of $\mathcal{O}_K$, giving a bijection
      \begin{equation*}
        \mathrm{P l}_K^{\mathrm{fin}} \leftrightarrow \operatorname{Max}(\mathcal{O}_K).
      \end{equation*}
      We may form the residue field $k_v := \mathcal{O}_K / \mathfrak{q}_v$.  We denote $q_v$ for the cardinality of $k_v$.  We write $\operatorname{char}(v)$ for the characteristic of $k_v$.  We denote by $\mathcal{O}_v = \varprojlim \mathcal{O} / \mathfrak{q}_v^n$, with fraction field $K_v$.  Also, we have a short exact sequence
      \begin{equation*}
        1 \rightarrow I_v \rightarrow G_v := \Gal_{K_v} \rightarrow \Gal_{k_v} \rightarrow 1.
      \end{equation*}
      A topological generator for $\Gal_{k_v}$ is given by
      \begin{equation*}
        \mathrm{Fr}_v : \alpha \mapsto \alpha^{q_v}.
      \end{equation*}
      We denote by $\mathrm{Frob}_v \in G_v$ some lift of $\mathrm{Fr}_v$.

      We write $S_\infty := \mathrm{P l}_K^\infty$ for the set of archimedean places, so that $K \otimes_{\mathbb{Q}} \mathbb{R} \cong \prod_{v \in S_\infty} K_v$.  For a rational prime $p$, we write $S_p$ for the set of places $v$ of $K$ such that $v \mid p$.
    \item We also need some local analogues for $E \supseteq \mathbb{Q}_p$ a $p$-adic field.  Let $\mathcal{O} = \mathcal{O}_E$ denote the ring of integers, $\pi = \pi_E$ a uniformizer, and $\mathbb{F} = \mathcal{O}_E / \pi$ the residue field, with $q = \# \mathbb{F}$.  Then $E \supseteq \mathbb{Q}_q = \mathbb{Q}_p[\zeta_{q - 1}] \supseteq \mathbb{Q}_p$.  We have $W(\mathbb{F}) = \mathbb{Z}_q = \mathbb{Z}_p[\zeta_{q - 1}]$.
    \end{enumerate}

    Contiuing the examples, which may serve as exercises:
    
  \item\label{enumerate:cq6r3g5d9c} Let $\zeta_t$ be a primitive $t$th root of $1$.  For $k$ a finite field, we have $G_k \cong \hat{\mathbb{Z}} = \overline{\langle \mathrm{Fr}_k \rangle}$, where $\mathrm{Fr}_k : \alpha \mapsto \alpha^{\lvert k \rvert}$.
  \item\label{enumerate:cq6r3g5e7y} Let $E \supseteq \mathbb{Q}_p$ (finite extension).  Then $G_E$ (Jannsen--Wingberg for $p \geq 2$).  Local class field theory: the Artin map $E^\times \rightarrow G_E^{\mathrm{ab}}$ is a continuous inclusion with dense image.  Writing $E^\times = \pi_E^{\mathbb{Z}} \times \mathcal{O}_E^\times = \pi_E^{\mathbb{Z}} \times \mathbb{F}^\times \times \mathcal{U}_E^1$.  Since the units are known to be a finitely generated $\mathbb{Z}_p$-module, we get as a corollary that
    \begin{equation*}
      \Hom_{\mathrm{cts}}(G_E, \mathbb{F}_p) = H^1_{\mathrm{cts}}(G_E, \mathbb{F}_p)
    \end{equation*}
    is finite.
  \item\label{enumerate:cq6r3g5cxb} We turn to the case of a number field $K$.  We fix an embedding $K^{\mathrm{sep}} \subseteq K_v^{\mathrm{sep}}$ for each place $v$, which gives an embedding of Galois groups $G_v \rightarrow G_K$.  For $S \subseteq \mathrm{P l}_K$ finite, we write
    \begin{equation*}
      K_S := \left\{ \alpha \in K^{\mathrm{sep}} : K(\alpha) \text{ is unramified outside } S \right\},
    \end{equation*}
    which is a normal (typically infinite) extension of $K$.  We write
    \begin{equation*}
      G_{K, S} := \Gal(K_S / K) = G_K / G_{K_S}
    \end{equation*}
    for its Galois group.  We remark that if we take $v \notin S$, then since $v$ does not ramify in $K_S$, we know that the map $G_v \rightarrow G_{K, S}$ factors via the quotient $G_v / I_v \cong G_{k_v}$, so that $\mathrm{Frob}_v \in G_{K, S}$ is independent of the choice of lift.  On the other hand, if $v \in S$, then we might ask whether the map $G_v \hookrightarrow G_{K, S}$ (see the work of Cheniever--Clozel).  The structure of $G_{K, S}$ is unknown, but global class field theory describes $G_{K, S}^{\mathrm{ab}}$.  A corollary is that
    \begin{equation*}
      H^{1}_{\mathrm{cts}}(G_{K, S}, \mathbb{F}_p) = \Hom_{\mathrm{cts}}(G_{K, S}, \mathbb{F}_p)
    \end{equation*}
    is finite whenever $S$ is finite.  (One can appeal to Hermite--Minkowski, or class field theory.)
  \item\label{enumerate:cq6r3g5bmp} Consider the tame quotient of $G_E$, for $E \supseteq \mathbb{Q}_p$.  Given $E \supseteq \mathbb{Q}_p$, we form the tower of extensions $E^{\mathrm{tame}} / E^{\mathrm{unr}} / E$, where
    \begin{equation*}
      E^{\mathrm{unr}} = \cup \left\{ E(\zeta_n) : p \nmid n \right\},
    \end{equation*}
    \begin{equation*}
      E^{\mathrm{tame}} = \cup \left\{ E^{\mathrm{unr}}(\sqrt[n]{\pi_E}) : p \nmid n \right\}.
    \end{equation*}
    It's a fact that $G_E^{\mathrm{tame}}$ may be expressed as the profinite completion of $\langle s t : s t s^{-1} = t^q \rangle$.
  \end{enumerate}
\end{example}
We finally come to \textbf{Galois representations}.  They will typically be called $\rho : G \rightarrow \mathrm{GL}_n(A)$, where $G$ is a topological group, $A$ is a topological ring, and $\rho$ is a continuous map.  The topology on $\mathrm{GL}_n(A)$ is the subspace topology coming from embedding inside $M_n(A) \times A$ via $g \mapsto(g, \det(g)^{-1})$, for instance.  We call $\rho$ a Galois representation if $G = G_F$ for some field $F$.  The main examples of interest for $A$ will be $\mathbb{C}$, finite fields, and $p$-adic fields, to interpolate $\mathrm{C N L}_{\mathcal{O}}$ (complete Noetherian local $\mathcal{O}$-algebras).

\begin{exercise}
  Let $G$ be profinite, and $\rho$ as above.
  \begin{enumerate}[(a)]
  \item If $A = \mathbb{C}$, then $\rho(G)$ is finite.
  \item If $A = \overline{\mathcal{O}_p}$, then there is a finite extension $E \supseteq \mathbb{Q}_p$ such that $\rho(G) \subseteq \mathrm{GL}_n(E)$ up to conjugation.
  \item\label{enumerate:cq6r3j15cy} If $A = E \supseteq \mathbb{Q}_p$ (finite extension), then after conjugation, we can assume that $\rho(G) \subseteq \mathrm{GL}_n(\mathcal{O})$.
  \end{enumerate}
\end{exercise}
In case \eqref{enumerate:cq6r3j15cy}, we have a $G$-stable lattice $\Lambda \cong \mathcal{O}^n \subseteq E^n$.  We can apply reduction $\mathcal{O} \rightarrow \mathbb{F}$.  This gives a reduction
\begin{equation*}
  \overline{\rho}_\Lambda : G \rightarrow \mathrm{GL}_n(\mathbb{F}).
\end{equation*}
Let's use the notation $\mathrm{c p}_\alpha$ for the characteristic polynomial of $\alpha \in M_n(A)$.
\begin{theorem}
  \begin{enumerate}[(a)]
  \item Given a representation $r : G \rightarrow \mathrm{GL}_n(\mathbb{F})$.  Then there exists a semisimple representation $r^{\mathrm{ss}} : G \rightarrow \mathrm{GL}_n(\mathbb{F})$ such that $\mathrm{c p}_r = \mathrm{c p}_{r^{\mathrm{ss}}}$ (Brauer--Hesbitt), where $r^{\mathrm{ss}}$ is unique up to isomorphism.
  \item We have $\mathrm{c p}_{\rho} \in \mathcal{O}[X]$ and $\mathrm{c p}_{\bar{\rho}_\Lambda} \in \mathbb{F}[X]$, independent of $\Lambda$.
  \end{enumerate}
\end{theorem}
\begin{theorem}
  For $\rho, \rho ' : G_{K, S} \rightarrow \mathrm{GL}_n(E)$ semisimple, we have that $\rho \sim \rho '$ (conjugate) if and only if  for all $v \in \mathrm{P l}_K^{\mathrm{fin}} \setminus S$, we have
  \begin{equation*}
    \mathrm{c p}_{\rho(\Frob_v)} = \mathrm{c p}_{\rho '(\Frob_v)}.
  \end{equation*}
\end{theorem}

\bibliography{refs}{} \bibliographystyle{plain}
\end{document}
