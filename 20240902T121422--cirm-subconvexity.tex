\documentclass[reqno]{amsart} \input{common.tex}

\begin{document}


\section{Some things to think about in the afternoons and beyond}

\begin{enumerate}
\item Understand the details of the consequences of subconvexity mentioned in lecture (reference: \cite[\S5]{MR2331346}):
  \begin{itemize}
  \item Subconvexity vs.\ geometry of numbers.
    % For a fundamental discriminant $d < 0$, $K = \mathbb{Q}(\sqrt{d})$, $H < \mathrm{Cl}(K)$ of index $\lvert d \rvert^s$ with $s > 0$ small enough, we can find an ideal $\mathfrak{a}$ with $[\mathfrak{a}] \subseteq H$ such that $\mathrm{Nr}(\mathfrak{a}) \leq \lvert d \rvert^{1/2 - s '}$ for some fixed $s' > 0$.
  \item Distinguishing modular forms.
  \item Duke theorem.  Reduction of supersingular elliptic curves.
  \item Quantum unique ergodicity.  One exercise here is to understand how this follows from subconvexity in the special case of Eisenstein series, as in \cite[\S2]{MR1361757}.
  \end{itemize}

\item Some ``classic'' papers (non-exhaustive):
  \begin{itemize}
  \item Bounds for Fourier coefficients of half-integral weight, applications to quadratic forms: \cite{MR870736}, \cite{MR931205}, \cite{MR1437494}
  \item Subconvexity for $\GL_2$: \cite{DFI93} \cite{DFI94} \cite{DFI01}.
  \item Moments and amplification via periods:  \cite{venkatesh-2005}, \cite{michel-2009}, \cite{iwan-sar}, \cite{MR780071}.
  \item Shifted convolution sums:
    \begin{itemize}
    \item via $\delta$-symbol: \cite{DFI93}
    \item via periods: \cite{Sar01}\cite{MR2437682}, \cite{2024arXiv2404.10692}.
    \end{itemize}
  \item Papers emphasizing variation of the test vector:   \cite{MR2373356, MR2726097, venkatesh-2005}.
  \end{itemize}
  One exercise is draw parallels, e.g., between
  \begin{itemize}
  \item \cite{michel-2009}[Thm 5.1] and \cite{DFI93},
  \item \cite[\S4]{venkatesh-2005} and \cite{DFI94}, or
  \item \cite{michel-2009}[Thm 5.2] and \cite{DFI01}.
  \end{itemize}
  Another is to reprove some results using different methods, e.g., by working out a ``classical'' proof in the style of \cite{DFI01} for subconvexity for Maass forms at special points, namely, for $L(1/2 + i t_f, f)$ with $f$ on $\SL_2(\mathbb{Z})$ of eigenvalue $1/4 + t_f^2$, by estimating an amplified fourth moment, e.g.,
  \begin{equation*}
    \sum_{f : t_f \in [T, T+1]}
    \left| \sum_{\ell \asymp L}
      c_{\ell}   \lambda_f(\ell)\right|^2
    \left| L(\tfrac{1}{2} + i t_f, f) \right|^4.    
  \end{equation*}

\item Study the proof of the convexity bound.  There are two steps:
  \begin{itemize}
  \item The Phragmen--Lindel\"{o}f convexity principle, to reduce estimates for $\Re(s) = 1/2$ to estiamtes for $\Re(s) = 1 + \eps$ and $\Re(s) = - \eps$.
  \item The functional equation, to reduce further to estimates for $\Re(s) = 1 + \eps$.
  \item Establishing the necessary bounds for $\Re(s) = 1 + \eps$, for which see \url{https://www.math.wsu.edu/faculty/scliu/papers/Convexity.pdf} and references.
  \end{itemize}

\item Some recent papers, concerning subconvexity or related problems, that haven't been fully explored (e.g., interpreted via integral representations):
  \begin{itemize}
  \item $\delta$-method papers such as \cite{MR4416133} and \cite{2022arXiv2206.06517}
  \item Higher moments over very large families, as in \cite{MR4216694}, \cite{MR4067357}
  \item Rankin--Selberg when the rank difference is larger than one, as in \cite{MR3996341}
  \end{itemize}

\item Higher rank subconvex bounds \cite{MR4203038, 2023arXiv2309.16667, 2020arXiv201202187N, 2021arXiv210915230N, 2023arXiv2309.06314}.  There are many ``exercises'' implicit in these papers; for instance, a half-dozen are suggested in \cite[Remark 1.4]{2020arXiv201202187N}.  Some other questions:
  \begin{itemize}
  \item These have all proceeded via arithmetic amplification.  Is it possible to succeed in some cases via ``family shortening'' (as in, e.g., \cite{Sar01})?  A natural case to try would be the $t$-aspect.  Some experiments with $\GL_2$ suggest this is difficult (see \url{https://ultronozm.github.io/math/20230522T174726__shrinking-archimedean-families-second-moment-gl2.html}).
  \end{itemize} ``Purely horizontal'' aspects remain open, e.g., twists by Dirichlet characters of prime conductor on $\GL_4$.
\end{enumerate}


\bibliography{refs}{} \bibliographystyle{plain}
\end{document}
