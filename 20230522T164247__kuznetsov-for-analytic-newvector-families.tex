\documentclass[reqno]{amsart} \input{common.tex}

\title{Kuznetsov heuristic for analytic newvector families}

\author{PN, TH}

\begin{document}

\begin{abstract}
  Writing down the shape of the Kuznetsov formula for analytic newvector families.
\end{abstract}

We consider the family of all $\pi$ on $\PGL_2(\mathbb{Z}) \backslash \PGL_2(\mathbb{R})$ with $C(\pi) \leq Q$, and aim to write down the shape of the Kuznetsov formula for
\begin{equation*}
  \frac{1}{Q}\sum _{C(\pi) \leq Q} \lambda(m) \lambda(n)
\end{equation*}
in the range
\begin{equation*}
m, n \asymp T,
\end{equation*}
where $T$ and $Q$ are parameters at our disposal.

We pick off the family using $f_0$, the normalized characteristic function of the archimedean variant $K_0(Q)$ of the standard congruence subgroup, like \cite{JN19a}:
\begin{equation*}
  K_0(Q)
  = \left\{
    \begin{pmatrix}
a & b \\
c & d \\
    \end{pmatrix}
    :
    a = 1 + o(1), \quad
    b = o(1),
    \quad
    c \lll 1/Q,
    \quad
    d = 1 + o(1)
  \right\}.
\end{equation*}
Then we have the pretrace formula
\begin{equation*}
  \sum _{\gamma \in \Gamma } f_0 (x  ^{-1} \gamma y)
  =
  \sum _{C(\pi) \leq Q } \sum _{
    \substack{
      \varphi \in \mathcal{B}(\pi) :  \\
      \text{analytic newvector}
    }
  } \varphi(x) \overline{\varphi(y)}.
\end{equation*}
Each such $\varphi$ will have roughly the Fourier expansion
\begin{equation*}
  \varphi( n(x) a(y)) = \sum _{n \geq 1} \frac{\lambda(n)}{ \sqrt{n} } W(n y) e(n x).
\end{equation*}
Here the test function $W(n y)$ roughly detects $n y \asymp 1$, i.e., $n \asymp 1/y$.  So if we integrate the above against $e(-m x)$ over $x \in \mathbb{R} / \mathbb{Z}$, we get
\begin{equation*}
  \int _{x \in \mathbb{R} / \mathbb{Z} } \varphi(n(x) a(y)) e(-n x) \, d x
  =
  1 _{n \asymp 1/y} \frac{\lambda(n)}{ \sqrt{n}}.
\end{equation*}
We choose $y := 1/T$.  Then we have the following:
\begin{equation*}
  \int _{u, v \in \Gamma_N \backslash N} \varphi(u a(1/T)) e(-m u) \overline{\varphi(v a(1/T)) e(-n v)}
  \approx
  1 _{m, n \asymp T} \frac{1}{ \sqrt{m n}} \lambda(m) \lambda(n).  
\end{equation*}
Thus for $m, n \asymp T$,
\begin{equation*}
  \sum _{C(\pi) \leq Q} \frac{\lambda(m) \lambda(n)}{\sqrt{m n}}
  \approx \int _{u , v \in \Gamma_N \backslash N} e(-m u + n v) \sum _{\gamma \in \Gamma } f _0 (a(T) u ^{-1} \gamma v a(1/T)) \, d u \, d v.
\end{equation*}
Here we think
\begin{equation*}
  u \leftrightarrow
  \begin{pmatrix}
    1 & u \\
    0 & 1
  \end{pmatrix},
  \quad
  v \leftrightarrow
  \begin{pmatrix}
    1 & v \\
    0 & 1
  \end{pmatrix}.
\end{equation*}
Now we do the Bruhat decomposition.  First, let's consider the diagonal contribution from $\gamma \in \Gamma _N$.  This unfolds to
\begin{equation*}
  1 _{m = n} \int _{u \in N} f _0 (a(T) u a(1/T)) e(m u) \, d u,
\end{equation*}
which for our $f_0$, which we recall has the shape
\begin{equation*}
  f_0 \approx Q 1 _{K_0(Q)}, \quad K_0(Q) = G \cap \left( 1 +
    \begin{pmatrix}
      o(1) & o(1) \\
      o(1/Q) & o(1)
    \end{pmatrix} \right).
\end{equation*}
should have size $\asymp Q/T$, because the upper-right entry of the following is $o(1/T)$:
\begin{equation*}
  J := a(1/T) K_0(Q) a(T) \subseteq \left( 1 +
    \begin{pmatrix}
      o(1) & o(1/T) \\
      o(T/Q) & o(1)
    \end{pmatrix} \right).
\end{equation*}
This is as expected, because for $n \asymp T$,
\begin{equation*}
  \sum _{C(\pi) \leq Q} \frac{|\lambda(n)|^2}{n} \approx \frac{Q}{T}.
\end{equation*}

Next we consider the off-diagonal.  So we write, for $\gamma \in \SL_2(\mathbb{Z})$ with
\begin{equation*}
  \gamma =
  \begin{pmatrix}
    a & b \\
    c & d
  \end{pmatrix} \text{ with } c \neq 0,
\end{equation*}
\begin{equation*}
  \gamma =
  \begin{pmatrix}
    1 & a/c \\
    0 & 1
  \end{pmatrix}
  \begin{pmatrix}
    0 & -1/c \\
    c & 0
  \end{pmatrix}
  \begin{pmatrix}
    1 & d/c \\
    0 & 1
  \end{pmatrix}
  =:
  n(a/c) w(c) n(d/c).
\end{equation*}
The off-diagonal contribution is
\begin{equation*}
  Q
  \sum _{c \neq 0}
  \sum _{a, d \in \mathbb{Z} : a d \equiv 1(c)}
  \int _{u, v \in \mathbb{R} / \mathbb{Z} }
  e (- m u + n v )
  1 _{
    n(a/c+u) w(c) n(d/c + v)
    \in
    J
  }
  \,d u \, d v.
\end{equation*}
We want to recognize Kloosterman sums and then evaluate everything else.  To that end, we split the sum over $a$ and $d$ into arithmetic progressions modulo $c$:
\begin{itemize}
\item $a = a_0 + c a_1$,
\item $d = d_0 + c d_1$,
\end{itemize}
where $a_0, d_0 \in (\mathbb{Z} / c)^\times$ and $a_1, d_1 \in \mathbb{Z}$.  Then the above is rewritten as
\begin{equation*}
  Q \sum _{c \neq 0} \sum _{
    \substack{
      a_0, d_0 \in (\mathbb{Z} / c)^\times :  \\
       a_0 d_0 \equiv  1(c)
    }
  }
  \sum _{a_1, d_1 \in \mathbb{Z} }
  \int _{u, v \in \mathbb{R} / \mathbb{Z} }
  e (- mu + n v )
  1 _{n (a_0/c +  a_1 - u) w(c) n(d_0/c + d_1 + v) \in J
  }
  \, d u \, d v.
\end{equation*}
Now we substitute $u \mapsto u - a_0/c$ and $v \mapsto v - d_0/c$:
\begin{equation*}
  Q \sum _{c \neq 0} \sum _{
    \substack{
      a_0, d_0 \in (\mathbb{Z} / c)^\times :  \\
       a_0 d_0 \equiv  1(c)
    }
  }
  e_c(-m a_0 - n d_0)
  \sum _{a_1, d_1 \in \mathbb{Z} }
  \int _{u, v \in \mathbb{R} / \mathbb{Z} }
  e (- mu + n v )
  1 _{n (a_1 + u) w(c) n(d_1 + v) \in J
  }
  \, d u \, d v.
\end{equation*}
Now for this last integral, we unfold $(a_1,u)$ and $(d_1,v)$:
\begin{equation*}
  Q \sum _{c \neq 0} \sum _{
    \substack{
      a_0, d_0 \in (\mathbb{Z} / c)^\times :  \\
       a_0 d_0 \equiv  1(c)
    }
  }
  e_c(-m a_0 - n d_0)
  \int _{u, v \in \mathbb{R}}
  e (- mu + n v )
  1 _{n (u) w(c) n(v) \in J
  }
  \, d u \, d v.
\end{equation*}
Now we rewrite this as
\begin{equation*}
  Q \sum _{c \neq 0}
  S(m,n,c)
  I(m,n,c),
\end{equation*}
where
\begin{equation}\label{eqn:20230522164836}
  I(m,n,c) :=
  \int _{u, v \in \mathbb{R}}
  e (- mu + n v )
  1 _{n (u) w(c) n(v) \in J
  }
  \, d u \, d v.
\end{equation}
So in summary, we have shown thus far that for $m, n \asymp T$,
\begin{equation*}
  \sum _{C(\pi) \leq Q}
  \frac{\lambda(m) \lambda(n)}{ \sqrt{m n }}
  \approx
  1 _{m = n} \frac{Q}{T}
  + Q \sum _{c \neq 0} S (m, n, c) I (m,n ,c).
\end{equation*}
In other words,
\begin{equation}\label{eqn:20230522165006}
  \frac{1}{Q}
  \sum _{C(\pi) \leq Q}
  \lambda(m) \lambda(n)
  \approx
  1 _{m = n}
  + T \sum _{c \neq 0} S (m, n, c) I (m,n ,c).
\end{equation}


To study \eqref{eqn:20230522164836}, we apply the matrix multiplication identity
\begin{equation*}
\begin{pmatrix}
1 & u/c \\
0 & 1 \\
\end{pmatrix}
\begin{pmatrix}
0 & -1/c \\
c & 0 \\
\end{pmatrix}
\begin{pmatrix}
1 & v/c \\
0 & 1 \\
\end{pmatrix}
=\begin{pmatrix}u & \frac{u v}{c} - \frac{1}{c}\\c & v\end{pmatrix}.
\end{equation*}
For this to lie in $J$, we should have
\begin{equation*}
u = o(1), \quad v = o(1), \quad c \lll T/Q,
\end{equation*}
say $c \asymp T / Q$, and then
\begin{equation*}
u v - 1 \ll 1/Q.
\end{equation*}
This essentially detects $v = 1 / u$ but we save a factor of $1/Q$ from the volume of the set of relevant $v$.  We arrive at, for $c \asymp T/Q$,
\begin{equation*}
  I(m,n,c)
  =
  \frac{1}{c^2 Q}
  \int_{u = 1  + o(1)}^{\text{smooth}}
  e \left( \frac{m u + n /u }{ c } \right) \, d u \, d v.
\end{equation*}
We have, as functions of $u$,
\begin{equation*}
(m u + n u^{-1} ) ' = m - n u^{-2}. 
\end{equation*}
The stationary points are
\begin{equation*}
  u_0 = \pm \sqrt{n/m}.
\end{equation*}
Near that point, we can approximate the phase using its second degree Taylor expansion.  The second derivative is
\begin{equation*}
(m u + n u^{-1} )'' = \frac{1}{2} n u^{-3}.
\end{equation*}
The value of the phase is
\begin{equation*}
  \frac{m u_0 + n u_0^{-1} }{c}
  =
  \pm \frac{m \sqrt{n / m } + n \sqrt{m / n} }{c}
  =
  \pm 2 \frac{\sqrt{m n }}{c}.
\end{equation*}
The second degree Taylor expansion then looks like
\begin{equation*}
  \frac{m u + n u^{-1} }{c}
  \approx
  \pm 2 \frac{\sqrt{m n }}{c}
  \pm \frac{1}{2} \frac{n u_0^{-3}}{2 c} (u - u_0)^2.
\end{equation*}
So we're reduced to the Fresnel integral
\begin{equation*}
\int_{u} e \left( \frac{1}{2} \frac{n u_0^{-3}}{2 c} u^2 \right) \, d u.
\end{equation*}
Doing $u \mapsto u (c/n)^{1/2}$ gets rid of the part of the phase that is not $\asymp 1$, so the above has size
\begin{equation*}
\asymp (c / n)^{1/2}.
\end{equation*}
So we arrive at
\begin{equation*}
I(m,n,c) \approx \frac{1}{c^2 Q} \frac{c^{1/2} }{ n^{1/2} } e \left( \pm 2 \frac{\sqrt{m n }}{c} \right).
\end{equation*}
For $c \asymp T/Q$ and $n \asymp T$, the above has magnitude
\begin{equation*}
\frac{1}{(T/Q)^2 Q^{3/2}} = \frac{ Q^{1/2}}{T^2}.
\end{equation*}
Substituting back into \eqref{eqn:20230522165006} gives now
\begin{equation*}
 \frac{1}{Q}
  \sum _{C(\pi) \leq Q}
  \lambda(m) \lambda(n)
  \approx
  1 _{m = n}
  +
  \frac{ Q^{1/2}}{T }
  \sum_{c \asymp T/Q}
  S(m,n,c) e \left( \pm 2 \frac{\sqrt{m n} }{c} \right).
\end{equation*}


\bibliography{refs}{} \bibliographystyle{plain}
\end{document}
