\documentclass[reqno]{amsart} \usepackage{graphicx, amsmath, amssymb, amsfonts, amsthm, stmaryrd, amscd}
\usepackage[usenames, dvipsnames]{xcolor}
\usepackage{tikz}
% \usepackage{tikzcd}
% \usepackage{comment}

% \let\counterwithout\relax
% \let\counterwithin\relax
% \usepackage{chngcntr}

\usepackage{enumerate}
% \usepackage{enumitem}
% \usepackage{times}
\usepackage[normalem]{ulem}
% \usepackage{minted}
% \usepackage{xypic}
% \usepackage{color}


% \usepackage{silence}
% \WarningFilter{latex}{Label `tocindent-1' multiply defined}
% \WarningFilter{latex}{Label `tocindent0' multiply defined}
% \WarningFilter{latex}{Label `tocindent1' multiply defined}
% \WarningFilter{latex}{Label `tocindent2' multiply defined}
% \WarningFilter{latex}{Label `tocindent3' multiply defined}
\usepackage{hyperref}
% \usepackage{navigator}


% \usepackage{pdfsync}
\usepackage{xparse}


\usepackage[all]{xy}
\usepackage{enumerate}
\usetikzlibrary{matrix,arrows,decorations.pathmorphing}



\makeatletter
\newcommand*{\transpose}{%
  {\mathpalette\@transpose{}}%
}
\newcommand*{\@transpose}[2]{%
  % #1: math style
  % #2: unused
  \raisebox{\depth}{$\m@th#1\intercal$}%
}
\makeatother


\makeatletter
\newcommand*{\da@rightarrow}{\mathchar"0\hexnumber@\symAMSa 4B }
\newcommand*{\da@leftarrow}{\mathchar"0\hexnumber@\symAMSa 4C }
\newcommand*{\xdashrightarrow}[2][]{%
  \mathrel{%
    \mathpalette{\da@xarrow{#1}{#2}{}\da@rightarrow{\,}{}}{}%
  }%
}
\newcommand{\xdashleftarrow}[2][]{%
  \mathrel{%
    \mathpalette{\da@xarrow{#1}{#2}\da@leftarrow{}{}{\,}}{}%
  }%
}
\newcommand*{\da@xarrow}[7]{%
  % #1: below
  % #2: above
  % #3: arrow left
  % #4: arrow right
  % #5: space left 
  % #6: space right
  % #7: math style 
  \sbox0{$\ifx#7\scriptstyle\scriptscriptstyle\else\scriptstyle\fi#5#1#6\m@th$}%
  \sbox2{$\ifx#7\scriptstyle\scriptscriptstyle\else\scriptstyle\fi#5#2#6\m@th$}%
  \sbox4{$#7\dabar@\m@th$}%
  \dimen@=\wd0 %
  \ifdim\wd2 >\dimen@
    \dimen@=\wd2 %   
  \fi
  \count@=2 %
  \def\da@bars{\dabar@\dabar@}%
  \@whiledim\count@\wd4<\dimen@\do{%
    \advance\count@\@ne
    \expandafter\def\expandafter\da@bars\expandafter{%
      \da@bars
      \dabar@ 
    }%
  }%  
  \mathrel{#3}%
  \mathrel{%   
    \mathop{\da@bars}\limits
    \ifx\\#1\\%
    \else
      _{\copy0}%
    \fi
    \ifx\\#2\\%
    \else
      ^{\copy2}%
    \fi
  }%   
  \mathrel{#4}%
}
\makeatother
% \DeclareMathOperator{\rg}{rg}

\usepackage{mathtools}
\DeclarePairedDelimiter{\paren}{(}{)}
\DeclarePairedDelimiter{\abs}{\lvert}{\rvert}
\DeclarePairedDelimiter{\norm}{\lVert}{\rVert}
\DeclarePairedDelimiter{\innerproduct}{\langle}{\rangle}
\newcommand{\Of}[2]{{\operatorname{#1}} {\paren*{#2}}}
\newcommand{\of}[2]{{{{#1}} {\paren*{#2}}}}

\DeclareMathOperator{\Shim}{Shim}
\DeclareMathOperator{\sgn}{sgn}
\DeclareMathOperator{\fdeg}{fdeg}
\DeclareMathOperator{\SL}{SL}
\DeclareMathOperator{\slLie}{\mathfrak{s}\mathfrak{l}}
\DeclareMathOperator{\soLie}{\mathfrak{s}\mathfrak{o}}
\DeclareMathOperator{\spLie}{\mathfrak{s}\mathfrak{p}}
\DeclareMathOperator{\glLie}{\mathfrak{g}\mathfrak{l}}
\newcommand{\pn}[1]{{\color{ForestGreen} \sf PN: [#1]}}
\DeclareMathOperator{\Mp}{Mp}
\DeclareMathOperator{\Mat}{Mat}
\DeclareMathOperator{\GL}{GL}
\DeclareMathOperator{\Gr}{Gr}
\DeclareMathOperator{\GU}{GU}
\def\gl{\mathfrak{g}\mathfrak{l}}
\DeclareMathOperator{\odd}{odd}
\DeclareMathOperator{\even}{even}
\DeclareMathOperator{\GO}{GO}
\DeclareMathOperator{\good}{good}
\DeclareMathOperator{\bad}{bad}
\DeclareMathOperator{\PGO}{PGO}
\DeclareMathOperator{\htt}{ht}
\DeclareMathOperator{\height}{height}
\DeclareMathOperator{\Ass}{Ass}
\DeclareMathOperator{\coheight}{coheight}
\DeclareMathOperator{\GSO}{GSO}
\DeclareMathOperator{\SO}{SO}
\DeclareMathOperator{\so}{\mathfrak{s}\mathfrak{o}}
\DeclareMathOperator{\su}{\mathfrak{s}\mathfrak{u}}
\DeclareMathOperator{\ad}{ad}
% \DeclareMathOperator{\sc}{sc}
\DeclareMathOperator{\Ad}{Ad}
\DeclareMathOperator{\disc}{disc}
\DeclareMathOperator{\inv}{inv}
\DeclareMathOperator{\Pic}{Pic}
\DeclareMathOperator{\uc}{uc}
\DeclareMathOperator{\Cl}{Cl}
\DeclareMathOperator{\Clf}{Clf}
\DeclareMathOperator{\Hom}{Hom}
\DeclareMathOperator{\hol}{hol}
\DeclareMathOperator{\Heis}{Heis}
\DeclareMathOperator{\Haar}{Haar}
\DeclareMathOperator{\h}{h}
\def\sp{\mathfrak{s}\mathfrak{p}}
\DeclareMathOperator{\heis}{\mathfrak{h}\mathfrak{e}\mathfrak{i}\mathfrak{s}}
\DeclareMathOperator{\End}{End}
\DeclareMathOperator{\JL}{JL}
\DeclareMathOperator{\image}{image}
\DeclareMathOperator{\red}{red}
\def\div{\operatorname{div}}
\def\eps{\varepsilon}
\def\cHom{\mathcal{H}\operatorname{om}}
\DeclareMathOperator{\Ops}{Ops}
\DeclareMathOperator{\Symb}{Symb}
\def\boldGL{\mathbf{G}\mathbf{L}}
\def\boldSO{\mathbf{S}\mathbf{O}}
\def\boldU{\mathbf{U}}
\DeclareMathOperator{\hull}{hull}
\DeclareMathOperator{\LL}{LL}
\DeclareMathOperator{\PGL}{PGL}
\DeclareMathOperator{\class}{class}
\DeclareMathOperator{\lcm}{lcm}
\DeclareMathOperator{\spann}{span}
\DeclareMathOperator{\Exp}{Exp}
\DeclareMathOperator{\ext}{ext}
\DeclareMathOperator{\Ext}{Ext}
\DeclareMathOperator{\Tor}{Tor}
\DeclareMathOperator{\et}{et}
\DeclareMathOperator{\tor}{tor}
\DeclareMathOperator{\loc}{loc}
\DeclareMathOperator{\tors}{tors}
\DeclareMathOperator{\pf}{pf}
\DeclareMathOperator{\smooth}{smooth}
\DeclareMathOperator{\prin}{prin}
\DeclareMathOperator{\Kl}{Kl}
\newcommand{\kbar}{\mathchar'26\mkern-9mu k}
\DeclareMathOperator{\der}{der}
% \DeclareMathOperator{\abs}{abs}
\DeclareMathOperator{\Sub}{Sub}
\DeclareMathOperator{\Comp}{Comp}
\DeclareMathOperator{\Err}{Err}
\DeclareMathOperator{\dom}{dom}
\DeclareMathOperator{\radius}{radius}
\DeclareMathOperator{\Fitt}{Fitt}
\DeclareMathOperator{\Sel}{Sel}
\DeclareMathOperator{\rad}{rad}
\DeclareMathOperator{\id}{id}
\DeclareMathOperator{\Center}{Center}
\DeclareMathOperator{\Der}{Der}
\DeclareMathOperator{\U}{U}
% \DeclareMathOperator{\norm}{norm}
\DeclareMathOperator{\trace}{trace}
\DeclareMathOperator{\Equid}{Equid}
\DeclareMathOperator{\Feas}{Feas}
\DeclareMathOperator{\bulk}{bulk}
\DeclareMathOperator{\tail}{tail}
\DeclareMathOperator{\sys}{sys}
\DeclareMathOperator{\atan}{atan}
\DeclareMathOperator{\temp}{temp}
\DeclareMathOperator{\Asai}{Asai}
\DeclareMathOperator{\glob}{glob}
\DeclareMathOperator{\Kuz}{Kuz}
\DeclareMathOperator{\Irr}{Irr}
\newcommand{\rsL}{ \frac{ L^{(R)}(\Pi \times \Sigma, \std, \frac{1}{2})}{L^{(R)}(\Pi \times \Sigma, \Ad, 1)}  }
\DeclareMathOperator{\GSp}{GSp}
\DeclareMathOperator{\PGSp}{PGSp}
\DeclareMathOperator{\BC}{BC}
\DeclareMathOperator{\Ann}{Ann}
\DeclareMathOperator{\Gen}{Gen}
\DeclareMathOperator{\SU}{SU}
\DeclareMathOperator{\PGSU}{PGSU}
% \DeclareMathOperator{\gen}{gen}
\DeclareMathOperator{\PMp}{PMp}
\DeclareMathOperator{\PGMp}{PGMp}
\DeclareMathOperator{\PB}{PB}
\DeclareMathOperator{\ind}{ind}
\DeclareMathOperator{\Jac}{Jac}
\DeclareMathOperator{\jac}{jac}
\DeclareMathOperator{\im}{im}
\DeclareMathOperator{\Aut}{Aut}
\DeclareMathOperator{\Int}{Int}
\DeclareMathOperator{\PSL}{PSL}
\DeclareMathOperator{\co}{co}
\DeclareMathOperator{\irr}{irr}
\DeclareMathOperator{\prim}{prim}
\DeclareMathOperator{\bal}{bal}
\DeclareMathOperator{\baln}{bal}
\DeclareMathOperator{\dist}{dist}
\DeclareMathOperator{\RS}{RS}
\DeclareMathOperator{\Ram}{Ram}
\DeclareMathOperator{\Sob}{Sob}
\DeclareMathOperator{\Sol}{Sol}
\DeclareMathOperator{\soc}{soc}
\DeclareMathOperator{\nt}{nt}
\DeclareMathOperator{\mic}{mic}
\DeclareMathOperator{\Gal}{Gal}
\DeclareMathOperator{\st}{st}
\DeclareMathOperator{\std}{std}
\DeclareMathOperator{\diag}{diag}
\DeclareMathOperator{\Sym}{Sym}
\DeclareMathOperator{\gr}{gr}
\DeclareMathOperator{\aff}{aff}
\DeclareMathOperator{\Dil}{Dil}
\DeclareMathOperator{\Lie}{Lie}
\DeclareMathOperator{\Symp}{Symp}
\DeclareMathOperator{\Stab}{Stab}
\DeclareMathOperator{\St}{St}
\DeclareMathOperator{\stab}{stab}
\DeclareMathOperator{\codim}{codim}
\DeclareMathOperator{\linear}{linear}
\newcommand{\git}{/\!\!/}
\DeclareMathOperator{\geom}{geom}
\DeclareMathOperator{\spec}{spec}
\def\O{\operatorname{O}}
\DeclareMathOperator{\Au}{Aut}
\DeclareMathOperator{\Fix}{Fix}
\DeclareMathOperator{\Opp}{Op}
\DeclareMathOperator{\opp}{op}
\DeclareMathOperator{\Size}{Size}
\DeclareMathOperator{\Save}{Save}
% \DeclareMathOperator{\ker}{ker}
\DeclareMathOperator{\coker}{coker}
\DeclareMathOperator{\sym}{sym}
\DeclareMathOperator{\mean}{mean}
\DeclareMathOperator{\elliptic}{ell}
\DeclareMathOperator{\nilpotent}{nil}
\DeclareMathOperator{\hyperbolic}{hyp}
\DeclareMathOperator{\newvector}{new}
\DeclareMathOperator{\new}{new}
\DeclareMathOperator{\full}{full}
\newcommand{\qr}[2]{\left( \frac{#1}{#2} \right)}
\DeclareMathOperator{\unr}{u}
\DeclareMathOperator{\ram}{ram}
% \DeclareMathOperator{\len}{len}
\DeclareMathOperator{\fin}{fin}
\DeclareMathOperator{\cusp}{cusp}
\DeclareMathOperator{\curv}{curv}
\DeclareMathOperator{\rank}{rank}
\DeclareMathOperator{\rk}{rk}
\DeclareMathOperator{\pr}{pr}
\DeclareMathOperator{\Transform}{Transform}
\DeclareMathOperator{\mult}{mult}
\DeclareMathOperator{\Eis}{Eis}
\DeclareMathOperator{\reg}{reg}
\DeclareMathOperator{\sing}{sing}
\DeclareMathOperator{\alt}{alt}
\DeclareMathOperator{\irreg}{irreg}
\DeclareMathOperator{\sreg}{sreg}
\DeclareMathOperator{\Wd}{Wd}
\DeclareMathOperator{\Weil}{Weil}
\DeclareMathOperator{\Th}{Th}
\DeclareMathOperator{\Sp}{Sp}
\DeclareMathOperator{\Ind}{Ind}
\DeclareMathOperator{\Res}{Res}
\DeclareMathOperator{\ini}{in}
\DeclareMathOperator{\ord}{ord}
\DeclareMathOperator{\osc}{osc}
\DeclareMathOperator{\fluc}{fluc}
\DeclareMathOperator{\size}{size}
\DeclareMathOperator{\ann}{ann}
\DeclareMathOperator{\equ}{eq}
\DeclareMathOperator{\res}{res}
\DeclareMathOperator{\pt}{pt}
\DeclareMathOperator{\src}{source}
\DeclareMathOperator{\Zcl}{Zcl}
\DeclareMathOperator{\Func}{Func}
\DeclareMathOperator{\Map}{Map}
\DeclareMathOperator{\Frac}{Frac}
\DeclareMathOperator{\Frob}{Frob}
\DeclareMathOperator{\ev}{eval}
\DeclareMathOperator{\pv}{pv}
\DeclareMathOperator{\eval}{eval}
\DeclareMathOperator{\Spec}{Spec}
\DeclareMathOperator{\Speh}{Speh}
\DeclareMathOperator{\Spin}{Spin}
\DeclareMathOperator{\GSpin}{GSpin}
\DeclareMathOperator{\Specm}{Specm}
\DeclareMathOperator{\Sphere}{Sphere}
\DeclareMathOperator{\Sqq}{Sq}
\DeclareMathOperator{\Ball}{Ball}
\DeclareMathOperator\Cond{\operatorname{Cond}}
\DeclareMathOperator\proj{\operatorname{proj}}
\DeclareMathOperator\Swan{\operatorname{Swan}}
\DeclareMathOperator{\Proj}{Proj}
\DeclareMathOperator{\bPB}{{\mathbf P}{\mathbf B}}
\DeclareMathOperator{\Projm}{Projm}
\DeclareMathOperator{\Tr}{Tr}
\DeclareMathOperator{\Type}{Type}
\DeclareMathOperator{\Prop}{Prop}
\DeclareMathOperator{\vol}{vol}
\DeclareMathOperator{\covol}{covol}
\DeclareMathOperator{\Rep}{Rep}
\DeclareMathOperator{\Cent}{Cent}
\DeclareMathOperator{\val}{val}
\DeclareMathOperator{\area}{area}
\DeclareMathOperator{\nr}{nr}
\DeclareMathOperator{\CM}{CM}
\DeclareMathOperator{\CH}{CH}
\DeclareMathOperator{\tr}{tr}
\DeclareMathOperator{\characteristic}{char}
\DeclareMathOperator{\supp}{supp}


\theoremstyle{plain} \newtheorem{theorem} {Theorem} \newtheorem{conjecture} [theorem] {Conjecture} \newtheorem{corollary} [theorem] {Corollary} \newtheorem{proposition} [theorem] {Proposition} \newtheorem{fact} [theorem] {Fact}
\theoremstyle{definition} \newtheorem{definition} [theorem] {Definition} \newtheorem{hypothesis} [theorem] {Hypothesis} \newtheorem{assumptions} [theorem] {Assumptions}
\newtheorem{example} [theorem] {Example}
\newtheorem{assertion}[theorem] {Assertion}
\newtheorem{note}[theorem] {Note}
\newtheorem{conclusion}[theorem] {Conclusion}
\newtheorem{claim}            {Claim}
\newtheorem{homework} {Homework}
\newtheorem{exercise} {Exercise}  \newtheorem{question}[theorem] {Question}    \newtheorem{answer} {Answer}  \newtheorem{problem} {Problem}    \newtheorem{remark} [theorem] {Remark}
\newtheorem{notation} [theorem]           {Notation}
\newtheorem{terminology}[theorem]            {Terminology}
\newtheorem{convention}[theorem]            {Convention}
\newtheorem{motivation}[theorem]            {Motivation}


\newtheoremstyle{itplain} % name
{6pt}                    % Space above
{5pt\topsep}                    % Space below
{\itshape}                   % Body font
{}                           % Indent amount
{\itshape}                   % Theorem head font
{.}                          % Punctuation after theorem head
{5pt plus 1pt minus 1pt}                       % Space after theorem head
% {.5em}                       % Space after theorem head
{}  % Theorem head spec (can be left empty, meaning ‘normal’)

% \theoremstyle{mytheoremstyle}


\theoremstyle{itplain} %--default
% \theoremheaderfont{\itshape}
% \newtheorem{lemma}{Lemma}
\newtheorem{lemma}[theorem]{Lemma}
% \newtheorem{lemma}{Lemma}[subsubsection]

\newtheorem*{lemma*}{Lemma}
\newtheorem*{proposition*}{Proposition}
\newtheorem*{definition*}{Definition}
\newtheorem*{example*}{Example}

\newtheorem*{results*}{Results}
\newtheorem{results} [theorem] {Results}


\usepackage[displaymath,textmath,sections,graphics]{preview}
\PreviewEnvironment{align*}
\PreviewEnvironment{multline*}
\PreviewEnvironment{tabular}
\PreviewEnvironment{verbatim}
\PreviewEnvironment{lstlisting}
\PreviewEnvironment*{frame}
\PreviewEnvironment*{alert}
\PreviewEnvironment*{emph}
\PreviewEnvironment*{textbf}



\title{Kuznetsov heuristic for analytic newvector families}

\author{PN, TH}

\begin{document}

\begin{abstract}
  Writing down the shape of the Kuznetsov formula for analytic newvector families.
\end{abstract}

We consider the family of all $\pi$ on $\PGL_2(\mathbb{Z}) \backslash \PGL_2(\mathbb{R})$ with $C(\pi) \leq Q$, and aim to write down the shape of the Kuznetsov formula for
\begin{equation*}
  \frac{1}{Q}\sum _{C(\pi) \leq Q} \lambda(m) \lambda(n)
\end{equation*}
in the range
\begin{equation*}
m, n \asymp T,
\end{equation*}
where $T$ and $Q$ are parameters at our disposal.

We pick off the family using $f_0$, the normalized characteristic function of the archimedean variant $K_0(Q)$ of the standard congruence subgroup, like \cite{JN19a}:
\begin{equation*}
  K_0(Q)
  = \left\{
    \begin{pmatrix}
a & b \\
c & d \\
    \end{pmatrix}
    :
    a = 1 + o(1), \quad
    b = 1 + o(1),
    \quad
    c \lll 1/Q,
    \quad
    d = 1 + o(1)
  \right\}.
\end{equation*}
Then we have the pretrace formula
\begin{equation*}
  \sum _{\gamma \in \Gamma } f_0 (x  ^{-1} \gamma y)
  =
  \sum _{C(\pi) \leq Q } \sum _{
    \substack{
      \varphi \in \mathcal{B}(\pi) :  \\
      \text{analytic newvector}
    }
  } \varphi(x) \overline{\varphi(y)}.
\end{equation*}
Each such $\varphi$ will have roughly the Fourier expansion
\begin{equation*}
  \varphi( n(x) a(y)) = \sum _{n \geq 1} \frac{\lambda(n)}{ \sqrt{n} } W(n y) e(n x).
\end{equation*}
Here the test function $W(n y)$ roughly detects $n y \asymp 1$, i.e., $n \asymp 1/y$.  So if we integrate the above against $e(-m x)$ over $x \in \mathbb{R} / \mathbb{Z}$, we get
\begin{equation*}
  \int _{x \in \mathbb{R} / \mathbb{Z} } \varphi(n(x) a(y)) e(-n x) \, d x
  =
  1 _{n \asymp 1/y} \frac{\lambda(n)}{ \sqrt{n}}.
\end{equation*}
We choose $y := 1/T$.  Then we have the following:
\begin{equation*}
  \int _{u, v \in \Gamma_N \backslash N} \varphi(u a(1/T)) e(-m u) \overline{\varphi(v a(1/T)) e(-n v)}
  \approx
  1 _{m, n \asymp T} \frac{1}{ \sqrt{m n}} \lambda(m) \lambda(n).  
\end{equation*}
Thus for $m, n \asymp T$,
\begin{equation*}
  \sum _{C(\pi) \leq Q} \frac{\lambda(m) \lambda(n)}{\sqrt{m n}}
  \approx \int _{u , v \in \Gamma_N \backslash N} e(-m u + n v) \sum _{\gamma \in \Gamma } f _0 (a(T) u ^{-1} \gamma v a(1/T)) \, d u \, d v.
\end{equation*}
Here we think
\begin{equation*}
  u \leftrightarrow
  \begin{pmatrix}
    1 & u \\
    0 & 1
  \end{pmatrix},
  \quad
  v \leftrightarrow
  \begin{pmatrix}
    1 & v \\
    0 & 1
  \end{pmatrix}.
\end{equation*}
Now we do the Bruhat decomposition.  First, let's consider the diagonal contribution from $\gamma \in \Gamma _N$.  This unfolds to
\begin{equation*}
  1 _{m = n} \int _{u \in N} f _0 (a(T) u a(1/T)) e(m u) \, d u,
\end{equation*}
which for our $f_0$, which we recall has the shape
\begin{equation*}
  f_0 \approx Q 1 _{K_0(Q)}, \quad K_0(Q) = G \cap \left( 1 +
    \begin{pmatrix}
      o(1) & o(1) \\
      o(1/Q) & o(1)
    \end{pmatrix} \right).
\end{equation*}
should have size $\asymp Q/T$, because the upper-right entry of the following is $o(1/T)$:
\begin{equation*}
  J := a(1/T) K_0(Q) a(T) \subseteq \left( 1 +
    \begin{pmatrix}
      o(1) & o(1/T) \\
      o(T/Q) & o(1)
    \end{pmatrix} \right).
\end{equation*}
This is as expected, because for $n \asymp T$,
\begin{equation*}
  \sum _{C(\pi) \leq Q} \frac{|\lambda(n)|^2}{n} \approx \frac{Q}{T}.
\end{equation*}

Next we consider the off-diagonal.  So we write, for $\gamma \in \SL_2(\mathbb{Z})$ with
\begin{equation*}
  \gamma =
  \begin{pmatrix}
    a & b \\
    c & d
  \end{pmatrix} \text{ with } c \neq 0,
\end{equation*}
\begin{equation*}
  \gamma =
  \begin{pmatrix}
    1 & a/c \\
    0 & 1
  \end{pmatrix}
  \begin{pmatrix}
    0 & -1/c \\
    c & 0
  \end{pmatrix}
  \begin{pmatrix}
    1 & d/c \\
    0 & 1
  \end{pmatrix}
  =:
  n(a/c) w(c) n(d/c).
\end{equation*}
The off-diagonal contribution is
\begin{equation*}
  Q
  \sum _{c \neq 0}
  \sum _{a, d \in \mathbb{Z} : a d \equiv 1(c)}
  \int _{u, v \in \mathbb{R} / \mathbb{Z} }
  e (- m u + n v )
  1 _{
    n(a/c+u) w(c) n(d/c + v)
    \in
    J
  }
  \,d u \, d v.
\end{equation*}
We want to recognize Kloosterman sums and then evaluate everything else.  To that end, we split the sum over $a$ and $d$ into arithmetic progressions modulo $c$:
\begin{itemize}
\item $a = a_0 + c a_1$,
\item $d = d_0 + c d_1$,
\end{itemize}
where $a_0, d_0 \in (\mathbb{Z} / c)^\times$ and $a_1, d_1 \in \mathbb{Z}$.  Then the above is rewritten as
\begin{equation*}
  Q \sum _{c \neq 0} \sum _{
    \substack{
      a_0, d_0 \in (\mathbb{Z} / c)^\times :  \\
       a_0 d_0 \equiv  1(c)
    }
  }
  \sum _{a_1, d_1 \in \mathbb{Z} }
  \int _{u, v \in \mathbb{R} / \mathbb{Z} }
  e (- mu + n v )
  1 _{n (a_0/c +  a_1 - u) w(c) n(d_0/c + d_1 + v) \in J
  }
  \, d u \, d v.
\end{equation*}
Now we substitute $u \mapsto u - a_0/c$ and $v \mapsto v - d_0/c$:
\begin{equation*}
  Q \sum _{c \neq 0} \sum _{
    \substack{
      a_0, d_0 \in (\mathbb{Z} / c)^\times :  \\
       a_0 d_0 \equiv  1(c)
    }
  }
  e_c(-m a_0 - n d_0)
  \sum _{a_1, d_1 \in \mathbb{Z} }
  \int _{u, v \in \mathbb{R} / \mathbb{Z} }
  e (- mu + n v )
  1 _{n (a_1 + u) w(c) n(d_1 + v) \in J
  }
  \, d u \, d v.
\end{equation*}
Now for this last integral, we unfold $(a_1,u)$ and $(d_1,v)$:
\begin{equation*}
  Q \sum _{c \neq 0} \sum _{
    \substack{
      a_0, d_0 \in (\mathbb{Z} / c)^\times :  \\
       a_0 d_0 \equiv  1(c)
    }
  }
  e_c(-m a_0 - n d_0)
  \int _{u, v \in \mathbb{R}}
  e (- mu + n v )
  1 _{n (u) w(c) n(v) \in J
  }
  \, d u \, d v.
\end{equation*}
Now we rewrite this as
\begin{equation*}
  Q \sum _{c \neq 0}
  S(m,n,c)
  I(m,n,c),
\end{equation*}
where
\begin{equation}\label{eqn:20230522164836}
  I(m,n,c) :=
  \int _{u, v \in \mathbb{R}}
  e (- mu + n v )
  1 _{n (u) w(c) n(v) \in J
  }
  \, d u \, d v.
\end{equation}
So in summary, we have shown thus far that for $m, n \asymp T$,
\begin{equation*}
  \sum _{C(\pi) \leq Q}
  \frac{\lambda(m) \lambda(n)}{ \sqrt{m n }}
  \approx
  1 _{m = n} \frac{Q}{T}
  + Q \sum _{c \neq 0} S (m, n, c) I (m,n ,c).
\end{equation*}
In other words,
\begin{equation}\label{eqn:20230522165006}
  \frac{1}{Q}
  \sum _{C(\pi) \leq Q}
  \lambda(m) \lambda(n)
  \approx
  1 _{m = n}
  + T \sum _{c \neq 0} S (m, n, c) I (m,n ,c).
\end{equation}


To study \eqref{eqn:20230522164836}, we apply the matrix multiplication identity
\begin{equation*}
\begin{pmatrix}
1 & u/c \\
0 & 1 \\
\end{pmatrix}
\begin{pmatrix}
0 & -1/c \\
c & 0 \\
\end{pmatrix}
\begin{pmatrix}
1 & v/c \\
0 & 1 \\
\end{pmatrix}
=\begin{pmatrix}u & \frac{u v}{c} - \frac{1}{c}\\c & v\end{pmatrix}.
\end{equation*}
For this to lie in $J$, we should have
\begin{equation*}
u = o(1), \quad v = o(1), \quad c \lll T/Q,
\end{equation*}
say $c \asymp T / Q$, and then
\begin{equation*}
u v - 1 \ll 1/Q.
\end{equation*}
This essentially detects $v = 1 / u$ but we save a factor of $1/Q$ from the volume of the set of relevant $v$.  We arrive at, for $c \asymp T/Q$,
\begin{equation*}
  I(m,n,c)
  =
  \frac{1}{c^2 Q}
  \int_{u = 1  + o(1)}^{\text{smooth}}
  e \left( \frac{m u + n /u }{ c } \right) \, d u \, d v.
\end{equation*}
We have, as functions of $u$,
\begin{equation*}
(m u + n u^{-1} ) ' = m - n u^{-2}. 
\end{equation*}
The stationary points are
\begin{equation*}
  u_0 = \pm \sqrt{n/m}.
\end{equation*}
Near that point, we can approximate the phase using its second degree Taylor expansion.  The second derivative is
\begin{equation*}
(m u + n u^{-1} )'' = \frac{1}{2} n u^{-3}.
\end{equation*}
The value of the phase is
\begin{equation*}
  \frac{m u_0 + n u_0^{-1} }{c}
  =
  \pm \frac{m \sqrt{n / m } + n \sqrt{m / n} }{c}
  =
  \pm 2 \frac{\sqrt{m n }}{c}.
\end{equation*}
The second degree Taylor expansion then looks like
\begin{equation*}
  \frac{m u + n u^{-1} }{c}
  \approx
  \pm 2 \frac{\sqrt{m n }}{c}
  \pm \frac{1}{2} \frac{n u_0^{-3}}{2 c} (u - u_0)^2.
\end{equation*}
So we're reduced to the Fresnel integral
\begin{equation*}
\int_{u} e \left( \frac{1}{2} \frac{n u_0^{-3}}{2 c} u^2 \right) \, d u.
\end{equation*}
Doing $u \mapsto u (c/n)^{1/2}$ gets rid of the part of the phase that is not $\asymp 1$, so the above has size
\begin{equation*}
\asymp (c / n)^{1/2}.
\end{equation*}
So we arrive at
\begin{equation*}
I(m,n,c) \approx \frac{1}{c^2 Q} \frac{c^{1/2} }{ n^{1/2} } e \left( \pm 2 \frac{\sqrt{m n }}{c} \right).
\end{equation*}
For $c \asymp T/Q$ and $n \asymp T$, the above has magnitude
\begin{equation*}
\frac{1}{(T/Q)^2 Q^{3/2}} = \frac{ Q^{1/2}}{T^2}.
\end{equation*}
Substituting back into \eqref{eqn:20230522165006} gives now
\begin{equation*}
 \frac{1}{Q}
  \sum _{C(\pi) \leq Q}
  \lambda(m) \lambda(n)
  \approx
  1 _{m = n}
  +
  \frac{ Q^{1/2}}{T }
  \sum_{c \asymp T/Q}
  S(m,n,c) e \left( \pm 2 \frac{\sqrt{m n} }{c} \right).
\end{equation*}


\bibliography{refs}{} \bibliographystyle{plain}
\end{document}
