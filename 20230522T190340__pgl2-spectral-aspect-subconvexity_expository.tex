\documentclass[reqno]{amsart} \usepackage{graphicx, amsmath, amssymb, amsfonts, amsthm, stmaryrd, amscd}
\usepackage[usenames, dvipsnames]{xcolor}
\usepackage{tikz}
% \usepackage{tikzcd}
% \usepackage{comment}

% \let\counterwithout\relax
% \let\counterwithin\relax
% \usepackage{chngcntr}

\usepackage{enumerate}
% \usepackage{enumitem}
% \usepackage{times}
\usepackage[normalem]{ulem}
% \usepackage{minted}
% \usepackage{xypic}
% \usepackage{color}


% \usepackage{silence}
% \WarningFilter{latex}{Label `tocindent-1' multiply defined}
% \WarningFilter{latex}{Label `tocindent0' multiply defined}
% \WarningFilter{latex}{Label `tocindent1' multiply defined}
% \WarningFilter{latex}{Label `tocindent2' multiply defined}
% \WarningFilter{latex}{Label `tocindent3' multiply defined}
\usepackage{hyperref}
% \usepackage{navigator}


% \usepackage{pdfsync}
\usepackage{xparse}


\usepackage[all]{xy}
\usepackage{enumerate}
\usetikzlibrary{matrix,arrows,decorations.pathmorphing}



\makeatletter
\newcommand*{\transpose}{%
  {\mathpalette\@transpose{}}%
}
\newcommand*{\@transpose}[2]{%
  % #1: math style
  % #2: unused
  \raisebox{\depth}{$\m@th#1\intercal$}%
}
\makeatother


\makeatletter
\newcommand*{\da@rightarrow}{\mathchar"0\hexnumber@\symAMSa 4B }
\newcommand*{\da@leftarrow}{\mathchar"0\hexnumber@\symAMSa 4C }
\newcommand*{\xdashrightarrow}[2][]{%
  \mathrel{%
    \mathpalette{\da@xarrow{#1}{#2}{}\da@rightarrow{\,}{}}{}%
  }%
}
\newcommand{\xdashleftarrow}[2][]{%
  \mathrel{%
    \mathpalette{\da@xarrow{#1}{#2}\da@leftarrow{}{}{\,}}{}%
  }%
}
\newcommand*{\da@xarrow}[7]{%
  % #1: below
  % #2: above
  % #3: arrow left
  % #4: arrow right
  % #5: space left 
  % #6: space right
  % #7: math style 
  \sbox0{$\ifx#7\scriptstyle\scriptscriptstyle\else\scriptstyle\fi#5#1#6\m@th$}%
  \sbox2{$\ifx#7\scriptstyle\scriptscriptstyle\else\scriptstyle\fi#5#2#6\m@th$}%
  \sbox4{$#7\dabar@\m@th$}%
  \dimen@=\wd0 %
  \ifdim\wd2 >\dimen@
    \dimen@=\wd2 %   
  \fi
  \count@=2 %
  \def\da@bars{\dabar@\dabar@}%
  \@whiledim\count@\wd4<\dimen@\do{%
    \advance\count@\@ne
    \expandafter\def\expandafter\da@bars\expandafter{%
      \da@bars
      \dabar@ 
    }%
  }%  
  \mathrel{#3}%
  \mathrel{%   
    \mathop{\da@bars}\limits
    \ifx\\#1\\%
    \else
      _{\copy0}%
    \fi
    \ifx\\#2\\%
    \else
      ^{\copy2}%
    \fi
  }%   
  \mathrel{#4}%
}
\makeatother
% \DeclareMathOperator{\rg}{rg}

\usepackage{mathtools}
\DeclarePairedDelimiter{\paren}{(}{)}
\DeclarePairedDelimiter{\abs}{\lvert}{\rvert}
\DeclarePairedDelimiter{\norm}{\lVert}{\rVert}
\DeclarePairedDelimiter{\innerproduct}{\langle}{\rangle}
\newcommand{\Of}[2]{{\operatorname{#1}} {\paren*{#2}}}
\newcommand{\of}[2]{{{{#1}} {\paren*{#2}}}}

\DeclareMathOperator{\Shim}{Shim}
\DeclareMathOperator{\sgn}{sgn}
\DeclareMathOperator{\fdeg}{fdeg}
\DeclareMathOperator{\SL}{SL}
\DeclareMathOperator{\slLie}{\mathfrak{s}\mathfrak{l}}
\DeclareMathOperator{\soLie}{\mathfrak{s}\mathfrak{o}}
\DeclareMathOperator{\spLie}{\mathfrak{s}\mathfrak{p}}
\DeclareMathOperator{\glLie}{\mathfrak{g}\mathfrak{l}}
\newcommand{\pn}[1]{{\color{ForestGreen} \sf PN: [#1]}}
\DeclareMathOperator{\Mp}{Mp}
\DeclareMathOperator{\Mat}{Mat}
\DeclareMathOperator{\GL}{GL}
\DeclareMathOperator{\Gr}{Gr}
\DeclareMathOperator{\GU}{GU}
\def\gl{\mathfrak{g}\mathfrak{l}}
\DeclareMathOperator{\odd}{odd}
\DeclareMathOperator{\even}{even}
\DeclareMathOperator{\GO}{GO}
\DeclareMathOperator{\good}{good}
\DeclareMathOperator{\bad}{bad}
\DeclareMathOperator{\PGO}{PGO}
\DeclareMathOperator{\htt}{ht}
\DeclareMathOperator{\height}{height}
\DeclareMathOperator{\Ass}{Ass}
\DeclareMathOperator{\coheight}{coheight}
\DeclareMathOperator{\GSO}{GSO}
\DeclareMathOperator{\SO}{SO}
\DeclareMathOperator{\so}{\mathfrak{s}\mathfrak{o}}
\DeclareMathOperator{\su}{\mathfrak{s}\mathfrak{u}}
\DeclareMathOperator{\ad}{ad}
% \DeclareMathOperator{\sc}{sc}
\DeclareMathOperator{\Ad}{Ad}
\DeclareMathOperator{\disc}{disc}
\DeclareMathOperator{\inv}{inv}
\DeclareMathOperator{\Pic}{Pic}
\DeclareMathOperator{\uc}{uc}
\DeclareMathOperator{\Cl}{Cl}
\DeclareMathOperator{\Clf}{Clf}
\DeclareMathOperator{\Hom}{Hom}
\DeclareMathOperator{\hol}{hol}
\DeclareMathOperator{\Heis}{Heis}
\DeclareMathOperator{\Haar}{Haar}
\DeclareMathOperator{\h}{h}
\def\sp{\mathfrak{s}\mathfrak{p}}
\DeclareMathOperator{\heis}{\mathfrak{h}\mathfrak{e}\mathfrak{i}\mathfrak{s}}
\DeclareMathOperator{\End}{End}
\DeclareMathOperator{\JL}{JL}
\DeclareMathOperator{\image}{image}
\DeclareMathOperator{\red}{red}
\def\div{\operatorname{div}}
\def\eps{\varepsilon}
\def\cHom{\mathcal{H}\operatorname{om}}
\DeclareMathOperator{\Ops}{Ops}
\DeclareMathOperator{\Symb}{Symb}
\def\boldGL{\mathbf{G}\mathbf{L}}
\def\boldSO{\mathbf{S}\mathbf{O}}
\def\boldU{\mathbf{U}}
\DeclareMathOperator{\hull}{hull}
\DeclareMathOperator{\LL}{LL}
\DeclareMathOperator{\PGL}{PGL}
\DeclareMathOperator{\class}{class}
\DeclareMathOperator{\lcm}{lcm}
\DeclareMathOperator{\spann}{span}
\DeclareMathOperator{\Exp}{Exp}
\DeclareMathOperator{\ext}{ext}
\DeclareMathOperator{\Ext}{Ext}
\DeclareMathOperator{\Tor}{Tor}
\DeclareMathOperator{\et}{et}
\DeclareMathOperator{\tor}{tor}
\DeclareMathOperator{\loc}{loc}
\DeclareMathOperator{\tors}{tors}
\DeclareMathOperator{\pf}{pf}
\DeclareMathOperator{\smooth}{smooth}
\DeclareMathOperator{\prin}{prin}
\DeclareMathOperator{\Kl}{Kl}
\newcommand{\kbar}{\mathchar'26\mkern-9mu k}
\DeclareMathOperator{\der}{der}
% \DeclareMathOperator{\abs}{abs}
\DeclareMathOperator{\Sub}{Sub}
\DeclareMathOperator{\Comp}{Comp}
\DeclareMathOperator{\Err}{Err}
\DeclareMathOperator{\dom}{dom}
\DeclareMathOperator{\radius}{radius}
\DeclareMathOperator{\Fitt}{Fitt}
\DeclareMathOperator{\Sel}{Sel}
\DeclareMathOperator{\rad}{rad}
\DeclareMathOperator{\id}{id}
\DeclareMathOperator{\Center}{Center}
\DeclareMathOperator{\Der}{Der}
\DeclareMathOperator{\U}{U}
% \DeclareMathOperator{\norm}{norm}
\DeclareMathOperator{\trace}{trace}
\DeclareMathOperator{\Equid}{Equid}
\DeclareMathOperator{\Feas}{Feas}
\DeclareMathOperator{\bulk}{bulk}
\DeclareMathOperator{\tail}{tail}
\DeclareMathOperator{\sys}{sys}
\DeclareMathOperator{\atan}{atan}
\DeclareMathOperator{\temp}{temp}
\DeclareMathOperator{\Asai}{Asai}
\DeclareMathOperator{\glob}{glob}
\DeclareMathOperator{\Kuz}{Kuz}
\DeclareMathOperator{\Irr}{Irr}
\newcommand{\rsL}{ \frac{ L^{(R)}(\Pi \times \Sigma, \std, \frac{1}{2})}{L^{(R)}(\Pi \times \Sigma, \Ad, 1)}  }
\DeclareMathOperator{\GSp}{GSp}
\DeclareMathOperator{\PGSp}{PGSp}
\DeclareMathOperator{\BC}{BC}
\DeclareMathOperator{\Ann}{Ann}
\DeclareMathOperator{\Gen}{Gen}
\DeclareMathOperator{\SU}{SU}
\DeclareMathOperator{\PGSU}{PGSU}
% \DeclareMathOperator{\gen}{gen}
\DeclareMathOperator{\PMp}{PMp}
\DeclareMathOperator{\PGMp}{PGMp}
\DeclareMathOperator{\PB}{PB}
\DeclareMathOperator{\ind}{ind}
\DeclareMathOperator{\Jac}{Jac}
\DeclareMathOperator{\jac}{jac}
\DeclareMathOperator{\im}{im}
\DeclareMathOperator{\Aut}{Aut}
\DeclareMathOperator{\Int}{Int}
\DeclareMathOperator{\PSL}{PSL}
\DeclareMathOperator{\co}{co}
\DeclareMathOperator{\irr}{irr}
\DeclareMathOperator{\prim}{prim}
\DeclareMathOperator{\bal}{bal}
\DeclareMathOperator{\baln}{bal}
\DeclareMathOperator{\dist}{dist}
\DeclareMathOperator{\RS}{RS}
\DeclareMathOperator{\Ram}{Ram}
\DeclareMathOperator{\Sob}{Sob}
\DeclareMathOperator{\Sol}{Sol}
\DeclareMathOperator{\soc}{soc}
\DeclareMathOperator{\nt}{nt}
\DeclareMathOperator{\mic}{mic}
\DeclareMathOperator{\Gal}{Gal}
\DeclareMathOperator{\st}{st}
\DeclareMathOperator{\std}{std}
\DeclareMathOperator{\diag}{diag}
\DeclareMathOperator{\Sym}{Sym}
\DeclareMathOperator{\gr}{gr}
\DeclareMathOperator{\aff}{aff}
\DeclareMathOperator{\Dil}{Dil}
\DeclareMathOperator{\Lie}{Lie}
\DeclareMathOperator{\Symp}{Symp}
\DeclareMathOperator{\Stab}{Stab}
\DeclareMathOperator{\St}{St}
\DeclareMathOperator{\stab}{stab}
\DeclareMathOperator{\codim}{codim}
\DeclareMathOperator{\linear}{linear}
\newcommand{\git}{/\!\!/}
\DeclareMathOperator{\geom}{geom}
\DeclareMathOperator{\spec}{spec}
\def\O{\operatorname{O}}
\DeclareMathOperator{\Au}{Aut}
\DeclareMathOperator{\Fix}{Fix}
\DeclareMathOperator{\Opp}{Op}
\DeclareMathOperator{\opp}{op}
\DeclareMathOperator{\Size}{Size}
\DeclareMathOperator{\Save}{Save}
% \DeclareMathOperator{\ker}{ker}
\DeclareMathOperator{\coker}{coker}
\DeclareMathOperator{\sym}{sym}
\DeclareMathOperator{\mean}{mean}
\DeclareMathOperator{\elliptic}{ell}
\DeclareMathOperator{\nilpotent}{nil}
\DeclareMathOperator{\hyperbolic}{hyp}
\DeclareMathOperator{\newvector}{new}
\DeclareMathOperator{\new}{new}
\DeclareMathOperator{\full}{full}
\newcommand{\qr}[2]{\left( \frac{#1}{#2} \right)}
\DeclareMathOperator{\unr}{u}
\DeclareMathOperator{\ram}{ram}
% \DeclareMathOperator{\len}{len}
\DeclareMathOperator{\fin}{fin}
\DeclareMathOperator{\cusp}{cusp}
\DeclareMathOperator{\curv}{curv}
\DeclareMathOperator{\rank}{rank}
\DeclareMathOperator{\rk}{rk}
\DeclareMathOperator{\pr}{pr}
\DeclareMathOperator{\Transform}{Transform}
\DeclareMathOperator{\mult}{mult}
\DeclareMathOperator{\Eis}{Eis}
\DeclareMathOperator{\reg}{reg}
\DeclareMathOperator{\sing}{sing}
\DeclareMathOperator{\alt}{alt}
\DeclareMathOperator{\irreg}{irreg}
\DeclareMathOperator{\sreg}{sreg}
\DeclareMathOperator{\Wd}{Wd}
\DeclareMathOperator{\Weil}{Weil}
\DeclareMathOperator{\Th}{Th}
\DeclareMathOperator{\Sp}{Sp}
\DeclareMathOperator{\Ind}{Ind}
\DeclareMathOperator{\Res}{Res}
\DeclareMathOperator{\ini}{in}
\DeclareMathOperator{\ord}{ord}
\DeclareMathOperator{\osc}{osc}
\DeclareMathOperator{\fluc}{fluc}
\DeclareMathOperator{\size}{size}
\DeclareMathOperator{\ann}{ann}
\DeclareMathOperator{\equ}{eq}
\DeclareMathOperator{\res}{res}
\DeclareMathOperator{\pt}{pt}
\DeclareMathOperator{\src}{source}
\DeclareMathOperator{\Zcl}{Zcl}
\DeclareMathOperator{\Func}{Func}
\DeclareMathOperator{\Map}{Map}
\DeclareMathOperator{\Frac}{Frac}
\DeclareMathOperator{\Frob}{Frob}
\DeclareMathOperator{\ev}{eval}
\DeclareMathOperator{\pv}{pv}
\DeclareMathOperator{\eval}{eval}
\DeclareMathOperator{\Spec}{Spec}
\DeclareMathOperator{\Speh}{Speh}
\DeclareMathOperator{\Spin}{Spin}
\DeclareMathOperator{\GSpin}{GSpin}
\DeclareMathOperator{\Specm}{Specm}
\DeclareMathOperator{\Sphere}{Sphere}
\DeclareMathOperator{\Sqq}{Sq}
\DeclareMathOperator{\Ball}{Ball}
\DeclareMathOperator\Cond{\operatorname{Cond}}
\DeclareMathOperator\proj{\operatorname{proj}}
\DeclareMathOperator\Swan{\operatorname{Swan}}
\DeclareMathOperator{\Proj}{Proj}
\DeclareMathOperator{\bPB}{{\mathbf P}{\mathbf B}}
\DeclareMathOperator{\Projm}{Projm}
\DeclareMathOperator{\Tr}{Tr}
\DeclareMathOperator{\Type}{Type}
\DeclareMathOperator{\Prop}{Prop}
\DeclareMathOperator{\vol}{vol}
\DeclareMathOperator{\covol}{covol}
\DeclareMathOperator{\Rep}{Rep}
\DeclareMathOperator{\Cent}{Cent}
\DeclareMathOperator{\val}{val}
\DeclareMathOperator{\area}{area}
\DeclareMathOperator{\nr}{nr}
\DeclareMathOperator{\CM}{CM}
\DeclareMathOperator{\CH}{CH}
\DeclareMathOperator{\tr}{tr}
\DeclareMathOperator{\characteristic}{char}
\DeclareMathOperator{\supp}{supp}


\theoremstyle{plain} \newtheorem{theorem} {Theorem} \newtheorem{conjecture} [theorem] {Conjecture} \newtheorem{corollary} [theorem] {Corollary} \newtheorem{proposition} [theorem] {Proposition} \newtheorem{fact} [theorem] {Fact}
\theoremstyle{definition} \newtheorem{definition} [theorem] {Definition} \newtheorem{hypothesis} [theorem] {Hypothesis} \newtheorem{assumptions} [theorem] {Assumptions}
\newtheorem{example} [theorem] {Example}
\newtheorem{assertion}[theorem] {Assertion}
\newtheorem{note}[theorem] {Note}
\newtheorem{conclusion}[theorem] {Conclusion}
\newtheorem{claim}            {Claim}
\newtheorem{homework} {Homework}
\newtheorem{exercise} {Exercise}  \newtheorem{question}[theorem] {Question}    \newtheorem{answer} {Answer}  \newtheorem{problem} {Problem}    \newtheorem{remark} [theorem] {Remark}
\newtheorem{notation} [theorem]           {Notation}
\newtheorem{terminology}[theorem]            {Terminology}
\newtheorem{convention}[theorem]            {Convention}
\newtheorem{motivation}[theorem]            {Motivation}


\newtheoremstyle{itplain} % name
{6pt}                    % Space above
{5pt\topsep}                    % Space below
{\itshape}                   % Body font
{}                           % Indent amount
{\itshape}                   % Theorem head font
{.}                          % Punctuation after theorem head
{5pt plus 1pt minus 1pt}                       % Space after theorem head
% {.5em}                       % Space after theorem head
{}  % Theorem head spec (can be left empty, meaning ‘normal’)

% \theoremstyle{mytheoremstyle}


\theoremstyle{itplain} %--default
% \theoremheaderfont{\itshape}
% \newtheorem{lemma}{Lemma}
\newtheorem{lemma}[theorem]{Lemma}
% \newtheorem{lemma}{Lemma}[subsubsection]

\newtheorem*{lemma*}{Lemma}
\newtheorem*{proposition*}{Proposition}
\newtheorem*{definition*}{Definition}
\newtheorem*{example*}{Example}

\newtheorem*{results*}{Results}
\newtheorem{results} [theorem] {Results}


\usepackage[displaymath,textmath,sections,graphics]{preview}
\PreviewEnvironment{align*}
\PreviewEnvironment{multline*}
\PreviewEnvironment{tabular}
\PreviewEnvironment{verbatim}
\PreviewEnvironment{lstlisting}
\PreviewEnvironment*{frame}
\PreviewEnvironment*{alert}
\PreviewEnvironment*{emph}
\PreviewEnvironment*{textbf}



\title{Spectral aspect subconvexity for $\PGL_2$: a couple approaches}

\usepackage{xr-hyper}
\externaldocument{20230522T174726__shrinking-archimedean-families-second-moment-gl2}
\externaldocument{var-quat-3-submitted}
\externaldocument{standard}

\numberwithin{equation}{section}
\numberwithin{theorem}{section}

\begin{document}

\maketitle
\tableofcontents


\begin{abstract}
  We discuss and compare two approaches for estimating a short second moment of $L$-functions on $\PGL_2$ in the spectral aspect: that of Iwaniec \cite{Iwaniec1992}, using the approximate functional equation and Kuznetsov formula, and that of \cite{2021arXiv210915230N} using period integrals.
\end{abstract}

\begin{remark}
  This is a lightly edited version of ``log-2022-03-08-a.tex''.  It needs a lot of polishing, but might be useful to someone even in its current state.
\end{remark}

\section{Goal}\label{sec:org64b74bb}
Take
\begin{equation*}
G := \PGL_2(\mathbb{R}),
\end{equation*}
\begin{equation*}
\Gamma := \PGL_2(\mathbb{Z}) \hookrightarrow G,
\end{equation*}
\begin{equation*}
  H := \GL_1(\mathbb{R}) \cong  \begin{pmatrix} \ast & 0 \\ 0 & 1 \end{pmatrix} \hookrightarrow G,
\end{equation*}
\begin{equation*}
\Gamma_H := \GL_1(\mathbb{Z}) = \{\pm 1\} \hookrightarrow  H.
\end{equation*}
Let $\pi \subset L^2_{\text{cusp}}(\Gamma \backslash G)$ be a cuspidal automorphic representation corresponding to a Maass form of eigenvalue $1/4 + T^2$.  We aim to show that
\begin{equation}\label{eqn:35ac318817}
  L(\pi,\tfrac{1}{2}) \ll T^{1/2 - \delta}
\end{equation}
for some fixed $\delta > 0$.


Such an estimate was first shown by Iwaniec \cite{Iwaniec1992} (in the Maass case, and initially under an additional hypothesis, which Iwaniec later observed could be removed using identity $\lambda(p)^2 - \lambda(p^2) = 1$).  Iwaniec's method consists of estimating an amplified second moment over $T$ in an interval of length a bit more than $1$.  This method was adapted to $\PGL_3$ by Blomer--Buttcane \cite{MR4203038, MR4039487}, using an amplified fourth moment, and then to $\GL_n$ in \cite{2021arXiv210915230N}  (\href{standard.pdf}{direct link}), using an amplified $2(n-1)$th moment.  In the works of Iwaniec and Blomer---Buttcane, the moment is estimated using the approximate functional equation, Kuznetsov formula, and several applications of Poisson summation.  In \cite{2021arXiv210915230N}, the moment is estimated implicitly by direct analysis of an integral representation for the $L$-function, with vectors chosen as in \cite{nelson-venkatesh-1} and the averaging implemented via the pretrace formula like in \cite{iwan-sar}.

\begin{remark}
  The best known bound for \eqref{eqn:35ac318817} is to due to Ivic \cite{MR1879668} , who showed (in the Maass case) that \eqref{eqn:35ac318817} holds for any fixed $\delta < 1/6$.  Ivic's method consists of estimating an unamplified fourth moment over $T$ in an interval of length a bit more than $T^{1/3}$ (compare with \cite{michel-2009, Nelson-EisCubic, 2021arXiv210112106B, balkanova2022weyls}).  Such approaches using ``higher moments'' have no known generalization beyond $\GL_2$.
\end{remark}

\section{Automorphic background}\label{sec:35ac3e5876}

\subsection{Cuspidal automorphic representations}\label{sec:35ac3e5878}
The space $L^2_{\cusp}(\Gamma \backslash G)$ consists of square-integrable functions on $\Gamma \backslash G$ whose constant term vanishes.  It is simultaneously a representation for the group $G$, acting via right translation, and the Hecke algebra, acting via left translation with respect to double cosets in $\Gamma \backslash \PGL_2(\mathbb{Q}) / \Gamma$.  Under these actions, it decomposes as a direct sum of irreducible representations, each occurring with multiplicity one.  We denote by $\pi$ the space of smooth vectors inside one such representation.

\subsection{Whittaker expansion}\label{sec:35ac3e587a}
The Hecke eigenvalues of $\pi$ are described by a multiplicative function $\lambda_\pi : \mathbb{N} \rightarrow \mathbb{C}$, which specifies the eigenvalues for a spanning set of double cosets.  For $\varphi \in \pi$, one defines the Whittaker function $W_\varphi : G \rightarrow \mathbb{C}$ by
\begin{equation*}
  W_\varphi (g) :=
  \int_{x \in \mathbb{R} / \mathbb{Z} } \varphi \left(
    \begin{pmatrix}
1 & x \\
0 & 1 \\
\end{pmatrix} g \right) e(-x) \, d x,
\end{equation*}
where we employ the standard abbreviation $e(t) := e^{2 \pi i t}$.  Conversely, $\varphi$ may be recovered from $\lambda_\pi$ and $W_\varphi$ via the formula
\begin{equation}\label{eqn:35ac31d7c1}
  \varphi(g) = \sum _{n \neq 0}
  \frac{\lambda_{\pi}(|n|)}{|n|^{1/2}}
  W_\varphi \left( \begin{pmatrix}
      n & 0 \\
      0 & 1
    \end{pmatrix} g \right),
\end{equation}
with the sum taken over all nonzero integers $n$.

\subsection{Kirillov model}\label{sec:35ac31dd2c}
The Whittaker model $\mathcal{W}(\pi)$ for $\pi$ is defined to consist of all functions of the form $W_\varphi$.  The theory of the Kirillov model implies that the restriction map
\begin{equation*}
\mathcal{W}(\pi) \rightarrow \{\text{functions } H \rightarrow \mathbb{C} \}
\end{equation*}
is injective, and that its image contains $C_c^\infty(H)$.  Combining this fact with \eqref{eqn:35ac31d7c1} tells us that
\begin{itemize}
\item any $\varphi \in \pi$ is determined by $W_\varphi : H \rightarrow \mathbb{C}$, and
\item every smooth compactly-supported function $H \rightarrow \mathbb{C}$ determines a unique $\varphi \in \pi$.
\end{itemize}



\section{Integral representation}\label{sec:org8f5cc20}
The $L$-function $L(\pi,s)$ may be defined for $\Re(s) > 1$ by the absolutely convergent Dirichlet series $\sum_{n \in \mathbb{N} } \lambda_\pi(n) n^{-s}$ and in general by meromorphic continuation.  By unfolding \eqref{eqn:35ac31d7c1}, one obtains the Hecke/Jacquet--Langlands integral representation
\begin{equation}\label{eq:int-_h-varphi}
  \int _{\Gamma_{H} \backslash H} \varphi = L(\pi,\tfrac{1}{2}) \int _H W_\varphi.
\end{equation}


\section{Coadjoint orbits}\label{sec:orge4ffe7c}
We denote by $\mathfrak{g}$ the Lie algebra of $G$ and by $\mathfrak{g}^*$ its linear dual.  Using the trace pairing, we may identify $\mathfrak{g}^*$ with $\mathfrak{g}$, which we identify further with the space $\slLie_2(\mathbb{R})$ of traceless $2 \times 2$ matrices $\xi$.  The coadjoint orbit
\begin{equation*}
  \mathcal{O}_\pi \subseteq \mathfrak{g}^*
\end{equation*}
attached to $\pi$ is the one-sheeted hyperboloid cut out by the equation
\begin{equation*}
  \det(\xi) = -T^2.
\end{equation*}
(See for instance \S\ref{sec:coadjoint-orbits}-\S\ref{sec:prelims-representations} of \href{var-quat-3-submitted.pdf}{Quantum variance III}. It comes with a natural symplectic volume form that describes the character of $\pi$ via the Kirillov formula (see \cite[\S6]{nelson-venkatesh-1}).  We note that in the coordinates
\begin{equation*}
  \xi =
  \begin{pmatrix}
    a & b \\
    c & -a
  \end{pmatrix}
  =
  \begin{pmatrix}
  x & y - z \\
  y + z & -x
\end{pmatrix},
\end{equation*}
we have
\begin{equation*}
  \det(\xi) = -a^2 - b c = z^2 - y^2 - x^2.
\end{equation*}
In particular, $\mathcal{O}_\pi$ contains the circle $\{(x,y,0) : x^2 + y^2 = T^2\}$.  The circular strip $\{(x,y,z) \in \mathcal{O}_\pi : |z| \leq 1/2\}$ has symplectic volume one; in the orbit method heuristic described in \cite[\S1.7]{nelson-venkatesh-1}, it corresponds to the weight zero vector in $\pi$.

The integrals \eqref{eq:int-_h-varphi} may be understood as describing how $\pi$ oscillates against the trivial character of $H$.  The orbit method suggests \cite[\S1.9]{nelson-venkatesh-1} that such oscillation may be understood in terms of the intersection
\begin{equation*}
  \mathcal{O}_\pi \cap \mathfrak{h}^\perp,
\end{equation*}
i.e., the preimage in $\mathcal{O}_\pi$ of the trivial element $0$ of $\mathfrak{h}^*$.  That intersection is given in $(x,y,z)$ coordinates by
\begin{equation*}
  \{(0,y,z) : y^2 - z^2 = T^2\}
\end{equation*}
and in $(a,b,c)$ coordinates by
\begin{equation*}
  \{(0,b,c) : b c = T^2\}.
\end{equation*}
It is a closed $H$-orbit, with trivial stabilizer.

We pick a point $\tau \in \mathcal{O}_\pi \cap \mathfrak{h}^\perp$ of size comparable to $T$.  For concreteness, let us take
\begin{equation*}
  \tau = \begin{pmatrix}
    0 & T \\
    T & 0
  \end{pmatrix}.
\end{equation*}

\begin{remark}
  For the variant problem concerning $\pi$ attached to a holomorphic form of weight $2 k$, we would take $T := k - 1/2$, we would take $\mathcal{O}_\pi$ cut out by $\det(\xi) = T^2$, and we would take
  \begin{equation*}
\tau =
\begin{pmatrix}
0 & -T \\
T & 0 \\
\end{pmatrix}.
\end{equation*}
See for instance \S\ref{sec:holomorphic-analogue-1} of \href{var-quat-3-submitted.pdf}{Quantum variance III}.
\end{remark}


\section{Choice of vector}\label{sec:org806411a}
We seek a unit vector $\varphi \in \pi$ that is ``localized at $\tau$.''  Informally, this means that for each fixed Lie algebra element $X \in \mathfrak{g}$, we have
\begin{equation}\label{eq:x-varphi-=}
  X \varphi = {i \langle X, \tau  \rangle} \varphi + \O(T^{1/2}).
\end{equation}
For further informal discussion of this concept, see \cite[\S1.7]{nelson-venkatesh-1} and \cite[\S2.5]{2020arXiv201202187N} .  For some precise definitions, see \S\ref{sec:states-approx-microlocal} of \href{var-quat-3-submitted.pdf}{Quantum variance III}, or \cite[\S14]{2021arXiv210915230N} (direct link: \S\ref{sec:abstr-study-local}).

Such a vector $\varphi$ may be described readily in the Kirillov model for $\pi$.  Recall (from \S\ref{sec:35ac31dd2c}) that this consists of identifying elements $W$ of the Whittaker model $\mathcal{W}(\pi)$ their restrictions to $H$.  It will be convenient to think of such $W$ as function on $\mathbb{R}^\times$ via the abbreviation
\begin{equation*}
W(y) := W\left(
  \begin{pmatrix}
y & 0 \\
0 & 1 \\
\end{pmatrix} \right).
\end{equation*}
We consider the following basis elements for $\mathfrak{g}$:
\begin{equation*}
  \partial_a =
  \begin{pmatrix}
    1 & 0 \\
    0 & -1
  \end{pmatrix},
  \quad
  \partial_b =
  \begin{pmatrix}
    0 & 1 \\
    0 & 0
  \end{pmatrix},
  \quad
  \partial_c =
  \begin{pmatrix}
    0 & 0 \\
    1 & 0
  \end{pmatrix}  .
\end{equation*}
The action of the first two of these elements on the Kirillov model is given very simply:
\begin{equation*}
  \partial_a W(y) = 2 y W'(y),
  \quad
  \partial_b W(y) = 2 \pi i y W(y).
\end{equation*}
Since $\langle \partial_a, \tau \rangle = 0$ and $\langle \partial_b, \tau \rangle = T$, the condition \eqref{eq:x-varphi-=} says in particular that
\begin{equation*}
  \partial_a W_\varphi = \O(T^{1/2}),
  \quad
  \partial_b W_\varphi = i T W_\varphi + \O(T^{1/2}).
\end{equation*}
These formulas suggest taking the following smoothened $L^2$-normalized characteristic function:
\begin{equation}\label{eqn:35ac3e31cb}
  W_\varphi(y) = T^{1/4} 1_{T / 2 \pi + \O(T^{1/2})}^{\text{smooth}}(y).
\end{equation}
Using that $W_\varphi$ is an eigenfunction under the Casimir operator (see \cite[\S12]{2021arXiv210915230N}, direct link: \S\ref{sec:overview-asymp-kirillov}), one can check also that
\begin{equation*}
  \partial_c W_\varphi = i T W_\varphi + \O(T^{1/2}).
\end{equation*}
Precise forms of all these estimates are established in \cite[Part 3]{2021arXiv210915230N} (direct link: Part \ref{part:asympt-analys-kirill}).

\begin{remark}
  This choice of $\varphi$ is closely related to the ``microlocal lift'' of Zelditch \cite{MR916129} et al., see \cite{MR1859345, MR2346281, MR2314452}.  More precisely, it is (asymptotically) a $G$-translate of the usual definition; the limit invariance for $L^2$-masses will be with respect to the stabilizer $G_\tau$ of $\tau$, namely
\begin{equation}\label{eqn:G-tau}
  G_\tau = \left\{ \begin{pmatrix}
      \cosh t & \sinh t \\
      \sinh t & \cosh t
    \end{pmatrix} \right\},
\end{equation}
rather than with respect to the diagonal subgroup.
\end{remark}

By chosing the smooth bump function implicit in \eqref{eqn:35ac3e31cb} to be nonnegative, we may arrange that
\begin{equation*}
  \int_{H} W_\varphi \asymp  T^{-1/4},
\end{equation*}
so that \eqref{eq:int-_h-varphi} reads
\begin{equation}\label{eqn:35ac3e32a4}
  T^{-1/4} L(\pi,\tfrac{1}{2}) \asymp  \int_{\Gamma_{H} \backslash H} \varphi.
\end{equation}
Our task is to show that the right hand side is $\ll T^{1/4-\delta}$.


\section{Truncation}\label{sec:orgc98f847}
We now indicate why the integral on the right hand side of \eqref{eqn:35ac3e32a4} of may be effectively truncated to a fixed compact set.  For a detailed discussion of this point, see \cite[\S5.3]{2021arXiv210915230N} (direct link: \ref{sec:reduct-peri-bounds}) or \cite[\S5.1.4]{michel-2009} or \cite[\S3]{2020arXiv200406791S} (or \S\ref{sec:20230522180027} of \href{20230522T174726__shrinking-archimedean-families-second-moment-gl2.pdf}{this informal note}).

Consider, for a fixed even ``truncation'' function $\mathcal{T} \in C_c^\infty(H)$, the map
\begin{equation*}
  I : H \rightarrow \mathbb{C} 
\end{equation*}
\begin{equation*}
  I(Y) := \int _{y \in \Gamma_{H} \backslash H}  \mathcal{T}(y/Y) \varphi(y) \, d^\times y
\end{equation*}
assigning to a parameter $Y$ the smoothened integral of $\varphi|_H$ over the corresponding dyadic range.  We have
\begin{equation*}
  \int _{\Gamma_{H} \backslash H} \varphi |.|^{s-1/2} = L(\pi,s) Z(W_\varphi,s),
\end{equation*}
where $Z(W_\varphi,s)$ denotes the local zeta integral
\begin{equation*}
  Z(W_\varphi,s) := \int _{H} W_\varphi |.|^{s-1/2}.
\end{equation*}
For $s = \O(1)$, we see by explicit calculation that
\begin{equation*}
  Z(W_\varphi,s) \approx T^{s-3/4}.
\end{equation*}
If moreover $|\Re(s)| \leq 1/2$, then the convexity bound reads
\begin{equation*}
  L(\pi,s) \ll T^{1 - s}.
\end{equation*}
These bounds become only polynomially worse if we relax the condition $s = \O(1)$ to $\Re(s) = \O(1)$.  Multiplying them together and convolving against the rapidly-decaying Mellin transform of the fixed test function $\mathcal{T}$, we deduce the Mellin transform estimate
\begin{equation*}
|\Re(s)| \leq 1/2 \implies \tilde{I}(s) \ll T^{1/4} (1 + |s|)^{-\infty}.
\end{equation*}
It follows readily that for fixed $\kappa > 0$, we incur the acceptable error $\O(T^{1/4 -\kappa/2})$ by smoothly truncating the integral $\int_H \varphi$ to the range $\{T^{-\kappa} < |y| < T^{\kappa}\}$.  If we seek only a qualitative subconvex bound, then we can take $\kappa$ as small as we like, so there is no harm in truncating to $\{ |y| = T^{o(1)} \}$.  Note that the relevant truncation is specific to our choice of vector $\varphi$.


The model problem is thus to bound
\begin{equation}\label{eq:int-_y-in}
  \int _{y \in H, y \asymp 1} \varphi(y) \, d^\times y.
\end{equation}

\section{Symmetries}\label{sec:org1580512}
We use a convolution kernel $\omega \in C_c^\infty(G)$ to ``remember'' many of the symmetries satisfied by $\varphi$.  (When amplifying, we really take $\omega \in C_c^\infty(\PGL_2(\mathbb{A}))$.)  Roughly speaking, we take $\omega$ to be a character multiple of an approximate subgroup of $G$:
\begin{equation*}
  \omega := \vol(J)^{-1} 1_J^{\text{smooth}} \chi_\tau^{-1},
\end{equation*}
where:
\begin{itemize}
\item $J$ is a subset of $G$ roughly of the shape
  \begin{equation*}
    J = (1 + \O(T^{-\eps})) \cap (G_\tau + \O(T^{-1/2-\eps})),
  \end{equation*}
  with $G_\tau$ the stabilizer of $\tau$, as described in \eqref{eqn:G-tau}.
\item $1_J^{\text{smooth}}$ is a smoothened characteristic function of $J$.
\item $\chi_\tau$ is the ``approximate character'' of $J$ attached to $\tau$, given near the identity in exponential coordinates by
  \begin{equation*}
    \chi_\tau(\exp(X)) = e^{i \langle X, \tau \rangle}.
  \end{equation*}  
\end{itemize}

We may also describe $\omega$ in terms of the function $\mathfrak{g}^* \rightarrow \mathbb{C}$ obtained by taking the Fourier transform of the pullback $\omega \circ \exp$.  This function is roughly a smoothened characteristic function of a ``coin-shaped'' neighborhood of $\tau$, of thickness $T^\eps$ (resp. $T^{1/2+\eps}$) in directions transverse (resp. tangential) to the coadjoint orbit $\mathcal{O}_\pi$ at $\tau$.  The intersection of this neighborhood with $\mathcal{O}_\pi$ has symplectic volume $\approx T^{2 \eps} \approx 1$.  The orbit method heuristic suggests that for an irreducible representation $\sigma$ of $G$, we have $\sigma(\omega) \approx 0$ unless $\sigma$ is a principal series representation of parameter $T + \O(T^{\eps})$, in which case $\sigma(\omega)$ is approximately a projection onto a rank $\approx T^{2 \eps} \approx 1$ ``subspace'' of vectors microlocalized at $\tau$.  In particular,
\begin{equation}\label{eqn:35ac3e34eb}
  \pi(\omega) \varphi \approx \varphi.
\end{equation}
These heuristics and definitions can be made precise, and the above approximation holds in an extremely strong sense (i.e., up to $\O(T^{-\infty})$ with respect to any fixed seminorm).

For further informal discussion concerning $\omega$ in a general setting, see \cite[\S2]{2020arXiv201202187N}.  For a precise discussion in the current rank one example, see \S\ref{sec:states-approx-microlocal} of \href{var-quat-3-submitted.pdf}{Quantum variance III}.


\section{The convexity threshold}\label{sec:org82a14f5}
Let's explain how to recover the convexity bound from here.  Our task is to show that
\begin{equation*}
  \int _{y \in H, y \asymp 1} \pi(\omega) \varphi(y) \, d^\times y \ll T^{1/4}.
\end{equation*}
We view the square of the left hand side as one term arising from an integrated pretrace formula, like in the sup norm story \cite{iwan-sar} (alternatively, we write the left hand side as the inner product of $\varphi$ against a Poincar{\'e} series and apply Cauchy--Schwarz; see \cite[\S5.3]{2020arXiv201202187N}).  We reduce to checking that
\begin{equation}\label{eq:int-_-substack-1}
  \int _{
    \substack{
      y_1, y_2 \asymp 1
    }
  }
  \sum _{\gamma \in \Gamma }
  \omega (y_1^{-1} \gamma y_2)
  \ll
  T^{1/2}.
\end{equation}
Since $\omega$ is supported on $1 + \O(T^{-\eps})$, we see that the only $\gamma$ that contribute are those in $\Gamma_{H}$.  Combining the $y_1$ and $y_2$ integrals, it remains to check that
\begin{equation}\label{eq:int-_-substack}
  \int _{
    \substack{
      y \in H : y\asymp 1
    }
  }
  \omega (y)
  \ll
  T^{1/2}.
\end{equation}
To that end, we observe that
\begin{equation*}
  H \cap G_\tau = \{1\},
\end{equation*}
i.e., that no nontrivial matrices are simultaneously diagonal and of the form \eqref{eqn:G-tau}.  (This is a baby case of the ``stability'' feature explained in \cite[\S1.9, \S14]{nelson-venkatesh-1}.)  It follows that (up to some $\eps$'s in the exponents)
\begin{equation*}
  H \cap J \subseteq \O(T^{-1/2}),
\end{equation*}
and so the volume of the integral in  \eqref{eq:int-_-substack} is $\O(T^{-1/2})$.

On the other hand, the magnitude of the integrand is
\begin{equation*}
  \vol(J)^{-1} \approx T.
\end{equation*}
Indeed, $J$ has dimensions roughly $1$ along one direction and $T^{-1/2}$ along the remaining two directions.

These observations combine to give the required estimate \eqref{eq:int-_-substack}.

\begin{remark}
  It's clear in retrospect that we should have obtained such an estimate: the orbit method heuristic applied to $\omega$ suggests that the left hand side of \eqref{eq:int-_-substack-1} is a proxy for the sum of $|L(\pi,\tfrac{1}{2})|^2$ over $T$ in an interval of width roughly $\O(T^\eps)$ (see \cite[\S2.3]{2020arXiv201202187N}), which is of the appropriate size for an averaged Lindel\"{o}f estimate to recover convexity.
\end{remark}

\section{Amplification}\label{sec:orgb296ba1}
This section is just a stub;  see \cite[\S1.5, \S2.7-2.10]{2020arXiv201202187N} for details and pictures relevant for this example.  To carry out the amplification step, we basically need to know that the vector $\varphi '$ obtained by averaging the translates of $\varphi$ under elements of $H$ of size $\asymp 1$ satisfies a matrix coefficient estimate $\langle g \varphi ', \varphi ' \rangle \ll T^{-\delta}$ except when $g$ is very close to $H$.  This is word-for-word what happens in the sup norm problem \cite{iwan-sar}, where, if we replace $H$ with $\SO(2)$, then $\varphi '$ becomes something like the weight zero vector.

\section{Comparison with Iwaniec}\label{sec:org60bf720}
The ``two arguments'' are ``equivalent.''  Recall that Iwaniec \cite{Iwaniec1992} starts with the approximate functional equation, applies Kuznetsov, and then applies Poisson summation to both variables.  The reduction to \eqref{eq:int-_y-in} is essentially the approximate functional equation (TODO: explain this in some detail?).  In the passage to \eqref{eq:int-_-substack-1}, rather than applying the pretrace formula, we could have instead applied the Fourier expansion of $\varphi$ and averaged each term in the resulting double sum using Kuznetsov.  A couple applications of Poisson to the geometric side of Kuznetsov would then bring us right back to \eqref{eq:int-_-substack-1}.

It's more interesting to compare the generalization of this argument to $\GL_3$ with Blomer--Buttcane \cite{MR4203038}.  The arguments are again ultimately ``equivalent,'' but the present approach seems less miraculous.



\bibliography{refs}{} \bibliographystyle{plain}
\end{document}
