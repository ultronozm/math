\documentclass[reqno]{amsart} \input{common.tex}

\numberwithin{equation}{section}
\numberwithin{theorem}{section}

\usepackage{xr}
\externaldocument{test2}

\title{Doesn't matter}

\begin{document}

\maketitle

\begin{abstract}
  We achieve very little.  More specifically, we consider the equations
  \(a^2 + b^2 = c^2\) and
  \begin{equation*}
a x^2 + b x + c = 0,
\end{equation*}
yet manage to say nothing interesting about either.
\end{abstract}

External reference to \href{test2.pdf}{Other File}, Theorem \ref{theorem:35ac204d35}

Oh hey, what's up.  We could take $x = y$, or we could solve
\begin{equation}\label{eqn:35ac136ef8}
  \begin{pmatrix}
    a & b \\
    c & d \\
  \end{pmatrix}^2 = 1.
\end{equation}
Equation \eqref{eqn:35ac136ef8} is standard.  See also \cite{2021arXiv210915230N}.

Let's add one more equation:
\begin{equation*}
\dim \PGL_2(\mathbb{R}) = 3.
\end{equation*}

Check out my lemma:
\begin{lemma}\label{lemma:35ac221761}
This is my lemma.
\end{lemma}
\begin{proof}
I said so.
\end{proof}
By Lemma \ref{lemma:35ac221761}, whatever.

They say that lunch time only comes once a day. Be prepared.
\setlength{\unitlength}{1.5cm}
\begin{figure}
  \begin{picture}(4,3)

    \put(-1,0){\vector(1,0){6}}
    \put(-1,0){\vector(0,1){1.5}}

    {%
      \thicklines
      \color{black}%
    }


    {%
      \thicklines
      \color{black}%

      {%
        \thicklines
        \color{black}%
        \qbezier(0,1)(2,0)(4,1)
        \put(-0.6,1.1){$\mathcal{O}_\pi$}
      }

      \color{black}
      \put(-1.4,1.4){$\xi''$}
      \put(4.9,-0.25){$\xi'$}
      \put(2,0.35){$\tau$}
      \put(1.95,0.5){\circle*{0.1}}
    }

    {%
      \thicklines
      \color{black}%
      \multiput(0.9,0.3)(0,0.1){4}{\line(0,1){0.05}}
      \multiput(2.8,0.3)(0,0.1){4}{\line(0,1){0.05}}
    }


    {%
      \thicklines
      \color{black}%
      \multiput(0.8,0.3)(0.1,0){20}{\line(1,0){0.05}}
      \multiput(0.8,0.7)(0.1,0){20}{\line(1,0){0.05}}
      \put(2.9, 0.4){$T^{\eps}$}
      \put(1.5,0.8){$T^{1/2+\eps}$}
    }



  \end{picture}
  \caption{ The coadjoint orbit $\mathcal{O}_\pi$ near $\tau$.  The dotted rectangle indicates the support of $a$.  }
  \label{fig:tau-coordinates-intro-0}
\end{figure}

\section{This is a section}

\begin{theorem}\label{theorem:d1a98951ccef}
This is a theorem in a section.
\end{theorem}

Test

\begin{exercise}\label{exercise:d1a990942cdb}
This is an exercise.
\end{exercise}

\begin{proposition}\label{proposition:d1a9909473de}
This is a proposition.
\end{proposition}


\bibliography{refs}{} \bibliographystyle{plain}
\end{document}
