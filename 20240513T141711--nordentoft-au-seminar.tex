\documentclass[reqno]{amsart} \usepackage{graphicx, amsmath, amssymb, amsfonts, amsthm, stmaryrd, amscd}
\usepackage[usenames, dvipsnames]{xcolor}
\usepackage{tikz}
% \usepackage{tikzcd}
% \usepackage{comment}

% \let\counterwithout\relax
% \let\counterwithin\relax
% \usepackage{chngcntr}

\usepackage{enumerate}
% \usepackage{enumitem}
% \usepackage{times}
\usepackage[normalem]{ulem}
% \usepackage{minted}
% \usepackage{xypic}
% \usepackage{color}


% \usepackage{silence}
% \WarningFilter{latex}{Label `tocindent-1' multiply defined}
% \WarningFilter{latex}{Label `tocindent0' multiply defined}
% \WarningFilter{latex}{Label `tocindent1' multiply defined}
% \WarningFilter{latex}{Label `tocindent2' multiply defined}
% \WarningFilter{latex}{Label `tocindent3' multiply defined}
\usepackage{hyperref}
% \usepackage{navigator}


% \usepackage{pdfsync}
\usepackage{xparse}


\usepackage[all]{xy}
\usepackage{enumerate}
\usetikzlibrary{matrix,arrows,decorations.pathmorphing}



\makeatletter
\newcommand*{\transpose}{%
  {\mathpalette\@transpose{}}%
}
\newcommand*{\@transpose}[2]{%
  % #1: math style
  % #2: unused
  \raisebox{\depth}{$\m@th#1\intercal$}%
}
\makeatother


\makeatletter
\newcommand*{\da@rightarrow}{\mathchar"0\hexnumber@\symAMSa 4B }
\newcommand*{\da@leftarrow}{\mathchar"0\hexnumber@\symAMSa 4C }
\newcommand*{\xdashrightarrow}[2][]{%
  \mathrel{%
    \mathpalette{\da@xarrow{#1}{#2}{}\da@rightarrow{\,}{}}{}%
  }%
}
\newcommand{\xdashleftarrow}[2][]{%
  \mathrel{%
    \mathpalette{\da@xarrow{#1}{#2}\da@leftarrow{}{}{\,}}{}%
  }%
}
\newcommand*{\da@xarrow}[7]{%
  % #1: below
  % #2: above
  % #3: arrow left
  % #4: arrow right
  % #5: space left 
  % #6: space right
  % #7: math style 
  \sbox0{$\ifx#7\scriptstyle\scriptscriptstyle\else\scriptstyle\fi#5#1#6\m@th$}%
  \sbox2{$\ifx#7\scriptstyle\scriptscriptstyle\else\scriptstyle\fi#5#2#6\m@th$}%
  \sbox4{$#7\dabar@\m@th$}%
  \dimen@=\wd0 %
  \ifdim\wd2 >\dimen@
    \dimen@=\wd2 %   
  \fi
  \count@=2 %
  \def\da@bars{\dabar@\dabar@}%
  \@whiledim\count@\wd4<\dimen@\do{%
    \advance\count@\@ne
    \expandafter\def\expandafter\da@bars\expandafter{%
      \da@bars
      \dabar@ 
    }%
  }%  
  \mathrel{#3}%
  \mathrel{%   
    \mathop{\da@bars}\limits
    \ifx\\#1\\%
    \else
      _{\copy0}%
    \fi
    \ifx\\#2\\%
    \else
      ^{\copy2}%
    \fi
  }%   
  \mathrel{#4}%
}
\makeatother
% \DeclareMathOperator{\rg}{rg}

\usepackage{mathtools}
\DeclarePairedDelimiter{\paren}{(}{)}
\DeclarePairedDelimiter{\abs}{\lvert}{\rvert}
\DeclarePairedDelimiter{\norm}{\lVert}{\rVert}
\DeclarePairedDelimiter{\innerproduct}{\langle}{\rangle}
\newcommand{\Of}[2]{{\operatorname{#1}} {\paren*{#2}}}
\newcommand{\of}[2]{{{{#1}} {\paren*{#2}}}}

\DeclareMathOperator{\Shim}{Shim}
\DeclareMathOperator{\sgn}{sgn}
\DeclareMathOperator{\fdeg}{fdeg}
\DeclareMathOperator{\SL}{SL}
\DeclareMathOperator{\slLie}{\mathfrak{s}\mathfrak{l}}
\DeclareMathOperator{\soLie}{\mathfrak{s}\mathfrak{o}}
\DeclareMathOperator{\spLie}{\mathfrak{s}\mathfrak{p}}
\DeclareMathOperator{\glLie}{\mathfrak{g}\mathfrak{l}}
\newcommand{\pn}[1]{{\color{ForestGreen} \sf PN: [#1]}}
\DeclareMathOperator{\Mp}{Mp}
\DeclareMathOperator{\Mat}{Mat}
\DeclareMathOperator{\GL}{GL}
\DeclareMathOperator{\Gr}{Gr}
\DeclareMathOperator{\GU}{GU}
\def\gl{\mathfrak{g}\mathfrak{l}}
\DeclareMathOperator{\odd}{odd}
\DeclareMathOperator{\even}{even}
\DeclareMathOperator{\GO}{GO}
\DeclareMathOperator{\good}{good}
\DeclareMathOperator{\bad}{bad}
\DeclareMathOperator{\PGO}{PGO}
\DeclareMathOperator{\htt}{ht}
\DeclareMathOperator{\height}{height}
\DeclareMathOperator{\Ass}{Ass}
\DeclareMathOperator{\coheight}{coheight}
\DeclareMathOperator{\GSO}{GSO}
\DeclareMathOperator{\SO}{SO}
\DeclareMathOperator{\so}{\mathfrak{s}\mathfrak{o}}
\DeclareMathOperator{\su}{\mathfrak{s}\mathfrak{u}}
\DeclareMathOperator{\ad}{ad}
% \DeclareMathOperator{\sc}{sc}
\DeclareMathOperator{\Ad}{Ad}
\DeclareMathOperator{\disc}{disc}
\DeclareMathOperator{\inv}{inv}
\DeclareMathOperator{\Pic}{Pic}
\DeclareMathOperator{\uc}{uc}
\DeclareMathOperator{\Cl}{Cl}
\DeclareMathOperator{\Clf}{Clf}
\DeclareMathOperator{\Hom}{Hom}
\DeclareMathOperator{\hol}{hol}
\DeclareMathOperator{\Heis}{Heis}
\DeclareMathOperator{\Haar}{Haar}
\DeclareMathOperator{\h}{h}
\def\sp{\mathfrak{s}\mathfrak{p}}
\DeclareMathOperator{\heis}{\mathfrak{h}\mathfrak{e}\mathfrak{i}\mathfrak{s}}
\DeclareMathOperator{\End}{End}
\DeclareMathOperator{\JL}{JL}
\DeclareMathOperator{\image}{image}
\DeclareMathOperator{\red}{red}
\def\div{\operatorname{div}}
\def\eps{\varepsilon}
\def\cHom{\mathcal{H}\operatorname{om}}
\DeclareMathOperator{\Ops}{Ops}
\DeclareMathOperator{\Symb}{Symb}
\def\boldGL{\mathbf{G}\mathbf{L}}
\def\boldSO{\mathbf{S}\mathbf{O}}
\def\boldU{\mathbf{U}}
\DeclareMathOperator{\hull}{hull}
\DeclareMathOperator{\LL}{LL}
\DeclareMathOperator{\PGL}{PGL}
\DeclareMathOperator{\class}{class}
\DeclareMathOperator{\lcm}{lcm}
\DeclareMathOperator{\spann}{span}
\DeclareMathOperator{\Exp}{Exp}
\DeclareMathOperator{\ext}{ext}
\DeclareMathOperator{\Ext}{Ext}
\DeclareMathOperator{\Tor}{Tor}
\DeclareMathOperator{\et}{et}
\DeclareMathOperator{\tor}{tor}
\DeclareMathOperator{\loc}{loc}
\DeclareMathOperator{\tors}{tors}
\DeclareMathOperator{\pf}{pf}
\DeclareMathOperator{\smooth}{smooth}
\DeclareMathOperator{\prin}{prin}
\DeclareMathOperator{\Kl}{Kl}
\newcommand{\kbar}{\mathchar'26\mkern-9mu k}
\DeclareMathOperator{\der}{der}
% \DeclareMathOperator{\abs}{abs}
\DeclareMathOperator{\Sub}{Sub}
\DeclareMathOperator{\Comp}{Comp}
\DeclareMathOperator{\Err}{Err}
\DeclareMathOperator{\dom}{dom}
\DeclareMathOperator{\radius}{radius}
\DeclareMathOperator{\Fitt}{Fitt}
\DeclareMathOperator{\Sel}{Sel}
\DeclareMathOperator{\rad}{rad}
\DeclareMathOperator{\id}{id}
\DeclareMathOperator{\Center}{Center}
\DeclareMathOperator{\Der}{Der}
\DeclareMathOperator{\U}{U}
% \DeclareMathOperator{\norm}{norm}
\DeclareMathOperator{\trace}{trace}
\DeclareMathOperator{\Equid}{Equid}
\DeclareMathOperator{\Feas}{Feas}
\DeclareMathOperator{\bulk}{bulk}
\DeclareMathOperator{\tail}{tail}
\DeclareMathOperator{\sys}{sys}
\DeclareMathOperator{\atan}{atan}
\DeclareMathOperator{\temp}{temp}
\DeclareMathOperator{\Asai}{Asai}
\DeclareMathOperator{\glob}{glob}
\DeclareMathOperator{\Kuz}{Kuz}
\DeclareMathOperator{\Irr}{Irr}
\newcommand{\rsL}{ \frac{ L^{(R)}(\Pi \times \Sigma, \std, \frac{1}{2})}{L^{(R)}(\Pi \times \Sigma, \Ad, 1)}  }
\DeclareMathOperator{\GSp}{GSp}
\DeclareMathOperator{\PGSp}{PGSp}
\DeclareMathOperator{\BC}{BC}
\DeclareMathOperator{\Ann}{Ann}
\DeclareMathOperator{\Gen}{Gen}
\DeclareMathOperator{\SU}{SU}
\DeclareMathOperator{\PGSU}{PGSU}
% \DeclareMathOperator{\gen}{gen}
\DeclareMathOperator{\PMp}{PMp}
\DeclareMathOperator{\PGMp}{PGMp}
\DeclareMathOperator{\PB}{PB}
\DeclareMathOperator{\ind}{ind}
\DeclareMathOperator{\Jac}{Jac}
\DeclareMathOperator{\jac}{jac}
\DeclareMathOperator{\im}{im}
\DeclareMathOperator{\Aut}{Aut}
\DeclareMathOperator{\Int}{Int}
\DeclareMathOperator{\PSL}{PSL}
\DeclareMathOperator{\co}{co}
\DeclareMathOperator{\irr}{irr}
\DeclareMathOperator{\prim}{prim}
\DeclareMathOperator{\bal}{bal}
\DeclareMathOperator{\baln}{bal}
\DeclareMathOperator{\dist}{dist}
\DeclareMathOperator{\RS}{RS}
\DeclareMathOperator{\Ram}{Ram}
\DeclareMathOperator{\Sob}{Sob}
\DeclareMathOperator{\Sol}{Sol}
\DeclareMathOperator{\soc}{soc}
\DeclareMathOperator{\nt}{nt}
\DeclareMathOperator{\mic}{mic}
\DeclareMathOperator{\Gal}{Gal}
\DeclareMathOperator{\st}{st}
\DeclareMathOperator{\std}{std}
\DeclareMathOperator{\diag}{diag}
\DeclareMathOperator{\Sym}{Sym}
\DeclareMathOperator{\gr}{gr}
\DeclareMathOperator{\aff}{aff}
\DeclareMathOperator{\Dil}{Dil}
\DeclareMathOperator{\Lie}{Lie}
\DeclareMathOperator{\Symp}{Symp}
\DeclareMathOperator{\Stab}{Stab}
\DeclareMathOperator{\St}{St}
\DeclareMathOperator{\stab}{stab}
\DeclareMathOperator{\codim}{codim}
\DeclareMathOperator{\linear}{linear}
\newcommand{\git}{/\!\!/}
\DeclareMathOperator{\geom}{geom}
\DeclareMathOperator{\spec}{spec}
\def\O{\operatorname{O}}
\DeclareMathOperator{\Au}{Aut}
\DeclareMathOperator{\Fix}{Fix}
\DeclareMathOperator{\Opp}{Op}
\DeclareMathOperator{\opp}{op}
\DeclareMathOperator{\Size}{Size}
\DeclareMathOperator{\Save}{Save}
% \DeclareMathOperator{\ker}{ker}
\DeclareMathOperator{\coker}{coker}
\DeclareMathOperator{\sym}{sym}
\DeclareMathOperator{\mean}{mean}
\DeclareMathOperator{\elliptic}{ell}
\DeclareMathOperator{\nilpotent}{nil}
\DeclareMathOperator{\hyperbolic}{hyp}
\DeclareMathOperator{\newvector}{new}
\DeclareMathOperator{\new}{new}
\DeclareMathOperator{\full}{full}
\newcommand{\qr}[2]{\left( \frac{#1}{#2} \right)}
\DeclareMathOperator{\unr}{u}
\DeclareMathOperator{\ram}{ram}
% \DeclareMathOperator{\len}{len}
\DeclareMathOperator{\fin}{fin}
\DeclareMathOperator{\cusp}{cusp}
\DeclareMathOperator{\curv}{curv}
\DeclareMathOperator{\rank}{rank}
\DeclareMathOperator{\rk}{rk}
\DeclareMathOperator{\pr}{pr}
\DeclareMathOperator{\Transform}{Transform}
\DeclareMathOperator{\mult}{mult}
\DeclareMathOperator{\Eis}{Eis}
\DeclareMathOperator{\reg}{reg}
\DeclareMathOperator{\sing}{sing}
\DeclareMathOperator{\alt}{alt}
\DeclareMathOperator{\irreg}{irreg}
\DeclareMathOperator{\sreg}{sreg}
\DeclareMathOperator{\Wd}{Wd}
\DeclareMathOperator{\Weil}{Weil}
\DeclareMathOperator{\Th}{Th}
\DeclareMathOperator{\Sp}{Sp}
\DeclareMathOperator{\Ind}{Ind}
\DeclareMathOperator{\Res}{Res}
\DeclareMathOperator{\ini}{in}
\DeclareMathOperator{\ord}{ord}
\DeclareMathOperator{\osc}{osc}
\DeclareMathOperator{\fluc}{fluc}
\DeclareMathOperator{\size}{size}
\DeclareMathOperator{\ann}{ann}
\DeclareMathOperator{\equ}{eq}
\DeclareMathOperator{\res}{res}
\DeclareMathOperator{\pt}{pt}
\DeclareMathOperator{\src}{source}
\DeclareMathOperator{\Zcl}{Zcl}
\DeclareMathOperator{\Func}{Func}
\DeclareMathOperator{\Map}{Map}
\DeclareMathOperator{\Frac}{Frac}
\DeclareMathOperator{\Frob}{Frob}
\DeclareMathOperator{\ev}{eval}
\DeclareMathOperator{\pv}{pv}
\DeclareMathOperator{\eval}{eval}
\DeclareMathOperator{\Spec}{Spec}
\DeclareMathOperator{\Speh}{Speh}
\DeclareMathOperator{\Spin}{Spin}
\DeclareMathOperator{\GSpin}{GSpin}
\DeclareMathOperator{\Specm}{Specm}
\DeclareMathOperator{\Sphere}{Sphere}
\DeclareMathOperator{\Sqq}{Sq}
\DeclareMathOperator{\Ball}{Ball}
\DeclareMathOperator\Cond{\operatorname{Cond}}
\DeclareMathOperator\proj{\operatorname{proj}}
\DeclareMathOperator\Swan{\operatorname{Swan}}
\DeclareMathOperator{\Proj}{Proj}
\DeclareMathOperator{\bPB}{{\mathbf P}{\mathbf B}}
\DeclareMathOperator{\Projm}{Projm}
\DeclareMathOperator{\Tr}{Tr}
\DeclareMathOperator{\Type}{Type}
\DeclareMathOperator{\Prop}{Prop}
\DeclareMathOperator{\vol}{vol}
\DeclareMathOperator{\covol}{covol}
\DeclareMathOperator{\Rep}{Rep}
\DeclareMathOperator{\Cent}{Cent}
\DeclareMathOperator{\val}{val}
\DeclareMathOperator{\area}{area}
\DeclareMathOperator{\nr}{nr}
\DeclareMathOperator{\CM}{CM}
\DeclareMathOperator{\CH}{CH}
\DeclareMathOperator{\tr}{tr}
\DeclareMathOperator{\characteristic}{char}
\DeclareMathOperator{\supp}{supp}


\theoremstyle{plain} \newtheorem{theorem} {Theorem} \newtheorem{conjecture} [theorem] {Conjecture} \newtheorem{corollary} [theorem] {Corollary} \newtheorem{proposition} [theorem] {Proposition} \newtheorem{fact} [theorem] {Fact}
\theoremstyle{definition} \newtheorem{definition} [theorem] {Definition} \newtheorem{hypothesis} [theorem] {Hypothesis} \newtheorem{assumptions} [theorem] {Assumptions}
\newtheorem{example} [theorem] {Example}
\newtheorem{assertion}[theorem] {Assertion}
\newtheorem{note}[theorem] {Note}
\newtheorem{conclusion}[theorem] {Conclusion}
\newtheorem{claim}            {Claim}
\newtheorem{homework} {Homework}
\newtheorem{exercise} {Exercise}  \newtheorem{question}[theorem] {Question}    \newtheorem{answer} {Answer}  \newtheorem{problem} {Problem}    \newtheorem{remark} [theorem] {Remark}
\newtheorem{notation} [theorem]           {Notation}
\newtheorem{terminology}[theorem]            {Terminology}
\newtheorem{convention}[theorem]            {Convention}
\newtheorem{motivation}[theorem]            {Motivation}


\newtheoremstyle{itplain} % name
{6pt}                    % Space above
{5pt\topsep}                    % Space below
{\itshape}                   % Body font
{}                           % Indent amount
{\itshape}                   % Theorem head font
{.}                          % Punctuation after theorem head
{5pt plus 1pt minus 1pt}                       % Space after theorem head
% {.5em}                       % Space after theorem head
{}  % Theorem head spec (can be left empty, meaning ‘normal’)

% \theoremstyle{mytheoremstyle}


\theoremstyle{itplain} %--default
% \theoremheaderfont{\itshape}
% \newtheorem{lemma}{Lemma}
\newtheorem{lemma}[theorem]{Lemma}
% \newtheorem{lemma}{Lemma}[subsubsection]

\newtheorem*{lemma*}{Lemma}
\newtheorem*{proposition*}{Proposition}
\newtheorem*{definition*}{Definition}
\newtheorem*{example*}{Example}

\newtheorem*{results*}{Results}
\newtheorem{results} [theorem] {Results}


\usepackage[displaymath,textmath,sections,graphics]{preview}
\PreviewEnvironment{align*}
\PreviewEnvironment{multline*}
\PreviewEnvironment{tabular}
\PreviewEnvironment{verbatim}
\PreviewEnvironment{lstlisting}
\PreviewEnvironment*{frame}
\PreviewEnvironment*{alert}
\PreviewEnvironment*{emph}
\PreviewEnvironment*{textbf}



\begin{document}

\title{Horizontal $p$-adic $L$-functions}

\begin{abstract}
  Talk by Asbj{\o}rn Nordentoft at the seminar at Aarhus on 13 May 2024.  Joint work with Daniel Kriz.
\end{abstract}

\maketitle

\section{Introduction}
Let $E$ be an elliptic curve over $\mathbb{Q}$.  Let $L/K$ be a Galois extension of number fields.  Then there is the following concept, due to Mazur and Rubin.
\begin{definition}
  We say that $E$ is \emph{diophantine stable} (abbreviated $\mathrm{D S}$) relative to $L/K$ if
  \begin{equation*}
    \rank_{\mathbb{Z}} E(L) = \rank_{\mathbb{Z}} E(K).
  \end{equation*}
\end{definition}
\begin{question}
  How often is $E$ diophantine stable for $L/K$?
\end{question}
\begin{remark}
  The notion of diophantine stability has applications to Hilbert's tenth problem.  This was part of the original motivation.  From our point of view, it's just a natural question.
\end{remark}
There is an analytic counterpart to this question.  Let $F /\mathbb{Q}$ be an abelian extension of $\mathbb{Q}$.  Assuming the BSD conjecture, the question of diophantine stability takes a different form:d
\begin{align*}
  \rank_{\mathbb{Z}} E(F)
  &= \ord_{s = 1} L(E/F, s) \\
  &=
    \sum_{\chi \in \widehat{\Gal(F / \mathbb{Q})}}
    \ord_{s = 1} L(E, \chi, s) \\
  &=
    \rank_{\mathbb{Z}} E(\mathbb{Q})
    \sum_{1 \neq \chi \in \widehat{\Gal(F / \mathbb{Q})}}
    \ord_{s = 1} L(E, \chi, s).
\end{align*}
where
\begin{equation*}
  L(E, \chi, s) = \sum_{n \geq 1} a_E(n) \chi(n) n^{- s}.
\end{equation*}
The upshot is that diophantine stability for $F/\mathbb{Q}$ is related, under BSD, to understanding when $L(E, \chi, 1) \neq 0$ for $\chi \in \widehat{\Gal(F / \mathbb{Q})}$ with $\chi \neq 1$.

\begin{question}
  How often is $L(E, \chi, 1) \neq 0$ for Dirichlet characters $\chi$?
\end{question}
\section{Vertical analysis}
Let $F_n = \mathbb{Q}(\mu_p)$, $F _\infty = \cup_{n \geq 1} F_n$ ($p$th cyclotomic extension of $\mathbb{Q}$).
\begin{theorem}[Mazur]
  If $E$ is good at $p$, then $\rank_{\mathbb{Z}} E(F _\infty) < \infty$.
\end{theorem}
\begin{remark}
  In our langauge of Diophantine stability, this means that if $n$ is sufficiently large, then $E$ is diophantine stable for $F _\infty / F_n$.
\end{remark}
\begin{theorem}[Rohrlich]
  We have $L(E, \chi, 1) \neq 0$ for all but finitely many $\chi$, a Dirichlet character of $p$-power conductor.  This is an exercise in class field theory.
\end{theorem}
\begin{remark}
  The results of Mazur and Rohrlich should be equivalent under BSD, but the speaker doesn't know of a direct way to go between them.
\end{remark}
\section{Horizontal analysis}
Fix $d \geq 2$.  Let $L / \mathbb{Q}$ be a cyclic extension, of order $d$.  (The corresponding Dirichlet characters $\chi$ then have order $d$.)

\begin{conjecture}[Goldfeld]
  Suppose $d = 2$, and let $\chi_D$ be a quadratic character of conductor $D$.  Then
  \begin{equation*}
    \ord_{s = 1} L(E, \chi_D, s) =
    \begin{cases}
      0      & \text{with probability } 50\%, \\
      1      & \text{with probability } 50\%, \\
      \geq 2      & \text{with probability } 0\%. \\
    \end{cases}
  \end{equation*}
\end{conjecture}
Previous work when $d = 2$:
\begin{itemize}
\item Friedberg--Hoffstein, 1990's
\item Murty--Murty, Ono--Skinner ($\gg X / \log X$);
\item Smith--Kriz
\end{itemize}
How about when $d > 2$:
\begin{conjecture}[David--Fearnley--Kisilenski]
  % We have $L(E, \chi, 1) \neq 0$ for 100\% of all $\chi$ of order $d$.  (Now there is no forced vanishing like before.)  Not much is known.  Fearnley--Kisilevsky--Kawata (restrictive-setting)

\end{conjecture}

\section{Results}
We denote by $F_d(X)$ the set of relevant Dirichlet characters $\chi \pmod{D}$ with $D \leq X$ of order $\tau$.  There exists $c_d > 0$ such that
\begin{equation*}
  \# F_d(X) \sim c_d X(\log X)^{\sigma_0(d) - 2}.
\end{equation*}
Here $\sigma_0(d)$ denotes the number of divisors of $d$.
\begin{theorem} [Kriz--N 23]
  Let $d \equiv 2\pod{4}$, $d > 2$.  Then there exists $\alpha = \alpha(E, d) > 0$ such that
  \begin{equation*}
    \# \left\{ x \in F_d(X) : L(E, \chi, 1) \neq 0 \right\} \gg \frac{X}{(\log X)^{- \alpha}}.
  \end{equation*}
\end{theorem}
\begin{remark}
  Not previously knwon in this setting for $n=6$.
\end{remark}

We can also study a more general case, by imposing further assumptions.
\begin{theorem}
  We get the same conclusions in the following cases:
  \begin{enumerate}
  \item $L(E, 1) \neq 0$ and there exists $p \mid d$ such that $\overline{\rho}_{E, p}$ is irreducible,
  \item $2 \mid d$ and $\overline{\rho}_{E, 2}$ is irreducible.
  \end{enumerate}
\end{theorem}


\begin{remark}
  $d = 2^m, \overline{\rho_{E, 2}}$. if and only if $E(\mathbb{Q})[2] = 0$ (satisfied for 100\%) and
  \begin{equation*}
    \alpha = \alpha(2^m, E)
    =
    \begin{cases}
      m/3        & \im(\bar{\rho}_{E, 2}) \cong S_3, \\
      2 m / 3                   &   \cong \mathbb{Z}/3.
    \end{cases}
  \end{equation*}
\end{remark}


For $d = 2$, this improves on Ono from 2000's.

\begin{theorem}[Simultaneous nonvanishing]
  For 100\% of tuples of elliptic curves $E_1, \dotsc, E_m$ such that $L(E_i, 1) \neq 0$, it holds that: for each $d$ and $\alpha > 0$, we have
  \begin{equation*}
    \# \left\{ \chi \in F_d(X) : L(E_i, \chi, 1) \neq 0 \,\forall i \right\} \gg \frac{X}{(\log X)^{1 - \alpha}}.
  \end{equation*}
\end{theorem}

The proofs uses horizontal $p$-adic $L$-functions.

\begin{definition}
  Fix a prime $p$.  Let $E/\mathbb{Q}$ be an elliptic curve, of conductor $N$.  We say that $\ell$ is a \emph{Taylor--Wiles prime} for $(E, p)$ if the following three conditions are met:
  \begin{enumerate}
  \item\label{enumerate:cnjgh144b4} $(\ell, N) = 1$
  \item\label{enumerate:cnjgh142qb} $\ell \equiv 1 \pod{p}$
  \item\label{enumerate:cnjgh14zi3} $a_E(\ell) \not \equiv 2 \pod{p}$.
  \end{enumerate}
  We say that $p$ is $E$\emph{-good} if Taylor--Wiles primes have positive density.
\end{definition}
\begin{remark}
  The last condition \eqref{enumerate:cnjgh14zi3} is really one concerning the mod $p$ Galois representation $\overline{\rho}_{E, p}$: if the latter is irreducible, then $p$ is $E$-good.
\end{remark}
\begin{remark}
  If $E$ is non-CM and $p \geq 13$, then $p$ is $E$-good (Zywina).
\end{remark}
Thus, let $p$ be $E$-good.  We consider $\mathcal{L} =(\ell_n)_n$, where $\ell_n \equiv 1 \pod{p}$ and $\ell_n$ is a Taylor--Wiles prime for all sufficiently large $n$.  Set
\begin{equation*}
  m_n := v_p(\ell_n - 1) \geq 1.
\end{equation*}
Defint eh \emph{horizontal Iwasawa algebra}
\begin{equation*}
  \Lambda^{\mathrm{hor}}
  :=
  \mathbb{Z}_p
  \left[ \left[ \prod_{ n \in \mathbb{N}} \mathbb{Z} / p^{m n} \right] \right]
  :=
  \varprojlim_{
    \substack{
      A \leq \mathbb{N}  \\
      \text{finite}      
    }
  }
  \cong \Hom_{\mathrm{cts}} \left( \mathcal{C}(\prod \mathbb{Z} / p^{m_n}, \mathbb{Z}_p), \mathbb{Z}_p \right).
\end{equation*}
Associated to the elliptic curve, we define (using modular symbols) a measure
\begin{equation*}
  \nu_E \in \Lambda^{\mathrm{}}
\end{equation*}
that interpolates the twistsby characters of $p$-power order, i.e.,
\begin{equation*}
  \chi : \prod_{n \in \mathbb{N}} \left( \mathbb{Z} / \ell_n \right)^\ast \rightarrow \bar{\mathbb{Q}} ^\times.
\end{equation*}<++>

Now if $\chi$ has $p$-power order, then it factors through some character
\begin{equation*}
  \tilde{\chi} : \prod_{n \in \mathbb{N} } \mathbb{Z} / p^{m_n} \rightarrow \bar{\mathbb{Q}}^\times.
\end{equation*}
The connection is that
\begin{equation*}
  \nu_E(\tilde{\chi}) = (\text{Euler factor}) \cdot  \frac{L(E, \chi, 1) \tau(\bar{\chi})}{\Omega_E}.
\end{equation*}
Pushing forward along
\begin{equation*}
  \prod \mathbb{Z} / p^{m_n} \twoheadrightarrow \prod \mathbb{Z} / p =(\mathbb{Z} / p)^\infty,
\end{equation*}
get
\begin{equation*}
  \overline{\nu_E} \in \mathbb{Z}_p[[(\mathbb{Z} / p)^\infty]]
  = \Lambda^{\mathrm{diag}}.
\end{equation*}


\begin{theorem}[K--N]
  Let $0 \neq\nu \in \Lambda^{\mathrm{diag}}$.  Then $\nu(x) \neq 0$ for a ``positive proportion'' of characters $\chi :(\mathbb{Z} / p)^\infty \rightarrow \bar{\mathbb{Q}}^\times$.  There exists $I \subseteq \mathbb{N}$, a finite set, such that for all $\chi :(\mathbb{Z} / p)^\infty \rightarrow \bar{\mathbb{Q}}^\times$, there exists $\chi_I : \prod_{n \in I} \mathbb{Z} / p \rightarrow \bar{\mathbb{Q}}^\times$ such that $\nu(\chi \chi_I) \neq 0.$
\end{theorem}

The \emph{upshot} is that it suffices to prove $\overline{\nu_E} \neq 0$.  This follows if we can show any of the following:
\begin{equation*}
  \begin{cases}
    &   L(E, 1) = \bar{\nu }_E(1) \neq 0 \\
    p = 2    & \text{Friedberg--Hoffstein}, \\
    \text{general } p                                          & \text{use Kurihara's conjecture.}
  \end{cases}
\end{equation*}
\bibliography{refs}{} \bibliographystyle{plain}
\end{document}
