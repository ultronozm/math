\documentclass[reqno]{amsart} \input{common.tex}

\begin{document}

\emph{Horizontal} $p$\emph{-adic} $L$\emph{-functions}, joint with Daniel Kriz.

\section{Introduction}
Let $E$ be an elliptic curve over $\mathbb{Q}$.  Let $L/K$ be a Galois extension of number fields.  Then there is the following concept, due to Mazur and Rubin.
\begin{definition}
  We say that $E$ is \emph{diophantine stable} (abbreviated $\mathrm{D S}$) relative to $L/K$ if
  \begin{equation*}
    \rank_{\mathbb{Z}} E(L) = \rank_{\mathbb{Z}} E(K).
  \end{equation*}
\end{definition}
\begin{question}
  How often is $E$ diophantine stable for $L/K$?
\end{question}
\begin{remark}
  The notion of diophantine stability has applications to Hilbert's tenth problem.  This was part of the original motivation.  From our point of view, it's just a natural question.
\end{remark}
There is an analytic counterpart to this question.  Let $F /\mathbb{Q}$ be an abelian extension of $\mathbb{Q}$.  Assuming the BSD conjecture, the question of diophantine stability takes a different form:d
\begin{align*}
  \rank_{\mathbb{Z}} E(F)
  &= \ord_{s = 1} L(E/F, s) \\
  &=
  \sum_{\chi \in \widehat{\Gal(F / \mathbb{Q})}}
    \ord_{s = 1} L(E, \chi, s) \\
  &=
    \rank_{\mathbb{Z}} E(\mathbb{Q})
    \sum_{1 \neq \chi \in \widehat{\Gal(F / \mathbb{Q})}}
    \ord_{s = 1} L(E, \chi, s).
\end{align*}
where
\begin{equation*}
  L(E, \chi, s) = \sum_{n \geq 1} a_E(n) \chi(n) n^{- s}.
\end{equation*}
The upshot is that diophantine stability for $F/\mathbb{Q}$ is related, under BSD, to understanding when $L(E, \chi, 1) \neq 0$ for $\chi \in \widehat{\Gal(F / \mathbb{Q})}$ with $\chi \neq 1$.

\begin{question}
  How often is $L(E, \chi, 1) \neq 0$ for Dirichlet characters $\chi$?
\end{question}
\section{Vertical analysis}
Let $F_n = \mathbb{Q}(\mu_p)$, $F _\infty = \cup_{n \geq 1} F_n$ ($p$th cyclotomic extension of $\mathbb{Q}$).
\begin{theorem}[Mazur]
  If $E$ is good at $p$, then $\rank_{\mathbb{Z}} E(F _\infty) < \infty$.
\end{theorem}
\begin{remark}
  In our langauge of Diophantine stability, this means that if $n$ is sufficiently large, then $E$ is diophantine stable  for $F _\infty / F_n$.
\end{remark}
\begin{theorem}[Rohrlich]
  We have $L(E, \chi, 1) \neq 0$ for all but finitely many $\chi$, a Dirichlet character of $p$-power conductor.  This is an exercise in class field theory.
\end{theorem}
\begin{remark}
  The results of Mazur and Rohrlich should be equivalent under BSD, but the speaker doesn't know of a direct way to go between them.
\end{remark}
\section{Horizontal analysis}
Fix $d \geq 2$.  Let $L / \mathbb{Q}$ be a cyclic extension, of order $d$.  (The corresponding Dirichlet characters $\chi$ then have order $d$.)

\begin{conjecture}[Goldfeld]
  Suppose $d = 2$, and let $\chi_D$ be a quadratic character of conductor $D$.  Then
  \begin{equation*}
    \ord_{s = 1} L(E, \chi_D, s) =
    \begin{cases}
      0      & \text{with probability } 50\%, \\
      1      & \text{with probability } 50\%, \\
      \geq 2      & \text{with probability } 0\%. \\
    \end{cases}
  \end{equation*}
\end{conjecture}
Previous work when $d = 2$:
\begin{itemize}
\item Friedberg--Hoffstein, 1990's
\item Murty--Murty, Ono--Skinner ($\gg X / \log X$); 
\item Smith--Kriz
\end{itemize}
How about when $d > 2$:
\begin{conjecture}[David--Fearnley--Kisilenski]
  % We have $L(E, \chi, 1) \neq 0$ for 100\% of all $\chi$ of order $d$.  (Now there is no forced vanishing like before.)  Not much is known.  Fearnley--Kisilevsky--Kawata  (restrictive-setting)

\end{conjecture}

\section{Results}
We denote by $F_d(X)$ the set of relevant Dirichlet characters $\chi \pmod{D}$ with $D \leq X$ of order $\tau$.  There exists $c_d > 0$ such that
\begin{equation*}
  \# F_d(X) \sim c_d X(\log X)^{\sigma_0(d) - 2}.
\end{equation*}
Here $\sigma_0(d)$ denotes the number of divisors of $d$.
\begin{theorem}
  [Kriz--N 23]
  Let $d \equiv 2\pod{4}$, $d > 2$.  Then there exists $\alpha = \alpha(E, d) > 0$ such that
  \begin{equation*}
    \# \left\{ x \in F_d(X) : L(E, \chi, 1) \neq 0 \right\} \gg \frac{X}{(\log X)^{- \alpha}}.
  \end{equation*}
\end{theorem}
\begin{remark}
  Not previously knwon in this setting for $n=6$.
\end{remark}

We can also study a more general case, by imposing further assumptions.
\begin{theorem}
  We get the same conclusions in the following cases:
  \begin{enumerate}
  \item $L(E, 1) \neq 0$ and there exists $p \mid d$ such that $\overline{\rho}_{E, p}$ is irreducible,
  \item $2 \mid d$ and $\overline{\rho}_{E, 2}$ is irreducible.
  \end{enumerate}  
\end{theorem}


\begin{remark}
  $d = 2^m, \overline{\rho_{E, 2}}$. if and only if $E(\mathbb{Q})[2] = 0$ (satisfied for 100\%) and
  \begin{equation*}
      \alpha = \alpha(2^m, E)
      =
      \begin{cases}
        m/3        & \im(\bar{\rho}_{E, 2}) \cong S_3, \\
        2 m / 3                   &   \cong \mathbb{Z}/3.
      \end{cases}
    \end{equation*}
\end{remark}


For $d = 2$, this improves on Ono from 2000's.

\begin{theorem}[Simultaneous nonvanishing]
  For 100\% of tuples of elliptic curves $E_1, \dotsc, E_m$ such that $L(E_i, 1) \neq 0$, it holds that: for each $d$ and $\alpha > 0$, we have
  \begin{equation*}
    \# \left\{ \chi \in F_d(X) : L(E_i, \chi, 1) \neq 0 \,\forall i \right\} \gg \frac{X}{(\log X)^{1 - \alpha}}.
  \end{equation*}
\end{theorem}

The proofs uses horizontal $p$-adic $L$-functions.

\begin{definition}
  Fix a prime $p$.  Let $E/\mathbb{Q}$ be an elliptic curve, of conductor $N$.  We say that $\ell$ is a \emph{Taylor--Wiles prime} for $(E, p)$ if the following three conditions are met:
  \begin{enumerate}
  \item\label{enumerate:cnjgh144b4} $(\ell, N) = 1$
  \item\label{enumerate:cnjgh142qb} $\ell \equiv 1 \pod{p}$
  \item\label{enumerate:cnjgh14zi3} $a_E(\ell) \not \equiv 2 \pod{p}$.
  \end{enumerate}
  We say that $p$ is $E$\emph{-good} if Taylor--Wiles primes have positive density.
\end{definition}
\begin{remark}
  The last condition \eqref{enumerate:cnjgh14zi3} is really one concerning the mod $p$ Galois representation $\overline{\rho}_{E, p}$: if the latter is irreducible, then $p$ is $E$-good.
\end{remark}
\begin{remark}
  If $E$ is non-CM and $p \geq 13$, then $p$ is $E$-good (Zywina).
\end{remark}
Thus, let $p$ be $E$-good.  We consider $\mathcal{L} =(\ell_n)_n$, where $\ell_n \equiv 1 \pod{p}$ and $\ell_n$ is a Taylor--Wiles prime for all sufficiently large $n$.  Set
\begin{equation*}
  m_n := v_p(\ell_n - 1) \geq 1.
\end{equation*}
Defint eh \emph{horizontal Iwasawa algebra}
\begin{equation*}
  \Lambda^{\mathrm{hor}}
  :=
  \mathbb{Z}_p
  \left[ \left[ \prod_{ n \in \mathbb{N}} \mathbb{Z} / p^{m n} \right] \right]
  :=
  \varprojlim_{
    \substack{
      A \leq \mathbb{N}  \\
      \text{finite}      
    }
  }
  \cong \Hom_{\mathrm{cts}} \left( \mathcal{C}(\prod \mathbb{Z} / p^{m_n}, \mathbb{Z}_p), \mathbb{Z}_p \right).
\end{equation*}
Associated to the elliptic curve, we define (using modular symbols) a measure
\begin{equation*}
  \nu_E \in \Lambda^{\mathrm{}}
\end{equation*}
that interpolates the twistsby characters of $p$-power order, i.e.,
\begin{equation*}
  \chi : \prod_{n \in \mathbb{N}} \left( \mathbb{Z} / \ell_n \right)^\ast \rightarrow \bar{\mathbb{Q}} ^\times.
\end{equation*}<++>

Now if $\chi$ has $p$-power order, then it factors through some character
\begin{equation*}
  \tilde{\chi} : \prod_{n \in \mathbb{N} } \mathbb{Z} / p^{m_n} \rightarrow \bar{\mathbb{Q}}^\times.
\end{equation*}
The connection is that
\begin{equation*}
  \nu_E(\tilde{\chi}) = (\text{Euler factor}) \cdot  \frac{L(E, \chi, 1) \tau(\bar{\chi})}{\Omega_E}.
\end{equation*}
Pushing forward along
\begin{equation*}
  \prod \mathbb{Z} / p^{m_n} \twoheadrightarrow \prod \mathbb{Z} / p =(\mathbb{Z} / p)^\infty,
\end{equation*}
get
\begin{equation*}
  \overline{\nu_E} \in \mathbb{Z}_p[[(\mathbb{Z} / p)^\infty]]
  = \Lambda^{\mathrm{diag}}.
\end{equation*}


\begin{theorem}[K--N]
  Let $0 \neq\nu \in \Lambda^{\mathrm{diag}}$.  Then $\nu(x) \neq 0$ for a ``positive proportion'' of characters $\chi :(\mathbb{Z} / p)^\infty \rightarrow \bar{\mathbb{Q}}^\times$.  There exists $I \subseteq \mathbb{N}$, a finite set, such that for all $\chi :(\mathbb{Z} / p)^\infty \rightarrow \bar{\mathbb{Q}}^\times$, there exists $\chi_I : \prod_{n \in I} \mathbb{Z} / p \rightarrow \bar{\mathbb{Q}}^\times$ such that $\nu(\chi \chi_I) \neq 0.$
\end{theorem}

The \emph{upshot} is that it suffices to prove $\overline{\nu_E} \neq 0$.  This follows if we can show any of the following:
\begin{equation*}
  \begin{cases}
    &   L(E, 1) = \bar{\nu }_E(1) \neq 0 \\
    p = 2    & \text{Friedberg--Hoffstein}, \\
    \text{general } p                                          & \text{use Kurihara's conjecture.}
  \end{cases}
\end{equation*}
\bibliography{refs}{} \bibliographystyle{plain}
\end{document}
