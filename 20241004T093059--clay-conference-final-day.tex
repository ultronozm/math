\documentclass[reqno]{amsart} \usepackage{graphicx, amsmath, amssymb, amsfonts, amsthm, stmaryrd, amscd}
\usepackage[usenames, dvipsnames]{xcolor}
\usepackage{tikz}
% \usepackage{tikzcd}
% \usepackage{comment}

% \let\counterwithout\relax
% \let\counterwithin\relax
% \usepackage{chngcntr}

\usepackage{enumerate}
% \usepackage{enumitem}
% \usepackage{times}
\usepackage[normalem]{ulem}
% \usepackage{minted}
% \usepackage{xypic}
% \usepackage{color}


% \usepackage{silence}
% \WarningFilter{latex}{Label `tocindent-1' multiply defined}
% \WarningFilter{latex}{Label `tocindent0' multiply defined}
% \WarningFilter{latex}{Label `tocindent1' multiply defined}
% \WarningFilter{latex}{Label `tocindent2' multiply defined}
% \WarningFilter{latex}{Label `tocindent3' multiply defined}
\usepackage{hyperref}
% \usepackage{navigator}


% \usepackage{pdfsync}
\usepackage{xparse}


\usepackage[all]{xy}
\usepackage{enumerate}
\usetikzlibrary{matrix,arrows,decorations.pathmorphing}



\makeatletter
\newcommand*{\transpose}{%
  {\mathpalette\@transpose{}}%
}
\newcommand*{\@transpose}[2]{%
  % #1: math style
  % #2: unused
  \raisebox{\depth}{$\m@th#1\intercal$}%
}
\makeatother


\makeatletter
\newcommand*{\da@rightarrow}{\mathchar"0\hexnumber@\symAMSa 4B }
\newcommand*{\da@leftarrow}{\mathchar"0\hexnumber@\symAMSa 4C }
\newcommand*{\xdashrightarrow}[2][]{%
  \mathrel{%
    \mathpalette{\da@xarrow{#1}{#2}{}\da@rightarrow{\,}{}}{}%
  }%
}
\newcommand{\xdashleftarrow}[2][]{%
  \mathrel{%
    \mathpalette{\da@xarrow{#1}{#2}\da@leftarrow{}{}{\,}}{}%
  }%
}
\newcommand*{\da@xarrow}[7]{%
  % #1: below
  % #2: above
  % #3: arrow left
  % #4: arrow right
  % #5: space left 
  % #6: space right
  % #7: math style 
  \sbox0{$\ifx#7\scriptstyle\scriptscriptstyle\else\scriptstyle\fi#5#1#6\m@th$}%
  \sbox2{$\ifx#7\scriptstyle\scriptscriptstyle\else\scriptstyle\fi#5#2#6\m@th$}%
  \sbox4{$#7\dabar@\m@th$}%
  \dimen@=\wd0 %
  \ifdim\wd2 >\dimen@
    \dimen@=\wd2 %   
  \fi
  \count@=2 %
  \def\da@bars{\dabar@\dabar@}%
  \@whiledim\count@\wd4<\dimen@\do{%
    \advance\count@\@ne
    \expandafter\def\expandafter\da@bars\expandafter{%
      \da@bars
      \dabar@ 
    }%
  }%  
  \mathrel{#3}%
  \mathrel{%   
    \mathop{\da@bars}\limits
    \ifx\\#1\\%
    \else
      _{\copy0}%
    \fi
    \ifx\\#2\\%
    \else
      ^{\copy2}%
    \fi
  }%   
  \mathrel{#4}%
}
\makeatother
% \DeclareMathOperator{\rg}{rg}

\usepackage{mathtools}
\DeclarePairedDelimiter{\paren}{(}{)}
\DeclarePairedDelimiter{\abs}{\lvert}{\rvert}
\DeclarePairedDelimiter{\norm}{\lVert}{\rVert}
\DeclarePairedDelimiter{\innerproduct}{\langle}{\rangle}
\newcommand{\Of}[2]{{\operatorname{#1}} {\paren*{#2}}}
\newcommand{\of}[2]{{{{#1}} {\paren*{#2}}}}

\DeclareMathOperator{\Shim}{Shim}
\DeclareMathOperator{\sgn}{sgn}
\DeclareMathOperator{\fdeg}{fdeg}
\DeclareMathOperator{\SL}{SL}
\DeclareMathOperator{\slLie}{\mathfrak{s}\mathfrak{l}}
\DeclareMathOperator{\soLie}{\mathfrak{s}\mathfrak{o}}
\DeclareMathOperator{\spLie}{\mathfrak{s}\mathfrak{p}}
\DeclareMathOperator{\glLie}{\mathfrak{g}\mathfrak{l}}
\newcommand{\pn}[1]{{\color{ForestGreen} \sf PN: [#1]}}
\DeclareMathOperator{\Mp}{Mp}
\DeclareMathOperator{\Mat}{Mat}
\DeclareMathOperator{\GL}{GL}
\DeclareMathOperator{\Gr}{Gr}
\DeclareMathOperator{\GU}{GU}
\def\gl{\mathfrak{g}\mathfrak{l}}
\DeclareMathOperator{\odd}{odd}
\DeclareMathOperator{\even}{even}
\DeclareMathOperator{\GO}{GO}
\DeclareMathOperator{\good}{good}
\DeclareMathOperator{\bad}{bad}
\DeclareMathOperator{\PGO}{PGO}
\DeclareMathOperator{\htt}{ht}
\DeclareMathOperator{\height}{height}
\DeclareMathOperator{\Ass}{Ass}
\DeclareMathOperator{\coheight}{coheight}
\DeclareMathOperator{\GSO}{GSO}
\DeclareMathOperator{\SO}{SO}
\DeclareMathOperator{\so}{\mathfrak{s}\mathfrak{o}}
\DeclareMathOperator{\su}{\mathfrak{s}\mathfrak{u}}
\DeclareMathOperator{\ad}{ad}
% \DeclareMathOperator{\sc}{sc}
\DeclareMathOperator{\Ad}{Ad}
\DeclareMathOperator{\disc}{disc}
\DeclareMathOperator{\inv}{inv}
\DeclareMathOperator{\Pic}{Pic}
\DeclareMathOperator{\uc}{uc}
\DeclareMathOperator{\Cl}{Cl}
\DeclareMathOperator{\Clf}{Clf}
\DeclareMathOperator{\Hom}{Hom}
\DeclareMathOperator{\hol}{hol}
\DeclareMathOperator{\Heis}{Heis}
\DeclareMathOperator{\Haar}{Haar}
\DeclareMathOperator{\h}{h}
\def\sp{\mathfrak{s}\mathfrak{p}}
\DeclareMathOperator{\heis}{\mathfrak{h}\mathfrak{e}\mathfrak{i}\mathfrak{s}}
\DeclareMathOperator{\End}{End}
\DeclareMathOperator{\JL}{JL}
\DeclareMathOperator{\image}{image}
\DeclareMathOperator{\red}{red}
\def\div{\operatorname{div}}
\def\eps{\varepsilon}
\def\cHom{\mathcal{H}\operatorname{om}}
\DeclareMathOperator{\Ops}{Ops}
\DeclareMathOperator{\Symb}{Symb}
\def\boldGL{\mathbf{G}\mathbf{L}}
\def\boldSO{\mathbf{S}\mathbf{O}}
\def\boldU{\mathbf{U}}
\DeclareMathOperator{\hull}{hull}
\DeclareMathOperator{\LL}{LL}
\DeclareMathOperator{\PGL}{PGL}
\DeclareMathOperator{\class}{class}
\DeclareMathOperator{\lcm}{lcm}
\DeclareMathOperator{\spann}{span}
\DeclareMathOperator{\Exp}{Exp}
\DeclareMathOperator{\ext}{ext}
\DeclareMathOperator{\Ext}{Ext}
\DeclareMathOperator{\Tor}{Tor}
\DeclareMathOperator{\et}{et}
\DeclareMathOperator{\tor}{tor}
\DeclareMathOperator{\loc}{loc}
\DeclareMathOperator{\tors}{tors}
\DeclareMathOperator{\pf}{pf}
\DeclareMathOperator{\smooth}{smooth}
\DeclareMathOperator{\prin}{prin}
\DeclareMathOperator{\Kl}{Kl}
\newcommand{\kbar}{\mathchar'26\mkern-9mu k}
\DeclareMathOperator{\der}{der}
% \DeclareMathOperator{\abs}{abs}
\DeclareMathOperator{\Sub}{Sub}
\DeclareMathOperator{\Comp}{Comp}
\DeclareMathOperator{\Err}{Err}
\DeclareMathOperator{\dom}{dom}
\DeclareMathOperator{\radius}{radius}
\DeclareMathOperator{\Fitt}{Fitt}
\DeclareMathOperator{\Sel}{Sel}
\DeclareMathOperator{\rad}{rad}
\DeclareMathOperator{\id}{id}
\DeclareMathOperator{\Center}{Center}
\DeclareMathOperator{\Der}{Der}
\DeclareMathOperator{\U}{U}
% \DeclareMathOperator{\norm}{norm}
\DeclareMathOperator{\trace}{trace}
\DeclareMathOperator{\Equid}{Equid}
\DeclareMathOperator{\Feas}{Feas}
\DeclareMathOperator{\bulk}{bulk}
\DeclareMathOperator{\tail}{tail}
\DeclareMathOperator{\sys}{sys}
\DeclareMathOperator{\atan}{atan}
\DeclareMathOperator{\temp}{temp}
\DeclareMathOperator{\Asai}{Asai}
\DeclareMathOperator{\glob}{glob}
\DeclareMathOperator{\Kuz}{Kuz}
\DeclareMathOperator{\Irr}{Irr}
\newcommand{\rsL}{ \frac{ L^{(R)}(\Pi \times \Sigma, \std, \frac{1}{2})}{L^{(R)}(\Pi \times \Sigma, \Ad, 1)}  }
\DeclareMathOperator{\GSp}{GSp}
\DeclareMathOperator{\PGSp}{PGSp}
\DeclareMathOperator{\BC}{BC}
\DeclareMathOperator{\Ann}{Ann}
\DeclareMathOperator{\Gen}{Gen}
\DeclareMathOperator{\SU}{SU}
\DeclareMathOperator{\PGSU}{PGSU}
% \DeclareMathOperator{\gen}{gen}
\DeclareMathOperator{\PMp}{PMp}
\DeclareMathOperator{\PGMp}{PGMp}
\DeclareMathOperator{\PB}{PB}
\DeclareMathOperator{\ind}{ind}
\DeclareMathOperator{\Jac}{Jac}
\DeclareMathOperator{\jac}{jac}
\DeclareMathOperator{\im}{im}
\DeclareMathOperator{\Aut}{Aut}
\DeclareMathOperator{\Int}{Int}
\DeclareMathOperator{\PSL}{PSL}
\DeclareMathOperator{\co}{co}
\DeclareMathOperator{\irr}{irr}
\DeclareMathOperator{\prim}{prim}
\DeclareMathOperator{\bal}{bal}
\DeclareMathOperator{\baln}{bal}
\DeclareMathOperator{\dist}{dist}
\DeclareMathOperator{\RS}{RS}
\DeclareMathOperator{\Ram}{Ram}
\DeclareMathOperator{\Sob}{Sob}
\DeclareMathOperator{\Sol}{Sol}
\DeclareMathOperator{\soc}{soc}
\DeclareMathOperator{\nt}{nt}
\DeclareMathOperator{\mic}{mic}
\DeclareMathOperator{\Gal}{Gal}
\DeclareMathOperator{\st}{st}
\DeclareMathOperator{\std}{std}
\DeclareMathOperator{\diag}{diag}
\DeclareMathOperator{\Sym}{Sym}
\DeclareMathOperator{\gr}{gr}
\DeclareMathOperator{\aff}{aff}
\DeclareMathOperator{\Dil}{Dil}
\DeclareMathOperator{\Lie}{Lie}
\DeclareMathOperator{\Symp}{Symp}
\DeclareMathOperator{\Stab}{Stab}
\DeclareMathOperator{\St}{St}
\DeclareMathOperator{\stab}{stab}
\DeclareMathOperator{\codim}{codim}
\DeclareMathOperator{\linear}{linear}
\newcommand{\git}{/\!\!/}
\DeclareMathOperator{\geom}{geom}
\DeclareMathOperator{\spec}{spec}
\def\O{\operatorname{O}}
\DeclareMathOperator{\Au}{Aut}
\DeclareMathOperator{\Fix}{Fix}
\DeclareMathOperator{\Opp}{Op}
\DeclareMathOperator{\opp}{op}
\DeclareMathOperator{\Size}{Size}
\DeclareMathOperator{\Save}{Save}
% \DeclareMathOperator{\ker}{ker}
\DeclareMathOperator{\coker}{coker}
\DeclareMathOperator{\sym}{sym}
\DeclareMathOperator{\mean}{mean}
\DeclareMathOperator{\elliptic}{ell}
\DeclareMathOperator{\nilpotent}{nil}
\DeclareMathOperator{\hyperbolic}{hyp}
\DeclareMathOperator{\newvector}{new}
\DeclareMathOperator{\new}{new}
\DeclareMathOperator{\full}{full}
\newcommand{\qr}[2]{\left( \frac{#1}{#2} \right)}
\DeclareMathOperator{\unr}{u}
\DeclareMathOperator{\ram}{ram}
% \DeclareMathOperator{\len}{len}
\DeclareMathOperator{\fin}{fin}
\DeclareMathOperator{\cusp}{cusp}
\DeclareMathOperator{\curv}{curv}
\DeclareMathOperator{\rank}{rank}
\DeclareMathOperator{\rk}{rk}
\DeclareMathOperator{\pr}{pr}
\DeclareMathOperator{\Transform}{Transform}
\DeclareMathOperator{\mult}{mult}
\DeclareMathOperator{\Eis}{Eis}
\DeclareMathOperator{\reg}{reg}
\DeclareMathOperator{\sing}{sing}
\DeclareMathOperator{\alt}{alt}
\DeclareMathOperator{\irreg}{irreg}
\DeclareMathOperator{\sreg}{sreg}
\DeclareMathOperator{\Wd}{Wd}
\DeclareMathOperator{\Weil}{Weil}
\DeclareMathOperator{\Th}{Th}
\DeclareMathOperator{\Sp}{Sp}
\DeclareMathOperator{\Ind}{Ind}
\DeclareMathOperator{\Res}{Res}
\DeclareMathOperator{\ini}{in}
\DeclareMathOperator{\ord}{ord}
\DeclareMathOperator{\osc}{osc}
\DeclareMathOperator{\fluc}{fluc}
\DeclareMathOperator{\size}{size}
\DeclareMathOperator{\ann}{ann}
\DeclareMathOperator{\equ}{eq}
\DeclareMathOperator{\res}{res}
\DeclareMathOperator{\pt}{pt}
\DeclareMathOperator{\src}{source}
\DeclareMathOperator{\Zcl}{Zcl}
\DeclareMathOperator{\Func}{Func}
\DeclareMathOperator{\Map}{Map}
\DeclareMathOperator{\Frac}{Frac}
\DeclareMathOperator{\Frob}{Frob}
\DeclareMathOperator{\ev}{eval}
\DeclareMathOperator{\pv}{pv}
\DeclareMathOperator{\eval}{eval}
\DeclareMathOperator{\Spec}{Spec}
\DeclareMathOperator{\Speh}{Speh}
\DeclareMathOperator{\Spin}{Spin}
\DeclareMathOperator{\GSpin}{GSpin}
\DeclareMathOperator{\Specm}{Specm}
\DeclareMathOperator{\Sphere}{Sphere}
\DeclareMathOperator{\Sqq}{Sq}
\DeclareMathOperator{\Ball}{Ball}
\DeclareMathOperator\Cond{\operatorname{Cond}}
\DeclareMathOperator\proj{\operatorname{proj}}
\DeclareMathOperator\Swan{\operatorname{Swan}}
\DeclareMathOperator{\Proj}{Proj}
\DeclareMathOperator{\bPB}{{\mathbf P}{\mathbf B}}
\DeclareMathOperator{\Projm}{Projm}
\DeclareMathOperator{\Tr}{Tr}
\DeclareMathOperator{\Type}{Type}
\DeclareMathOperator{\Prop}{Prop}
\DeclareMathOperator{\vol}{vol}
\DeclareMathOperator{\covol}{covol}
\DeclareMathOperator{\Rep}{Rep}
\DeclareMathOperator{\Cent}{Cent}
\DeclareMathOperator{\val}{val}
\DeclareMathOperator{\area}{area}
\DeclareMathOperator{\nr}{nr}
\DeclareMathOperator{\CM}{CM}
\DeclareMathOperator{\CH}{CH}
\DeclareMathOperator{\tr}{tr}
\DeclareMathOperator{\characteristic}{char}
\DeclareMathOperator{\supp}{supp}


\theoremstyle{plain} \newtheorem{theorem} {Theorem} \newtheorem{conjecture} [theorem] {Conjecture} \newtheorem{corollary} [theorem] {Corollary} \newtheorem{proposition} [theorem] {Proposition} \newtheorem{fact} [theorem] {Fact}
\theoremstyle{definition} \newtheorem{definition} [theorem] {Definition} \newtheorem{hypothesis} [theorem] {Hypothesis} \newtheorem{assumptions} [theorem] {Assumptions}
\newtheorem{example} [theorem] {Example}
\newtheorem{assertion}[theorem] {Assertion}
\newtheorem{note}[theorem] {Note}
\newtheorem{conclusion}[theorem] {Conclusion}
\newtheorem{claim}            {Claim}
\newtheorem{homework} {Homework}
\newtheorem{exercise} {Exercise}  \newtheorem{question}[theorem] {Question}    \newtheorem{answer} {Answer}  \newtheorem{problem} {Problem}    \newtheorem{remark} [theorem] {Remark}
\newtheorem{notation} [theorem]           {Notation}
\newtheorem{terminology}[theorem]            {Terminology}
\newtheorem{convention}[theorem]            {Convention}
\newtheorem{motivation}[theorem]            {Motivation}


\newtheoremstyle{itplain} % name
{6pt}                    % Space above
{5pt\topsep}                    % Space below
{\itshape}                   % Body font
{}                           % Indent amount
{\itshape}                   % Theorem head font
{.}                          % Punctuation after theorem head
{5pt plus 1pt minus 1pt}                       % Space after theorem head
% {.5em}                       % Space after theorem head
{}  % Theorem head spec (can be left empty, meaning ‘normal’)

% \theoremstyle{mytheoremstyle}


\theoremstyle{itplain} %--default
% \theoremheaderfont{\itshape}
% \newtheorem{lemma}{Lemma}
\newtheorem{lemma}[theorem]{Lemma}
% \newtheorem{lemma}{Lemma}[subsubsection]

\newtheorem*{lemma*}{Lemma}
\newtheorem*{proposition*}{Proposition}
\newtheorem*{definition*}{Definition}
\newtheorem*{example*}{Example}

\newtheorem*{results*}{Results}
\newtheorem{results} [theorem] {Results}


\usepackage[displaymath,textmath,sections,graphics]{preview}
\PreviewEnvironment{align*}
\PreviewEnvironment{multline*}
\PreviewEnvironment{tabular}
\PreviewEnvironment{verbatim}
\PreviewEnvironment{lstlisting}
\PreviewEnvironment*{frame}
\PreviewEnvironment*{alert}
\PreviewEnvironment*{emph}
\PreviewEnvironment*{textbf}



\usepackage{tikz-cd}


\begin{document}


\part{Matthew Emerton, \emph{Aspects of $p$-adic categorical local Langlands for $\GL_2(\mathbb{Q}_p)$}}

\begin{abstract}
  The categorical p-adic local Langlands correspondence has been established for the group $\GL_2(\mathbb{Q}_p)$ in joint work of the speaker with Andrea Dotto and Toby Gee. In this talk I will describe some aspects of the categorical correspondence, including its relationship to Taylor--Wiles--Kisin patching, to the work of Colmez and Paskunas, and to recent work of Johansson--Newton--Wang-Erickson. I also hope to illustrate what aspects of the $\GL_2(\mathbb{Q}_p)$ situation are expected to carry over to the case of $\GL_2(\mathbb{Q}_{p^f})$, and what aspects are not.
\end{abstract}

Fix $p \geq 5$ and $G := \GL_2(\mathbb{Q}_p)$.  Let $E / \mathbb{Q}_p$ be a finite extension, with ring of integers $\mathcal{O}$ and residue field $\mathbb{F}$.  Let $\zeta : G_{\mathbb{Q}_p} \rightarrow \mathcal{O}^\times$ be trivial (or $\eps$, the cyclotomic character, if you want).  It will play the role of a central character.

Let $\mathcal{A}$ denote the category of smooth representations of $G$, with central character $\zeta$, on locally torsion $\mathcal{O}$-modules, i.e., any element of the module is killed by some $p^n$.

We denote by $\mathfrak{X}$ the ``$E G$ stack'' parametrizing families of projective rank $2$ {\'e}tale $(\varphi, \Gamma)$-modules over Fontaine's ring
\begin{equation*}
  \mathbf{A}_{\mathbb{Q}_p} = \widehat{\mathbb{Z}_p((\tau))},
\end{equation*}
where $\varphi(1 + \tau) =(1 + \tau)^p$ and $\gamma_a(1 + \tau) =(1 + \tau)^a$.  Here ``family'' means family over a $p$-adically complete $\mathcal{O}$-algebra.  To match the central character condition, we'll have a determinant condition.  The determinant of our rank two $(\varphi, \Gamma)$-modules will be a rank one $(\varphi, \Gamma)$-module, and we want this determine to be to be $\zeta \eps^{-1}$.  Those are the objects in play.

\begin{theorem}[Dotto--Emerton--Gee]
  There exists a fully faithful functor
  \begin{equation*}
    \mathfrak{a} : D_{f .g.}^{\mathrm{b}}(\mathcal{A}) \hookrightarrow D_{\mathrm{coh}}^{\mathrm{b}}(\mathfrak{X}).
  \end{equation*}
\end{theorem}
\begin{remark}
  It might be that what we've written doesn't characterize the functor.  We don't claim that there is a unique functor.  It is going to satisfy certain properties.  We don't know that they characterize it, but it is in some sense reasonable.  Maybe at first it's a bit hard to see what we mean at all, but a very rough idea is that an object in $\mathcal{A}$, which is a smooth representation, well, one could at least think of a smooth representation that is not irreducible (or at least admissible) as some family, or integral, of representations that are irreducible (or admissible).  There is a classical mod $p$ Langlands that relates irreducible mod $p$ representations (the building blocks in $\mathcal{A}$) to semisimple mod $p$ $2$-dimensional Galois representations of $G_{\mathbb{Q}_p}$, which are somehow closed points in $\mathfrak{X}$.  So the building block of the functor will be that the building blocks in $\mathcal{A}$ go to some sort of skyscrapers in $\mathfrak{X}$, and a typical element of $\mathcal{A}$ will be some direct integral of building blocks, while a typical element of $\mathfrak{X}$ will be a direct integral of its fibers.  That's how you're supposed to imagine this functor going.
\end{remark}

When you have a derived category and put a subscript at the bottom (indicating cohomologies in a subcategory), this condition should be a reasonably behaved one for this notion to give you a reasonable subcategory.  One would want to know that ``finitely generated'' is a reasonable condition to put on $\mathcal{A}$-modules.  There's kind of a fact which is that in $\mathcal{A}$, ``finitely generated'' is equivalent to ``finitely presented''.  So $\mathcal{A}$ has a kind of ``Noetherian'' nature.  We want to say a bit more about this, partly to introduce more notation and concepts, but also, it's clear what it means for a representation to be finitely generated, but less so for finitely presented in this setting, since $\mathcal{A}$ is not a category of modules over a ring.  So let's say a bit more about what ``finitely presented'' means here.  The building blocks of the finitely generated modules are the \emph{free} finite rank modules.  The corresponding building blocks in $\mathcal{A}$

If you have a finitely generated element of $\mathcal{A}$, so it's some $G$-representation with a finite number of generators, then what you can do is you can act using $K = \GL_2(\mathbb{Z}_p)$ and the center $Z \leq G$ (which just acts by the scalar $\zeta$) and act on your finitely many generators with $K Z$.  This will give you a finitely-generated $K Z$ module, but since the representation is smooth, you'll actually get a finite $\mathcal{O}$-module.  A typical finitely generated $G$ module will be generated over $G$ by some finite $\mathcal{O}$-module $V$, which is a representation of $K Z$.  So what you can do is, your representation that you're thinking about will be a quotient of a compact induction
\begin{equation}\label{eq:cnpp1dqjrk}
  \ind_{K Z}^{G} V \twoheadrightarrow \pi.
\end{equation}
Thus such compact inductions are the building blocks of the finitely generated representations.  Now $\pi$ being ``finitely presented'' is equivalent, by definition if you'd like, to the kernel of the map \eqref{eq:cnpp1dqjrk} being finitely generated.  (We note that if it is true for one such $V$, it will be true for all $V$.)  It's true for $\GL_2(\mathbb{Q}_p)$ that finitely generated implies finitely presented (by Berthel--Livre (?) and Breuil), but that's not an obvious fact, nor a general one (e.g., it fails for $\GL_2$ over a quadratic extension of $\mathbb{Q}_p$).

We want to come back to these representations in a little bit, but first, let's say what we mean by coherent. $\mathfrak{X}$ is a formal algebraic stack, and is in fact Noetherian, so you can write it as an increasing union, or colimit, of algebraic stacks $\mathfrak{X}_n$:
\begin{equation*}
  \mathfrak{X} = \operatorname{colim}_n \mathfrak{X}_n.
\end{equation*}
Here $\mathfrak{X}_n \hookrightarrow \mathfrak{X}_{n + 1}$ are closed immersions, which we can think of as a ``thickening''.  You can push forward a coherent sheaf on $\mathfrak{X}_n$ to think of it as a coherent sheaf on $\mathfrak{X}_{n + 1}$, so you can take the colimit of those maps at the level of derived categories:
\begin{equation*}
  D_{\mathrm{coh}}^{\mathrm{b}}(\mathfrak{X}) :=
  \operatorname{colim}_n D_{\mathrm{coh}}^{\mathrm{b}}(\mathfrak{X}_n).
\end{equation*}
You have to be a bit careful thinking about such things, however.  For instance, the structure sheaf $\mathcal{O}_{\mathfrak{X}}$ is \emph{not} coherent (by this definition).  The structure sheaf $\mathcal{O}_{\mathfrak{X}_n}$ \emph{is} a coherent sheaf on $\mathfrak{X}$, and so $\mathcal{O}_{\mathfrak{X}}$ is kind of an inverse limit $\varinjlim_n \mathcal{O}_{\mathfrak{X}_n}$, so in our framework, it will be a \emph{pro-}coherent sheaf, but not itself a coherent sheaf:
\begin{equation*}
  \mathcal{O}_{\mathfrak{X}} =  \varinjlim_n \mathcal{O}_{\mathfrak{X}_n} \in \operatorname{Pro}
  D_{\mathrm{coh}}^{\mathrm{b}}(\mathfrak{X}).
\end{equation*}

So that's at least some slight orientation about what the categories are.  Now what we'd like to say is, how would you produce such a functor?  

We should have said earlier that one key property of this functor $\mathfrak{a}$ is that it is \emph{exact}, i.e., preserves distinguished triangles.  We actually use the language of stable $\infty$ categories rather than derived categories, which offers some technical advantages, but in the language of derived categories, we certainly want the property of exactness.  This functor will not be $t$ exact

There are abelian categories $A$ and $X$ underlying $\mathcal{A}$ and $\mathfrak{X}$, but these will not be preserved by our functor, so working with derived categories is out of necessity.  But if we were working with such abelian categories, say thinking of them as categories of modules over a ring, we could try to understand functors between them by understanding where the ring goes.  What you discover is that any such functor is given by some sort of tensor, which you might call the \emph{kernel} of such a functor (in the sense of an integral kernel).  So if you're given a functor and want to find the kernel, you just apply the functor to the ring, and this gives you the kernel to describe the functor in general.  This gives a clue as to how you should think  of this functor, but like we said, it's not the category of modules over a ring.  Why not?  Well, roughly for two reasons.  What would the ring be?  Well, you're looking at the category of representations of a group $G$, so it'd be the category of modules over the group ring of $G$, but why isn't it that?  Well, roughly, we're looking at $p$-power torsion modules, but not with any bounded amount of torsion necessarily, so the only coefficient ring that would make sense is $\mathcal{O}$, but we're not allowed to take $\mathcal{O}[G]$ itself or even the trivial representation of $G$ on $\mathcal{O}$ as a module of $A$, because $\mathcal{O} [G]$ acts on modules that are too $p$-adically big to be in our category, but also there's no continuity condition built into that group ring, so that group ring acts on representations that are not continuous or smooth, so it doesn't take into account the topology.

So again, any finitely generated module is killed by some $\mathcal{O} / p^n$, and then you might have your group $K$ and in fact $K Z$ acting (with $Z$ acting via $\zeta$), and then some $H$ that is open and normal in $K$ might be acting trivially because the action is smooth, so the group algebra
\begin{equation*}
  \mathcal{O} / p^n[K Z / H]_\zeta 
\end{equation*}
would be acting, but then the group ring of $G$ would act compatibly with the group ring of $K Z$, leading to
\begin{equation*}
  \mathcal{O}[G]_\zeta  \otimes_{  \mathcal{O}[K Z]_\zeta }  \mathcal{O} / p^n[K Z / H]_\zeta
  = \ind_{K Z}^G \mathcal{O} / p^n[K Z / H]_\zeta,
\end{equation*}
and then we might take an inverse limit, giving
\begin{equation}\label{eq:cnpp1gy22u}
  \mathcal{O}[[G]]_\zeta
  =
  \varprojlim 
  \widehat{\mathcal{O}[G]_\zeta  \otimes_{  \mathcal{O}[K Z]_\zeta }  \mathcal{O} / p^n[K Z / H]_\zeta }
  = \widehat{\ind_{K Z}^G \mathcal{O}[[K Z]]_\zeta}.
\end{equation}
So it's this inverse limit of the gadgets that were used for checking ``finitely presented''.  You can either think of this as literally the inverse limit of a topological ring, or as a formal inverse limit -- not an object of $\mathcal{A}$, but an object of $\operatorname{Pro} \mathcal{A}$.

So again, if we actually were looking at the category of modules over a ring, this would be projective over that, but instead it's a projective pro-generator.  In fact, it is the case that this category fully faithfully embeds in the category of finitely presented modules over this last ring \eqref{eq:cnpp1gy22u}.  The key thing is that, to the extent that our category $\mathcal{A}$ is a category of modules over anything, it is a category of modules over this ring \eqref{eq:cnpp1gy22u}, and so to find a kernel, we should apply our functor to this ring, leading us to consider
\begin{equation*}
  L_\infty := \mathfrak{a} \left( \mathcal{O}[[G]]_\zeta \right).
\end{equation*}
Applying our functor to this pro-object should lead to another pro-object.  One of the first properties (not obvious) is that what this will give will be a pro-coherent sheaf (rather than a pro-complex).  The name $L_\infty$ goes back to a paper (with Caraiani, Gee, Shin, others) studying patching, where we posited a hypothetical true $p$-adic local Langlands object; ``$L$'' was chosen because it came next to ``$M$'' in the alphabet.

The Morita-theoretic yoga we were just recalling is that $\mathfrak{a}$ is given by
\begin{equation*}
  \pi \mapsto L_\infty \otimes_{\mathcal{O}[[G]]_\zeta} \pi.
\end{equation*}
Here $\pi$ is an object of $\mathcal{A}$, or more generally $D_{\mathrm{fg}}^{\mathrm{b}}(\mathcal{A})$.  You can resolve $\pi$ by considering free modules over this ring or via compact inductions at finite levels.  If you just resolve $\pi$ by finitely presented modules over this ring (which you can do), then you just get copies of $L_\infty$, which is a pro-object.  So something you have to prove is that you actually land in coherent sheaves when you put in a $\pi$, rather than just pro-coherent sheaves.  Note that $L_\infty$ is pro-coherent (rather than coherent), so this is something special about $\pi$.

Okay, but first, we have to know what $L_\infty$ actually is.  For this, you have to guess.  It's given by the ideas of Colmez, who explained how to convert $2$-dimensional Galois representations into representations of $G_{\mathbb{Q}_p}$ (a rather small object of $\mathcal{A}$).  He did it by a formula: take your Galois rep, turn it into a $(\varphi, \Gamma)$-module $V$, then take
\begin{equation*}
  D^{\natural} \otimes _{\zeta \eps^2} \mathbb{P}^1.
\end{equation*}
(This is notation Colmez introduced.)  Here $\mathbb{P} ^1$ has an action of $\GL_2(\mathbb{Q}_p)$.  We do this for the universal $(\varphi, \Gamma)$-module over $\mathfrak{X}$.  That is to say, our stack has a universal $(\varphi, \Gamma)$-module, and we check that we can implement Colmez's construction naturally, producing this object.

What this object will look like is sort of, if this formal stack $\mathfrak{X}$ were some formal spectrum $\operatorname{Spf} A$ of some ring $A$ (say locally in the fppf topology), then the $(\varphi, \Gamma)$-module is some finite rank module over something like $\mathbb{A}_{\mathbb{Q}_p}$, but with $\mathbb{Z}_p$ replaced by this other algebra $A$.  Then, roughly, $\mathbb{P}^1$ looks like a power series ring.  So if you're on a chart on $\mathfrak{X}$ of the form $\operatorname{Spf} A$, then $L_\infty$ is roughly $A[[T]]$.  So we're already seeing the coefficients $A$ (which is a bit like the structure sheaf of $\mathfrak{X}$, pro just along the formal directions in $\mathfrak{X}$) but also this $[[T]]$, which is another pro direction.  That's the pro nature of this $L_\infty$.

Having produced this $L_\infty$, it's not clear that the resulting functor does anything that you want, e.g., that it lands in $D_{\mathrm{coh}}^{\mathrm{b}}(\mathfrak{X})$ rather than some more pro version, and it's not clear that it's fully faithful.  So to study this functor, we need some tools.  We want to conclude the talking by saying what those tools are.  For that, we need to come back to $\mathfrak{X}$, which we recall is this stack of $(\varphi, \Gamma)$-module.  Fontaine tells us that if we take a point of $\mathfrak{X}$ over some (literally) finite $\mathcal{O}$-algebra, that's a rank two $(\varphi, \Gamma)$-module, then this gives a Galois representation.  So if you probe $\mathfrak{X}$ with small rings, you get Galois representations.  But if you probe $\mathfrak{X}$ with \emph{big} rings, you get stuff that's too big.  But somehow
\begin{equation*}
  \mathfrak{X}[\overline{\mathbb{F}_p}]
  \leftrightarrow
  \bar{\rho} : G_{\mathbb{Q}_p} \rightarrow \GL_2(\overline{\mathbb{F}_p}).
\end{equation*}
Now $\mathfrak{X}$ is an algebraic stack, so it has these $\overline{\mathbb{F}_p}$-points (what you might call ``finite type'' points), which, if it were a variety, would be the closed points, but here we have an Artin rather than Deligne--Mumford stack, so such points are not in general closed, but instead specialize to semisimplifications.  Thus the semisimple $\bar{\rho}$'s correspond to the closed $\overline{\mathbb{F}_p}$-points:
\begin{equation*}
  \mathfrak{X}(\overline{\mathbb{F}_p})_{\mathrm{closed}}
  \leftrightarrow \bar{\rho}^{\mathrm{ss}}.
\end{equation*}

There's another thing, a scheme $X$ (that will in the minute be a formal scheme over $\operatorname{Spf} \mathcal{O}$), such that $X(\overline{\mathbb{F}_p})$ is in bijection with $\mathfrak{X}(\overline{\mathbb{F}_p})_{\mathrm{closed}}$.  The underlying reduce scheme $X_{\red}$ will be a chain of copies of the projective line, where the number of them is $\tfrac{p \mp 1}{2}$, where the sign is given by $\zeta(- 1) = \pm 1$.  What we claim is that semisimple Galois representations naturally arrange themselves in this way.  It's an observation that in some sense goes back to computations in the beginning of this century that were later brought out in conversations of Berget and Breuil, and partly pointed out by Kisin, but kind of carefully noted and promoted to thinking you could really think about by Emerton and Gee, as part of this $E G$ stack, and so we'll maybe tell you how this goes, at least briefly.

We'll be looking at the reduction $\overline{\eps}$ of our cyclotomic character mod $p$.  Maybe we're looking at a direct sum of two characters, say
\begin{equation}\label{eq:cnpp1hxgb1}
  \begin{pmatrix}
    w_\alpha \cdot \bar{\eps}^i    &  \\
                                   & w_{\alpha^{-1}} \cdot \bar{\eps}^{-(i + 1)} \\
  \end{pmatrix}.
\end{equation}
Here $i$ is a discrete variable, while $\alpha$ lives in some $\mathbb{G}_m$.  This kind of a thing is a common thing you'll see in number theory talks on Galois representations.  A general principle in mathematics, going back at least to Arabic algebraists (pre-Renaissance), is that a piece of notation can refer either to a number or a variable, and this leads to the idea of the $E G$ stack, which is that \eqref{eq:cnpp1hxgb1} should really be a family of things.  Now, what happens if you let $\alpha$ go to $\infty$?  There's a natural way to think about it coming from the theory of reduction of crystalline theorems (or Serre's conjectures, etc).  If you send $\alpha$ to $0$, then the above will go to an irreducible induction, then take the inverse to get what happens near $\infty$.

So what we said is tha tthe closed points of $\mathfrak{X}$ are in bijection with these semisimple guys, and once you know that, there's a unique way to produce a morphism of underlying topological spaces,
\begin{equation*}
  \lvert \mathfrak{X} \rvert \rightarrow \lvert X \rvert,
\end{equation*}
once you've matched the closed points.  Then you can essentially push forward the structure sheaf on $\mathfrak{X}$ to get a structure sheaf $\mathcal{O}_X = f_\ast \mathcal{O}_{\mathfrak{X}}$ under $f : \mathfrak{X} \rightarrow X$.  By Paskunas, semisimple $\bar{\rho}$ correspond to \emph{blocks} of $A$ that are locally admissible (colimits of admissible).  One can make the following:
\begin{definition}
  Given a closed subset $Y \hookrightarrow \lvert X \rvert$, we define
  \begin{equation*}
    \mathcal{A}_Y := \left\{
      \begin{array}{@{}l@{}}
        \pi \in \mathcal{A} : \text{all nonzero subquotients} \\
        \text{of $\pi$ lie in blocks} \\
        \text{induced by $Y \in \mathcal{Y}$}
      \end{array}
    \right\} \subseteq \mathcal{A}.
  \end{equation*}
  We then define, for $U := \lvert X \rvert - Y$, the quotient
  \begin{equation*}
    \mathcal{A}_U := \mathcal{A} / \mathcal{A}_Y.
  \end{equation*}
\end{definition}
We then have the following result \cite[Theorem 1.1.1]{2022arXiv2207.04671}:
\begin{theorem}[DEG]
  $\tilde{\mathcal{A}} : U \mapsto A_U$ is a stack of abelian categories over $\lvert X \rvert$, with $\Gamma(X, \tilde{\mathcal{A}}) = \mathcal{A}$.
\end{theorem}

Now for $x \in X(\overline{\mathbb{F}_p})$, we have a formal substack $D_{\mathrm{fg}}^{\mathrm{b}}(\mathcal{A}_x) \xrightarrow{\mathfrak{a}} D_{\mathrm{fg}}^{\mathrm{b}}(\mathfrak{X}_x )$.  In characteristic $p$, you can thicken up pseudorepresentations, but not deform them nontrivially.  These sort of stacks were first studied by Carl(?)--Wang--Erickson.  Using Colmez--Paskunas's work, and then the work of Johansson--Newton--Wang--Erickson, we can rewrite
\begin{equation*}
  \mathfrak{a} |_{D_{\mathrm{fg}}^{\mathrm{b}}(\mathcal{A}_x)}
\end{equation*}
in terms of Colmez's functors, and then using the mentioned work in this rewritten form, we deduce that this functor is fully faithful.  This version of the functor was thus already studied in somewhat disguised form by those authors.  We thus have some ``fully faithfulness'' available.  With that in hand, our arguments need to combine with this localization theory and diagrams like
\begin{equation*}
  \begin{CD}         
    D^{\mathrm{b}}_{\mathrm{fg}}(\mathcal{A})    @> \mathfrak{a}>> D_{\mathrm{coh}}^{\mathrm{b}}(\mathfrak{X})\\
    @VVV  @VVV \\
    D_{\mathrm{fg}}^{\mathrm{b}} @>> \mathfrak{a}_U> D_{\mathrm{coh}}^{\mathrm{b}}(f^{-1}(U)).\\
  \end{CD}
\end{equation*}
to get to the whole setting.  That's roughly how we prove the theorem.

We note that nearly everything in the proof is specific to $\GL_2(\mathbb{Q}_p)$, in a very fundamental way.

\begin{remark}
  $\mathfrak{X}$ makes sense in great generality, but $X$ does not.
\end{remark}

[next talk at 11:00!!!!]

\part{Andrew Wiles (University of Oxford), \emph{Non-abelian descent and modularity}}

\begin{abstract}
  I will present a new approach to modularity based on the trace formula and using some ergodic
  and analytic arguments.
\end{abstract}

So I've been thinking for a long time now about how to prove modularity more generally.  Modularity lifting is in very good shape, whereas proving residual modularity is much more hard, and besides the theorem of Khare and Wintenberger, there isn't something very general.  Seems extremely difficult to generalize Khare--Wintenberger directly, so I've been trying to take the approach using potential modularity, which is known quite generally.  Suppose you have a representation ($\ell$-adic, or a compatible system) $\rho$, over some totally real field $F_0$.  You know that if you take some not-necessarily-solvable totally real field $F$ of $F_0$, passing from $\rho$ to $\tilde{\rho}$, then you can find an automorphic form $f$ that corresponds to $\tilde{\rho}$.  This $f$ will be invariant (since $\tilde{\rho}$ comes from $F_0$), and the question then arises: does $f$ equal to the lift $\tilde{f}_0$ of some $f_0$?  Namely, can we find the $f_0$ over $F_0$ such that the natural diagram commutes?  That will be a way of finding modularity.  Thus, potential modularity reduces the problem to descent, but unfortunately, it's nonsolvable descent.

One example where we can do this is cyclic base change, where we do get descent, but cyclic base changes depends upon the trace formula, and this works in contexts where you have (inner) forms or a group, or when you have a subgroup given by invariance under one element, as is the case for cyclic base change.  For non-solvable base change, invariance is much more subtle, so we trade one problem for a much harder one.

What we're going to do is cyclic base change in a slightly different context from normal and see how to replace the arguments in that.

Let's take $F$ and $F_0$ totally real number fields.  We're going  to assume that $F / F_0$ is a cyclic extension, with Galois group generated by some element $\sigma$.  (This generality is sufficient for what we want to use it for.)  We assume moreover that $F / F_0$ is unramified.  Let $D_0$ be a quaternion algebra over $F_0$ that is ramified at all the infinite places and unramified at the finite ones.  To do this, we need to assume that $[F_0 : \mathbb{Q}] \equiv 0 \pmod{2}$.  This restriction on the finite places is not really essential, but as we'll explain later, there's a reason why we wanted to include it.

We then form a Shimura variety (really a set)
\begin{equation*}
  X_0 := X_0(U_0) =
  D_0^\times \backslash D_{0, f}^\times  / U_0,
\end{equation*}
where
\begin{itemize}
\item $D_0^\times$ is the multiplicative group of our quaternion algebra,
\item $D_{0, f}^\ast$ is its group of finite adelic points, and
\item $U_0$ is some level structure, say $U_0 = U_0(\mathfrak{n}_0)$ (``$\Gamma_0(\bullet)$'' structure) with $\mathfrak{n}_0$ square-free.
\end{itemize}
We then have a space
\begin{equation*}
  S(U_0) = \left\{ f : X_0(U_0) \rightarrow \mathbb{C}  \right\}. 
\end{equation*}
It might seem like this is a very simple structure, but it carries the action of the Hecke operators in the usual way, and moreover, by the theorem of Jacquet--Langlands and Shimizu, this $S(U_0)$ is isomorphic as a Hecke module to the space of weight two Hilbert modular forms of level $U_0$.  I'm going to work with these spaces instead of with $\GL_2$ and Hilbert modular forms.  From the point of view of the trace formula, basically we're focusing on the elliptic elements in the trace formula.  I think one could do this for $\GL_2$ instead, but it's simpler and easier to handle if we do it in this context.

Let's turn to the trace formula.  Trace of what?  Let $(\pi_0) \subseteq \mathcal{O}_{F_0}$ be a prime that splits completely in $F$, thus $\pi_0 = \prod_i \pi^{\sigma^i}$, with $(\pi) \subseteq \mathcal{O}_F$.  What we want to compute is the trace on the space of forms here.  We're going to consider the vector space $\mathbb{C}[X_0]$, which has two important bases:
\begin{enumerate}
\item\label{enumerate:cnpp1no6c8} $X_0$ itself, given characteristic functions of points $t_{0, i}$, and
\item\label{enumerate:cnpp1npes7} a basis of eigenforms $\mathcal{F}_0$ for the Hecke operators.
\end{enumerate}
What we want to compute is the trace of $T_{\pi_0}$.  We can express this analytically in terms of $\mathcal{F}_0$, and geometrically in terms of the points.  We write this as
\begin{equation*}
  \trace T_{\pi_0} |_{\mathcal{F}_0}
  = \lvert T_{\pi_0} \cdot \Delta_{X_0} \rvert,
  \qquad
  \Delta_{X_0} = \Delta \subset X_0 \times X_0.
\end{equation*}

What we need to do is to relate the trace of $T_{\pi_0}$ on $\mathcal{F}_0$ to $T_{\pi}$ acting on the space of invariant forms over $F$.  But the way of getting from $X_0$ to $X$ is going to be via the geometric side.  So we want to understand what this set $T_{\pi_0} \cdot \Delta_{X_0}$ is.

So, what is an element $Q \in T_{\pi}$?  We pick some $t_{0, i}$, some coset representative $\varpi_{\pi_0}$, and we write 
\begin{equation*}
  t_{0, i} \varpi_{\pi_0 } = \gamma_0 t_{0, i} u_0,
\end{equation*}
where $\gamma_0 \in D_0^\times$ and $u_0 \in U_0$.  So we get some expression like that, and reading off from this, we can see for instance that $\gamma_0$ has determinant (or norm) of the same ideal as $\pi_0$.

So associated to this $Q$, we have some $t_{0, i}$ (some representative of $X_0$) and an element $\gamma_0$ of $D_0^\times$.  It moreover satisfies a quadratic equation over $F_0$, so it actually lies in some quadratic extension
\begin{equation*}
  L_0 = F_0(\gamma_0).
\end{equation*}
That quadratic extension is necessarily CM (coming from the fact that $D_0$ is ramified at all infinite places).

Now, what we want to do is associate a similar kind of thing over $F$.  So, to get over $F$ from the element $\gamma_0$, we choose
\begin{equation*}
  \gamma \in L = F(\gamma_0)
\end{equation*}
so that
\begin{equation}\label{eq:cnpp1n6c66}
  \mathcal{N}_{L / L_0}(\gamma) = \gamma_0.
\end{equation}
(Because it's a cyclic extension, class field theory tells you that you only need to check this locally.  It's easy to see from our construction and Hasse's theorem that such a $\gamma$ exists.)  In view of \eqref{eq:cnpp1n6c66}, it's easy to see that
\begin{equation*}
  (\gamma) = \mathfrak{b}^{\sigma - 1} \mathfrak{p},
\end{equation*}
where $n(\mathfrak{p}) =(\pi)$ is a prime divisor of $\pi$, lying above the corresponding factorization of $\gamma_0$.  So we have a prime factorization like this and can turn it back into adelic language.  If we pick adelic representatives for the ideals, then we can write this as
\begin{equation*}
  \beta^\sigma \omega_{\pi} = \gamma \beta u.
\end{equation*}
Now, what does this say?  Thinking about it, it gives us an element of $T_\pi \sigma \cdot \Delta_X$, where $X = X(U_0)$ is defined like $X_0$, but now over $F$, in terms of $D = D_0 \otimes_{F_0} F$.

So what we've ended up with is a map
\begin{equation}\label{eq:cnpp1oiwza}
  T_{\pi_0} \cdot \Delta_{X_0} \hookrightarrow T_{\pi^\sigma} \cdot \Delta_X.
\end{equation}
Now we want to analyze surjectivity.  In the trace formula (for cyclic base change, anyway), enormous use is made of the norm operator, which is defined like the usual norm operator, but defined on the non-abelian group $D^\times$ rather than just on the multiplicative group of some field such as $F^\times$.  This requires making a choice, such as
\begin{equation*}
  \mathcal{N}_\sigma : D \ni y \mapsto y y^{\sigma} y^{\sigma^2} \dotsb.
\end{equation*}
Obviously the norm of $\gamma$ is $\gamma_0$.  What you find is that
\begin{equation*}
  \mathcal{N}_\sigma(\gamma_1) \sim \mathcal{N}_\sigma(\gamma_2) \implies \gamma_1 \sim^\sigma \gamma_2.
\end{equation*}
Here
\begin{itemize}
\item $\sim$ denotes conjugacy,
\item $\sim^\sigma$ denotes $\sigma$-conjugacy.
\end{itemize}
You can show in this way that there's actually a converse map to \eqref{eq:cnpp1oiwza}.  That's the mechanism that the proof of cyclic base change will use.  Actually, this refines \eqref{eq:cnpp1oiwza} to an isomorphism.

Now in the usual development of cyclic base change, it's important to note that the norm map only gives you a norm map on conjugacy classes, i.e. it's not well-defined on elements.  You can thus compare orbits and, more generally, orbital integrals, using this.  But we're not going to get into it, because this method doesn't work in the non-solvable case at all.

We get that far, and with what we have, we can now complete the proof of base change in this case.  We have
\begin{equation*}
  \trace T_{\pi_0} |_{\mathcal{F}_0} =
  \lvert T_{\pi_0} \cdot \Delta_{X_0} \rvert
  =
  \lvert T_{\pi^\sigma} \cdot \Delta_X \rvert
  =
  \trace T_{\pi^\sigma} |_{\mathcal{F}}
  = \trace T_\pi |_{\mathcal{F}^{\mathrm{inv}}}.
\end{equation*}
Here
\begin{itemize}
\item the second equality comes from the geometric argument that we just gave,
\item the third equality comes from the trace formula for $X$, 
\item the fourth equality uses that $\sigma$ permutes the non-invariant forms, but is trivial on the invariant ones.
\end{itemize}
Okay, we've only done this for certain $\pi$ (splitting completely into principal ideals), but now Cebotarev, using the associated Galois representations, tells us that we get this for all places.

So that's what we want to model this argument about generalizing the trace formula on.

The first key problem in doing so is that, as I said, we don't have one $\sigma$, so we can't apply this argument.  But also, the cohomological method simply doesn't work more generally.

We replace this now, as in class field theory, we find an analytic argument to replace this cohomological argument.  Let me explain that first, since this is much simpler.  Let's focus on the case of a non-solvable extension, with no abelian subquotient.  We'll continue to use $F$ and $F_0$ for the two fields, and we write $d =[F : F_0]$ for their degree.  We again choose primes $\pi_0$ in $F_0$ that split completely into principal primes in $F$, and consider the following sum
\begin{equation}\label{eq:cnpp1o4jp4}
  \sum_{\pi_0} \frac{1}{d}
  \sum_{\pi \mid \pi_0}
  \sum_{f \in \mathcal{F}^{\mathrm{inv}}}
  a_\pi(f) \cdot \operatorname{Norm}(\pi)^{- s}.
\end{equation}
Here, again, $\mathcal{F} ^{\mathrm{inv}}$ is (a basis for) the space of forms invariant under the action of the Galois group.

\begin{remark}
  ``Invariant'' is in a suitably defined sense -- there is a choice here regarding whether to take things up to twist.  The thing that's confusing is that invariant cyclic Galois representations do not descend, but that's the only obstruction.  For instance, projective representations always descend.
\end{remark}

We're going to compare that, on the one hand, to
\begin{equation}\label{eq:cnpp1pcyts}
  \sum_{\pi_0} \sum_{f \in \mathcal{F}_0}
  a_{\pi_0}(f_0)
  \operatorname{Norm}(\pi_0)^{- s}.
\end{equation}
Here, as before, $\mathcal{F}_0$ is a basis for eigenforms on $F_0$.

In the above, we further restrict $\pi_0$ to primes of degree $1$ that split completely in $M$, a totally real field that contains the Hilbert class field $H_F$ of $F$.  The first sum \eqref{eq:cnpp1o4jp4} would then be the main part contributing to these $L$-functions for the supposed base changes to $M$ of these forms.  That is to say, if we actually had base changes to $M$, then this would be the main part, at least as far as convergence goes; the rest over $M$ (if they existed) would have Dirichlet series that converge at $\Re(s) = 3/2$.

By potential automorphy, we can choose $M$ such that $\mathcal{F}_0$ and $\mathcal{F}$ are simultaneously automorphic over $M$.

If we make a similar construction as in the discussion of cyclic base change, then the coefficients here in the non-solvable case would be
\begin{equation*}
  \frac{1}{d} \sum_{\pi_0} \sum_{\pi \mid \pi_0} \sum_{f \in \mathcal{F}^{\mathrm{inv}}}
  a_{\pi}(f)  
\end{equation*}
will be greater than or equal to
\begin{equation*}
  \sum_{\pi_0}  \sum_{f_0 \in \mathcal{F}_0 }
  a_{\pi_0}(f),
\end{equation*}
so the difference is nonnegative.  Now they're not cuspidal these two, but we can match up the Eisenstein parts of \eqref{eq:cnpp1pcyts} and \eqref{eq:cnpp1pcyts} easily.  We thus obtain an $L$-function that is cuspidal (i.e., no pole) and has positive coefficients.  It's not difficult to deduce that for almost all $\pi_0$, we get equality.

That side of the argument, also as in class field theory, can be done relatively simply.  The problem is the construction.  How do we get this map between the two?  If we go back to the same kind of argument here (try and construct a $\gamma$ in the same way), we can find that the norm of $\gamma$ is $\gamma_0$, but what happens is that you get the identity, of ideals in $L = F(\gamma_0) = F(\gamma)$,
\begin{equation*}
  (\gamma) = \prod_t \mathfrak{b}_t^{\sigma_t - 1} \cdot \mathfrak{p}.
\end{equation*}
(This uses some class field theory.  You have to set it up right.  It's not trivial, but you can put yourself in this situation.  It's not just Hasse's theorem, as in the cyclic case.)  We don't have a $\sigma$ that is picking out the invariant forms.  What I want to do now is, instead of what I did before, let's view this as one entry of 
\begin{equation}\label{eq:cnpp1ptgfj}
  (\prod T_{b_t} \sigma_t) T_\pi t_{0, i}
  = \prod T_{b_t} \cdot t_{0, i},
\end{equation}
where $b_t = n_{L / F}(\mathfrak{b}_t)$.

\textbf{Step 1}. Replace $T_\pi$ by $T_\pi T_\alpha$, where $T_\alpha$ projects to the invariant part.  Once you've done that, because $T_\alpha$ projects, the identity \eqref{eq:cnpp1ptgfj} becomes
\begin{equation}\label{eq:cnpp1pvo6p}
  T_\pi T_\alpha 
  \prod_t T_{b_t}   t_{0, i}
  = \prod_t T_{b_t}  t_{0, i}.
\end{equation}
Now what we do is choose our $T_{b_t}$ so that they're invertible.  We have to make sure we can do that in the original thing.  Then we view the products over $t$ in \eqref{eq:cnpp1pvo6p} as new basis elements.  Now we're getting something that looks more like the trace of of $T_\pi T_\alpha$.  For this, we need that $\prod T_{b_t}$ is invertible in $\mathbb{C}[X]$.

Okay, so that looks nice, but the issue is that in doing \eqref{eq:cnpp1ptgfj}, I only gave you one coset representative for the Hecke operator, but we need the whole Hecke operator, not just one entry.  This is where it becomes more complicated.  I'm not going to go into it in detail, but the idea is that you make the $b_t$'s complicated ideals (products of many ideals) and you consider different factorizations of $b_t$ and you get different representatives, and you have to do it in such a way that you can make sure you get all the representatives on both sides of \eqref{eq:cnpp1pvo6p}.  So you choose your $b_t$'s so that there are many prime factors, then you can always change each prime factor to its complex conjugate, which gives you a huge number of choices.  Making sure this works is a little like the ergodic method that people have used.  I don't want to go into the argument because it is rather involved, but for this argument to work, we need a version of Cebotarev in this context.  What it says is that
\begin{enumerate}
\item\label{enumerate:cnpp1p8snz} If you pick one of these CM fields $L_f^\times \rightarrow D_f^\times \rightarrow X(U_0)$, then this factors through some class group
  \begin{equation*}
    C_0 = L^\times \backslash L_f^\times / U_{\mathcal{O}}^\times \rightarrow X(U_0).
  \end{equation*}
  (Here $U_{\mathcal{O}}^\times$ is some level structure in $L_f^\times$, corresponding to the class group of an order.)  We need this to be surjective and equidistributed.  This can be proven using Waldspurger formula (Zhang, Tian, ...).  This kind of argument has been used in this context by Zhang in simpler settings, with some of the new refinements of Waldspurger.
\item\label{enumerate:cnpp1p8trm} If you have a map from $L_f^\times$ to \emph{two } copies of $X(U_0)$ that is twisted, say
  \begin{equation*}
    L_f^\times \rightarrow
    D_f^\times \times D_f^\times
    \rightarrow 
    X(U_0) \times X(U_0)
  \end{equation*}
  \begin{equation*}
    \ell_f \mapsto(\ell_f, \ell_f a_f)
  \end{equation*}
  where $a_f$ is some element of $L_f^\times$ corresponding to some ideal $\mathfrak{a}_f$ of sufficiently large norm, then you want this map to be surjective and equidistributed as the discriminant $D_{\mathcal{O}}$ tends to $\infty$.  This is the mixing conjecture (in this context) of Michel--Venkatesh \cite{MichelVenkateshICM}.
\end{enumerate}

Assertion \eqref{enumerate:cnpp1p8trm} is not yet proved.  Over $\mathbb{Q}$, Khayutin  has proved this.  Robinson, in his thesis, and generalized some of this to totally real fields.  Unfortunately, we actually need something a bit stronger: for the subgroup of squares $L_f^{\times 2} $, and even the subgroup of fourth powers $L_f^{\times 4}$.

How does Khayutin do this?  This proof depends initially upon a classification of measures of Einsiedler--Lindenstrauss.  Basically, the idea is quite $p$-adic.  Rather than doing it with level $U_0$, they allow the level at two primes to go to $\infty$ (as the discriminant goes to $\infty$) and get some invariance.  If you assume that the projections are surjective and equidistributed (which we know from the first method \eqref{enumerate:cnpp1p8snz}), then the joinings theorem of Einsiedler--Lindenstrauss says that the support must be on some joining.  Khayutin finds some nice system of neighborhoods to calculate whether there could be any.  Want to know that the measure comes from the full space rather than some Hecke operator.  Want to know that no Hecke operator captures too much of the image.  In other words, you're trying to bound the intersection of the Hecke operator with the image of this CM orbit.  Khayutin does this by reducing it to a counting argument involving counting numbers of points in a certain Diophantine geometric setup.  He manages to bound it using sieve theory, a relative trace formula, and arguments from Diophantine geometry.  It's got a strong $p$-adic sense to it, by taking level structures of this kind.  It does seem a bit reminiscent of the arguments of Cornut--Vatsal proving Mazur's conjecture.  But the part that needs proving is not in measure theory, but instead in number theory and Diophantine geometry.

Assuming this, you get modularity of all compatible systems as in Serre's conjecture.

\begin{remark}
  There is another approach to the mixing conjecture by Blomer--Brumley assuming GRH.
\end{remark}

\begin{remark}
  The reason we used ``split at all finite primes'' is that we think this should work for the Artin conjecture in simpler cases, because Artin representations are potentially automorphic -- go up to somewhere it's dihedral.  You might think I'm using potential automorphy in this analytic argument, and we are, because it's simpler, but you can always make that analytic argument over a solvable extension of the base field, using Brauer's theorem.  So we think it may be possible to understand the Artin conjecture from this point of view.
\end{remark}

\part{Pol van Hoften (VU Amsterdam), \emph{Igusa stacks and exotic Hecke correspondences}}

\begin{abstract}
  Xiao and Zhu have conjectured the existence of exotic Hecke correspondences between the mod $p$ fibers of different Shimura varieties. In this talk I will present a conjectural relationship between the Igusa stacks for different Shimura varieties, which implies the conjecture of Xiao—Zhu. I will then discuss a proof of this conjecture for a large class of Shimura varieties, and give applications to the global Jacquet—Langlands correspondence. This is joint work in progress with Jack Sempliner.
\end{abstract}

Let's say something about exotic Hecke correspondences.  This phrase was perhaps coined by Zhu, but it's a kind of theme in the geometry of Shimura varieties, which is that mod $p$ Shimura varieties for ``related groups'' have ``related geometry''.  Certainly in work of Ribet this already plays a role -- his proof of the $\eps$-conjecture involves passing because modular curves (such as in the ``$p$-$q$ switch'').  This was taken up by Helm, who considered unitary Shimura varieties.  There's then the work of Tian--Xiao; in the work of these two people, it was all a bit of an art.  The work of Xiao--Zhu turned it into more of a science by giving a rigorous formulation of when these sorts of things should exist, explaining where they should come from.

We would like to start by giving some examples so that we have a feeling for what these things are.  As a kind of bonus, the result that we'll state will apply to these examples.
\begin{example}\label{example:cnpp15n4tc}
  Let $F$ be a real quadratic field, with $p$ split.  To be precise, maybe we should fix an ordering $\{\infty_1, \infty_2 \}$ of the real place, and $\{ \mathfrak{p}_1, \mathfrak{p}_2\}$ for the $p$-adic places.  Let $S$ be the mod $p$ Hilbert modular surface, choosing the level in such a way that it has good reduction.  This is some nice smooth surface over $\mathbb{F}_p$.  We also need our Shimura curves.  For $i = 1, 2$, consider $B_i := B_{\infty_i \mathfrak{p}_i}$.  This is the quaternion algebra over $F$ ramified precisely at $\infty_i$ and $\mathfrak{p}_i$.  We denote the corresponding Shimura curve by $X_{B_i}$, over $\mathbb{F}_p$.

  What we're being vague about is that these Shimura curves are defined over $F$, the reflex field, and I'm not telling you what $p$-adic places I'm taking.  Secretly, we're fixing an isomorphism
  \begin{equation*}
    (B_i \otimes \mathbb{A}_f^p )^\times \cong \GL_2(\mathbb{A}_f^p),
  \end{equation*}
  and everything is canonical relative to this choice.

  Tian--Xiao tells us that
  \begin{equation*}
    X_{B_i, \overline{\mathbb{F}_p}} \hookrightarrow S_{\overline{\mathbb{F}_p}}
  \end{equation*}
  as a closed \emph{Goren--Oort stratum}.  (This surface has two Goren--Oort strata.)  These curves moreover sit inside here Hecke equivariantly.
\end{example}
\begin{example}\label{example:cnpp15w7ik}
  Let $E$ be imaginary quadratic.  Let $W$ be a Hermitian $E$-module of rank $2 n$.  Choose some Hermitian $E$-module $W'$ of rank $2 n$, with $W \otimes \mathbb{A}_f \cong W ' \otimes \mathbb{A}_f$.  So somehow, as Hermitian $E$-modules, they're isomorphic to each other.  Then at an inert prime $p > 2$ in $E$, we can consider the Shimura variety for $X$ over $\overline{\mathbb{F}_p}$, as well as the Shimura variety for $W'$, and there exists a correspondence (rather than just a map)
  \begin{equation*}
    \begin{tikzcd}
      & \operatorname{Sh}(W | W') \arrow[ld, "t"'] \arrow[rd, "s"] & \\
      \operatorname{Sh}(W')_{\overline{\mathbb{F}_p}} & & \operatorname{Sh}(W)_{\overline{\mathbb{F}_p}}
    \end{tikzcd}
  \end{equation*}
  and $s, t$ have image a closed nonempty union of Newton strata.  

  Note that if we were in the definite case for $W'$, then the corresponding Shimura variety would be a set; the picture thus depends upon the archimedean situation.

  There's thus some sort of art .
\end{example}

\begin{remark}
  If $\operatorname{Sh}(W)_{\overline{\mathbb{F}_p}}$ is a moduli of certain abelian varieties $A$ with $E$-action, and $\operatorname{Sh}(W')_{\overline{\mathbb{F}_p}}$ is a moduli of certain abelian varieties $B$, then
  \begin{equation*}
    \operatorname{Sh}(W | W')_{\overline{\mathbb{F}_p}} = \left\{ \text{$p$-power $E$-linear isogenies } A \rightarrow B \right\}.
  \end{equation*}
  Note that these things don't lift to characteristic zero, i.e., to $\overline{\mathbb{Q}}$, due to issues involving the signature when $W \not \cong W'$.
\end{remark}
This remark doesn't apply to the first example, Example \ref{example:cnpp15n4tc} but you have to be a bit careful because these Shimura varieties are \emph{not} of PEL type.  To construct these, you have to put yourself in a situation where, by passing to auxiliary Shimura varieties, you can do something related to abelian varieties.  This theme will also appear in the rest of the talk.

Given that we've identified $W$ and $W'$ over the adeles, given $\alpha$, we get in particular that
\begin{equation*}
  \operatorname{GU}(W)_{\mathbb{Q}_p}
  \cong 
  \operatorname{GU}(W')_{\mathbb{Q}_p} =: G.
\end{equation*}
Newton strata are indexed by the subsets $B(G, - \mu)$ and $B(G, - \mu ')$ of $B(G)$.  Here $\mu$ and $\mu '$ are cocharacters coming from the Shimura datum, recordeing the signature.  The Newton strata hit by the correspondence $(s,t)$ are precisely the intersection $B(G, - \mu ) \cap B(G, - \mu ')$.

Let $(\mathbf{G}, \mathbf{X})$ be a Shimura datum.  Fix a prime $p$, together with an isomorphism $\mathbb{C} \cong \overline{\mathbb{Q}_p}$.  Let $\mathbf{P}$ be a $\mathbf{G}$-torsor over $\Spec \mathbb{Q}$, and let $\mathbf{G} ' := \Aut_{\mathbf{G}}(\mathbf{P})$.

Note that in the setting of Example \ref{example:cnpp15w7ik}, the torsors in the unitary setting classify unitary spaces, and the $\mathbf{P}$ that we're taking corresponds to $W'$.  Fix an isomorphism
\begin{equation*}
  \mathbf{P} \otimes \mathbb{A}_f^p \cong \mathbf{G} \otimes \mathbb{A}_f^p.
\end{equation*}
This corresponds to fixing a $W'$ and fixing this isomorphism, although we're happy to fix it at the adeles away from $p$.

Let $\mathbf{X} '$ be a Shimura datum for $\mathbf{G}'$.  We're assuming it exists.  In the case of unitary Shimura varieties, there's not much choice, because it's pinned down by what happens at $\infty$.  But we can certainly write down an orthogonal group with some signature condition for which there are no Shimura data.  We thus need to put some condition here.
We can consider
\begin{equation*}
  \operatorname{Sh}(\mathbf{G}, \mathbf{X})_{\mathbf{E}}, \qquad
  \operatorname{Sh}(\mathbf{G} ', \mathbf{X} )_{\mathbf{E} '},
\end{equation*}
where maybe we're secretly taking an inverse limit over levels away from $p$.
\begin{conjecture}[Scholze]
  There are Igusa stacks $\operatorname{IGS}(\mathbf{G}, \mathbf{X}) \rightarrow \mathrm{Bun}_{G}$ and $\operatorname{IGS}(\mathbf{G} ', \mathbf{X}) \rightarrow \mathrm{Bun}_{G'}$.
\end{conjecture}

You might argue that there are two or three possible Igusa stacks, but that cannot be the case -- these are actually unique if they exist (work in progress of Kim).  So we can really work with these objects as if they're unique.  They also exist in the Hodge type case.

In order to compare these two Igusa stacks, well, one lives over $\mathrm{Bun}_G$, the other over $\mathrm{Bun}_{G'}$.  Set $G := \mathbf{G}_{\mathbb{Q}_p}$ and $G' := \mathbf{G} '_{\mathbb{Q}_p}$.  Then $P := \mathbf{P}_{\mathbb{Q}_p}$ is a $G$-torsor over $\mathbb{Q}_p$, and you can think of this as a basic element of the Kottwitz set, giving an element of $\mathrm{Bun}_G(\mathbb{F}_p)_{\mathrm{basic}}$, and Fargues--Scholze will tell you that there is an isomorphism that somehow sends the class of $P$ to the class of $1$:
\begin{equation*}
  \mathrm{Bun}_G \rightarrow \mathrm{Bun}_{G '}
\end{equation*}
\begin{equation*}
  [P] \mapsto [1].
\end{equation*}
One Igusa stack lives over $\mathrm{Bun}_G$, the other over $\mathrm{Bun}_{G'}$, so we can relate them.  In fact, Igusa stacks live over $\mathrm{Bun}_{G, \mu} \subset \mathrm{Bun}_G$ and $\mathrm{Bun}_{G', \mu '} \subset \mathrm{Bun}_{G'}$, so we let $\mathrm{Bun}_{G, \mu, \mu '}$ and $\mathrm{Bun}_{G', \mu, \mu '}$ be their intersections.  We'll then have our Igusa stacks
\begin{equation*}
  \operatorname{IGS}(\mathbf{G}, \mathbf{X}) \mu '
  \rightarrow
  \mathrm{Bun}_{G, \mu, \mu '},
\end{equation*}
\begin{equation*}
  \operatorname{IGS}(\mathbf{G}', \mathbf{X}') \mu
  \rightarrow
  \mathrm{Bun}_{G', \mu, \mu '}.
\end{equation*}

\begin{conjecture}[Sempliner + E]\label{conjecture:cnpp17dreg}
  There is a $\mathbf{G}(\mathbb{A}_f^p)$-equivariant isomorphism
  \begin{equation*}
    \begin{tikzcd}
      & \operatorname{IGS}(\mathbf{G}, \mathbf{X}) \mu ' \arrow[ld, "t"'] \arrow[rd, "s"] \arrow[r] 
      & \operatorname{IGS}(\mathbf{G} ', \mathbf{X} '){\mu} \arrow[d] 
      \\
      \mathrm{Bun}_{G, \mu, \mu '} & & \mathrm{Bun}_{G', \mu, \mu '}
    \end{tikzcd}
  \end{equation*}

\end{conjecture}

[speaker in the way of his conjecture]

\begin{remark}
  For $b \in B(G, \mu, \mu ')$, we get $G(\mathbb{A}_f^p) \times \tilde{G_b}$-equivariant isomorphisms
  \begin{equation*}
    \operatorname{IG}^b(\mathbf{G}, \mathbf{X}) \times \operatorname{IG}^b(\mathbf{G} ', \mathbf{X} '),
  \end{equation*}
  geometrizing work of G?--T?.
\end{remark}

\begin{remark}
  There should be a version with $\mathbb{P}$ allowed to be a Kottwitz cocycle (Sempliner--Taylor).
\end{remark}

\begin{remark}
  Note that $H^1(\Res_{F / \mathbb{Q}} \GL_2) = 0$.  From our point of view, this explains why you have to go to an auxiliary unitary Shimura variety in this work of Tian--Xiao.
\end{remark}

\begin{remark}
  If the Igusa stacks moreover satisfy the integral diagram
  \begin{equation*}
    \begin{CD}         
      S_{K_p}(\mathbf{G}, \mathbf{X})^{\mathrm{an}} @>>> \operatorname{Sht}_{g, \mu}^{\mathrm{an}}\\
      @VVV  @VVV \\
      \operatorname{IGS}(\mathbf{G}, \mathbf{X}) @>>> \mathrm{Bun}_G\\
    \end{CD}
  \end{equation*}
  then this implies and generalizes a conjecture (or hypothesis) of Xiao--Zhu.  There's just a small diagram chase to relate this conjecture to theirs.  Gives
  \begin{equation}\label{eq:cnpp17s7i9}
    \mathbf{P}_{\mathbb{Q}_p} \xrightarrow{\cong} \mathbf{G}_{\mathbb{Q}_p}.
  \end{equation}
\end{remark}

\begin{remark}
  Often, there aren't so many of these $\mathbb{P}$'s.  The situation is most interesting for adjoint groups.  Most adjoint groups will just have these cohomology classes lying around.  
\end{remark}

Let's give one more example.

\begin{example}[New]
  Let $F$ be real quadratic.  Let $V$ over $F$ be a quadratic space (something with a quadratic form).  We want the signatures to be $(2, n - 2)$ and $(2, n - 2)$, with $n$ \emph{odd} (odd orthogonal groups are nicer than even orthogonal groups for many reasons).  We take $\mathbf{G} = R_{F / \mathbb{Q}} \SO(V)$, and $p$ split in $F$ such that $\mathbf{G}_{\mathbb{Q}_p}$ is split.  We take $\mathbf{G} ' = R_{F/ Q} \SO(V')$, where $V \otimes \mathbb{A}_f^p \cong V' \otimes \mathbb{A}_f^p$, and $V_{\mathbb{R}}'$ has signature $(2, n - 2)$, $(0, n)$ and $V_{\mathfrak{p}_2}'$ is non-split, $V_{\mathfrak{p}_1}'$ is split.

  There's a similar story in type C, where you take symplectic groups, change them at one infinite place and one corresponding $p$-adic place if $p$ is split.
\end{example}

We need two assumptions.
\begin{theorem}[vH--Sempliner]
  Suppose $\mathbf{G}_{\mathbb{Q}_p}$ is split over an unramified extension (not necessarily quasi-split), $p > 2$, and some technical condition is satisfied, and $(\mathbf{G}, \mathbf{X})$ is of Hodge type, then Conjecture \ref{conjecture:cnpp17dreg} holds.

  This result is conditional on the conjecture of Xiao--Zhu concerning Hecke correspondences, although they have announced the proof as of last year.  (For $\mathbf{G}_{\mathbb{Q}_p}$ quasi-split and $\mathbf{P}_{\mathbb{Q}_p} \cong \mathbf{G}_{\mathbb{Q}_p}$ trivial.)
\end{theorem}

Now, what we can do is go back to this comment that the conjecture is most interesting for adjoint groups.  Given adjoint abelian type $\mathbf{G}, \mathbf{X}, \mathbf{P}$, you can ask if there is a Hodge type
\begin{equation*}
  (\mathbf{G}_2, \mathbf{X}_2, \mathbf{P}_2) \rightarrow(\mathbf{G} , \mathbf{X} , \mathbf{P}).
\end{equation*}
It's hard to answer this question in the negative, because then you need to classify Hodge type Shimura varieties or something.  This is always true in type B and type C.  It's sometimes true in type $\mathbb{D}^{\mathbb{R}}$, $\mathbb{D}^{\mathbb{H}}$, and $A$.  (If your group is locally $\GL_n$ at $p$, then we don't know how to show that.)  This is work in progress.  In the function field case there are no conditions and you can write anything down, more-or-less.

In my remaining time, I don't want to say too much about the proof.  If you really want to know it, there's a YouTube video from Bonn last year.  The point is to go in the Xiao--Zhu case ($G_{\mathbb{Q}_p}$ is quasi-split and the torsor is trivial) and look at what their theorem on exotic correspondences tell you, then with some effort, you realize that you can reverse the arrow in \eqref{eq:cnpp17s7i9}.

Leads to a fixed point theorem for Igusa stacks.

Let's conclude by discussing \emph{applications to cohomology}.  We give the theorem first.
\begin{theorem}[vH--Sempliner]
  If the conjecture holds for $(\mathbf{G}, \mathbf{X}, \mathbf{P})$, then we can compare the cohomology of the two different Shimura varieties: for $\phi$ a supercuspidal Fargues--Scholze $L$-parameter
  \begin{equation*}
    W_{\mathbb{Q}_p} \rightarrow {}^L G = {}^L G'
  \end{equation*}
  from the Weil group of $\mathbb{Q}_p$ to the dual group of $G$ (or equivalently, of $G'$), we can prove that
  \begin{equation*}
    R \Gamma_{\acute{e}t}(\operatorname{Sh}(\mathbf{G}, \mathbf{X}), \overline{\mathbb{Q}_p})_{\phi}
    \neq 0 \iff
    R f_{\acute{e}t}(\operatorname{Sh}(\mathbf{G} ', \mathbf{X} '))_{\phi} \neq 0.
  \end{equation*}
  More precisely, if $\phi$ satisfies
  \begin{equation*}
    \Stab_\phi = Z(\hat{G})^{\Gamma_p}
  \end{equation*}
  and if $\mathbf{P}_{\mathbb{Q}_p}$ is trivial, then what happens is that
  \begin{equation*}
    R f_{\acute{e}t}(\operatorname{Sh}(\mathbf{G}, \mathbf{X}), \mathbb{Q}_{\ell})
  \end{equation*}
  is a direct summand over $G(\mathbb{A}_f^p)$ of
  \begin{equation*}
    R \Gamma_{\acute{e}t}(\operatorname{Sh}(\mathbf{G}, \mathbf{X}), \overline{\mathbb{Q}_p})_{\phi},
  \end{equation*}
  or vice-versa.  But this inclusion is not Galois-equivariant.  

  If you like, you can put $\overline{\mathbb{F}_{\ell}}$-coefficients there.  
\end{theorem}
\begin{proof}
  Glue together Igusa stacks to build
  \begin{equation*}
    \tilde{\mathcal{F}}
    \in D_{\acute{e}t}(\mathrm{Bun}_G)
  \end{equation*}
  such that
  \begin{equation*}
    R \Gamma_{\acute{e}t}(\mathbf{G}) = i_1^\ast
    \mathcal{V}_\mu  \tilde{\mathcal{F}}
  \end{equation*}
  and
  \begin{equation*}
    i_1^\ast \mathcal{V}_{\mu '} \tilde{\mathcal{F}} =
    R \Gamma_{\acute{e}t}(\mathbf{G} ')
  \end{equation*}
  and

  \begin{equation*}
    \begin{tikzcd}
      \Hom_{X_{\hat{G}}}(V_\mu, V_{\mu'}) \arrow[dd, "f"] \arrow[rr] & & \Hom(R \Gamma_{\acute{e}t}(\mathbf{G}), R \Gamma_{\acute{e}t}(\mathbf{G}'))_{\phi} \\
      & & \\
      \Hom_{X_{\hat{G}, \phi}}(V_{\mu}, V_{\mu '}) \arrow[uurr, "g"] & & 
    \end{tikzcd}
  \end{equation*}
  Something about $X_{\hat{G}, \phi} = \ast / Z(\hat{G})^\Gamma$.
\end{proof}

\part{Jack Thorne, \emph{The Ramanujan Conjecture for Bianchi Modular Forms}}

\begin{abstract}
  The Ramanujan Conjecture for a Bianchi modular form over an imaginary quadratic field was proved in the lowest weight (“weight 2”) case in the 10-author paper, by proving the potential automorphy of the symmetric powers of associated 2-dimensional compatible systems of Galois representations. Treating the higher weight situation presents difficulties due to the paucity of congruences between Bianchi modular forms of different weights. I will explain these difficulties and how they can be avoided to indeed prove the Ramanujan Conjecture in the general case. This is joint work with Boxer, Calegari, Gee, and Newton.
\end{abstract}

Let's begin by saying a bit about the classical story.  Let's think first about the Ramanujan $\Delta$ function
\begin{equation*}
  \Delta(z) = q \prod_{n = 1}^\infty(1 - q^n)^{2 4} \in S_{12}(\SL_2(\mathbb{Z})).
\end{equation*}
Ramanujan (1916): if $p$ is prime, then $\lvert \tau(p) \rvert \leq 2 p^{11/2}$.  An equivalent formulation is that the Hecke polynomial $X^2 - \tau(p) X + p^{11}$, when written as a product of linear factors $(X - \alpha_p)( X - \beta_p)$, has the property that the matrix
\begin{equation*}
  \gamma_p := p^{- 11/2}
  \begin{pmatrix}
    \alpha_p    & 0 \\
    0 & \beta_p  \\
  \end{pmatrix}
\end{equation*}
lies in $\SU_2(\mathbb{R})$.

The Sato--Tate conjecture (maybe first observed in some cases by Serre in 1966) says that the elements $[\gamma_p] \in \SU_2(\mathbb{R}) / \mathrm{conjugacy}$ are equidistributed with respect to the natural measure on the space of conjugacy classes, namely, the pushforward of the probability Haar measure.

Deligne (1969) associates to $\Delta$ the compatible system $(\rho_{\Delta, \ell} : G_{\mathbb{Q}} \rightarrow \GL_2(\mathbb{Q}_{\ell}))_{\ell}$, occurring in the {\'e}tale cohomology
\begin{equation*}
  H^\ast_{\acute{e}t}(\mathrm{K S}, \mathbb{Q}_{\ell}),
\end{equation*}
where $\mathrm{K S}$ stands for Kuga--Sato.  Using this, he showed that the Weil conjectures imply the Ramanujan conjecture.

A bit earlier, Serre observed that to prove Sato--Tate (assuming Ramanujan), it is enough to show that for each $m \geq 1$, the symmetric power $L$-functions $L(\sym^m \rho_{\Delta,_{\ell}}, s)$ have a meromorphic continuation to the complex numbers that is holomorphic and nonvanishing on the boundary of the region of absolute convergence, namely when $\Re(s) = 1 + \tfrac{11m}{2}$.

The proof was completed by Barnet--Lamb, Geraghty, Harris and Taylor in 2011, following earier work of Clozel and Shepherd--Barron.  They proved Sato--Tate for $\Delta$ by showing that for each $m \geq 1$, $\sym^m \rho_{\Delta,_{\ell}}$ is potentially automorphic, i.e., there exists an extension $F_m | \mathbb{Q}$ and cuspidal automorphic $\Pi_m \circlearrowleft \GL_{m + 1}(\mathbb{A}_{F_m})$ with the property that the $L$-function of the restricted representation is equal to the standard $L$-function of that automorphic representation:
\begin{equation*}
  L(\sym^m \rho_{\Delta,_{\ell}} | _{G_{F_m}}, s)
  =
  L(\Pi_m, s).
\end{equation*}
That was a very exciting result when it was proved.  It also gave a ``new'' proof of the Ramanujan conjecture for $\Delta$.  Why is that?  Well, there's a standard argument that if you know the potential automorphy for infintiely many symmetric powers, then combining the trivial bound for the symmetric powers yields the Ramanujan bound for the form you start with.  (When the speaker and Newton first did symmetric power functoriality over $\mathbb{Q}$ for $\Delta$, people asked if it gave a new proof of Ramanujan, but the answer was probably no, because we used purity.  But I think it should be possible to prove potential modularity without using purity, so it deserves to be called a ``new'' proof.)

What's the generalized Ramanujan conjecture (GRC) for $\GL_2$?  Fix a number field $F$, and a cuspidal automorphic representation $\pi \circlearrowleft \GL_2(\mathbb{A}_F)$.  Just for normalization, let's suppose that the central character $\omega_\pi$ is unitary.  Then the generalized Ramanujan conjecture is the statement that for each place $v$ of your number field, the representation $\pi_v$ is tempered.  Put another way, this says that if you apply the local Langlands correspondence to get a parameter and take the image, giving
\begin{equation*}
  \operatorname{rec}_{F_v}(\lambda_v)(L_{F_v}),
\end{equation*}
you get a relatively compact subset (i.e., of compact closure).  This corresponds in the unramified case to the eigenvalues or Satake parameters being of absolute value one.

What is the generalized Sato--Tate conjecture (GST)?  The assertion is that there is some compact subgroup $M_\pi \leq \GL_2(\mathbb{C})$ such that for each finite unramified place $v \nmid N(\pi) \infty$, the element
\begin{equation*}
  \operatorname{rec}_{F_v} \left( \pi_v \right)(\Frob_v) =: \gamma_v
\end{equation*}
conjugate to an element of $M_\pi$, and the conjugacy classes $[\gamma_v] \in M_\pi / \mathrm{conjugacy}$ is equidistributed as $v$ varies.  What's known?  We know GRC when $\pi$ is CM (``automorphic induction''), where it is easy.  Another case is when $F$ is a totally real number field and $\pi$, our automorphic representation, is associated to Hilbert modular forms of regular weight.  That means of weights $(k_v)_{v \mid \infty}$, $v$ ranging over the set of infinite places of our totally real field, with $k_v \geq 2$.  The methods used to prove this are similar to what were used with $\Delta$.

Beyond that setting, I don't think there are other examples, besides those we're about to talk about, where we know the Ramanujan conjecture.  In general, we have only partial information coming from known cases of symmetric power lifting applied to small rank general linear groups.  To give an example of the kinds of things you can say, there's a rather charming theorem of Dinakar Ramakrishnan (1997) which says that over any number field, for any cusp form $\pi$ over $\GL_2$, the Ramanujan conjecture is true for $\pi _v$ for at least $90\%$ of the places $v$ of your number field.  That is to say, the set of such places $v$ has lower Dirichlet density at least $9/10$.

Now let's talk about some cases where we can actually say more.


Suppose $F$ is CM and $\pi$ is of weight $2$, and in particular, regular algebraic.  This means that
\begin{equation*}
  (\pi^\infty)^K \hookrightarrow H^\ast_{\cusp} \left( \Gamma \backslash \mathbb{H}_3^{[F^+ : \mathbb{Q}]}, \mathbb{C} \right).
\end{equation*}
Then GRC and GST are known for $\pi$.  How did we prove this?  Well, we in fact showed that for each $m \geq 1$, the representation
\begin{equation*}
  \sym^m \rho_{\pi, \ell} : G_F \rightarrow \GL_{m + 1}(\mathbb{Q}_{\ell})
\end{equation*}
is potentially automorphic.  There's one surprising feature of this that's worth drawing your attention to.  In the setting of the $\Delta$ function or more generally a Hilbert modular form, one knows how to realize the Galois representation in almost all cases inside the cohomology of Shimura variety.  That's how one can prove purity first.  But for these ones over CM fields, we only know how to construct them using congruences.  So we don't know purity any way other than by using the potential automorphy of the associated symmetric power representations.

The work we want to talk about today is a generalization of this that removes the ``weight $2$'' condition.
\begin{theorem}[Boxer--Calegari--Gee--Newton--Thorne, \cite{2023arXiv2309.15880}]
  Suppose $F$ is a CM field, and $\pi$ has parallel weight.  Then GRC and GST hold for $\pi$.
\end{theorem}
Our assumption says that
\begin{equation*}
  (\pi^\infty )^K \hookrightarrow H^\ast_{\cusp} \left( \bigoplus_{\tau : F \hookrightarrow \mathbb{C}}
    \sym^{k - 2} \mathbb{C}^2\right),
\end{equation*}
or
\begin{equation*}
  \operatorname{H T}_\tau\left(\rho_{\pi, \ell}\right) = \{0, k - 1\}.
\end{equation*}

\begin{remark}
  \begin{enumerate}
  \item\label{enumerate:cnpp2drvjk} This parallel weight condition is vacuous if $[F: \mathbb{Q}] = 2$, the reason being that there's this so-called purity lemma of Clozel that says that the weights of any regular algebraic form on $\GL_n$ over any totally complex field are invariant under complex conjugation.  That's why we called the lecture ``the Ramanujan conjecture for Bianchi modular forms'', and there we have no restrictions.
  \item\label{enumerate:cnpp2dtmbo} What's far from obvious is that the theorem is in fact non-empty, in the sense that you can find interesting examples of automorphic representations of this type of parallel weight.  Over a totally real field, we have dimension formulas that make such existence obvious.  But we can find an example.  The simplest one might be that there exists such a $\pi$ over $\mathbb{Q}(\sqrt{- 2})$ that has level $\Gamma_0(5 + 7 \sqrt{- 2})$ and parallel weight .
  \item Matsumoto \cite{2023arXiv2312.01551} has proved the same theorem, with no restriction on the weight.
  \end{enumerate}
\end{remark}


Let's now talk about the proof strategy.  Let $F$ be a CM field, $\pi$ on $\GL_2(\mathbb{A}_F)$ of parallel weight, non-CM.  Fix $m \geq 1$, and consider $R_{\ell} = \Sym^m \rho_{\pi, \ell} : G_F \rightarrow \GL_{m + 1}(\mathbb{Q}_{\ell})$.  Approach to proving potential automorphy of $R_{\ell}$:
\begin{enumerate}
\item Write down a compatible system $\left( S_{\ell} : G_{F(t)} \rightarrow \GL_{n}(\mathbb{Q}_{\ell}) \right)$, where $n := m + 1$.
\item Choose a prime $p \neq q$
\item Find a Galois extension $F ' / F$, $t_0 \in F'$ such that we have the first congruence
  \begin{equation*}
    \left(     t_0^\ast \overline{S_p} : G_{F '} \rightarrow \GL_n(F_{\ell}) \right)
    \cong \overline{R}_p |_{ G_{F '}}
  \end{equation*}
  and such that $t_0^\ast \overline{S}_q$ is residually automorphic.
\item Apply automorphy lifting theorem (ALTs) twice to see first that
  $t_0^\ast S_q$ is automorphic, then that the $p$-adic member $t_0^\ast S_p$ is automorphic.  We then apply the automorphy lifting theorem to deduce that $R_p |_{G_F '}$ is automorphic.
\end{enumerate}
This argument has been applied in many examples.  Whether you can make it work or not in a given case is clearly very dependent on the strength of the ALTs in your arsenal.  Let us thus first review a bit about what we know about ALTs in this setting.

The setting is that $F$ is a CM field, and we're thinking of $n$-dimensional Galois representations
\begin{equation*}
  \rho_1, \rho_2 : G_F \rightarrow \GL_n(\mathbb{Q}_{\ell})
\end{equation*}
which have isomorphic residual representations $\overline{\rho_1} \cong \overline{\rho_2}$ that we might assume to be irreducible.  We assume moreover that both are de Rham and Hodge--Tate regular, and that $\rho_2$ is automorphic.  We then want to know, what additional conditions do we need to put to ensure that $\rho_1$ is automorphic?  The most important condition that we're \emph{not} adding here is conjugate self-duality.  We would otherwise be in what's called the defect zero setting, which is for example the one used to understand the symmetric powers of $\Delta$.  That theory is now very mature and general and powerful, and could give a result in this case with almost no additional hypotheses.  But we're interested in the \emph{positive} defect setting, which is the one relevant for this theorem, and the technology is nowhere near as advanced.

So that's the setup.

(A) The first result we want to mention is what's sometimes called the ``natural output of the Taylor--Wiles--Kisin method'' (not sure it's in the literature in this form, but you can deduce it using the standard kinds of argument from the recent work of Caraiani--Newton, who in particular proved particular cases of local-global compatibility at primes $\ell = p$).
In this setting, we can deduce the automorphy of $\rho _1$ assuming the further condition that for every finite place $v \nmid \infty$, we have the relation denoted
\begin{equation*}
  \rho_2 |_{G_{F_v}} \rightsquigarrow \rho_1 |_{G_{F_v}}
\end{equation*}
introduced by BLGGT in 2014, whose meaning depends upon whether we're talking about an $\ell$-adic place or not.
\begin{itemize}
\item If it's non-$\ell$-adic, then it says that the two restrictions define points in the \emph{same} irreducible component of $R_v^D$, plus the additional condition that this be the unique component containing the representation associated to $\rho_2$.
\item If we're talking about an $\ell$-adic place, then the condition is slightly more complicated, but what we want to say is something like that both of these representations are crystalline and have the same Hodge--Tate weights, which means that they both define points of the same Kisin ring, and we then ask that these two representations again define points on the same irreducible component of that crystalline fixed weight lifting ring $R_v^{\mathrm{cr}, \lambda}$ of the residual representation.
\end{itemize}

(B) That's one thing we know how to do.  Here's another.  This is a theorem that's proved in the 10-author paper.  Suppose either that either
\begin{itemize}
\item for every $v \mid \ell$, both $\rho_1 |_{G_{F_v}}$ and $\rho_2 |_{G_{F_v}}$ are Fontaine--Laffaille, or
\item for all $v \mid \ell$, both restrictions are ordinary.
\end{itemize}
Then $\rho_1$ is automorphic.  This is what happens when you implement the so-called ``Ihara-avoidance'' trick of Taylor.  Somehow you put yourself in a situation where, above each irreducible component in the special fiber of the lifting that you have to worry about, there is exactly one irreducible component of the lifting ring itself.

BLGGT start with something like these (A) and (B), and input some additional tools in the 
conjugate self-dual setting.  These include Khare--Wintenberger, which gives a very strong local-global principle allowing you to produce auxiliary representations that you want.  But can't apply this in general, because the lifts you won't simply don't exist.  You'd also like to be able to use the so-called ``Harris tensor product trick'', which in some sense \emph{is} available here, but is less effective for reasons we'll come to in a bit.

(C) We're now ready to mention the first new tool that we introduce in our paper.  This is a new automorphy lifting theorem.  Suppose that for each $v \mid \ell$, the two representations $\rho_1 |_{G_{F_v}}$ and $\rho_2 |_{G_{F_v}}$ are crystalline, of Hodge--Tate weights the minimum possible, namely $\{0, 1, \dotsc, n - 1\}$.  Let's suppose also that we have this relation $\rho_2 |_{G_{F_v}} \rightsquigarrow \rho_1 |_{G_{F_v}}$.  This $\rho_1$ is automorphic.  This is a bit of a hybrid between (A) and (B), because you have no conditions to verify at places away from $\ell$, but there \emph{are} conditions to verify at the $\ell$-adic places.  We can prove this by using Ihara-avoidance \emph{provided} that these local lifting rings that are going to appear, namely $R_v^{\mathrm{cr}, \{0, \dotsc, n - 1\}}$ have this unique lifting property, namely, that each minimal prime of $R_v^{\mathrm{cr}, \{0, \dotsc, n - 1\}} / (\ell)$ is the specialization of a unique minimal prime $R_v^{\mathrm{cr}, \{0, \dotsc, n - 1\}}$.  (So that's a special proprety that would be true if it were smooth, but is true more generally.)  This follows if $R_v^{\mathrm{cr}, \{0, \dotsc, n - 1\}} / (\ell)$ is generically reduced.  We prove this provided that the prime is bigger than the dimension, i.e., $\ell > n$.  This is a completely separate result.  How do we prove it?  Well, we prove it using the Emerton--Gee stack, which makes studying this kind of question much easier.  Let's just describe the structure in words.  This special fiber of the lifting ring, we realize it as a versal ring and can reduce this statement to proving that the associated substack of the EG stack is generically reduced, but then because that's an algebraic stack you have much more room to wiggle around, and then you can reduce that question to finding one point on each irreducible component that's formally smooth, and that's something you can do by explicit computation.

Let's now come back to the general situation of $R_{\ell} = \sym^m \rho_{\pi, \ell}$, which we are assuming has Hodge--Tate weights $\operatorname{H T}(\rho_{\pi, \ell}) = \{0, k - 1\}$.  That means that
\begin{equation}\label{eq:cnpp2hsxjl}
  \operatorname{H T}(R_{\ell}) = \left\{ 0, k - 1, 2(k - 1), \dotsc, m(k - 1) \right\}.
\end{equation}
Then any (interesting) parameter $S_{\ell} : G_{F(t)} \rightarrow \GL_n(\mathbb{Q}_{\ell})$ that you'll produce is coming from geometry, and the Hodge--Tate weights will be consecutive, and that's because of the conditions imposed by Griffiths transversality.  That means that if we want to apply this automorphy lifting theorem, we can't, because the Hodge--Tate weights in \eqref{eq:cnpp2hsxjl} are not consecutive.  So instead, we're going to do something different: we show the potential automorphy of another representation, namely, of
\begin{equation*}
  R_{\ell} \otimes T_{\ell} : G_F \rightarrow \GL_{(k - 1)(m + 1)}(\mathbb{Q}_{\ell}),
\end{equation*}

where $T_{\ell} : G_F \rightarrow \GL_{k - 1}(\mathbb{Q}_{\ell})$ is some auxiliary thing, with $\operatorname{H T}(T_{\ell}) = \{0, \dotsc, k - 2\}$ (consecutive, so we stand a chance of proving that it's potentially automorphic)
\begin{remark}
  There's a separate problem of relating this to what we started with, which amounts to removing the $T_{\ell}$ somehow, but we don't want to talk about this.
\end{remark}
Where does $S_{\ell}$ come from?  We'll find it inside the cohomology of the Dwork family, which is the family of algebraic varieties depending upon the parameter $t$ and given by the equation
\begin{equation*}
  Y_t : x_1^N + \dotsb + x_N^N = t N x_1 \dotsb x_N,
\end{equation*}
which is a family of hypersurfaces inside the projective space $\mathbb{P}_{F(t)}^{N - 1}$.  What's going on here is there's an action of a certain group $H = \ker \left( \Pi : \mu_N^N \rightarrow \mu_N \right)$, acting on projective space by multiplication on coordinates and acting on $Y_t$, hence on the cohomology $H^\ast_{\acute{e}t}(Y_{t, \overline{F(t)}, \mathbb{Q}_{\ell}})$, and $S_{\ell}$ is a certain eigenspace.  We want to mention the work of Qian, a student of Taylor, who is the first person to look at pieces of the Dwork family that are not conjugate self-dual.  We're using the construction in his paper.

The main problem now is that to apply this new automorphy lifting theorem (C), we need to find all the data $p$, $T_p$, $F' | F$ and $t_0 \in F'$ such that for every place $v' \mid p$ in $F'$, all the representations $R_p \otimes T_p |_{G_{F'_{v '}}}$ and $t_0^\ast S_p  |_{G_{F_v '}}$ lie on a common irreducible component of the local lifting ring $R^{\mathrm{cr}}_{v '}$.  That's exactly this next relation that we need to verify to prove the theorem.  That's \emph{a priori} quite tricky, because this ring, when the base field is highly ramified, we don't know anything about the components.  Related work of Kisin.  We need to be able to control things coming from the cohomology of the Dwork family.  In practice, there are two things you might look for.  Either
\begin{itemize}
\item want everything to be ordinary (and crystalline), or
\item everything ``of niveau $2$''.
\end{itemize}
Surprisingly, the hard thing is finding enough specializations of the cohomology of the Dwork family that are both crystalline and ordinary.  That's the other main technical thing which we show how to do.

So the second main ingredient in our paper is finding enough crystalline ordinary representations of $S_p$, i.e., in the Dwork family.  Let's conclude by saying a few words about how we do that.  These representations are constructed inside the cohomology of an algebraic variety, so this is really a question about the action of Frobenius.  That is, we need to find $\overline{t_0}$ in the algebraic closure of the residue field $\overline{k(v)}$ such that if we look at the crystalline cohomology of the fiber of the Dwork family,
\begin{equation*}
  H^\ast_{\mathrm{cr}} \left( Y_{\overline{t_0}} \right)^{\chi}
\end{equation*}
has to be ordinary.  How do we do that?  Well, there's this very nice recent theorem of Drinfeld--Kedlaya which is very general -- it's about families of isocrystals -- that says in particular that there exists a Zariski open, nonempty subset of the affine line over the residue field
\begin{equation*}
  U \subset \mathbb{A}^1_{k(v)}
\end{equation*}
such that the Newton polygon is, first of all, constant for $\overline{t_0} \in \overline{k(v)}$, and moreover, if we write $s_1 \geq \dotsb \geq s_n$ for the slopes of this generic Newton polygon (these are what you need to control if you're showing that there's a point where the polygon is ordinary), then they can't increase too quickly:
\begin{equation*}
  s_i \leq s_{i + 1} + 1
\end{equation*}
for all $i = 1, \dotsc, n - 1$.  You can think of this as an analogue of Griffiths transversality but for the slopes of the generic Newton polygon.  This means that to show that you have an ordinary point, it's enough to show that the lowest slope vanishes: $s_n = 0$.  Equivalently, if you look at the trace of Frobenius, you have to show that that's not zero mod $p$.  We do that by an explicit computation.


\bibliography{refs}{} \bibliographystyle{plain}
\end{document}
