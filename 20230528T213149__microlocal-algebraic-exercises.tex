\documentclass[reqno]{amsart} \input{common.tex}

\title{Some algebra-flavored exercises}

\begin{document}

\maketitle

\begin{abstract}
  In this evolving note, we record some algebra-flavored exercises relevant for the Oberwolfach seminar.
\end{abstract}


\section{Cyclic and regular matrices}
Let $F$ be a field, let $V$ be a vector space over $F$, and let $M := \End(V)$ denote the space of linear maps $V \to V$.

\begin{definition}
  Let $\tau \in M$ and $v \in V$.  We denote by $F[\tau] v$ the set of elements of $V$ that may be written as a polynomial in $\tau$ applied to $v$, or equivalently, the span of the elements
  \begin{equation*}
v, \quad \tau v, \quad \tau^2 v, \quad (\dotsc).
\end{equation*}
We say that a vector $v \in V$ is $\tau$\emph{-cyclic}, or that $v$ is a \emph{cyclic vector} for $\tau$, if
\begin{equation*}
F[\tau] v = V.
\end{equation*}
We say that $\tau$ is \emph{cyclic} if it admits a cyclic vector.
\end{definition}

\begin{exercise}
  Show that $\tau$ is cyclic if and only if there is a basis with respect to which it is of the form, e.g., for $\dim(V) = 4$,
  \begin{equation*}
\begin{pmatrix}
0 & 0 & 0 & \ast \\
1 & 0 & 0 & \ast \\
0 & 1 & 0 & \ast \\
0 & 0 & 1 & \ast \\
\end{pmatrix}.
\end{equation*}
\end{exercise}

\begin{exercise}
  For each monic polynomial of degree $\dim(V)$, show that there exists a cyclic element $\tau \in M$ whose characteristic polynomial is that polynomial.  Show moreover that any two cyclic elements with the same characteristic polynomial are conjugate.
\end{exercise}

\begin{exercise}
  Show that a matrix given in Jordan form is cyclic precisely when the eigenvalues pertaining to different Jordan blocks are distinct.  In particular, a diagonal matrix is cyclic precisely when its diagonal entries are distinct.
\end{exercise}

\begin{definition}
  We say that $\tau \in M$ is \emph{regular} if $\dim M_\tau = \dim V$, where
  \begin{equation*}
    M_\tau := \left\{ x \in M : x \tau = \tau x \right\}
  \end{equation*}
  denotes the \emph{centralizer} of $\tau$ in $M$.
\end{definition}

\begin{exercise}
  Show that for $\tau \in M$, the following are equivalent.
  \begin{enumerate}[(i)]
\item $\tau$ is cyclic.
\item $M_\tau = F[\tau]$.
\item $\tau$ is regular.
\end{enumerate}
\end{exercise}

\begin{definition}
We recall that $\tau \in M$ is \emph{nilpotent} if some power of $\tau$ vanishes.
\end{definition}

\begin{exercise}
  Suppose that $F = \mathbb{R}$.  Fix a norm $\lVert . \rVert$ on $M$.  Let
  \begin{equation*}
    \mathcal{O} \subseteq M
  \end{equation*}
  be a conjugacy class consisting of regular (equivalently, cyclic) elements.  Let $x_j \in \mathcal{O}$ be a sequence whose matrix norms $\lVert x_j \rVert$ tend to infinity.
\begin{enumerate}[(i)]
\item Show that, after passing to a subsequence if necessary, the normalized limit
  \begin{equation*}
x := \lim_{j \rightarrow \infty } \frac{x_j}{ \lVert x_j \rVert}
\end{equation*}
exists, and is nilpotent.
\item Show that there exists a sequence $x_j$ as above for which the normalized limit $x$ is regular nilpotent.
\end{enumerate}  
\end{exercise}






\bibliography{refs}{} \bibliographystyle{plain}
\end{document}
