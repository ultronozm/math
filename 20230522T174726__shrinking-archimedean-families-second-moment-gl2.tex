\documentclass[reqno]{amsart} \input{common.tex}

\title{Shrinking archimedean families: second moment for $\GL_2$}
\author{TH, PN}

\begin{document}

\maketitle
\tableofcontents

\begin{abstract}
  We attempt, unsuccessfully, to estimate certain second moments for $\GL_2$ involving conductor-truncated families.
\end{abstract}

\section{Overview}\label{sec:20230522180023}
We consider $\pi $ on $\PGL_2(\mathbb{Z}) \backslash \PGL_2(\mathbb{R})$ and try to estimate
\begin{equation*}
\sum_{C(\pi) \leq Q} \left\lvert L (\pi, \tfrac{1}{2} + i T ) \right\rvert^2.
\end{equation*}
Here $Q$ and $T$ are asymptotic parameters.  We have in mind the range $Q \ll T$.  In this range (or indeed, for $Q \lll T^2$), the analytic conductors for the individual $L$-functions are $\asymp T^2$, so the convexity bound for the the squared $L$-function is $\ll T$.  It is straightforward to obtain an asymptotic formula for the above moment in the range where $Q \ggg T$.  We would like to obtain an essentially sharp upper bound for some $Q \lll T$, ideally $Q \ll T^{1-\delta}$.  This seems hard.

Note that the range $Q \asymp T$ is critical: a sharp bound for the moment in this range recovers the convexity bound for the individual $L$-values, while a sharp bound in any shorter range would give a subconvex bound.

\section{Test functions}\label{sec:20230522180025}
Let's set things up.  We take a test function $f_0$ to be a normalized smoothened characteristic function of $K_0(Q)$, the archimedean variant of the standard congruence subgroup, like in \cite{JN19a}:
\begin{equation*}
  K_0(Q)
  = \left\{
    \begin{pmatrix}
a & b \\
c & d \\
    \end{pmatrix}
    :
    a = 1 + o(1), \quad
    b = o(1),
    \quad
    c \lll 1/Q,
    \quad
    d = 1 + o(1)
  \right\}.
\end{equation*}

This should typically pick off something like an ``analytic newvector'' $W_0 \in \pi$ for $\pi$ with $C(\pi) \leq Q$.  In the Kirillov model, $W_0$ could be taken to look like a smooth bump supported near $1$:
\begin{equation*}
  W_0(y) \approx 1_{y \asymp 1}^{\text{smooth}}.
\end{equation*}
We then defined $f$ to be the conjugate of $f_0$ by $n(T)$.  On the spectral side, the contribution from $\pi$ will be
\begin{equation*}
  \left\lvert L(\pi, \tfrac{1}{2} + i T) \right\rvert^2
  \sum _{W_0 \in \mathcal{B}(\pi)} 
  \left\lvert
    \int _{y \in \mathbb{R} ^\times }
    \pi(f) W_0(y)
    |y|^{i T}
    \, d^\times y
  \right\rvert^2.
\end{equation*}
Now consider the contribution from an ``analytic newvector'' $W_0$ as above.  The local weight will be, with
\begin{equation*}
  W := n(T) W_0, \quad W(y) = e(T y) W_0(y),
\end{equation*}
\begin{equation*}
  \left\lvert \int _{y \in \mathbb{R}^\times } W (y) |y|^{i T} \, d^\times y \right\rvert^2 \asymp T^{-1}.
\end{equation*}
So far, so good.

\section{Geometric approximate functional equation}\label{sec:20230522180027}
The problem is that we're looking at the global period integral: for $\varphi \in \pi$,
\begin{equation*}
  \int _{y \in \mathbb{R}^\times / \mathbb{Z} ^\times }
  \varphi(a(y)) |y|^{i T} \, d^\times y, \quad
  a(y) :=
  \begin{pmatrix}
    y & 0 \\
    0 & 1
  \end{pmatrix}.
\end{equation*}
We really want to replace this with an integral over a compact subset of $\mathbb{R}^\times / \mathbb{Z}^\times$, so that we can later apply Cauchy---Schwarz productively.  We argue like in \cite[\S5.1.4]{michel-2009} (see also \cite[\S5.3]{2021arXiv210915230N}).  The idea is that if we smoothen this integral out, then we get quite bounds away from some critical dyadic range, and then we focus on that range.

Let's get started by fixing $h \in C_c^\infty(\mathbb{R}^\times_+)$.  We Mellin expand $h$:
\begin{equation*}
  h(t) = \int _{(\sigma)} H(s) t^s \, \frac{d s }{ 2 \pi i}.
\end{equation*}
We assume $h$ normalized to have integral one, so that $H(0) = 1$.

For each positive parameter $Y \in \mathbb{R}^\times_+$, consider
\begin{equation*}
  I(Y) :=
  \int _{y \in \mathbb{R}^\times / \mathbb{Z} ^\times }
  h \left( \frac{\lvert y \rvert}{Y} \right)
  \varphi(a(y)) |y|^{i T} \, d^\times y.
\end{equation*}
Then we aim to bound $I(Y)$ using the convexity bound for $L(\pi,s)$.  We have
\begin{equation}\label{eqn:cool-integral-rep-of-I-of-Y}
  I (Y) = \int _{(\sigma)} Y^{-s} \tilde{I}(s)
  \, \frac{d s}{2 \pi i},
\end{equation}
where
\begin{equation*}
  \tilde{I}(s)
  :=H(s)
    Z(\varphi,\tfrac{1}{2} + s+ iT),
\end{equation*}
\begin{equation*}
  Z(\varphi,\tfrac{1}{2} + s) := \int _{y \in \mathbb{R}^\times / \mathbb{Z}^\times } \varphi (a (y)) \lvert y \rvert ^s \, d^\times y.
\end{equation*}
$H(s)$ decays rapidly, so we can think of it informally as truncating to $s = O(1)$.

Strategy: eventually we will bound $\tilde{I}(0) = Z(\varphi, \tfrac{1}{2} + i T)$ by applying Cauchy's theorem:
\begin{equation*}
  \tilde{I}(0)
  =
  \oint \frac{\tilde{I}(s)}{s} \, \frac{d s }{2 \pi i},
\end{equation*}
where, since $\tilde{I}$ decays rapidly, we can take the contour to consist of a vertical line at $\Re(s) = \eps$ going up followed by a vertical line at $\Re(s) = - \eps$ going down, i.e., we consider the ``box''
\begin{equation*}
  \eps - i \infty \rightarrow \eps + i \infty \rightarrow - \eps + i \infty \rightarrow - \eps - i \infty \rightarrow \eps - i \infty.
\end{equation*}
Here we will have $s \gg 1$, and also $\tilde{I}(s)$ will decay rapidly, so the main point is to bound, for $\Re(s) = \pm \eps$,
\begin{equation*}
  \tilde{I}(s) = \int _{Y \in \mathbb{R}^\times_+} Y^{-s} I(Y) \, \frac{d Y }{ Y},
\end{equation*}
which, by the triangle inequality, satisfies
\begin{equation*}
  \left\lvert \tilde{I}(s) \right\rvert \leq \int _{Y \in \mathbb{R}^\times_+}
  \max(Y,1/Y)^{\eps} \lvert I(Y) \rvert \, \frac{d Y }{Y}.
\end{equation*}

Note: the bound that we seek for $\tilde{I}(0)$ should be compared to the trivial bound following from convexity, which is
\begin{equation*}
  \tilde{I}(0) \asymp T^{-1/2} L(\pi, \tfrac{1}{2} + iT) \prec 1.
\end{equation*}


We need to bound $Z(\varphi,s)$.  We do this via interpolation.  In general,
\begin{equation*}
  Z(\varphi,\tfrac{1}{2} + s) = L(\pi, \tfrac{1}{2} + s) Z(W, \tfrac{1}{2} + s),
\end{equation*}
where $W = W_\varphi$ is as constructed above and
\begin{equation*}
  Z(W, \tfrac{1}{2} + s) = \int _{\mathbb{R}^\times } W_0(y) e(T y) \lvert y \rvert ^s \, d ^\times y.
\end{equation*}
For $\Re(s) \ll 1$, since $W$ is a bump near $1$, we have
\begin{equation*}
  Z(W,\tfrac{1}{2} + s) \approx T^{-1/2 - \Im(s)} 1_{\Im(s) \asymp T}.
\end{equation*}
So if we take $\Re(s) = 1/2 + \eps$, then we get a bound of $\ll T^{-1/2}$.  On the other hand, by the convexity bound,
\begin{equation*}
  L(\pi, \tfrac{1}{2} + s) \prec 1 \text{ for } \Re(s) = 1/2 + \eps.
\end{equation*}
So this tells us that
\begin{equation*}
  Z(\varphi, \tfrac{1}{2} + s) \prec T^{-1/2} \text{ for } \Re(s) = 1/2 + \eps.
\end{equation*}

What does this tell us concretely?  Look back at the integral representation \eqref{eqn:cool-integral-rep-of-I-of-Y}.  If we shift to $\sigma = 1/2 + \eps$, then the function $H(s)$ will truncate us to $s \ll 1$, so we can bound the integral by something like its pointwise values at $s \ll 1$, which will be
\begin{equation*}
  \prec Y^{-1/2} T^{-1/2}.
\end{equation*}
What this is saying is that if $Y$ is a bit larger than $T^{-1}$, then the ``trivial bound'' for $I(Y)$ that we just sketched is stronger than $1$.  So it suggests that the main range to consider will be when $Y \lessapprox T^{-1}$.

Now we should do the same thing but shifting in the opposite direction to find a complementary upper bound on the range of $Y$ that we need to consider.  Let's shift to $\Re(s) = -1/2 - \eps$ for small $\eps > 0$.  Then we have, for $s \ll 1$,
\begin{equation*}
  L(\pi, \tfrac{1}{2} + s + i T) \prec T,
\end{equation*}
while we get the same bound $Z(W, \tfrac{1}{2} + s + iT) \prec T^{-1/2}$ as before.  Thus 
\begin{equation*}
  Z(\varphi, \tfrac{1}{2} + s) \prec T^{1/2} \text{ for } \Re(s) = -1/2 - \eps.
\end{equation*}
Now again shifting to $\sigma = -1/2 - \eps$ in \eqref{eqn:cool-integral-rep-of-I-of-Y}, we get
\begin{equation*}
  I(Y) \prec Y ^{1/2} T ^{1/2}.
\end{equation*}
This bound will be stronger than ``$\prec 1$'' if $Y$ is a bit smaller than $T^{-1}$.

Thus, the moral is that if we just want a subconvex bound for $L(\pi,\tfrac{1}{2} + i T)$, then it suffices to nontrivially estimate $I(Y)$ for $Y \approx 1/T$, i.e., up to $T^\eps$ factors.  Of course to actually recover the Weyl bound we need to consider a wider range of $Y$ and make the analysis uniform in that.  We would have had to do the same thing in the ``classical'' approach; the corresponding feature there is that the approximate functional equation has smaller dyadic ranges than the main one, i.e., we have
\begin{equation*}
  L(\pi, \tfrac{1}{2} + i T)
  \approx \sum _{n \ll T}
  \frac{\lambda(n) }{n ^{1/2 + i T}},
\end{equation*}
which we can't altogether approximate by the contribution from $n \asymp T$.

\section{Applying relative trace formula}\label{sec:20230522180028}
So we should now, I think, study $I(Y)$, for $Y \approx 1/T$, via relative trace formula whatever stuff.  That means we should write down the double integral ($H = \GL_1 \hookrightarrow \PGL_2$)
\begin{equation*}
  \int _{
    \substack{
      x, y \in H :  \\
       x, y \asymp 1/T
    }
  }
  \lvert x/y \rvert ^{i T}
  \sum _{\gamma \in \Gamma }
  f (x ^{-1} \gamma y) \, d x \, d y.
\end{equation*}
Here $d x$ and $d y$ denote Haar measures on $H$, i.e., of the form $d t / |t|$ with respect to Lebesgue measure, so that the integral over $x$ and $y$ is roughly a probability measure.  This sum should correspond very roughly to
\begin{equation*}
  \sum _{C(\pi) \leq Q}
  T^{-1} \left\lvert L(\pi, \tfrac{1}{2} + i T) \right\rvert^2,
\end{equation*}
or at least the ``main dyadic part'' of those $L$-values.  It may be useful to write $x, y$ as multiples by $a(1/T)$ over elements in $H$ of size $\asymp 1$, so that the main thing to consider becomes
\begin{equation*}
  \int _{
    \substack{
      x, y \in H :  \\
       x, y \asymp 1
    }
  }
  \lvert x/y \rvert ^{i T}
  \sum _{\gamma \in \Gamma }
  f (a(T) x ^{-1}  \gamma y a(1/T) ) \, d x \, d y.
\end{equation*}
We want to bound this by $\ll Q/T$.

\begin{remark}
More precisely, here an expression like
\begin{equation*}
  \int_{
    \substack{
      x \in H :  \\
       x \asymp 1
    }
  }
  f(x) \, d x
\end{equation*}
means
\begin{equation*}
  \int_{x \in H \cong \mathbb{R}^\times }
  f(x) V(x) \, d x,
\end{equation*}
where $V$ lies in some fixed bounded subset of $C_c^\infty(\mathbb{R}^\times)$.  For example, we could take $V$ to be a fixed element of that space, such as a smooth bump function supported on the interval $(1,2)$.
\end{remark}


\section{Writing stuff out}\label{sec:20230522180029}
We remember that
\begin{equation*}
  f (g) = f _0 (n (- T) g n (T)).
\end{equation*}
Thus
\begin{equation*}
  f (a(T) x ^{-1}  \gamma y a(1/T) )
  =
  f_0 (n(-T) a(T) x ^{-1}  \gamma y a(1/T)  n(T)).
\end{equation*}
We can do some conjugation:
\begin{equation*}
  n(-T) a(T) x ^{-1}  \gamma y a(1/T)  n(T)
  =
  a(T) n(-1) x ^{-1}  \gamma y n(1) a(1/T).
\end{equation*}
$f_0$ should detect when this lands in $K_0(Q)$.
\begin{equation*}
  K_0(Q) =
  G \cap 
  \left( 1 + \begin{pmatrix}
    o(1) & o(1) \\
    o(1/Q) & o(1)
  \end{pmatrix}  \right),
\end{equation*}
so
\begin{equation*}
  a(1/T) K_0(Q) a(T) =
  K_0(Q) =
  G \cap 
  \left( 1 +
    \begin{pmatrix}
      o(1) & o(1/T) \\
      o(T/Q) & o(1)
    \end{pmatrix}  \right).
\end{equation*}
So the main condition to work with is now that
\begin{equation*}
  n(-1) x ^{-1} \gamma y n (1) \in
  1 +
  \begin{pmatrix}
    o(1) & o(1/T) \\
    o(T/Q) & o(1)
  \end{pmatrix} =: J.
\end{equation*}

There's the contribution from $\gamma \in \Gamma_H \cong \{\pm 1\}$.  For this, we're basically looking at
\begin{equation*}
  Q
  \int _{
    \substack{
      x \in H :  \\
       x \asymp  1
    }
  }
  1 _{n(-1) x n (1) \in J} \, d x.
\end{equation*}
We have
\begin{equation*}
  n (-1) x n (1)
  =
  \begin{pmatrix}
    x & x-1 \\
    0 & 1
  \end{pmatrix}.
\end{equation*}
This lies in $J$ only if $x = 1 + o(1/T)$, which happens with probability $\lll T$, so we get the required bound $Q/T$.

It remains to estimate the contribution of the off-diagonal:
\begin{equation*}
  Q
  \sum _{\gamma \in \Gamma - \Gamma_H}
  \int _{
    \substack{
      x, y \in H :  \\
       x, y \asymp 1
    }
  }
  \lvert x/y \rvert^{i T}
  1 _{n (- 1 ) x ^{-1} \gamma y n (1) \in J} \, d x \, d y.
\end{equation*}
We'll see below that we're in a range where it's not possible to extract oscillation from the integrals over $x$ and $y$.

\section{Matrices}\label{sec:20230522180034}
Thus
\begin{equation*}
  f (a(T) x ^{-1}  \gamma y a(1/T) )
  =
  f_0 (n(-T) a(T) x ^{-1}  \gamma y a(1/T)  n(T)).
\end{equation*}
We can do some conjugation:
\begin{equation*}
  n(-T) a(T) x ^{-1}  \gamma y a(1/T)  n(T)
  =
  a(T) n(-1) x ^{-1}  \gamma y n(1) a(1/T).
\end{equation*}
$f_0$ should detect when this lands in $K_0(Q)$.
\begin{equation*}
  K_0(Q) =
  G \cap 
  \left( 1 + \begin{pmatrix}
    o(1) & o(1) \\
    o(1/Q) & o(1)
  \end{pmatrix}  \right),
\end{equation*}
so
\begin{equation*}
  a(1/T) K_0(Q) a(T)
  =
  G \cap 
  \left( 1 +
    \begin{pmatrix}
      o(1) & o(1/T) \\
      o(T/Q) & o(1)
    \end{pmatrix}  \right).
\end{equation*}
So the main condition to work with is now that
\begin{equation*}
  n(-1) x ^{-1} \gamma y n (1) \in
  1 +
  \begin{pmatrix}
    o(1) & o(1/T) \\
    o(T/Q) & o(1)
  \end{pmatrix} =: J.
\end{equation*}

There's the contribution from $\gamma \in \Gamma_H \cong \{\pm 1\}$.  For this, we're basically looking at
\begin{equation*}
  Q
  \int _{
    \substack{
      x \in H :  \\
       x \asymp  1
    }
  }
  1 _{n(-1) x n (1) \in J} \, d x.
\end{equation*}
We have
\begin{equation*}
  n (-1) x n (1)
  =
  \begin{pmatrix}
    x & x-1 \\
    0 & 1
  \end{pmatrix}.
\end{equation*}
This lies in $J$ only if $x = 1 + o(1/T)$, which happens with probability $\lll T$, so we get the required bound $Q/T$.

It remains to estimate the contribution of the off-diagonal:
\begin{equation*}
  Q
  \sum _{\gamma \in \Gamma - \Gamma_H}
  \int _{
    \substack{
      x, y \in H :  \\
       x, y \asymp 1
    }
  }
  \lvert x / y \rvert^{i T}
  1 _{n (- 1 ) x ^{-1} \gamma y n (1) \in J} \, d x \, d y.
\end{equation*}
We want to bound this by $\ll Q/T$?  The convexity bound for $\lvert L \rvert^2$ is $\ll T$, so we need to bound the sum by $\lll 1$ to improve upon convexity.  So we really need to show
\begin{equation*}
  \sum _{\gamma \in \Gamma - \Gamma_H}
  \int _{
    \substack{
      x, y \in H :  \\
       x, y \asymp 1
    }
  }
  \lvert x / y \rvert^{i T}
  1 _{n (- 1 ) x ^{-1} \gamma y n (1) \in J} \, d x \, d y \lll 1/Q
\end{equation*}
but we might hope to be able to show (for certain ranges of $Q$)
\begin{equation*}
  \sum _{\gamma \in \Gamma - \Gamma_H}
  \int _{
    \substack{
      x, y \in H :  \\
       x, y \asymp 1
    }
  }
  \lvert x / y \rvert^{i T}
  1 _{n (- 1 ) x ^{-1} \gamma y n (1) \in J} \, d x \, d y \ll 1/T.
\end{equation*}
We recall that, writing
\begin{equation*}
  \gamma =
  \begin{pmatrix}
    a & b \\
    c & d
  \end{pmatrix},
\end{equation*}
we have
\begin{equation*}
  x ^{-1} \gamma y =
  \begin{pmatrix}
    a y/x & b/x \\
    c y & d
  \end{pmatrix},
\end{equation*}
hence
\begin{equation*}
  n (-1) x ^{-1} \gamma y n (1)
  =
  \begin{pmatrix}
    a y/x - c y & b/x - d + a y / x - c y \\
    c y & d + c y
  \end{pmatrix}.
\end{equation*}
We arrive at the following conditions:
\begin{enumerate}[(i)]
\item\label{enumerate:20230522174144} $c y = o(T/Q)$, or equivalently, $c  \lll T/Q$ (because $y \asymp 1$),
\item\label{enumerate:20230522174157} $ay / x - c y = 1 + o(1)$, which determines $a$ up to $o(1)$ if we know $(x,y,c)$,
\item\label{enumerate:20230522174208} $d + c y = 1 + o(1)$, which determines $d$ up to $o(1)$ if we know $(x,y,c)$,
\item $b /x - d + a y / x - c y = o(1/T)$, which should be satisfied about a proportion $1/T$ of the time.
\end{enumerate}
So that would lead to an overall bound of $1 / Q$.  We need to do a bit better than that.  Seems tough!




\bibliography{refs}{} \bibliographystyle{plain}
\end{document}
