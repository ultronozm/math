\documentclass[reqno]{amsart} \usepackage{graphicx, amsmath, amssymb, amsfonts, amsthm, stmaryrd, amscd}
\usepackage[usenames, dvipsnames]{xcolor}
\usepackage{tikz}
% \usepackage{tikzcd}
% \usepackage{comment}

% \let\counterwithout\relax
% \let\counterwithin\relax
% \usepackage{chngcntr}

\usepackage{enumerate}
% \usepackage{enumitem}
% \usepackage{times}
\usepackage[normalem]{ulem}
% \usepackage{minted}
% \usepackage{xypic}
% \usepackage{color}


% \usepackage{silence}
% \WarningFilter{latex}{Label `tocindent-1' multiply defined}
% \WarningFilter{latex}{Label `tocindent0' multiply defined}
% \WarningFilter{latex}{Label `tocindent1' multiply defined}
% \WarningFilter{latex}{Label `tocindent2' multiply defined}
% \WarningFilter{latex}{Label `tocindent3' multiply defined}
\usepackage{hyperref}
% \usepackage{navigator}


% \usepackage{pdfsync}
\usepackage{xparse}


\usepackage[all]{xy}
\usepackage{enumerate}
\usetikzlibrary{matrix,arrows,decorations.pathmorphing}



\makeatletter
\newcommand*{\transpose}{%
  {\mathpalette\@transpose{}}%
}
\newcommand*{\@transpose}[2]{%
  % #1: math style
  % #2: unused
  \raisebox{\depth}{$\m@th#1\intercal$}%
}
\makeatother


\makeatletter
\newcommand*{\da@rightarrow}{\mathchar"0\hexnumber@\symAMSa 4B }
\newcommand*{\da@leftarrow}{\mathchar"0\hexnumber@\symAMSa 4C }
\newcommand*{\xdashrightarrow}[2][]{%
  \mathrel{%
    \mathpalette{\da@xarrow{#1}{#2}{}\da@rightarrow{\,}{}}{}%
  }%
}
\newcommand{\xdashleftarrow}[2][]{%
  \mathrel{%
    \mathpalette{\da@xarrow{#1}{#2}\da@leftarrow{}{}{\,}}{}%
  }%
}
\newcommand*{\da@xarrow}[7]{%
  % #1: below
  % #2: above
  % #3: arrow left
  % #4: arrow right
  % #5: space left 
  % #6: space right
  % #7: math style 
  \sbox0{$\ifx#7\scriptstyle\scriptscriptstyle\else\scriptstyle\fi#5#1#6\m@th$}%
  \sbox2{$\ifx#7\scriptstyle\scriptscriptstyle\else\scriptstyle\fi#5#2#6\m@th$}%
  \sbox4{$#7\dabar@\m@th$}%
  \dimen@=\wd0 %
  \ifdim\wd2 >\dimen@
    \dimen@=\wd2 %   
  \fi
  \count@=2 %
  \def\da@bars{\dabar@\dabar@}%
  \@whiledim\count@\wd4<\dimen@\do{%
    \advance\count@\@ne
    \expandafter\def\expandafter\da@bars\expandafter{%
      \da@bars
      \dabar@ 
    }%
  }%  
  \mathrel{#3}%
  \mathrel{%   
    \mathop{\da@bars}\limits
    \ifx\\#1\\%
    \else
      _{\copy0}%
    \fi
    \ifx\\#2\\%
    \else
      ^{\copy2}%
    \fi
  }%   
  \mathrel{#4}%
}
\makeatother
% \DeclareMathOperator{\rg}{rg}

\usepackage{mathtools}
\DeclarePairedDelimiter{\paren}{(}{)}
\DeclarePairedDelimiter{\abs}{\lvert}{\rvert}
\DeclarePairedDelimiter{\norm}{\lVert}{\rVert}
\DeclarePairedDelimiter{\innerproduct}{\langle}{\rangle}
\newcommand{\Of}[2]{{\operatorname{#1}} {\paren*{#2}}}
\newcommand{\of}[2]{{{{#1}} {\paren*{#2}}}}

\DeclareMathOperator{\Shim}{Shim}
\DeclareMathOperator{\sgn}{sgn}
\DeclareMathOperator{\fdeg}{fdeg}
\DeclareMathOperator{\SL}{SL}
\DeclareMathOperator{\slLie}{\mathfrak{s}\mathfrak{l}}
\DeclareMathOperator{\soLie}{\mathfrak{s}\mathfrak{o}}
\DeclareMathOperator{\spLie}{\mathfrak{s}\mathfrak{p}}
\DeclareMathOperator{\glLie}{\mathfrak{g}\mathfrak{l}}
\newcommand{\pn}[1]{{\color{ForestGreen} \sf PN: [#1]}}
\DeclareMathOperator{\Mp}{Mp}
\DeclareMathOperator{\Mat}{Mat}
\DeclareMathOperator{\GL}{GL}
\DeclareMathOperator{\Gr}{Gr}
\DeclareMathOperator{\GU}{GU}
\def\gl{\mathfrak{g}\mathfrak{l}}
\DeclareMathOperator{\odd}{odd}
\DeclareMathOperator{\even}{even}
\DeclareMathOperator{\GO}{GO}
\DeclareMathOperator{\good}{good}
\DeclareMathOperator{\bad}{bad}
\DeclareMathOperator{\PGO}{PGO}
\DeclareMathOperator{\htt}{ht}
\DeclareMathOperator{\height}{height}
\DeclareMathOperator{\Ass}{Ass}
\DeclareMathOperator{\coheight}{coheight}
\DeclareMathOperator{\GSO}{GSO}
\DeclareMathOperator{\SO}{SO}
\DeclareMathOperator{\so}{\mathfrak{s}\mathfrak{o}}
\DeclareMathOperator{\su}{\mathfrak{s}\mathfrak{u}}
\DeclareMathOperator{\ad}{ad}
% \DeclareMathOperator{\sc}{sc}
\DeclareMathOperator{\Ad}{Ad}
\DeclareMathOperator{\disc}{disc}
\DeclareMathOperator{\inv}{inv}
\DeclareMathOperator{\Pic}{Pic}
\DeclareMathOperator{\uc}{uc}
\DeclareMathOperator{\Cl}{Cl}
\DeclareMathOperator{\Clf}{Clf}
\DeclareMathOperator{\Hom}{Hom}
\DeclareMathOperator{\hol}{hol}
\DeclareMathOperator{\Heis}{Heis}
\DeclareMathOperator{\Haar}{Haar}
\DeclareMathOperator{\h}{h}
\def\sp{\mathfrak{s}\mathfrak{p}}
\DeclareMathOperator{\heis}{\mathfrak{h}\mathfrak{e}\mathfrak{i}\mathfrak{s}}
\DeclareMathOperator{\End}{End}
\DeclareMathOperator{\JL}{JL}
\DeclareMathOperator{\image}{image}
\DeclareMathOperator{\red}{red}
\def\div{\operatorname{div}}
\def\eps{\varepsilon}
\def\cHom{\mathcal{H}\operatorname{om}}
\DeclareMathOperator{\Ops}{Ops}
\DeclareMathOperator{\Symb}{Symb}
\def\boldGL{\mathbf{G}\mathbf{L}}
\def\boldSO{\mathbf{S}\mathbf{O}}
\def\boldU{\mathbf{U}}
\DeclareMathOperator{\hull}{hull}
\DeclareMathOperator{\LL}{LL}
\DeclareMathOperator{\PGL}{PGL}
\DeclareMathOperator{\class}{class}
\DeclareMathOperator{\lcm}{lcm}
\DeclareMathOperator{\spann}{span}
\DeclareMathOperator{\Exp}{Exp}
\DeclareMathOperator{\ext}{ext}
\DeclareMathOperator{\Ext}{Ext}
\DeclareMathOperator{\Tor}{Tor}
\DeclareMathOperator{\et}{et}
\DeclareMathOperator{\tor}{tor}
\DeclareMathOperator{\loc}{loc}
\DeclareMathOperator{\tors}{tors}
\DeclareMathOperator{\pf}{pf}
\DeclareMathOperator{\smooth}{smooth}
\DeclareMathOperator{\prin}{prin}
\DeclareMathOperator{\Kl}{Kl}
\newcommand{\kbar}{\mathchar'26\mkern-9mu k}
\DeclareMathOperator{\der}{der}
% \DeclareMathOperator{\abs}{abs}
\DeclareMathOperator{\Sub}{Sub}
\DeclareMathOperator{\Comp}{Comp}
\DeclareMathOperator{\Err}{Err}
\DeclareMathOperator{\dom}{dom}
\DeclareMathOperator{\radius}{radius}
\DeclareMathOperator{\Fitt}{Fitt}
\DeclareMathOperator{\Sel}{Sel}
\DeclareMathOperator{\rad}{rad}
\DeclareMathOperator{\id}{id}
\DeclareMathOperator{\Center}{Center}
\DeclareMathOperator{\Der}{Der}
\DeclareMathOperator{\U}{U}
% \DeclareMathOperator{\norm}{norm}
\DeclareMathOperator{\trace}{trace}
\DeclareMathOperator{\Equid}{Equid}
\DeclareMathOperator{\Feas}{Feas}
\DeclareMathOperator{\bulk}{bulk}
\DeclareMathOperator{\tail}{tail}
\DeclareMathOperator{\sys}{sys}
\DeclareMathOperator{\atan}{atan}
\DeclareMathOperator{\temp}{temp}
\DeclareMathOperator{\Asai}{Asai}
\DeclareMathOperator{\glob}{glob}
\DeclareMathOperator{\Kuz}{Kuz}
\DeclareMathOperator{\Irr}{Irr}
\newcommand{\rsL}{ \frac{ L^{(R)}(\Pi \times \Sigma, \std, \frac{1}{2})}{L^{(R)}(\Pi \times \Sigma, \Ad, 1)}  }
\DeclareMathOperator{\GSp}{GSp}
\DeclareMathOperator{\PGSp}{PGSp}
\DeclareMathOperator{\BC}{BC}
\DeclareMathOperator{\Ann}{Ann}
\DeclareMathOperator{\Gen}{Gen}
\DeclareMathOperator{\SU}{SU}
\DeclareMathOperator{\PGSU}{PGSU}
% \DeclareMathOperator{\gen}{gen}
\DeclareMathOperator{\PMp}{PMp}
\DeclareMathOperator{\PGMp}{PGMp}
\DeclareMathOperator{\PB}{PB}
\DeclareMathOperator{\ind}{ind}
\DeclareMathOperator{\Jac}{Jac}
\DeclareMathOperator{\jac}{jac}
\DeclareMathOperator{\im}{im}
\DeclareMathOperator{\Aut}{Aut}
\DeclareMathOperator{\Int}{Int}
\DeclareMathOperator{\PSL}{PSL}
\DeclareMathOperator{\co}{co}
\DeclareMathOperator{\irr}{irr}
\DeclareMathOperator{\prim}{prim}
\DeclareMathOperator{\bal}{bal}
\DeclareMathOperator{\baln}{bal}
\DeclareMathOperator{\dist}{dist}
\DeclareMathOperator{\RS}{RS}
\DeclareMathOperator{\Ram}{Ram}
\DeclareMathOperator{\Sob}{Sob}
\DeclareMathOperator{\Sol}{Sol}
\DeclareMathOperator{\soc}{soc}
\DeclareMathOperator{\nt}{nt}
\DeclareMathOperator{\mic}{mic}
\DeclareMathOperator{\Gal}{Gal}
\DeclareMathOperator{\st}{st}
\DeclareMathOperator{\std}{std}
\DeclareMathOperator{\diag}{diag}
\DeclareMathOperator{\Sym}{Sym}
\DeclareMathOperator{\gr}{gr}
\DeclareMathOperator{\aff}{aff}
\DeclareMathOperator{\Dil}{Dil}
\DeclareMathOperator{\Lie}{Lie}
\DeclareMathOperator{\Symp}{Symp}
\DeclareMathOperator{\Stab}{Stab}
\DeclareMathOperator{\St}{St}
\DeclareMathOperator{\stab}{stab}
\DeclareMathOperator{\codim}{codim}
\DeclareMathOperator{\linear}{linear}
\newcommand{\git}{/\!\!/}
\DeclareMathOperator{\geom}{geom}
\DeclareMathOperator{\spec}{spec}
\def\O{\operatorname{O}}
\DeclareMathOperator{\Au}{Aut}
\DeclareMathOperator{\Fix}{Fix}
\DeclareMathOperator{\Opp}{Op}
\DeclareMathOperator{\opp}{op}
\DeclareMathOperator{\Size}{Size}
\DeclareMathOperator{\Save}{Save}
% \DeclareMathOperator{\ker}{ker}
\DeclareMathOperator{\coker}{coker}
\DeclareMathOperator{\sym}{sym}
\DeclareMathOperator{\mean}{mean}
\DeclareMathOperator{\elliptic}{ell}
\DeclareMathOperator{\nilpotent}{nil}
\DeclareMathOperator{\hyperbolic}{hyp}
\DeclareMathOperator{\newvector}{new}
\DeclareMathOperator{\new}{new}
\DeclareMathOperator{\full}{full}
\newcommand{\qr}[2]{\left( \frac{#1}{#2} \right)}
\DeclareMathOperator{\unr}{u}
\DeclareMathOperator{\ram}{ram}
% \DeclareMathOperator{\len}{len}
\DeclareMathOperator{\fin}{fin}
\DeclareMathOperator{\cusp}{cusp}
\DeclareMathOperator{\curv}{curv}
\DeclareMathOperator{\rank}{rank}
\DeclareMathOperator{\rk}{rk}
\DeclareMathOperator{\pr}{pr}
\DeclareMathOperator{\Transform}{Transform}
\DeclareMathOperator{\mult}{mult}
\DeclareMathOperator{\Eis}{Eis}
\DeclareMathOperator{\reg}{reg}
\DeclareMathOperator{\sing}{sing}
\DeclareMathOperator{\alt}{alt}
\DeclareMathOperator{\irreg}{irreg}
\DeclareMathOperator{\sreg}{sreg}
\DeclareMathOperator{\Wd}{Wd}
\DeclareMathOperator{\Weil}{Weil}
\DeclareMathOperator{\Th}{Th}
\DeclareMathOperator{\Sp}{Sp}
\DeclareMathOperator{\Ind}{Ind}
\DeclareMathOperator{\Res}{Res}
\DeclareMathOperator{\ini}{in}
\DeclareMathOperator{\ord}{ord}
\DeclareMathOperator{\osc}{osc}
\DeclareMathOperator{\fluc}{fluc}
\DeclareMathOperator{\size}{size}
\DeclareMathOperator{\ann}{ann}
\DeclareMathOperator{\equ}{eq}
\DeclareMathOperator{\res}{res}
\DeclareMathOperator{\pt}{pt}
\DeclareMathOperator{\src}{source}
\DeclareMathOperator{\Zcl}{Zcl}
\DeclareMathOperator{\Func}{Func}
\DeclareMathOperator{\Map}{Map}
\DeclareMathOperator{\Frac}{Frac}
\DeclareMathOperator{\Frob}{Frob}
\DeclareMathOperator{\ev}{eval}
\DeclareMathOperator{\pv}{pv}
\DeclareMathOperator{\eval}{eval}
\DeclareMathOperator{\Spec}{Spec}
\DeclareMathOperator{\Speh}{Speh}
\DeclareMathOperator{\Spin}{Spin}
\DeclareMathOperator{\GSpin}{GSpin}
\DeclareMathOperator{\Specm}{Specm}
\DeclareMathOperator{\Sphere}{Sphere}
\DeclareMathOperator{\Sqq}{Sq}
\DeclareMathOperator{\Ball}{Ball}
\DeclareMathOperator\Cond{\operatorname{Cond}}
\DeclareMathOperator\proj{\operatorname{proj}}
\DeclareMathOperator\Swan{\operatorname{Swan}}
\DeclareMathOperator{\Proj}{Proj}
\DeclareMathOperator{\bPB}{{\mathbf P}{\mathbf B}}
\DeclareMathOperator{\Projm}{Projm}
\DeclareMathOperator{\Tr}{Tr}
\DeclareMathOperator{\Type}{Type}
\DeclareMathOperator{\Prop}{Prop}
\DeclareMathOperator{\vol}{vol}
\DeclareMathOperator{\covol}{covol}
\DeclareMathOperator{\Rep}{Rep}
\DeclareMathOperator{\Cent}{Cent}
\DeclareMathOperator{\val}{val}
\DeclareMathOperator{\area}{area}
\DeclareMathOperator{\nr}{nr}
\DeclareMathOperator{\CM}{CM}
\DeclareMathOperator{\CH}{CH}
\DeclareMathOperator{\tr}{tr}
\DeclareMathOperator{\characteristic}{char}
\DeclareMathOperator{\supp}{supp}


\theoremstyle{plain} \newtheorem{theorem} {Theorem} \newtheorem{conjecture} [theorem] {Conjecture} \newtheorem{corollary} [theorem] {Corollary} \newtheorem{proposition} [theorem] {Proposition} \newtheorem{fact} [theorem] {Fact}
\theoremstyle{definition} \newtheorem{definition} [theorem] {Definition} \newtheorem{hypothesis} [theorem] {Hypothesis} \newtheorem{assumptions} [theorem] {Assumptions}
\newtheorem{example} [theorem] {Example}
\newtheorem{assertion}[theorem] {Assertion}
\newtheorem{note}[theorem] {Note}
\newtheorem{conclusion}[theorem] {Conclusion}
\newtheorem{claim}            {Claim}
\newtheorem{homework} {Homework}
\newtheorem{exercise} {Exercise}  \newtheorem{question}[theorem] {Question}    \newtheorem{answer} {Answer}  \newtheorem{problem} {Problem}    \newtheorem{remark} [theorem] {Remark}
\newtheorem{notation} [theorem]           {Notation}
\newtheorem{terminology}[theorem]            {Terminology}
\newtheorem{convention}[theorem]            {Convention}
\newtheorem{motivation}[theorem]            {Motivation}


\newtheoremstyle{itplain} % name
{6pt}                    % Space above
{5pt\topsep}                    % Space below
{\itshape}                   % Body font
{}                           % Indent amount
{\itshape}                   % Theorem head font
{.}                          % Punctuation after theorem head
{5pt plus 1pt minus 1pt}                       % Space after theorem head
% {.5em}                       % Space after theorem head
{}  % Theorem head spec (can be left empty, meaning ‘normal’)

% \theoremstyle{mytheoremstyle}


\theoremstyle{itplain} %--default
% \theoremheaderfont{\itshape}
% \newtheorem{lemma}{Lemma}
\newtheorem{lemma}[theorem]{Lemma}
% \newtheorem{lemma}{Lemma}[subsubsection]

\newtheorem*{lemma*}{Lemma}
\newtheorem*{proposition*}{Proposition}
\newtheorem*{definition*}{Definition}
\newtheorem*{example*}{Example}

\newtheorem*{results*}{Results}
\newtheorem{results} [theorem] {Results}


\usepackage[displaymath,textmath,sections,graphics]{preview}
\PreviewEnvironment{align*}
\PreviewEnvironment{multline*}
\PreviewEnvironment{tabular}
\PreviewEnvironment{verbatim}
\PreviewEnvironment{lstlisting}
\PreviewEnvironment*{frame}
\PreviewEnvironment*{alert}
\PreviewEnvironment*{emph}
\PreviewEnvironment*{textbf}

 \numberwithin{theorem}{section} \numberwithin{equation}{section}

\begin{document}

These are notes for an ongoing Fall 2023 course on the Riemann zeta function and its generalizations, $L$-functions.  These notes will be filled in as we go.

\newpage
\section{Background}

\subsection{General notation}
$\mathbb{R}^+ := (0,\infty)$.

\subsection{Asymptotic notation}
We use the equivalent notations
\begin{equation*}
  A = \O(B), \qquad A \ll B,
  \qquad B \gg A
\end{equation*}
to denote that
\begin{equation*}
\lvert A \rvert \leq C \lvert B \rvert
\end{equation*}
for some ``constant'' $C$.  The precise meaning of ``constant'' will either be specified or clear from context.

\subsection{Holomorphic continuation}

\begin{theorem}[Identity principle for holomorphic functions]
  Let $U \subset \mathbb{C} $ be a connected open set.  Let $f, g : U \rightarrow \mathbb{C} $ be holomorphic functions.  If $f = g$ on a set with a limit point in $U$, then $f = g$ on all of $U$.
\end{theorem}
\begin{corollary}\label{corollary:cj3vqbthht}
  Let $U \subset \Omega \subseteq \mathbb{C} $ be open subsets, with $U$ nonempty and $\Omega$ connected.  Let $f : U \rightarrow \mathbb{C}$ be a holomorphic function.  Then there is at most one extension of $f$ to a holomorphic function $\Omega \rightarrow \mathbb{C}$.
\end{corollary}

\subsection{Cauchy's integral formula}
\begin{theorem}
  Let $f : U \rightarrow \mathbb{C} $ be a holomorphic function defined on an open subset $U$.  Let $\gamma$ be a closed rectifiable curve in $U$.  Then $\int_\gamma f(z) \, d z = 0$.
\end{theorem}

\begin{theorem}\label{theorem:cj3vqbjd26}
  Let $0 \leq a < b \leq \infty$.  Let $f(z)$ be a holomorphic function on the annulus $\{z \in \mathbb{C} : a < \lvert z \rvert < b\}$ given by a convergent Laurent series
  \begin{equation*}
    f(z) = \sum_{n \in \mathbb{Z} } c_n z^n.
  \end{equation*}
  \begin{enumerate}
  \item For any $r \in (a,b)$ and $n \in \mathbb{Z}$, we have
    \begin{align*}
      c_n &=  \oint_{\lvert z \rvert = r} \frac{f(z)}{z^{n}} \, \frac{d z}{2 \pi i z} \\
          &= \frac{1}{2 \pi r^n } \int_{\theta = 0 }^{2 \pi } f (r e^{i \theta }) e^{- i n \theta } \,d \theta.
    \end{align*}
  \item For each compact subset $E$ of $(a,b)$, there exists $M \geq 0$ so that for all $r \in E$, we have
    \begin{equation}\label{eq:cj3vqbiupy}
      \sum_{n \in \mathbb{Z}} \lvert c_n \rvert r^n \leq M.
    \end{equation}
  \end{enumerate}
\end{theorem}

\begin{theorem}\label{theorem:cj3wnb89dd}
  Let $U$ be an open subset of $\mathbb{C}$, let $f : U \rightarrow \mathbb{C} $ be meromorphic.  Let $\gamma$ be a smooth closed curve in $U$, oriented counterclockwise, that does not pass through any pole of $f$.  Then
  \begin{equation*}
    \int_\gamma f(z) \, d z = 2 \pi i \sum_{\substack{z \in \operatorname{interior}(\gamma) \\ \text{pole of $f$}}} \operatorname{res}_z(f).
\end{equation*}
\end{theorem}
\begin{remark}
  Let $0 < r < R$.  Let $f$ be a meromorphic function on a neighborhood of the annulus $\{z : r < |z| < R\}$ that has no poles on either of the circles $|z| = r, R$.  Then
  \begin{equation*}
    \oint_{|z| = R} f(z) \, d z
    =
    \oint_{|z| = r} f(z) \, d z
    + 2 \pi i \sum_{
      \substack{
        r < |z| < R 
        \\
        \text{pole of $f$}
      }
    }
    \res_z(f).
\end{equation*}
\end{remark}

\subsection{Holomorphy of limits and series}
\begin{theorem}\label{theorem:cj3vqa91ti}
  Let $U$ be an open subset of the complex plane.  Let $f_n$ be a sequence of holomorphic functions on $U$.
  \begin{enumerate}
  \item Suppose that the sequence $f_n$ converges pointwise to some function $f$, uniformly on compact subsets of $U$.  Then $f$ is holomorphic.
  \item Suppose that the partial sums $\sum_{n \leq N} f_n$ converge pointwise to some function $f$, uniformly on compact subsets of $U$.  Then the sum $\sum_n f_n$ is holomorphic.
  \end{enumerate}
\end{theorem}

\newpage
\section{Asymptotics and meromorphic continuation}

Reference: generatingfunctionology, section 5.2.


We consider a Laurent series
\begin{equation*}
f (z) = \sum_{n \in \mathbb{Z} } c_n z^n.
\end{equation*}
Here the $c_n$ are complex coefficients, while $z$ is a nonzero complex argument.  We assume that this series converges absolutely for at least one value of $z$.

\begin{lemma}\label{lemma:cj3vqafpa6}
  There is a unique maximal open subinterval $(a,b)$ of $\mathbb{R}^+$ on which $f$ converges absolutely.  Its endpoints are given explicitly by
  \begin{equation*}
a = \inf \left\{ r \in \mathbb{R}^+ : \sum_n \lvert c_n \rvert r^n < \infty  \right\},
\end{equation*}

\begin{equation*}
 b = \sup \left\{ r \in \mathbb{R}^+ : \sum_n \lvert c_n \rvert r^n < \infty  \right\}.
\end{equation*}
\end{lemma}
We refer to the interval $(a,b)$ as the \emph{fundamental interval} for $f$ (or for the $c_n$).

The fundamental interval controls the growth of the coefficients $c_n$ as $n \rightarrow \pm \infty$:
\begin{lemma}
  Let $b^- < b$ and $a^+ > a$.  Then
  \begin{equation*}
    c_n \ll {(b^-)}^{-n} \quad \text{ as } n \rightarrow \infty
  \end{equation*}
  and
  \begin{equation*}
    c_n \ll {(a^+)}^{-n} \quad \text{ as } n \rightarrow -\infty.
  \end{equation*}
\end{lemma}

Set
\begin{equation*}
\mathcal{C} (a, b) := \left\{ z \in \mathbb{C} : \lvert z  \rvert \in (a,b) \right\}.
\end{equation*}

\begin{lemma}
$f(z)$ defines a holomorphic function on $\mathcal{C}(a,b)$.
\end{lemma}
\begin{proof}
  Follows from Theorem~\ref{theorem:cj3vqa91ti}.
\end{proof}

\begin{lemma}\label{lemma:cj3vqbs30d}
$f$ does not extend to a holomorphic function on $\mathcal{C}(A,B)$ for any strictly larger interval $(A,B) \supsetneq (a,b)$.
\end{lemma}
\begin{proof}
  Suppose otherwise.  Let $r \in (A,B) - (a,b)$.  Then by Cauchy's integral formula (specifically, the estimate~\eqref{eq:cj3vqbiupy} of Theorem~\ref{theorem:cj3vqbjd26}), we see that $\sum_{n \in \mathbb{Z}} \lvert c_n \rvert r^n < \infty$.  This contradicts the formula for $a$ and $b$ given in Lemma~\ref{lemma:cj3vqafpa6}.
\end{proof}

\begin{note}
  It can happen that $f$ extends to a \emph{meromorphic} function on some strictly larger annulus (unique, in view of Corollary~\ref{corollary:cj3vqbthht}).  By Lemma~\ref{lemma:cj3vqbs30d}, this can only happen if $f$ has a pole at some point on the boundary of the fundamental annulus.
\end{note}

\begin{example}
  Take
  \begin{equation*}
    c_n =
    \begin{cases}
      2^n & \text{ if } n \geq 0, \\
      0 & \text{ if } n < 0.
    \end{cases}
  \end{equation*}
  Then the fundamental interval is $(a, b) = (0, 1/2)$.  However, the function $f(z)$, defined initially for $\lvert z \rvert < 1/2$, evaluates to a rational function:
  \begin{equation*}
    f(z) = \sum_{n \geq 0} 2^n z^n
    = \frac{1}{1 - 2 z}.
  \end{equation*}
  This is meromorphic on the entire complex plane; the only pole is a simple one at $z = 1/2$, with residue $-1/2$.
\end{example}

The possibility of meromorphically extending $f$ corresponds to the coefficients $c_n$ having asymptotic expansions as $n \rightarrow \pm \infty$.  For example:
\begin{lemma}[Meromorphic continuation vs.\ asymptotic expansion, special case]\label{lemma:cj3vqfrerl}
  Let $f$ and $(a,b)$ be as above.  Let $\beta \in \mathbb{C}$ with $\lvert \beta \rvert = b$.  Let $B > b$ and $\gamma \in \mathbb{C} $.  Then the following are equivalent:
  \begin{enumerate}[(i)]
  \item\label{enumerate:cj3vqef009} $f$ extends to a meromorphic function on $\mathcal{C}(a, B)$ with a unique simple pole at $z = \beta$ with residue $\gamma$.
  \item\label{enumerate:cj3vqef2m9} For each $B^- < B$, we have as $n \rightarrow \infty$ that
    \begin{equation}\label{eq:cj3wnbmxaw}
      c_n = - \gamma \beta ^{-n-1} + \O \left( {(B^-)}^{-n} \right),
    \end{equation}
    
  \end{enumerate}
\end{lemma}
\begin{proof}
  To see that~\eqref{enumerate:cj3vqef009} implies~\eqref{enumerate:cj3vqef2m9}, we start with Cauchy's integral formula on the disc of radius $b^-$ for some $b^- \in (a,b)$, then shift the contour, picking up the contribution of the unique pole:
  \begin{align}
    c_n
    &= \oint_{\lvert z \rvert = b^-} \frac{f(z)}{z^{n}} \, \frac{d z}{2 \pi i z} \nonumber
    \\
    &= \oint_{\lvert z \rvert = B^-} \frac{f(z)}{z^{n}} \, \frac{d z}{2 \pi i z} \label{align:cj3vqg2jvd}
      - \frac{\gamma}{\beta^{n+1}}.
  \end{align}
  We then estimate this last integral using that $f$ is bounded on compact sets.

  Conversely, to verify that~\eqref{enumerate:cj3vqef2m9} implies~\eqref{enumerate:cj3vqef009}, we define the coefficients
  \begin{equation*}
    b_n :=
    \begin{cases}
      - \gamma \beta^{- n - 1 } &  \text{ if } n \geq 0, \\
      0 & \text{ if } n < 0,
    \end{cases}
  \end{equation*}
  The corresponding series
  \begin{equation*}
    f_+(z) := \sum_{n \in \mathbb{Z} } b_n z^n
  \end{equation*}
  may be evaluated explicitly: a simple geometric series calculation, left to the reader, gives
  \begin{equation*}
    f_+(z) = \frac{\gamma}{z - \beta }.
  \end{equation*}
  Our hypothesis concerning the $c_n$ reads
  \begin{equation}\label{eq:cj3vqey6o9}
    b_n - c_n = \O \left( {(B^-)}^{-n} \right) \quad \text{ as } n \rightarrow \infty.
  \end{equation}
  On the other hand, because $f$ has fundamental interval $(a,b)$ and $b_n$ vanishes as $n \rightarrow -\infty$, we have for each $a^+ > a$ that
  \begin{equation}\label{eq:cj3vqey9fd}
    b_n - c_n = \O \left( {(a^+)}^{-n} \right) \quad \text{ as } n \rightarrow -\infty.
  \end{equation}
  From~\eqref{eq:cj3vqey6o9} and~\eqref{eq:cj3vqey9fd}, we deduce that the series $f - f_+$ with coefficients $c_n - b_n$ has fundamental interval containing $(a,B)$.  This implies that the function
  \begin{equation*}
    f(z) - \frac{\gamma }{z - \beta },
  \end{equation*}
defined initially as a holomorphic function on $\mathcal{C}(a,b)$, extends to a holomorphic function on $\mathcal{C}(a,B)$.  Equivalently, $f$ extends to a meromorphic function on $\mathcal{C}(a,B)$ with polar behavior as described in~\eqref{enumerate:cj3vqef2m9}.
\end{proof}

\begin{example}
  Suppose that
  \begin{equation*}
    c_n =
    \begin{cases}
      \beta^{- n} & \text{ if } n \geq 0, \\
      0 & \text{ if } n < 0,
    \end{cases}
  \end{equation*}
  so that, initially for $\lvert z \rvert < |\beta|$,
  \begin{equation*}
    f (z) = \sum_{n \geq 0} \beta^{-n} z^n = \frac{1}{1 - z / \beta }.
  \end{equation*}
  The function $f$ extends meromorphically, having a simple pole at $z = \beta$ with residue $-\beta$.  The sequence $c_n$ has the asymptotic behavior indicated in~\eqref{eq:cj3wnbmxaw}, in a very strong sense: the sequence is \emph{equal} to the asymptotic.
\end{example}

\begin{exercise}
  Generalize the above lemma to describe in terms of the coefficients $c_n$ what it means for $f$ to extend to a meromorphic function on $\mathcal{C}(A,B)$ for some $A < a$ and $B > b$, allowing the possibility of multiple poles of arbitrary order.
\end{exercise}

\begin{exercise}
  Let $c_n$ denote the Fibonacci sequence, thus $c_n = 0$ for $n < 0$ and
  \begin{equation*}
    c_0 = 1, \quad c_1 = 1, \quad
    c_{n+2} - c_{n+1} - c_n = 0.
  \end{equation*}
  This exercise rederives a standard formula for this sequence in a way that is intended to illustrate the technique of Lemma~\ref{lemma:cj3vqfrerl}.
  \begin{enumerate}
  \item Verify by crude estimation that the fundamental interval for the series $f(z) = \sum_n c_n z^n$ contains $(0,1/2)$.
  \item Show that
    \begin{equation*}
      f(z) = \frac{1}{1 - z - z^2} =
      \frac{1}{(1 - z/\varphi) ( 1 - z / \varphi ')},
    \end{equation*}
    where
    \begin{equation*}
      \varphi = \frac{1 + \sqrt{5}}{2} = 1.618 \dotsb, \quad
      \varphi ' = \frac{1 - \sqrt{5}}{2} = -0.618 \dotsb.
    \end{equation*}
  \item Following the proof of Lemma~\ref{lemma:cj3vqfrerl}, show that
    \begin{equation*}
      c_n = \frac{\varphi^n - {(\varphi ')}^n }{\varphi - \varphi ' }.
    \end{equation*}
    (Use that $f(z) \ll |z|^2$ for $|z| \geq 2$ to show that the ``remainder term'', namely the integral in~\eqref{align:cj3vqg2jvd}, tends to zero as $B^- \rightarrow \infty $.)
  \end{enumerate}
\end{exercise}

\begin{example}
  Let $\beta \in \mathbb{C} - \{0\}$ and $a \in \mathbb{Z}_{\geq 0}$.  Then one verifies by induction on $a$, using differentiation, that
  \begin{equation*}
    \frac{1}{(z - \beta)^{a+1}} =
    (-\beta)^{-a-1}
    \sum_{n \geq 0}
    \binom{n + a}{n} \beta^{-n} z^n,
  \end{equation*}
  where the binomial coefficient expands to a polynomial of degree $a$ in $n$:
  \begin{equation*}
    \binom{n+a}{a} = \frac{(n+a)!}{a! n!} =
    \frac{(n+1) (n+2) \dotsb (n+a)}{a!}.
  \end{equation*}
  More generally, given any coefficients $c_0, c_1\dotsc, c_k$, we have
  \begin{equation*}
    \sum_{k=0}^{a}
    \frac{c_k}{(z - \beta)^{k+1}} = \sum_{n \geq 0} P(n) \beta^{-n} z^n
  \end{equation*}
  for some polynomial $P(n)$ of degree at most $a$.  Conversely, given such a polynomial, we may find coefficients so that the above identity holds.
\end{example}


\begin{example}
  Take
  \begin{equation*}
    c_n:= e^{- 2^n }.
  \end{equation*}
  Observe that
  \begin{equation*}
    c_n \rightarrow
    \begin{cases}
      0 &  \text{ if } n \rightarrow \infty, \\
      1 & \text{ if } n \rightarrow - \infty.
    \end{cases}
  \end{equation*}
  Moreover, as $n \rightarrow \infty$, the convergence of the $c_n$ to zero is rapid in the sense that for each $B < \infty$, we have
  \begin{equation*}
    c_n \ll B^{-n}.
  \end{equation*}
  The fundamental interval is thus $(1,\infty)$: the series $f(z) = \sum_{n} c_n z^n$ converges absolutely for $|z| > 1$ and defines a holomorphic function there.  We will show that $f$ \emph{extends to a meromorphic function on} $\mathbb{C} - \{0\}$\emph{, which is holomorphic away from simple poles at} $1 / 2^k$ \emph{(for} $k \in \mathbb{Z}_{\geq 0}$\emph{) with residue} $(-1/4)^k / k!$\emph{.}  To that end, observe first that the contribution to $f$ from $n \geq 0$, namely
  \begin{equation*}
f_+(z) := \sum_{n > 0} c_n z^n,
\end{equation*}
converges absolutely and is thus holomorphic on the entire complex plane.  The meromorphic continuation of $f$ thereby reduces to that of the complementary sum
\begin{equation*}
f_-(z) := \sum_{n \leq  0} c_n z^n.
\end{equation*}
Inspired by Lemma~\ref{lemma:cj3vqfrerl}, we study the asymptotics of the coefficients $c_n$ as $n \rightarrow - \infty$.  These are described by the Taylor series of the exponential functions:
  \begin{equation*}
e^{x} = \sum_{k \geq 0} \frac{x^k }{k!}.
\end{equation*}
By estimating the tail of this series, we sees that for $x = \O(1)$ and $M = \O(1)$, we have
\begin{equation*}
e^x = \sum_{k = 0}^{N-1} \frac{x^k}{k!} + \O (x^M).
\end{equation*}
It follows that for $n \leq 0$,
\begin{equation}\label{eq:cj3wnu8lqf}
c_n = \sum_{k = 0}^{N-1} \frac{{(-2^n)}^k}{k!} + \O (2^{nM}).
\end{equation}
By the method of proof of Lemma~\ref{lemma:cj3vqfrerl}, we deduce from this estimate that $f_-$ the required assertions concerning the meromorphic continuation of $f$.  Let us spell this deduction out for the sake of practice.  Set
\begin{equation*}
 g_k(z) := 
  \sum_{n \leq 0}
  \frac{{(-2^n)}^k}{k!} z^n.
\end{equation*}
The estimate~\eqref{eq:cj3wnu8lqf} implies that the modified series
\begin{equation}\label{eq:cj3wnu7518}
   f_-(z)
  -
  \sum_{k = 0}^{N-1}
  g_k(z)
  = 
  \sum_{n \leq 0}
  \left( c_n -
    \sum_{k = 0}^{N-1} \frac{{(-2^n)}^k}{k!}
  \right) z^n
\end{equation}
converges absolutely for $\lvert z \rvert > 1 / 2^M$, hence defines a holomorphic function there.  On the other hand, for $\lvert z \rvert > 1$, we see by summing the geometric series that
\begin{equation*}
  g_k(z)
  =
  \frac{{(-1)}^k}{k!}
  \frac{1}{1 - 1 / 2^k z}
  =
  \frac{{(-1/2)}^k}{k!}
  \frac{z}{z - 1 / 2^k}.
\end{equation*}
Thus $g_k$ extends to a meromorphic function whose only pole is a simple one at $z = 1/2^k$ with residue ${(-1/4)}^k  / k!$.  It follows that $f$ has the claimed meromorphic properties.
\end{example}

TODO: I'll add some more systematic notes on regularization later.

\begin{remark}
  We can cases view $f(1)$ as the ``regularized sum'' of the (possibly divergent) series $\sum_n c_n$:
  \begin{equation*}
\sum_{n}^{\reg} c_n := \left( \sum_n c_n z^n \right)|_{z=1}, 
\end{equation*}
keeping in mind here that the series may initially convergent away from the point $z=1$, so that the specialization is understood as the result of analytic continuation.  We make this definition whenever the series is holomorphic at $z=1$.

For example,
  \begin{equation*}
    \sum_{n \geq 0}^{\reg} (-1)^n
    =
    \left( \sum_{n \geq 0} (-1)^n z^n \right)|_{z=1}
    = \frac{1}{1 + z} \vert_{z=1} = \frac{1}{2}.
\end{equation*}
In this example, we may understand $f(1)$ as the limit of the quantities $f(z)$ for $z < 1$ as $z \rightarrow 1$, and also as the Cesaro mean of the partial sums of the series $\sum_{n \geq 0} (-1)^n$, so the interpretation of $f(1)$ as a regularized sum makes intuitive sense.

In other examples, the interpretation may be less clear.  For example,
\begin{equation*}
  \sum_{n \geq 0}^{\reg} 10^n =
  \left( \sum_{n \geq 0} 10^n z^n \right)|_{z=1}
  =
  \frac{1}{1 - 10 z} \vert_{z=1} =
  \frac{-1}{9},
\end{equation*}
the intuitive meaning of which may be less clear.  One way to understand the regularization is as follows: the value $f(1)$ is insensitive to replacing the sequence $(c_n)_n$ by any of its shifts $(c_{n+k})_n$.  Setting
\begin{equation*}
S := \sum_{n \geq 0}^{\reg} 10^n,
\end{equation*}
we should thus have
\begin{equation*}
  10 S = \sum_{n \geq 0}^{\reg} 10^{n+1}
= 1 + S,
\end{equation*}
from which it follows that $S = -1/9$.
\end{remark}

\begin{example}
  Take
  \begin{equation*}
c_n = n^3 e^{- 2^{- n^2 }}.
\end{equation*}
Then the series $f$ converges absolutely nowhere: the fundamental interval is empty.  But we can regularize it to get a well-defined $f$.  TODO: expand.

Idea: meromorphically continue positive and negative parts separately, take the sum of these on intersections of domains of definition.

TODO: expansion of $c_n$.

Only pole for both $f_+$ and $f_-$ is at one; quadruple pole, with residues that cancel.
\end{example}

\begin{example}
  TODO.  For $a \in \mathbb{Z}_{\geq 0}$ and $\beta \in \mathbb{C}^\times$, take
  \begin{equation*}
c_n := n^a \beta^{- n}.
\end{equation*}
Then we claim that the regularized generating function $f$ vanishes identically.

TODO: one way to see this is to derive that $f(z) = (z / \beta) f(z)$.
\end{example}

\newpage
\section{The zeta function}

\subsection{Overview}
The Riemann zeta function is defined for a complex number $s$ by the series
\begin{equation*}
\zeta (s) = \sum_{n \geq 1} \frac{1}{n^s }.
\end{equation*}
\begin{lemma}
The series converges absolutely for $\Re(s) > 1$, uniformly for $\Re(s) \geq 1 + \eps$ for each $\eps > 0$.
\end{lemma}
\begin{proof}
  Using the identity
  \begin{equation*}
    \left\lvert \frac{1}{n^s} \right\rvert = \frac{1}{n^{\Re(s)}},
  \end{equation*}
  we reduce to the case that $s$ is real, in which this is a familiar consequence of the integral test.
\end{proof}

Our first main goal in the course is to explain the following basic facts.
\begin{theorem}
The Riemann zeta function admits a meromorphic continuation to the entire complex plane.  It is holomorphic away from a simple pole at $s = 1$, where it has residue $1$.  It admits a functional equation relating $\zeta (s)$ to $\zeta (1-s)$.
\end{theorem}

One historical motivation for considering the zeta function at complex arguments comes from the prime number theorem.
\begin{theorem}[Prime number theorem]
  Let $\pi(x) := \# \left\{ \text{primes } p \leq x \right\}$ denote the prime counting function.  Then
  \begin{equation*}
    \frac{\pi(x)}{x / \log x} \rightarrow 1
    \text{ as } x \rightarrow \infty.
  \end{equation*}
\end{theorem}
This is related to the following analytic fact concerning the zeros of the zeta function.
\begin{theorem}[Prime number theorem, formulated in terms of $\zeta$]\label{theorem:cj3vp9l79t}
We have $\zeta(s) = 0$ only if $\Re(s) < 1$.
\end{theorem}
\begin{remark}
  Even the statement of Theorem~\ref{theorem:cj3vp9l79t} is not clear without knowing the meromorphic continuation of $\zeta$.  This may offer some motivation for understanding the latter.
\end{remark}
We expect stronger nonvanishing properties:
\begin{conjecture}[Riemann Hypothesis]
We have $\zeta(s) = 0$ only if $\Re(s) < 1/2$.
\end{conjecture}
This corresponds to a conjectural stronger form of the prime number theorem, namely that
\begin{equation*}
  \pi(x) = \int_2^x \frac{t}{\log t} \, d t
  + \O (x^{1/2} \log x).
\end{equation*}



\bibliography{refs}{} \bibliographystyle{plain}
\end{document}
