\documentclass[reqno]{amsart} \usepackage{graphicx, amsmath, amssymb, amsfonts, amsthm, stmaryrd, amscd}
\usepackage[usenames, dvipsnames]{xcolor}
\usepackage{tikz}
% \usepackage{tikzcd}
% \usepackage{comment}

% \let\counterwithout\relax
% \let\counterwithin\relax
% \usepackage{chngcntr}

\usepackage{enumerate}
% \usepackage{enumitem}
% \usepackage{times}
\usepackage[normalem]{ulem}
% \usepackage{minted}
% \usepackage{xypic}
% \usepackage{color}


% \usepackage{silence}
% \WarningFilter{latex}{Label `tocindent-1' multiply defined}
% \WarningFilter{latex}{Label `tocindent0' multiply defined}
% \WarningFilter{latex}{Label `tocindent1' multiply defined}
% \WarningFilter{latex}{Label `tocindent2' multiply defined}
% \WarningFilter{latex}{Label `tocindent3' multiply defined}
\usepackage{hyperref}
% \usepackage{navigator}


% \usepackage{pdfsync}
\usepackage{xparse}


\usepackage[all]{xy}
\usepackage{enumerate}
\usetikzlibrary{matrix,arrows,decorations.pathmorphing}



\makeatletter
\newcommand*{\transpose}{%
  {\mathpalette\@transpose{}}%
}
\newcommand*{\@transpose}[2]{%
  % #1: math style
  % #2: unused
  \raisebox{\depth}{$\m@th#1\intercal$}%
}
\makeatother


\makeatletter
\newcommand*{\da@rightarrow}{\mathchar"0\hexnumber@\symAMSa 4B }
\newcommand*{\da@leftarrow}{\mathchar"0\hexnumber@\symAMSa 4C }
\newcommand*{\xdashrightarrow}[2][]{%
  \mathrel{%
    \mathpalette{\da@xarrow{#1}{#2}{}\da@rightarrow{\,}{}}{}%
  }%
}
\newcommand{\xdashleftarrow}[2][]{%
  \mathrel{%
    \mathpalette{\da@xarrow{#1}{#2}\da@leftarrow{}{}{\,}}{}%
  }%
}
\newcommand*{\da@xarrow}[7]{%
  % #1: below
  % #2: above
  % #3: arrow left
  % #4: arrow right
  % #5: space left 
  % #6: space right
  % #7: math style 
  \sbox0{$\ifx#7\scriptstyle\scriptscriptstyle\else\scriptstyle\fi#5#1#6\m@th$}%
  \sbox2{$\ifx#7\scriptstyle\scriptscriptstyle\else\scriptstyle\fi#5#2#6\m@th$}%
  \sbox4{$#7\dabar@\m@th$}%
  \dimen@=\wd0 %
  \ifdim\wd2 >\dimen@
    \dimen@=\wd2 %   
  \fi
  \count@=2 %
  \def\da@bars{\dabar@\dabar@}%
  \@whiledim\count@\wd4<\dimen@\do{%
    \advance\count@\@ne
    \expandafter\def\expandafter\da@bars\expandafter{%
      \da@bars
      \dabar@ 
    }%
  }%  
  \mathrel{#3}%
  \mathrel{%   
    \mathop{\da@bars}\limits
    \ifx\\#1\\%
    \else
      _{\copy0}%
    \fi
    \ifx\\#2\\%
    \else
      ^{\copy2}%
    \fi
  }%   
  \mathrel{#4}%
}
\makeatother
% \DeclareMathOperator{\rg}{rg}

\usepackage{mathtools}
\DeclarePairedDelimiter{\paren}{(}{)}
\DeclarePairedDelimiter{\abs}{\lvert}{\rvert}
\DeclarePairedDelimiter{\norm}{\lVert}{\rVert}
\DeclarePairedDelimiter{\innerproduct}{\langle}{\rangle}
\newcommand{\Of}[2]{{\operatorname{#1}} {\paren*{#2}}}
\newcommand{\of}[2]{{{{#1}} {\paren*{#2}}}}

\DeclareMathOperator{\Shim}{Shim}
\DeclareMathOperator{\sgn}{sgn}
\DeclareMathOperator{\fdeg}{fdeg}
\DeclareMathOperator{\SL}{SL}
\DeclareMathOperator{\slLie}{\mathfrak{s}\mathfrak{l}}
\DeclareMathOperator{\soLie}{\mathfrak{s}\mathfrak{o}}
\DeclareMathOperator{\spLie}{\mathfrak{s}\mathfrak{p}}
\DeclareMathOperator{\glLie}{\mathfrak{g}\mathfrak{l}}
\newcommand{\pn}[1]{{\color{ForestGreen} \sf PN: [#1]}}
\DeclareMathOperator{\Mp}{Mp}
\DeclareMathOperator{\Mat}{Mat}
\DeclareMathOperator{\GL}{GL}
\DeclareMathOperator{\Gr}{Gr}
\DeclareMathOperator{\GU}{GU}
\def\gl{\mathfrak{g}\mathfrak{l}}
\DeclareMathOperator{\odd}{odd}
\DeclareMathOperator{\even}{even}
\DeclareMathOperator{\GO}{GO}
\DeclareMathOperator{\good}{good}
\DeclareMathOperator{\bad}{bad}
\DeclareMathOperator{\PGO}{PGO}
\DeclareMathOperator{\htt}{ht}
\DeclareMathOperator{\height}{height}
\DeclareMathOperator{\Ass}{Ass}
\DeclareMathOperator{\coheight}{coheight}
\DeclareMathOperator{\GSO}{GSO}
\DeclareMathOperator{\SO}{SO}
\DeclareMathOperator{\so}{\mathfrak{s}\mathfrak{o}}
\DeclareMathOperator{\su}{\mathfrak{s}\mathfrak{u}}
\DeclareMathOperator{\ad}{ad}
% \DeclareMathOperator{\sc}{sc}
\DeclareMathOperator{\Ad}{Ad}
\DeclareMathOperator{\disc}{disc}
\DeclareMathOperator{\inv}{inv}
\DeclareMathOperator{\Pic}{Pic}
\DeclareMathOperator{\uc}{uc}
\DeclareMathOperator{\Cl}{Cl}
\DeclareMathOperator{\Clf}{Clf}
\DeclareMathOperator{\Hom}{Hom}
\DeclareMathOperator{\hol}{hol}
\DeclareMathOperator{\Heis}{Heis}
\DeclareMathOperator{\Haar}{Haar}
\DeclareMathOperator{\h}{h}
\def\sp{\mathfrak{s}\mathfrak{p}}
\DeclareMathOperator{\heis}{\mathfrak{h}\mathfrak{e}\mathfrak{i}\mathfrak{s}}
\DeclareMathOperator{\End}{End}
\DeclareMathOperator{\JL}{JL}
\DeclareMathOperator{\image}{image}
\DeclareMathOperator{\red}{red}
\def\div{\operatorname{div}}
\def\eps{\varepsilon}
\def\cHom{\mathcal{H}\operatorname{om}}
\DeclareMathOperator{\Ops}{Ops}
\DeclareMathOperator{\Symb}{Symb}
\def\boldGL{\mathbf{G}\mathbf{L}}
\def\boldSO{\mathbf{S}\mathbf{O}}
\def\boldU{\mathbf{U}}
\DeclareMathOperator{\hull}{hull}
\DeclareMathOperator{\LL}{LL}
\DeclareMathOperator{\PGL}{PGL}
\DeclareMathOperator{\class}{class}
\DeclareMathOperator{\lcm}{lcm}
\DeclareMathOperator{\spann}{span}
\DeclareMathOperator{\Exp}{Exp}
\DeclareMathOperator{\ext}{ext}
\DeclareMathOperator{\Ext}{Ext}
\DeclareMathOperator{\Tor}{Tor}
\DeclareMathOperator{\et}{et}
\DeclareMathOperator{\tor}{tor}
\DeclareMathOperator{\loc}{loc}
\DeclareMathOperator{\tors}{tors}
\DeclareMathOperator{\pf}{pf}
\DeclareMathOperator{\smooth}{smooth}
\DeclareMathOperator{\prin}{prin}
\DeclareMathOperator{\Kl}{Kl}
\newcommand{\kbar}{\mathchar'26\mkern-9mu k}
\DeclareMathOperator{\der}{der}
% \DeclareMathOperator{\abs}{abs}
\DeclareMathOperator{\Sub}{Sub}
\DeclareMathOperator{\Comp}{Comp}
\DeclareMathOperator{\Err}{Err}
\DeclareMathOperator{\dom}{dom}
\DeclareMathOperator{\radius}{radius}
\DeclareMathOperator{\Fitt}{Fitt}
\DeclareMathOperator{\Sel}{Sel}
\DeclareMathOperator{\rad}{rad}
\DeclareMathOperator{\id}{id}
\DeclareMathOperator{\Center}{Center}
\DeclareMathOperator{\Der}{Der}
\DeclareMathOperator{\U}{U}
% \DeclareMathOperator{\norm}{norm}
\DeclareMathOperator{\trace}{trace}
\DeclareMathOperator{\Equid}{Equid}
\DeclareMathOperator{\Feas}{Feas}
\DeclareMathOperator{\bulk}{bulk}
\DeclareMathOperator{\tail}{tail}
\DeclareMathOperator{\sys}{sys}
\DeclareMathOperator{\atan}{atan}
\DeclareMathOperator{\temp}{temp}
\DeclareMathOperator{\Asai}{Asai}
\DeclareMathOperator{\glob}{glob}
\DeclareMathOperator{\Kuz}{Kuz}
\DeclareMathOperator{\Irr}{Irr}
\newcommand{\rsL}{ \frac{ L^{(R)}(\Pi \times \Sigma, \std, \frac{1}{2})}{L^{(R)}(\Pi \times \Sigma, \Ad, 1)}  }
\DeclareMathOperator{\GSp}{GSp}
\DeclareMathOperator{\PGSp}{PGSp}
\DeclareMathOperator{\BC}{BC}
\DeclareMathOperator{\Ann}{Ann}
\DeclareMathOperator{\Gen}{Gen}
\DeclareMathOperator{\SU}{SU}
\DeclareMathOperator{\PGSU}{PGSU}
% \DeclareMathOperator{\gen}{gen}
\DeclareMathOperator{\PMp}{PMp}
\DeclareMathOperator{\PGMp}{PGMp}
\DeclareMathOperator{\PB}{PB}
\DeclareMathOperator{\ind}{ind}
\DeclareMathOperator{\Jac}{Jac}
\DeclareMathOperator{\jac}{jac}
\DeclareMathOperator{\im}{im}
\DeclareMathOperator{\Aut}{Aut}
\DeclareMathOperator{\Int}{Int}
\DeclareMathOperator{\PSL}{PSL}
\DeclareMathOperator{\co}{co}
\DeclareMathOperator{\irr}{irr}
\DeclareMathOperator{\prim}{prim}
\DeclareMathOperator{\bal}{bal}
\DeclareMathOperator{\baln}{bal}
\DeclareMathOperator{\dist}{dist}
\DeclareMathOperator{\RS}{RS}
\DeclareMathOperator{\Ram}{Ram}
\DeclareMathOperator{\Sob}{Sob}
\DeclareMathOperator{\Sol}{Sol}
\DeclareMathOperator{\soc}{soc}
\DeclareMathOperator{\nt}{nt}
\DeclareMathOperator{\mic}{mic}
\DeclareMathOperator{\Gal}{Gal}
\DeclareMathOperator{\st}{st}
\DeclareMathOperator{\std}{std}
\DeclareMathOperator{\diag}{diag}
\DeclareMathOperator{\Sym}{Sym}
\DeclareMathOperator{\gr}{gr}
\DeclareMathOperator{\aff}{aff}
\DeclareMathOperator{\Dil}{Dil}
\DeclareMathOperator{\Lie}{Lie}
\DeclareMathOperator{\Symp}{Symp}
\DeclareMathOperator{\Stab}{Stab}
\DeclareMathOperator{\St}{St}
\DeclareMathOperator{\stab}{stab}
\DeclareMathOperator{\codim}{codim}
\DeclareMathOperator{\linear}{linear}
\newcommand{\git}{/\!\!/}
\DeclareMathOperator{\geom}{geom}
\DeclareMathOperator{\spec}{spec}
\def\O{\operatorname{O}}
\DeclareMathOperator{\Au}{Aut}
\DeclareMathOperator{\Fix}{Fix}
\DeclareMathOperator{\Opp}{Op}
\DeclareMathOperator{\opp}{op}
\DeclareMathOperator{\Size}{Size}
\DeclareMathOperator{\Save}{Save}
% \DeclareMathOperator{\ker}{ker}
\DeclareMathOperator{\coker}{coker}
\DeclareMathOperator{\sym}{sym}
\DeclareMathOperator{\mean}{mean}
\DeclareMathOperator{\elliptic}{ell}
\DeclareMathOperator{\nilpotent}{nil}
\DeclareMathOperator{\hyperbolic}{hyp}
\DeclareMathOperator{\newvector}{new}
\DeclareMathOperator{\new}{new}
\DeclareMathOperator{\full}{full}
\newcommand{\qr}[2]{\left( \frac{#1}{#2} \right)}
\DeclareMathOperator{\unr}{u}
\DeclareMathOperator{\ram}{ram}
% \DeclareMathOperator{\len}{len}
\DeclareMathOperator{\fin}{fin}
\DeclareMathOperator{\cusp}{cusp}
\DeclareMathOperator{\curv}{curv}
\DeclareMathOperator{\rank}{rank}
\DeclareMathOperator{\rk}{rk}
\DeclareMathOperator{\pr}{pr}
\DeclareMathOperator{\Transform}{Transform}
\DeclareMathOperator{\mult}{mult}
\DeclareMathOperator{\Eis}{Eis}
\DeclareMathOperator{\reg}{reg}
\DeclareMathOperator{\sing}{sing}
\DeclareMathOperator{\alt}{alt}
\DeclareMathOperator{\irreg}{irreg}
\DeclareMathOperator{\sreg}{sreg}
\DeclareMathOperator{\Wd}{Wd}
\DeclareMathOperator{\Weil}{Weil}
\DeclareMathOperator{\Th}{Th}
\DeclareMathOperator{\Sp}{Sp}
\DeclareMathOperator{\Ind}{Ind}
\DeclareMathOperator{\Res}{Res}
\DeclareMathOperator{\ini}{in}
\DeclareMathOperator{\ord}{ord}
\DeclareMathOperator{\osc}{osc}
\DeclareMathOperator{\fluc}{fluc}
\DeclareMathOperator{\size}{size}
\DeclareMathOperator{\ann}{ann}
\DeclareMathOperator{\equ}{eq}
\DeclareMathOperator{\res}{res}
\DeclareMathOperator{\pt}{pt}
\DeclareMathOperator{\src}{source}
\DeclareMathOperator{\Zcl}{Zcl}
\DeclareMathOperator{\Func}{Func}
\DeclareMathOperator{\Map}{Map}
\DeclareMathOperator{\Frac}{Frac}
\DeclareMathOperator{\Frob}{Frob}
\DeclareMathOperator{\ev}{eval}
\DeclareMathOperator{\pv}{pv}
\DeclareMathOperator{\eval}{eval}
\DeclareMathOperator{\Spec}{Spec}
\DeclareMathOperator{\Speh}{Speh}
\DeclareMathOperator{\Spin}{Spin}
\DeclareMathOperator{\GSpin}{GSpin}
\DeclareMathOperator{\Specm}{Specm}
\DeclareMathOperator{\Sphere}{Sphere}
\DeclareMathOperator{\Sqq}{Sq}
\DeclareMathOperator{\Ball}{Ball}
\DeclareMathOperator\Cond{\operatorname{Cond}}
\DeclareMathOperator\proj{\operatorname{proj}}
\DeclareMathOperator\Swan{\operatorname{Swan}}
\DeclareMathOperator{\Proj}{Proj}
\DeclareMathOperator{\bPB}{{\mathbf P}{\mathbf B}}
\DeclareMathOperator{\Projm}{Projm}
\DeclareMathOperator{\Tr}{Tr}
\DeclareMathOperator{\Type}{Type}
\DeclareMathOperator{\Prop}{Prop}
\DeclareMathOperator{\vol}{vol}
\DeclareMathOperator{\covol}{covol}
\DeclareMathOperator{\Rep}{Rep}
\DeclareMathOperator{\Cent}{Cent}
\DeclareMathOperator{\val}{val}
\DeclareMathOperator{\area}{area}
\DeclareMathOperator{\nr}{nr}
\DeclareMathOperator{\CM}{CM}
\DeclareMathOperator{\CH}{CH}
\DeclareMathOperator{\tr}{tr}
\DeclareMathOperator{\characteristic}{char}
\DeclareMathOperator{\supp}{supp}


\theoremstyle{plain} \newtheorem{theorem} {Theorem} \newtheorem{conjecture} [theorem] {Conjecture} \newtheorem{corollary} [theorem] {Corollary} \newtheorem{proposition} [theorem] {Proposition} \newtheorem{fact} [theorem] {Fact}
\theoremstyle{definition} \newtheorem{definition} [theorem] {Definition} \newtheorem{hypothesis} [theorem] {Hypothesis} \newtheorem{assumptions} [theorem] {Assumptions}
\newtheorem{example} [theorem] {Example}
\newtheorem{assertion}[theorem] {Assertion}
\newtheorem{note}[theorem] {Note}
\newtheorem{conclusion}[theorem] {Conclusion}
\newtheorem{claim}            {Claim}
\newtheorem{homework} {Homework}
\newtheorem{exercise} {Exercise}  \newtheorem{question}[theorem] {Question}    \newtheorem{answer} {Answer}  \newtheorem{problem} {Problem}    \newtheorem{remark} [theorem] {Remark}
\newtheorem{notation} [theorem]           {Notation}
\newtheorem{terminology}[theorem]            {Terminology}
\newtheorem{convention}[theorem]            {Convention}
\newtheorem{motivation}[theorem]            {Motivation}


\newtheoremstyle{itplain} % name
{6pt}                    % Space above
{5pt\topsep}                    % Space below
{\itshape}                   % Body font
{}                           % Indent amount
{\itshape}                   % Theorem head font
{.}                          % Punctuation after theorem head
{5pt plus 1pt minus 1pt}                       % Space after theorem head
% {.5em}                       % Space after theorem head
{}  % Theorem head spec (can be left empty, meaning ‘normal’)

% \theoremstyle{mytheoremstyle}


\theoremstyle{itplain} %--default
% \theoremheaderfont{\itshape}
% \newtheorem{lemma}{Lemma}
\newtheorem{lemma}[theorem]{Lemma}
% \newtheorem{lemma}{Lemma}[subsubsection]

\newtheorem*{lemma*}{Lemma}
\newtheorem*{proposition*}{Proposition}
\newtheorem*{definition*}{Definition}
\newtheorem*{example*}{Example}

\newtheorem*{results*}{Results}
\newtheorem{results} [theorem] {Results}


\usepackage[displaymath,textmath,sections,graphics]{preview}
\PreviewEnvironment{align*}
\PreviewEnvironment{multline*}
\PreviewEnvironment{tabular}
\PreviewEnvironment{verbatim}
\PreviewEnvironment{lstlisting}
\PreviewEnvironment*{frame}
\PreviewEnvironment*{alert}
\PreviewEnvironment*{emph}
\PreviewEnvironment*{textbf}

 \numberwithin{theorem}{section} \numberwithin{equation}{section}

\usepackage{xr-hyper}
\externaldocument{20230907T142550--generating-functions-asymptotics}
\externaldocument{20230907T143130--fourier-and-mellin-transforms}
\externaldocument{20230907T143521--complex-analysis-preliminaries}
\externaldocument{20230907T144219--bernoulli-numbers-euler-maclaurin}

\begin{document}
 
These are notes for an ongoing Fall 2023 course on the Riemann zeta function and its generalizations, $L$-functions.  These notes will be filled in as we go.

\section{References thus far}
\begin{itemize}
\item Generating functions and asymptotics: \cite[\S5.2]{MR2172781}
\item Mellin transform and asymptotics: \cite{zagier-mellin}
\end{itemize}

\section{Outlines thus far}
(Some links here refer to external files; I'll have to think of a good way to notate that.)
\begin{itemize}
\item Tuesday, 29 Aug: parts of \S\ref{sec:cj4unj5r3k}; \S\ref{sec:cj4vkkg5ah}, \S\ref{sec:cj4vkkg9es}
\item Thursday, 31 Aug: \S\ref{sec:cj4unj3scf}, \S\ref{sec:cj4unj4gyo}
\item Friday, 1 Sep: \S\ref{sec:cj4unj04kx} and \S\ref{sec:cj4unj06mw}
\item Tuesday, 5 Sep: \S\ref{sec:cj4unjziaz}
\item Thursday, 7 Sep: \S\ref{sec:cj4vkkark8}, \S\ref{sec:cj4vkfaant}
\end{itemize}


\section{Course notes that I've since split off into separate files}
\begin{itemize}
\item \href{20230907T143521--complex-analysis-preliminaries.pdf}{Complex-analytic preliminaries}
\item \href{20230907T142550--generating-functions-asymptotics.pdf}{Generating functions and asymptotics}
\item \href{20230907T143130--fourier-and-mellin-transforms.pdf}{Fourier and Mellin transforms}
\item \href{20230907T144219--bernoulli-numbers-euler-maclaurin.pdf}{Bernoulli numbers and Euler--Maclaurin summation}
\end{itemize}


\newpage


\section{Background}\label{sec:cj4unj5r3k}

\subsection{General notation}
$\mathbb{R}^+ := (0,\infty)$.

\subsection{Asymptotic notation}
We use the equivalent notations
\begin{equation*}
  A = \O(B), \qquad A \ll B,
  \qquad B \gg A
\end{equation*}
to denote that
\begin{equation*}
  \lvert A \rvert \leq C \lvert B \rvert
\end{equation*}
for some ``constant'' $C$.  The precise meaning of ``constant'' will either be specified or clear from context.


\newpage
\subsection{Definition and basic properties of $\zeta$: overview}
The Riemann zeta function is defined for a complex number $s$ by the series
\begin{equation*}
  \zeta (s) = \sum_{n \geq 1} \frac{1}{n^s }.
\end{equation*}
\begin{lemma}
  The series converges absolutely for $\Re(s) > 1$, uniformly for $\Re(s) \geq 1 + \eps$ for each $\eps > 0$.
\end{lemma}
\begin{proof}
  Using the identity
  \begin{equation*}
    \left\lvert \frac{1}{n^s} \right\rvert = \frac{1}{n^{\Re(s)}},
  \end{equation*}
  we reduce to the case that $s$ is real, in which this is a familiar consequence of the integral test.
\end{proof}

Our first main goal in the course is to explain the following basic facts.
\begin{theorem}
  The Riemann zeta function admits a meromorphic continuation to the entire complex plane.  It is holomorphic away from a simple pole at $s = 1$, where it has residue $1$.  It admits a functional equation relating $\zeta (s)$ to $\zeta (1-s)$.
\end{theorem}

One historical motivation for considering the zeta function at complex arguments comes from the prime number theorem.
\begin{theorem}[Prime number theorem]
  Let $\pi(x) := \# \left\{ \text{primes } p \leq x \right\}$ denote the prime counting function.  Then
  \begin{equation*}
    \frac{\pi(x)}{x / \log x} \rightarrow 1
    \text{ as } x \rightarrow \infty.
  \end{equation*}
\end{theorem}
This is related to the following analytic fact concerning the zeros of the zeta function.
\begin{theorem}[Prime number theorem, formulated in terms of $\zeta$]\label{theorem:cj3vp9l79t}
  We have $\zeta(s) = 0$ only if $\Re(s) < 1$.
\end{theorem}
\begin{remark}
  Even the statement of Theorem~\ref{theorem:cj3vp9l79t} is not clear without knowing the meromorphic continuation of $\zeta$.  This may offer some motivation for understanding the latter.
\end{remark}
We expect stronger nonvanishing properties:
\begin{conjecture}[Riemann Hypothesis]
  We have $\zeta(s) = 0$ only if $\Re(s) < 1/2$.
\end{conjecture}
This corresponds to a conjectural stronger form of the prime number theorem, namely that
\begin{equation*}
  \pi(x) = \int_2^x \frac{t}{\log t} \, d t
  + \O (x^{1/2} \log x).
\end{equation*}

\newpage

\newpage
\section{Basic analytic properties of $\zeta$}\label{sec:cj4unjziaz}


\subsection{Meromorphic continuation and evaluation at negative integers}
Take
\begin{equation*}
  h(y) = \frac{1}{e^y - 1}.
\end{equation*}
The function $y h(y)$ extends to a holomorphic function of $y$ on the disc $\{y \in \mathbb{C} : \lvert y \rvert < 2 \pi \}$, so $h$ is represented for small $y > 0$ by an absolutely convergent Laurent series of the following form:
\begin{equation*}
  h(y) = \frac{1}{y} + \sum_{n = 1 }^\infty \frac{B_n  }{ n !} y^{n - 1}.
\end{equation*}
Here the $B_n$ are complex coefficients, called the \emph{Bernoulli numbers} (see \href{20230907T144219--bernoulli-numbers-euler-maclaurin.pdf}{this note}).  On the other hand, $h$ decays rapidly (like $\O(y^N)$ for any fixed $N$) as $y \rightarrow \infty$.  By the analysis we've now seen many times, we deduce that the Mellin transform $H(s)$ of $h(y)$ converges absolutely for $\Re(s) > 1$, where it defines a holomorphic function, and extends to a meromorphic function on the complex plane whose only poles are simple poles at $s = 1$ (corresponding to the $1/y$ term in the asymptotic expansion as $y \rightarrow 0$) and at $s = -n$ for each $n \in \mathbb{Z}_{\geq 0}$, with residue given by $1$ in the former case and by $B_{n+1} / (n+!)!$ in the latter.

On the other hand, we can rewrite
\begin{equation*}
  h(y) = \frac{e^{- y}}{ 1 - e^{- y}}
  = e^{- y } + e^{- 2 y } + e^{- 3 y} + \dotsb,
\end{equation*}
giving
\begin{equation*}
  H(s) = \int_{\mathbb{R}^+ } y^s \sum_{n = 1 }^\infty e^{- n y } \,d^\times y.
\end{equation*}
The following calculations will show that the doubly integral/sum converges absolutely for $\Re(s) > 1$, so we may rearrange it as
\begin{equation*}
  \sum_{n = 1 }^\infty \int_{\mathbb{R}^+ } y^s e^{- n y } \,d^\times y.
\end{equation*}
The inner integral may be simplified by the substitution $y \mapsto y /n$.  This has no effect on the measure $d^\times y$, but replaces $y^s$ by $n^{-s} y^s$, giving
\begin{equation*}
  \sum_{n = 1 }^\infty n^{- s} \int_{\mathbb{R}^+ } y^s e^{- y } \,d^\times y
  = \zeta(s) \Gamma(s).
\end{equation*}
We see in this way that the $\zeta$ function admits a meromorphic continuation.  Since the $\Gamma$-function \href{https://www.proofwiki.org/wiki/Zeroes_of_Gamma_Function}{does not vanish}, we deduce that the only pole of $\zeta$ is the simple one at $s=1$, with residue $s=0$.  Also, since we have computed the residues of both $\Gamma(s)$ and $\zeta(s) \Gamma(s)$ at the nonpositive integers, we may calculate in this way the values of $\zeta(-n)$ for each $n \in \mathbb{Z}_{\geq 0}$:
\begin{equation*}
  \zeta(-n) = (-1)^n \frac{B_{n + 1}}{ n + 1}.
\end{equation*}

\subsection{Another perspective on the meromorphic continuation}\label{sec:cj4vkkark8}
(TODO: think of better section titles)

Given $f : \mathbb{R} \rightarrow \mathbb{C}$ of sufficient decay at infinity, we define $g : \mathbb{R}^+ \rightarrow \mathbb{C} $ by
\begin{equation*}
g(y) := \sum_{n \in \mathbb{Z} } f(n y).
\end{equation*}
We assume henceforth that $f$ lies in the \href{https://en.wikipedia.org/wiki/Schwartz_space}{Schwartz space} $\mathcal{S}(\mathbb{R})$.

The asymptotics of $g(y)$ are described as follows.
\begin{lemma}
  Let $f \in \mathcal{S}(R)$.
  \begin{enumerate}
\item 
  As $y \rightarrow \infty$, we have
  \begin{equation}\label{eq:cj4vkfq4ku}
g(y) = f(0) + \O(y^{-N})
  \end{equation}
  for each fixed $N$.
\item As $y \rightarrow 0$, we have
  \begin{equation}\label{eq:cj4vkfrkjf}
g(y) = y^{-1} \int_{\mathbb{R} } f(x) \, d x  + \O (y^{N})
  \end{equation}
  for each fixed $N$.
\end{enumerate}
\end{lemma}
\begin{proof}
The first estimate \eqref{eq:cj4vkfq4ku} is an easy exercise using the definition of the Schwartz space.  The second estimate \eqref{eq:cj4vkfrkjf} may be proved either via Euler--Maclaurin summation (see external \S\ref{sec:cj4vkfaant}) or Poisson summation (see external \S\ref{sec:cj4vkkp26q}).
\end{proof}

Assuming that $f(0)$ and $\int f$ are nonzero, it follows that the Mellin transform of $g$ does not converge absolutely at any point.  We can still define a regularized Mellin transform
\begin{equation*}
G(s) = \int_{\mathbb{R}^+ }^{\reg} y^s g (y) \,d^\times y,
\end{equation*}
like in \S\ref{sec:cj4unj4gyo}, by splitting the integral into two pieces (e.g, the contributions of $(0,1)$ and $(1,\infty$)), meromorphically continuing each piece, and then summing the results on their domain of overlap (if any).

\begin{lemma}
  $G(s)$ defines a meromorphic function on the complex plane whose only poles at simple ones:
  \begin{itemize}
\item at $s = 0$, with residue $-f(0)$, and
\item at $s = 1$, with residue $\int_{\mathbb{R}} f(x) \,d x $.
\end{itemize}
\end{lemma}
\begin{proof}
  Let's carry this out in detail, applying the recipe of external \S\ref{sec:cj4vkf4yvq}.
  \begin{itemize}
  \item We see from \eqref{eq:cj4vkfq4ku} that the integral
    \begin{equation*}
      G_+(s) := \int_1^\infty y^s g(y) \,d^\times y
    \end{equation*}
    converges absolutely for $\Re(s) < 0$ and extends to a meromorphic function of $s$, whose only pole is a simple pole at $s = 0$ of residue $-f(0)$.  Indeed,
    \begin{equation*}
      G_+(s) = \int_1^\infty y^s \left(g (y) - f(0) \right) \,d^\times y
      + f(0)
      \underbrace
      {
        \int_1^\infty y^{s} \,d^\times y
      }_{
        -1/s
      },
    \end{equation*}
    where the first integral on the right hand side converges absolutely for all $s$.
  \item We see from \eqref{eq:cj4vkfrkjf} that the integral
    \begin{equation*}
      G_-(s) := \int_0^1 y^s g(y) \,d^\times y
    \end{equation*}
    converges absolutely for $\Re(s) > 1$ and extends to a meromorphic function of $s$, whose only pole is a simple pole at $s = 0$ of residue $I_f$, where $I_f := \int_{\mathbb{R} } f (x) \, d x$.  Indeed,
    \begin{equation*}
      G_-(s) = \int_0^1 y^s \left( g (y) - y^{-1} I_f \right) \,d^\times y + I_f \underbrace
      {
        \int_0^1 y^{s-1} \,d^\times y
      }_{
        1/(s-1)
      }.
    \end{equation*}
  \end{itemize}
\end{proof}

We denote henceforth by $\mathcal{F} f$ the (normalized) Fourier transform
  \begin{equation*}
\mathcal{F} f (\xi) := \int_{x \in \mathbb{R} } f(x) e^{- 2 \pi i x \xi } \, d x.
\end{equation*}

\begin{lemma}\label{lemma:cj4vkjxr7m}
  Write $G(s) = G_f(s)$ to indicate the dependence upon $f$.  Then we have the following functional equation:
  \begin{equation*}
G_f(s) = G_{\mathcal{F} f}(1-s).
  \end{equation*}
\end{lemma}
\begin{proof}
  This is a consequence of the Poisson summation formula (see external \eqref{eq:cj4vkktmw8}) 
  \begin{equation*}
    \sum_{n \in \mathbb{Z} } f (n y) = y^{-1} \sum_{n \in \mathbb{Z} } \mathcal{F} f (n / y).
  \end{equation*}
  Writing $g_f$ to indicate the dependence of $g$ upon $f$, the above identity reads
  \begin{equation*}
    g_f(y) = y^{-1} g_{\mathcal{F} f}(1/y).
  \end{equation*}
  Taking the (regularized) Mellin transform of both sides yields
  \begin{equation*}
    G_f(s) = \int_{\mathbb{R}^+}^{\reg} y^{s-1} g_{\mathcal{F} f}(1/y) \,d^\times y.
  \end{equation*}
  To evaluate this last integral, we substitute $y \mapsto 1/y$, which leaves the measure $d^\times y$ invariant (and is unaffected by the regularization).  This gives
  \begin{equation*}
    G_f(s) = \int_{\mathbb{R}^+}^{\reg} y^{1-s} g_{\mathcal{F} f}(y) \,d^\times y = G_{\mathcal{F} f}(1-s),
  \end{equation*}
  as required.
\end{proof}


We suppose henceforth that $f$ is even.  This is without much loss of generality -- any function can be written as a sum of even and odd functions, and if $f$ is odd, then $g$ vanishes identically.

\begin{exercise}
  Let $f \in \mathcal{S}(\mathbb{R})$ be even.  Show that the Mellin transform
  \begin{equation*}
F(s) := \int y^s f (y) \,d^\times y
  \end{equation*}
  converges initially for $\Re(s) > 0$ and extends to a meromorphic function on the complex plane, whose only poles are simple ones at $s = - 2n$ for $n \in \mathbb{Z}_{\geq 0}$ with residue
  \begin{equation*}
\res_{s = - 2 n} F(s) = \frac{1}{(2 n)!} f^{(2 n)}(0).
  \end{equation*}
  [Use Taylor's theorem with remainder, and note that the odd Taylor coefficients vanish in view of the evenness assumption on $f$.]
\end{exercise}
\begin{example}
  Take $f(x) := e^{- \pi x^2}$.  Then
  \begin{equation}\label{eq:cj4vkj6hym}
F(s) = \pi^{- s / 2 } \Gamma (s / 2),
  \end{equation}
  as one sees by substituting $y \mapsto \sqrt{y}$ and then $y \mapsto y / \pi$ in the defining integral.  This has poles in the expected places.
\end{example}

For $f$ even, we have
\begin{equation*}
g(y) = f(0) + 2 \sum_{n = 1 }^\infty f(n y).
\end{equation*}
Using that the constant function $1$ has vanishing regularized Mellin transform (for reasons similar to external Exercise \ref{exercise:cj4ss5ia8l}), we see that $G(s)$ admits the following absolutely convergent integral representation for $\Re(s) > 1$:
\begin{equation*}
G(s) = 2 \int_{\mathbb{R}^+ } y^s \sum_{n = 1 }^\infty f (n y ) \,d^\times y.
\end{equation*}
By substituting $y \mapsto y / n$, we see that
\begin{equation*}
G(s) = 2 \zeta(s) F(s), \qquad F(s) := \int y^s f(y) \,d^\times y.
\end{equation*}
This gives another proof of the meromorphic behavior of $\zeta$.  TODO: say more.

We can also deduce the functional equation:
\begin{theorem}
  We have
  \begin{equation*}
\xi(s) := \pi^{- s / 2 } \Gamma (s/ 2 ) \zeta(s) = \xi (1 - s).
\end{equation*}
\end{theorem}
\begin{proof}
  Take $f(x) := e^{- \pi x^2}$.  By \eqref{eq:cj4vkj6hym}, we then have $G_f(s) = \xi(s)$.  On the other hand, by the well-known \href{https://mathworld.wolfram.com/FourierTransformGaussian.html}{formula} for the Fourier transform of a Gaussian, we have $\mathcal{F} f = f$.  By Lemma \ref{lemma:cj4vkjxr7m}, it follows that $G_f (s) = G_f (1 - s)$.  The claimed formula follows.
\end{proof}




\bibliography{refs}{} \bibliographystyle{plain}
\end{document}
