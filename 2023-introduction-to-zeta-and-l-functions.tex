\documentclass[reqno]{amsart} \input{common.tex} \numberwithin{theorem}{section} \numberwithin{equation}{section}

\usepackage{xr-hyper}
\externaldocument{20230907T142550--generating-functions-asymptotics}
\externaldocument{20230907T143130--fourier-and-mellin-transforms}
\externaldocument{20230907T143521--complex-analysis-preliminaries}

\begin{document}
 
These are notes for an ongoing Fall 2023 course on the Riemann zeta function and its generalizations, $L$-functions.  These notes will be filled in as we go.

\section{References thus far}
\begin{itemize}
\item Generating functions and asymptotics: \cite[\S5.2]{MR2172781}
\item Mellin transform and asymptotics: \cite{zagier-mellin}
\end{itemize}

\section{Outlines thus far}
\begin{itemize}
\item Tuesday, 29 Aug: parts of \S\ref{sec:cj4unj5r3k}; \S\ref{sec:cj4unj5r3k} and \S\ref{sec:cj4unj5r3k}.
\item Thursday, 31 Aug: \S\ref{sec:cj4unj3scf} and \S\ref{sec:cj4unj4gyo}
\item Friday, 1 Sep: \S\ref{sec:cj4unj04kx} and \S\ref{sec:cj4unj06mw}
\item Tuesday, 5 Sep: \S\ref{sec:cj4unjziaz}
\end{itemize}


\section{Course notes that I've since split off into separate files}
\begin{itemize}
\item \href{20230907T143521--complex-analysis-preliminaries.pdf}{Complex-analytic preliminaries}
\item \href{20230907T142550--generating-functions-asymptotics.pdf}{Generating functions and asymptotics}
\item \href{20230907T143130--fourier-and-mellin-transforms.pdf}{Fourier and Mellin transforms}
\end{itemize}


\newpage


\section{Background}\label{sec:cj4unj5r3k}

\subsection{General notation}
$\mathbb{R}^+ := (0,\infty)$.

\subsection{Asymptotic notation}
We use the equivalent notations
\begin{equation*}
  A = \O(B), \qquad A \ll B,
  \qquad B \gg A
\end{equation*}
to denote that
\begin{equation*}
  \lvert A \rvert \leq C \lvert B \rvert
\end{equation*}
for some ``constant'' $C$.  The precise meaning of ``constant'' will either be specified or clear from context.


\newpage
\subsection{Definition and basic properties of $\zeta$: overview}
The Riemann zeta function is defined for a complex number $s$ by the series
\begin{equation*}
  \zeta (s) = \sum_{n \geq 1} \frac{1}{n^s }.
\end{equation*}
\begin{lemma}
  The series converges absolutely for $\Re(s) > 1$, uniformly for $\Re(s) \geq 1 + \eps$ for each $\eps > 0$.
\end{lemma}
\begin{proof}
  Using the identity
  \begin{equation*}
    \left\lvert \frac{1}{n^s} \right\rvert = \frac{1}{n^{\Re(s)}},
  \end{equation*}
  we reduce to the case that $s$ is real, in which this is a familiar consequence of the integral test.
\end{proof}

Our first main goal in the course is to explain the following basic facts.
\begin{theorem}
  The Riemann zeta function admits a meromorphic continuation to the entire complex plane.  It is holomorphic away from a simple pole at $s = 1$, where it has residue $1$.  It admits a functional equation relating $\zeta (s)$ to $\zeta (1-s)$.
\end{theorem}

One historical motivation for considering the zeta function at complex arguments comes from the prime number theorem.
\begin{theorem}[Prime number theorem]
  Let $\pi(x) := \# \left\{ \text{primes } p \leq x \right\}$ denote the prime counting function.  Then
  \begin{equation*}
    \frac{\pi(x)}{x / \log x} \rightarrow 1
    \text{ as } x \rightarrow \infty.
  \end{equation*}
\end{theorem}
This is related to the following analytic fact concerning the zeros of the zeta function.
\begin{theorem}[Prime number theorem, formulated in terms of $\zeta$]\label{theorem:cj3vp9l79t}
  We have $\zeta(s) = 0$ only if $\Re(s) < 1$.
\end{theorem}
\begin{remark}
  Even the statement of Theorem~\ref{theorem:cj3vp9l79t} is not clear without knowing the meromorphic continuation of $\zeta$.  This may offer some motivation for understanding the latter.
\end{remark}
We expect stronger nonvanishing properties:
\begin{conjecture}[Riemann Hypothesis]
  We have $\zeta(s) = 0$ only if $\Re(s) < 1/2$.
\end{conjecture}
This corresponds to a conjectural stronger form of the prime number theorem, namely that
\begin{equation*}
  \pi(x) = \int_2^x \frac{t}{\log t} \, d t
  + \O (x^{1/2} \log x).
\end{equation*}

\newpage

\newpage
\section{Basic analytic properties of $\zeta$}\label{sec:cj4unjziaz}





\begin{example}
  Take
  \begin{equation*}
h(y) = \frac{1}{e^y - 1}.
  \end{equation*}
  The function $y h(y)$ extends to a holomorphic function of $y$ on the disc $\{y \in \mathbb{C} : \lvert y \rvert < 2 \pi \}$, so $h$ is represented for small $y > 0$ by an absolutely convergent Laurent series of the following form:
  \begin{equation*}
h(y) = \frac{1}{y} + \sum_{n = 1 }^\infty \frac{B_n  }{ n !} y^{n - 1}.
  \end{equation*}
  Here the $B_n$ are complex coefficients, called the \emph{Bernoulli numbers}.  On the other hand, $h$ decays rapidly (like $\O(y^N)$ for any fixed $N$) as $y \rightarrow \infty$.  By the analysis we've now seen many times, we deduce that the Mellin transform $H(s)$ of $h(y)$ converges absolutely for $\Re(s) > 1$, where it defines a holomorphic function, and extends to a meromorphic function on the complex plane whose only poles are simple poles at $s = 1$ (corresponding to the $1/y$ term in the asymptotic expansion as $y \rightarrow 0$) and at $s = -n$ for each $n \in \mathbb{Z}_{\geq 0}$, with residue given by $1$ in the former case and by $B_{n+1} / (n+!)!$ in the latter.

  On the other hand, we can rewrite
  \begin{equation*}
    h(y) = \frac{e^{- y}}{ 1 - e^{- y}}
    = e^{- y } + e^{- 2 y } + e^{- 3 y} + \dotsb,
  \end{equation*}
  giving
  \begin{equation*}
    H(s) = \int_{\mathbb{R}^+ } y^s \sum_{n = 1 }^\infty e^{- n y } \,d^\times y.
  \end{equation*}
  The following calculations will show that the doubly integral/sum converges absolutely for $\Re(s) > 1$, so we may rearrange it as
  \begin{equation*}
\sum_{n = 1 }^\infty \int_{\mathbb{R}^+ } y^s e^{- n y } \,d^\times y.
  \end{equation*}
  The inner integral may be simplified by the substitution $y \mapsto y /n$.  This has no effect on the measure $d^\times y$, but replaces $y^s$ by $n^{-s} y^s$, giving
  \begin{equation*}
    \sum_{n = 1 }^\infty n^{- s} \int_{\mathbb{R}^+ } y^s e^{- y } \,d^\times y
    = \zeta(s) \Gamma(s).
  \end{equation*}
  We see in this way that the $\zeta$ function admits a meromorphic continuation.  Since the $\Gamma$-function \href{https://www.proofwiki.org/wiki/Zeroes_of_Gamma_Function}{does not vanish}, we deduce that the only pole of $\zeta$ is the simple one at $s=1$, with residue $s=0$.  Also, since we have computed the residues of both $\Gamma(s)$ and $\zeta(s) \Gamma(s)$ at the nonpositive integers, we may calculate in this way the values of $\zeta(-n)$ for each $n \in \mathbb{Z}_{\geq 0}$:
  \begin{equation*}
\zeta(-n) = (-1)^n \frac{B_{n + 1}}{ n + 1}.
\end{equation*}
\end{example}





\bibliography{refs}{} \bibliographystyle{plain}
\end{document}
