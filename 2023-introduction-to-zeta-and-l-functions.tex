\documentclass[reqno]{amsart} \usepackage{graphicx, amsmath, amssymb, amsfonts, amsthm, stmaryrd, amscd}
\usepackage[usenames, dvipsnames]{xcolor}
\usepackage{tikz}
% \usepackage{tikzcd}
% \usepackage{comment}

% \let\counterwithout\relax
% \let\counterwithin\relax
% \usepackage{chngcntr}

\usepackage{enumerate}
% \usepackage{enumitem}
% \usepackage{times}
\usepackage[normalem]{ulem}
% \usepackage{minted}
% \usepackage{xypic}
% \usepackage{color}


% \usepackage{silence}
% \WarningFilter{latex}{Label `tocindent-1' multiply defined}
% \WarningFilter{latex}{Label `tocindent0' multiply defined}
% \WarningFilter{latex}{Label `tocindent1' multiply defined}
% \WarningFilter{latex}{Label `tocindent2' multiply defined}
% \WarningFilter{latex}{Label `tocindent3' multiply defined}
\usepackage{hyperref}
% \usepackage{navigator}


% \usepackage{pdfsync}
\usepackage{xparse}


\usepackage[all]{xy}
\usepackage{enumerate}
\usetikzlibrary{matrix,arrows,decorations.pathmorphing}



\makeatletter
\newcommand*{\transpose}{%
  {\mathpalette\@transpose{}}%
}
\newcommand*{\@transpose}[2]{%
  % #1: math style
  % #2: unused
  \raisebox{\depth}{$\m@th#1\intercal$}%
}
\makeatother


\makeatletter
\newcommand*{\da@rightarrow}{\mathchar"0\hexnumber@\symAMSa 4B }
\newcommand*{\da@leftarrow}{\mathchar"0\hexnumber@\symAMSa 4C }
\newcommand*{\xdashrightarrow}[2][]{%
  \mathrel{%
    \mathpalette{\da@xarrow{#1}{#2}{}\da@rightarrow{\,}{}}{}%
  }%
}
\newcommand{\xdashleftarrow}[2][]{%
  \mathrel{%
    \mathpalette{\da@xarrow{#1}{#2}\da@leftarrow{}{}{\,}}{}%
  }%
}
\newcommand*{\da@xarrow}[7]{%
  % #1: below
  % #2: above
  % #3: arrow left
  % #4: arrow right
  % #5: space left 
  % #6: space right
  % #7: math style 
  \sbox0{$\ifx#7\scriptstyle\scriptscriptstyle\else\scriptstyle\fi#5#1#6\m@th$}%
  \sbox2{$\ifx#7\scriptstyle\scriptscriptstyle\else\scriptstyle\fi#5#2#6\m@th$}%
  \sbox4{$#7\dabar@\m@th$}%
  \dimen@=\wd0 %
  \ifdim\wd2 >\dimen@
    \dimen@=\wd2 %   
  \fi
  \count@=2 %
  \def\da@bars{\dabar@\dabar@}%
  \@whiledim\count@\wd4<\dimen@\do{%
    \advance\count@\@ne
    \expandafter\def\expandafter\da@bars\expandafter{%
      \da@bars
      \dabar@ 
    }%
  }%  
  \mathrel{#3}%
  \mathrel{%   
    \mathop{\da@bars}\limits
    \ifx\\#1\\%
    \else
      _{\copy0}%
    \fi
    \ifx\\#2\\%
    \else
      ^{\copy2}%
    \fi
  }%   
  \mathrel{#4}%
}
\makeatother
% \DeclareMathOperator{\rg}{rg}

\usepackage{mathtools}
\DeclarePairedDelimiter{\paren}{(}{)}
\DeclarePairedDelimiter{\abs}{\lvert}{\rvert}
\DeclarePairedDelimiter{\norm}{\lVert}{\rVert}
\DeclarePairedDelimiter{\innerproduct}{\langle}{\rangle}
\newcommand{\Of}[2]{{\operatorname{#1}} {\paren*{#2}}}
\newcommand{\of}[2]{{{{#1}} {\paren*{#2}}}}

\DeclareMathOperator{\Shim}{Shim}
\DeclareMathOperator{\sgn}{sgn}
\DeclareMathOperator{\fdeg}{fdeg}
\DeclareMathOperator{\SL}{SL}
\DeclareMathOperator{\slLie}{\mathfrak{s}\mathfrak{l}}
\DeclareMathOperator{\soLie}{\mathfrak{s}\mathfrak{o}}
\DeclareMathOperator{\spLie}{\mathfrak{s}\mathfrak{p}}
\DeclareMathOperator{\glLie}{\mathfrak{g}\mathfrak{l}}
\newcommand{\pn}[1]{{\color{ForestGreen} \sf PN: [#1]}}
\DeclareMathOperator{\Mp}{Mp}
\DeclareMathOperator{\Mat}{Mat}
\DeclareMathOperator{\GL}{GL}
\DeclareMathOperator{\Gr}{Gr}
\DeclareMathOperator{\GU}{GU}
\def\gl{\mathfrak{g}\mathfrak{l}}
\DeclareMathOperator{\odd}{odd}
\DeclareMathOperator{\even}{even}
\DeclareMathOperator{\GO}{GO}
\DeclareMathOperator{\good}{good}
\DeclareMathOperator{\bad}{bad}
\DeclareMathOperator{\PGO}{PGO}
\DeclareMathOperator{\htt}{ht}
\DeclareMathOperator{\height}{height}
\DeclareMathOperator{\Ass}{Ass}
\DeclareMathOperator{\coheight}{coheight}
\DeclareMathOperator{\GSO}{GSO}
\DeclareMathOperator{\SO}{SO}
\DeclareMathOperator{\so}{\mathfrak{s}\mathfrak{o}}
\DeclareMathOperator{\su}{\mathfrak{s}\mathfrak{u}}
\DeclareMathOperator{\ad}{ad}
% \DeclareMathOperator{\sc}{sc}
\DeclareMathOperator{\Ad}{Ad}
\DeclareMathOperator{\disc}{disc}
\DeclareMathOperator{\inv}{inv}
\DeclareMathOperator{\Pic}{Pic}
\DeclareMathOperator{\uc}{uc}
\DeclareMathOperator{\Cl}{Cl}
\DeclareMathOperator{\Clf}{Clf}
\DeclareMathOperator{\Hom}{Hom}
\DeclareMathOperator{\hol}{hol}
\DeclareMathOperator{\Heis}{Heis}
\DeclareMathOperator{\Haar}{Haar}
\DeclareMathOperator{\h}{h}
\def\sp{\mathfrak{s}\mathfrak{p}}
\DeclareMathOperator{\heis}{\mathfrak{h}\mathfrak{e}\mathfrak{i}\mathfrak{s}}
\DeclareMathOperator{\End}{End}
\DeclareMathOperator{\JL}{JL}
\DeclareMathOperator{\image}{image}
\DeclareMathOperator{\red}{red}
\def\div{\operatorname{div}}
\def\eps{\varepsilon}
\def\cHom{\mathcal{H}\operatorname{om}}
\DeclareMathOperator{\Ops}{Ops}
\DeclareMathOperator{\Symb}{Symb}
\def\boldGL{\mathbf{G}\mathbf{L}}
\def\boldSO{\mathbf{S}\mathbf{O}}
\def\boldU{\mathbf{U}}
\DeclareMathOperator{\hull}{hull}
\DeclareMathOperator{\LL}{LL}
\DeclareMathOperator{\PGL}{PGL}
\DeclareMathOperator{\class}{class}
\DeclareMathOperator{\lcm}{lcm}
\DeclareMathOperator{\spann}{span}
\DeclareMathOperator{\Exp}{Exp}
\DeclareMathOperator{\ext}{ext}
\DeclareMathOperator{\Ext}{Ext}
\DeclareMathOperator{\Tor}{Tor}
\DeclareMathOperator{\et}{et}
\DeclareMathOperator{\tor}{tor}
\DeclareMathOperator{\loc}{loc}
\DeclareMathOperator{\tors}{tors}
\DeclareMathOperator{\pf}{pf}
\DeclareMathOperator{\smooth}{smooth}
\DeclareMathOperator{\prin}{prin}
\DeclareMathOperator{\Kl}{Kl}
\newcommand{\kbar}{\mathchar'26\mkern-9mu k}
\DeclareMathOperator{\der}{der}
% \DeclareMathOperator{\abs}{abs}
\DeclareMathOperator{\Sub}{Sub}
\DeclareMathOperator{\Comp}{Comp}
\DeclareMathOperator{\Err}{Err}
\DeclareMathOperator{\dom}{dom}
\DeclareMathOperator{\radius}{radius}
\DeclareMathOperator{\Fitt}{Fitt}
\DeclareMathOperator{\Sel}{Sel}
\DeclareMathOperator{\rad}{rad}
\DeclareMathOperator{\id}{id}
\DeclareMathOperator{\Center}{Center}
\DeclareMathOperator{\Der}{Der}
\DeclareMathOperator{\U}{U}
% \DeclareMathOperator{\norm}{norm}
\DeclareMathOperator{\trace}{trace}
\DeclareMathOperator{\Equid}{Equid}
\DeclareMathOperator{\Feas}{Feas}
\DeclareMathOperator{\bulk}{bulk}
\DeclareMathOperator{\tail}{tail}
\DeclareMathOperator{\sys}{sys}
\DeclareMathOperator{\atan}{atan}
\DeclareMathOperator{\temp}{temp}
\DeclareMathOperator{\Asai}{Asai}
\DeclareMathOperator{\glob}{glob}
\DeclareMathOperator{\Kuz}{Kuz}
\DeclareMathOperator{\Irr}{Irr}
\newcommand{\rsL}{ \frac{ L^{(R)}(\Pi \times \Sigma, \std, \frac{1}{2})}{L^{(R)}(\Pi \times \Sigma, \Ad, 1)}  }
\DeclareMathOperator{\GSp}{GSp}
\DeclareMathOperator{\PGSp}{PGSp}
\DeclareMathOperator{\BC}{BC}
\DeclareMathOperator{\Ann}{Ann}
\DeclareMathOperator{\Gen}{Gen}
\DeclareMathOperator{\SU}{SU}
\DeclareMathOperator{\PGSU}{PGSU}
% \DeclareMathOperator{\gen}{gen}
\DeclareMathOperator{\PMp}{PMp}
\DeclareMathOperator{\PGMp}{PGMp}
\DeclareMathOperator{\PB}{PB}
\DeclareMathOperator{\ind}{ind}
\DeclareMathOperator{\Jac}{Jac}
\DeclareMathOperator{\jac}{jac}
\DeclareMathOperator{\im}{im}
\DeclareMathOperator{\Aut}{Aut}
\DeclareMathOperator{\Int}{Int}
\DeclareMathOperator{\PSL}{PSL}
\DeclareMathOperator{\co}{co}
\DeclareMathOperator{\irr}{irr}
\DeclareMathOperator{\prim}{prim}
\DeclareMathOperator{\bal}{bal}
\DeclareMathOperator{\baln}{bal}
\DeclareMathOperator{\dist}{dist}
\DeclareMathOperator{\RS}{RS}
\DeclareMathOperator{\Ram}{Ram}
\DeclareMathOperator{\Sob}{Sob}
\DeclareMathOperator{\Sol}{Sol}
\DeclareMathOperator{\soc}{soc}
\DeclareMathOperator{\nt}{nt}
\DeclareMathOperator{\mic}{mic}
\DeclareMathOperator{\Gal}{Gal}
\DeclareMathOperator{\st}{st}
\DeclareMathOperator{\std}{std}
\DeclareMathOperator{\diag}{diag}
\DeclareMathOperator{\Sym}{Sym}
\DeclareMathOperator{\gr}{gr}
\DeclareMathOperator{\aff}{aff}
\DeclareMathOperator{\Dil}{Dil}
\DeclareMathOperator{\Lie}{Lie}
\DeclareMathOperator{\Symp}{Symp}
\DeclareMathOperator{\Stab}{Stab}
\DeclareMathOperator{\St}{St}
\DeclareMathOperator{\stab}{stab}
\DeclareMathOperator{\codim}{codim}
\DeclareMathOperator{\linear}{linear}
\newcommand{\git}{/\!\!/}
\DeclareMathOperator{\geom}{geom}
\DeclareMathOperator{\spec}{spec}
\def\O{\operatorname{O}}
\DeclareMathOperator{\Au}{Aut}
\DeclareMathOperator{\Fix}{Fix}
\DeclareMathOperator{\Opp}{Op}
\DeclareMathOperator{\opp}{op}
\DeclareMathOperator{\Size}{Size}
\DeclareMathOperator{\Save}{Save}
% \DeclareMathOperator{\ker}{ker}
\DeclareMathOperator{\coker}{coker}
\DeclareMathOperator{\sym}{sym}
\DeclareMathOperator{\mean}{mean}
\DeclareMathOperator{\elliptic}{ell}
\DeclareMathOperator{\nilpotent}{nil}
\DeclareMathOperator{\hyperbolic}{hyp}
\DeclareMathOperator{\newvector}{new}
\DeclareMathOperator{\new}{new}
\DeclareMathOperator{\full}{full}
\newcommand{\qr}[2]{\left( \frac{#1}{#2} \right)}
\DeclareMathOperator{\unr}{u}
\DeclareMathOperator{\ram}{ram}
% \DeclareMathOperator{\len}{len}
\DeclareMathOperator{\fin}{fin}
\DeclareMathOperator{\cusp}{cusp}
\DeclareMathOperator{\curv}{curv}
\DeclareMathOperator{\rank}{rank}
\DeclareMathOperator{\rk}{rk}
\DeclareMathOperator{\pr}{pr}
\DeclareMathOperator{\Transform}{Transform}
\DeclareMathOperator{\mult}{mult}
\DeclareMathOperator{\Eis}{Eis}
\DeclareMathOperator{\reg}{reg}
\DeclareMathOperator{\sing}{sing}
\DeclareMathOperator{\alt}{alt}
\DeclareMathOperator{\irreg}{irreg}
\DeclareMathOperator{\sreg}{sreg}
\DeclareMathOperator{\Wd}{Wd}
\DeclareMathOperator{\Weil}{Weil}
\DeclareMathOperator{\Th}{Th}
\DeclareMathOperator{\Sp}{Sp}
\DeclareMathOperator{\Ind}{Ind}
\DeclareMathOperator{\Res}{Res}
\DeclareMathOperator{\ini}{in}
\DeclareMathOperator{\ord}{ord}
\DeclareMathOperator{\osc}{osc}
\DeclareMathOperator{\fluc}{fluc}
\DeclareMathOperator{\size}{size}
\DeclareMathOperator{\ann}{ann}
\DeclareMathOperator{\equ}{eq}
\DeclareMathOperator{\res}{res}
\DeclareMathOperator{\pt}{pt}
\DeclareMathOperator{\src}{source}
\DeclareMathOperator{\Zcl}{Zcl}
\DeclareMathOperator{\Func}{Func}
\DeclareMathOperator{\Map}{Map}
\DeclareMathOperator{\Frac}{Frac}
\DeclareMathOperator{\Frob}{Frob}
\DeclareMathOperator{\ev}{eval}
\DeclareMathOperator{\pv}{pv}
\DeclareMathOperator{\eval}{eval}
\DeclareMathOperator{\Spec}{Spec}
\DeclareMathOperator{\Speh}{Speh}
\DeclareMathOperator{\Spin}{Spin}
\DeclareMathOperator{\GSpin}{GSpin}
\DeclareMathOperator{\Specm}{Specm}
\DeclareMathOperator{\Sphere}{Sphere}
\DeclareMathOperator{\Sqq}{Sq}
\DeclareMathOperator{\Ball}{Ball}
\DeclareMathOperator\Cond{\operatorname{Cond}}
\DeclareMathOperator\proj{\operatorname{proj}}
\DeclareMathOperator\Swan{\operatorname{Swan}}
\DeclareMathOperator{\Proj}{Proj}
\DeclareMathOperator{\bPB}{{\mathbf P}{\mathbf B}}
\DeclareMathOperator{\Projm}{Projm}
\DeclareMathOperator{\Tr}{Tr}
\DeclareMathOperator{\Type}{Type}
\DeclareMathOperator{\Prop}{Prop}
\DeclareMathOperator{\vol}{vol}
\DeclareMathOperator{\covol}{covol}
\DeclareMathOperator{\Rep}{Rep}
\DeclareMathOperator{\Cent}{Cent}
\DeclareMathOperator{\val}{val}
\DeclareMathOperator{\area}{area}
\DeclareMathOperator{\nr}{nr}
\DeclareMathOperator{\CM}{CM}
\DeclareMathOperator{\CH}{CH}
\DeclareMathOperator{\tr}{tr}
\DeclareMathOperator{\characteristic}{char}
\DeclareMathOperator{\supp}{supp}


\theoremstyle{plain} \newtheorem{theorem} {Theorem} \newtheorem{conjecture} [theorem] {Conjecture} \newtheorem{corollary} [theorem] {Corollary} \newtheorem{proposition} [theorem] {Proposition} \newtheorem{fact} [theorem] {Fact}
\theoremstyle{definition} \newtheorem{definition} [theorem] {Definition} \newtheorem{hypothesis} [theorem] {Hypothesis} \newtheorem{assumptions} [theorem] {Assumptions}
\newtheorem{example} [theorem] {Example}
\newtheorem{assertion}[theorem] {Assertion}
\newtheorem{note}[theorem] {Note}
\newtheorem{conclusion}[theorem] {Conclusion}
\newtheorem{claim}            {Claim}
\newtheorem{homework} {Homework}
\newtheorem{exercise} {Exercise}  \newtheorem{question}[theorem] {Question}    \newtheorem{answer} {Answer}  \newtheorem{problem} {Problem}    \newtheorem{remark} [theorem] {Remark}
\newtheorem{notation} [theorem]           {Notation}
\newtheorem{terminology}[theorem]            {Terminology}
\newtheorem{convention}[theorem]            {Convention}
\newtheorem{motivation}[theorem]            {Motivation}


\newtheoremstyle{itplain} % name
{6pt}                    % Space above
{5pt\topsep}                    % Space below
{\itshape}                   % Body font
{}                           % Indent amount
{\itshape}                   % Theorem head font
{.}                          % Punctuation after theorem head
{5pt plus 1pt minus 1pt}                       % Space after theorem head
% {.5em}                       % Space after theorem head
{}  % Theorem head spec (can be left empty, meaning ‘normal’)

% \theoremstyle{mytheoremstyle}


\theoremstyle{itplain} %--default
% \theoremheaderfont{\itshape}
% \newtheorem{lemma}{Lemma}
\newtheorem{lemma}[theorem]{Lemma}
% \newtheorem{lemma}{Lemma}[subsubsection]

\newtheorem*{lemma*}{Lemma}
\newtheorem*{proposition*}{Proposition}
\newtheorem*{definition*}{Definition}
\newtheorem*{example*}{Example}

\newtheorem*{results*}{Results}
\newtheorem{results} [theorem] {Results}


\usepackage[displaymath,textmath,sections,graphics]{preview}
\PreviewEnvironment{align*}
\PreviewEnvironment{multline*}
\PreviewEnvironment{tabular}
\PreviewEnvironment{verbatim}
\PreviewEnvironment{lstlisting}
\PreviewEnvironment*{frame}
\PreviewEnvironment*{alert}
\PreviewEnvironment*{emph}
\PreviewEnvironment*{textbf}

 \numberwithin{theorem}{section} \numberwithin{equation}{section}

\usepackage{xr-hyper}
\externaldocument{20230907T142550--generating-functions-asymptotics}
\externaldocument{20230907T143130--fourier-and-mellin-transforms}
\externaldocument{20230907T143521--complex-analysis-preliminaries}
\externaldocument{20230907T144219--bernoulli-numbers-euler-maclaurin}
\externaldocument{20230919T144827--discrete-fourier-transform}

\begin{document}
 
These are notes for an ongoing Fall 2023 course on the Riemann zeta function and its generalizations, $L$-functions, taught with Sergey Arkhipov.  These notes will be filled in as we go.

\section{References thus far}
\begin{itemize}
\item Generating functions and asymptotics: \cite[\S5.2]{MR2172781}
\item Mellin transform and asymptotics: \cite{zagier-mellin}
\item Zeros of zeta and the prime number theorem: \cite{Dav80}, \cite{MR2061214}, \href{https://terrytao.wordpress.com/2014/12/09/254a-notes-2-complex-analytic-multiplicative-number-theory/}{Tao's notes}
\end{itemize}

\section{Outlines thus far}
(Some links here refer to external files; I'll have to think of a good way to notate that.)
\begin{itemize}
\item Tuesday, 29 Aug: parts of \S\ref{sec:cj4unj5r3k}; \S\ref{sec:cj4vkkg5ah}, \S\ref{sec:cj4vkkg9es}
\item Thursday, 31 Aug: \S\ref{sec:cj4unj3scf}, \S\ref{sec:cj4unj4gyo}
\item Friday, 1 Sep: \S\ref{sec:cj4unj04kx} and \S\ref{sec:cj4unj06mw}
\item Tuesday, 5 Sep: \S\ref{sec:cj4unjziaz}
\item Thursday, 7 Sep: \S\ref{sec:cj4vkkark8}, \S\ref{sec:cj4vkfaant}
\item Tuesday, 12 Sep and Thursday, 14 Sep: Fourier transform, Pontryagin duality, Poisson summation (see \href{20230919T144827--discrete-fourier-transform.pdf}{these notes} and the book \cite{MR3289059})
\item Tuesday, 19 Sep: Euler products, relevance of zeta zeros to asymptotics \S\ref{sec:cj4020t3kt}
\item Thursday, 21 Sep: harmonic functions (external \S\ref{sec:cj41z468dg}), approximate factorizations of holomorphic functions (external \S\ref{sec:cj41z5fwla}), crude control over zeta zeros (\S\ref{sec:cj410a1fzy}) 
\end{itemize}


\section{Course notes that I've since split off into separate files}
\begin{itemize}
\item \href{20230907T143521--complex-analysis-preliminaries.pdf}{Complex-analytic preliminaries}
\item \href{20230907T142550--generating-functions-asymptotics.pdf}{Generating functions and asymptotics}
\item \href{20230907T143130--fourier-and-mellin-transforms.pdf}{Fourier and Mellin transforms}
\item \href{20230907T144219--bernoulli-numbers-euler-maclaurin.pdf}{Bernoulli numbers and Euler--Maclaurin summation}
\end{itemize}


\newpage


\section{Background}\label{sec:cj4unj5r3k}

\subsection{General notation}
$\mathbb{R}^+ := (0,\infty)$.

\subsection{Asymptotic notation}
We use the equivalent notations
\begin{equation*}
  A = \O(B), \qquad A \ll B,
  \qquad B \gg A
\end{equation*}
to denote that
\begin{equation*}
  \lvert A \rvert \leq C \lvert B \rvert
\end{equation*}
for some ``constant'' $C$.  The precise meaning of ``constant'' will either be specified or clear from context.


\newpage
\subsection{Definition and basic properties of $\zeta$: overview}
The Riemann zeta function is defined for a complex number $s$ by the series
\begin{equation*}
  \zeta (s) = \sum_{n \geq 1} \frac{1}{n^s }.
\end{equation*}
\begin{lemma}
  The series converges absolutely for $\Re(s) > 1$, uniformly for $\Re(s) \geq 1 + \eps$ for each $\eps > 0$.
\end{lemma}
\begin{proof}
  Using the identity
  \begin{equation*}
    \left\lvert \frac{1}{n^s} \right\rvert = \frac{1}{n^{\Re(s)}},
  \end{equation*}
  we reduce to the case that $s$ is real, in which this is a familiar consequence of the integral test.
\end{proof}

Our first main goal in the course is to explain the following basic facts.
\begin{theorem}
  The Riemann zeta function admits a meromorphic continuation to the entire complex plane.  It is holomorphic away from a simple pole at $s = 1$, where it has residue $1$.  It admits a functional equation relating $\zeta (s)$ to $\zeta (1-s)$.
\end{theorem}

One historical motivation for considering the zeta function at complex arguments comes from the prime number theorem.
\begin{theorem}[Prime number theorem]
  Let $\pi(x) := \# \left\{ \text{primes } p \leq x \right\}$ denote the prime counting function.  Then
  \begin{equation*}
    \frac{\pi(x)}{x / \log x} \rightarrow 1
    \text{ as } x \rightarrow \infty.
  \end{equation*}
\end{theorem}
This is related to the following analytic fact concerning the zeros of the zeta function.
\begin{theorem}[Prime number theorem, formulated in terms of $\zeta$]\label{theorem:cj3vp9l79t}
  We have $\zeta(s) = 0$ only if $\Re(s) < 1$.
\end{theorem}
\begin{remark}
  Even the statement of Theorem~\ref{theorem:cj3vp9l79t} is not clear without knowing the meromorphic continuation of $\zeta$.  This may offer some motivation for understanding the latter.
\end{remark}
We expect stronger nonvanishing properties:
\begin{conjecture}[Riemann Hypothesis]
  We have $\zeta(s) = 0$ only if $\Re(s) < 1/2$.
\end{conjecture}
This corresponds to a conjectural stronger form of the prime number theorem, namely that
\begin{equation*}
  \pi(x) = \int_2^x \frac{t}{\log t} \, d t
  + \O (x^{1/2} \log x).
\end{equation*}

\newpage

\newpage
\section{Basic analytic properties of $\zeta$}\label{sec:cj4unjziaz}


\subsection{Meromorphic continuation and evaluation at negative integers}
Take
\begin{equation*}
  h(y) = \frac{1}{e^y - 1}.
\end{equation*}
The function $y h(y)$ extends to a holomorphic function of $y$ on the disc $\{y \in \mathbb{C} : \lvert y \rvert < 2 \pi \}$, so $h$ is represented for small $y > 0$ by an absolutely convergent Laurent series of the following form:
\begin{equation*}
  h(y) = \frac{1}{y} + \sum_{n = 1 }^\infty \frac{B_n  }{ n !} y^{n - 1}.
\end{equation*}
Here the $B_n$ are complex coefficients, called the \emph{Bernoulli numbers} (see \href{20230907T144219--bernoulli-numbers-euler-maclaurin.pdf}{this note}).  On the other hand, $h$ decays rapidly (like $\O(y^N)$ for any fixed $N$) as $y \rightarrow \infty$.  By the analysis we've now seen many times, we deduce that the Mellin transform $H(s)$ of $h(y)$ converges absolutely for $\Re(s) > 1$, where it defines a holomorphic function, and extends to a meromorphic function on the complex plane whose only poles are simple poles at $s = 1$ (corresponding to the $1/y$ term in the asymptotic expansion as $y \rightarrow 0$) and at $s = -n$ for each $n \in \mathbb{Z}_{\geq 0}$, with residue given by $1$ in the former case and by $B_{n+1} / (n+!)!$ in the latter.

On the other hand, we can rewrite
\begin{equation*}
  h(y) = \frac{e^{- y}}{ 1 - e^{- y}}
  = e^{- y } + e^{- 2 y } + e^{- 3 y} + \dotsb,
\end{equation*}
giving
\begin{equation*}
  H(s) = \int_{\mathbb{R}^+ } y^s \sum_{n = 1 }^\infty e^{- n y } \,d^\times y.
\end{equation*}
The following calculations will show that the doubly integral/sum converges absolutely for $\Re(s) > 1$, so we may rearrange it as
\begin{equation*}
  \sum_{n = 1 }^\infty \int_{\mathbb{R}^+ } y^s e^{- n y } \,d^\times y.
\end{equation*}
The inner integral may be simplified by the substitution $y \mapsto y /n$.  This has no effect on the measure $d^\times y$, but replaces $y^s$ by $n^{-s} y^s$, giving
\begin{equation*}
  \sum_{n = 1 }^\infty n^{- s} \int_{\mathbb{R}^+ } y^s e^{- y } \,d^\times y
  = \zeta(s) \Gamma(s).
\end{equation*}
We see in this way that the $\zeta$ function admits a meromorphic continuation.  Since the $\Gamma$-function \href{https://www.proofwiki.org/wiki/Zeroes_of_Gamma_Function}{does not vanish}, we deduce that the only pole of $\zeta$ is the simple one at $s=1$, with residue $s=0$.  Also, since we have computed the residues of both $\Gamma(s)$ and $\zeta(s) \Gamma(s)$ at the nonpositive integers, we may calculate in this way the values of $\zeta(-n)$ for each $n \in \mathbb{Z}_{\geq 0}$:
\begin{equation*}
  \zeta(-n) = (-1)^n \frac{B_{n + 1}}{ n + 1}.
\end{equation*}

\subsection{Another perspective on the meromorphic continuation}\label{sec:cj4vkkark8}
(TODO: think of better section titles)

Given $f : \mathbb{R} \rightarrow \mathbb{C}$ of sufficient decay at infinity, we define $g : \mathbb{R}^+ \rightarrow \mathbb{C} $ by
\begin{equation*}
g(y) := \sum_{n \in \mathbb{Z} } f(n y).
\end{equation*}
We assume henceforth that $f$ lies in the \href{https://en.wikipedia.org/wiki/Schwartz_space}{Schwartz space} $\mathcal{S}(\mathbb{R})$.

The asymptotics of $g(y)$ are described as follows.
\begin{lemma}
  Let $f \in \mathcal{S}(R)$.
  \begin{enumerate}
\item 
  As $y \rightarrow \infty$, we have
  \begin{equation}\label{eq:cj4vkfq4ku}
g(y) = f(0) + \O(y^{-N})
  \end{equation}
  for each fixed $N$.
\item As $y \rightarrow 0$, we have
  \begin{equation}\label{eq:cj4vkfrkjf}
g(y) = y^{-1} \int_{\mathbb{R} } f(x) \, d x  + \O (y^{N})
  \end{equation}
  for each fixed $N$.
\end{enumerate}
\end{lemma}
\begin{proof}
The first estimate \eqref{eq:cj4vkfq4ku} is an easy exercise using the definition of the Schwartz space.  The second estimate \eqref{eq:cj4vkfrkjf} may be proved either via Euler--Maclaurin summation (see external \S\ref{sec:cj4vkfaant}) or Poisson summation (see external \S\ref{sec:cj4vkkp26q}).
\end{proof}

Assuming that $f(0)$ and $\int f$ are nonzero, it follows that the Mellin transform of $g$ does not converge absolutely at any point.  We can still define a regularized Mellin transform
\begin{equation*}
G(s) = \int_{\mathbb{R}^+ }^{\reg} y^s g (y) \,d^\times y,
\end{equation*}
like in \S\ref{sec:cj4unj4gyo}, by splitting the integral into two pieces (e.g, the contributions of $(0,1)$ and $(1,\infty$)), meromorphically continuing each piece, and then summing the results on their domain of overlap (if any).

\begin{lemma}
  $G(s)$ defines a meromorphic function on the complex plane whose only poles at simple ones:
  \begin{itemize}
\item at $s = 0$, with residue $-f(0)$, and
\item at $s = 1$, with residue $\int_{\mathbb{R}} f(x) \,d x $.
\end{itemize}
\end{lemma}
\begin{proof}
  Let's carry this out in detail, applying the recipe of external \S\ref{sec:cj4vkf4yvq}.
  \begin{itemize}
  \item We see from \eqref{eq:cj4vkfq4ku} that the integral
    \begin{equation*}
      G_+(s) := \int_1^\infty y^s g(y) \,d^\times y
    \end{equation*}
    converges absolutely for $\Re(s) < 0$ and extends to a meromorphic function of $s$, whose only pole is a simple pole at $s = 0$ of residue $-f(0)$.  Indeed,
    \begin{equation*}
      G_+(s) = \int_1^\infty y^s \left(g (y) - f(0) \right) \,d^\times y
      + f(0)
      \underbrace
      {
        \int_1^\infty y^{s} \,d^\times y
      }_{
        -1/s
      },
    \end{equation*}
    where the first integral on the right hand side converges absolutely for all $s$.
  \item We see from \eqref{eq:cj4vkfrkjf} that the integral
    \begin{equation*}
      G_-(s) := \int_0^1 y^s g(y) \,d^\times y
    \end{equation*}
    converges absolutely for $\Re(s) > 1$ and extends to a meromorphic function of $s$, whose only pole is a simple pole at $s = 1$ of residue $I_f$, where $I_f := \int_{\mathbb{R} } f (x) \, d x$.  Indeed,
    \begin{equation*}
      G_-(s) = \int_0^1 y^s \left( g (y) - y^{-1} I_f \right) \,d^\times y + I_f \underbrace
      {
        \int_0^1 y^{s-1} \,d^\times y
      }_{
        1/(s-1)
      }.
    \end{equation*}
  \end{itemize}
\end{proof}

We denote henceforth by $\mathcal{F} f$ the (normalized) Fourier transform
  \begin{equation*}
\mathcal{F} f (\xi) := \int_{x \in \mathbb{R} } f(x) e^{- 2 \pi i x \xi } \, d x.
\end{equation*}

\begin{lemma}\label{lemma:cj4vkjxr7m}
  Write $G(s) = G_f(s)$ to indicate the dependence upon $f$.  Then we have the following functional equation:
  \begin{equation*}
G_f(s) = G_{\mathcal{F} f}(1-s).
  \end{equation*}
\end{lemma}
\begin{proof}
  This is a consequence of the Poisson summation formula (see external \eqref{eq:cj4vkktmw8}) 
  \begin{equation*}
    \sum_{n \in \mathbb{Z} } f (n y) = y^{-1} \sum_{n \in \mathbb{Z} } \mathcal{F} f (n / y).
  \end{equation*}
  Writing $g_f$ to indicate the dependence of $g$ upon $f$, the above identity reads
  \begin{equation*}
    g_f(y) = y^{-1} g_{\mathcal{F} f}(1/y).
  \end{equation*}
  Taking the (regularized) Mellin transform of both sides yields
  \begin{equation*}
    G_f(s) = \int_{\mathbb{R}^+}^{\reg} y^{s-1} g_{\mathcal{F} f}(1/y) \,d^\times y.
  \end{equation*}
  To evaluate this last integral, we substitute $y \mapsto 1/y$, which leaves the measure $d^\times y$ invariant (and is unaffected by the regularization).  This gives
  \begin{equation*}
    G_f(s) = \int_{\mathbb{R}^+}^{\reg} y^{1-s} g_{\mathcal{F} f}(y) \,d^\times y = G_{\mathcal{F} f}(1-s),
  \end{equation*}
  as required.
\end{proof}


We suppose henceforth that $f$ is even.  This is without much loss of generality -- any function can be written as a sum of even and odd functions, and if $f$ is odd, then $g$ vanishes identically.

\begin{exercise}
  Let $f \in \mathcal{S}(\mathbb{R})$ be even.  Show that the Mellin transform
  \begin{equation*}
F(s) := \int y^s f (y) \,d^\times y
  \end{equation*}
  converges initially for $\Re(s) > 0$ and extends to a meromorphic function on the complex plane, whose only poles are simple ones at $s = - 2n$ for $n \in \mathbb{Z}_{\geq 0}$ with residue
  \begin{equation*}
\res_{s = - 2 n} F(s) = \frac{1}{(2 n)!} f^{(2 n)}(0).
  \end{equation*}
  [Use Taylor's theorem with remainder, and note that the odd Taylor coefficients vanish in view of the evenness assumption on $f$.]
\end{exercise}
\begin{example}
  Take $f(x) := e^{- \pi x^2}$.  Then
  \begin{equation}\label{eq:cj4vkj6hym}
F(s) = \pi^{- s / 2 } \Gamma (s / 2),
  \end{equation}
  as one sees by substituting $y \mapsto \sqrt{y}$ and then $y \mapsto y / \pi$ in the defining integral.  This has poles in the expected places.
\end{example}

For $f$ even, we have
\begin{equation*}
g(y) = f(0) + 2 \sum_{n = 1 }^\infty f(n y).
\end{equation*}
Using that the constant function $1$ has vanishing regularized Mellin transform (for reasons similar to external Exercise \ref{exercise:cj4ss5ia8l}), we see that $G(s)$ admits the following absolutely convergent integral representation for $\Re(s) > 1$:
\begin{equation*}
G(s) = 2 \int_{\mathbb{R}^+ } y^s \sum_{n = 1 }^\infty f (n y ) \,d^\times y.
\end{equation*}
By substituting $y \mapsto y / n$, we see that
\begin{equation*}
G(s) = 2 \zeta(s) F(s), \qquad F(s) := \int y^s f(y) \,d^\times y.
\end{equation*}
This gives another proof of the meromorphic behavior of $\zeta$.  TODO: say more.

We can also deduce the functional equation:
\begin{theorem}
  We have
  \begin{equation*}
\xi(s) := \pi^{- s / 2 } \Gamma (s/ 2 ) \zeta(s) = \xi (1 - s).
\end{equation*}
\end{theorem}
\begin{proof}
  Take $f(x) := e^{- \pi x^2}$.  By \eqref{eq:cj4vkj6hym}, we then have $G_f(s) = \xi(s)$.  On the other hand, by the well-known \href{https://mathworld.wolfram.com/FourierTransformGaussian.html}{formula} for the Fourier transform of a Gaussian, we have $\mathcal{F} f = f$.  By Lemma \ref{lemma:cj4vkjxr7m}, it follows that $G_f (s) = G_f (1 - s)$.  The claimed formula follows.
\end{proof}

\subsection{Euler product}\label{sec:cj4020t3kt}
The link between the zeta function and the prime numbers is given by the following alternative formula:
\begin{lemma}
  For $\Re(s) > 1$, we have
\begin{equation*}
\zeta(s) = \prod_p \frac{1}{1 - p^{- s}},
\end{equation*}
where the product ranges over the primes $p$ and converges absolutely.  In particular,
\begin{equation*}
\zeta(s) \neq 0 \text{ for } \Re(s) > 1.
\end{equation*}
Quantitatively, for $\sigma > 1$, we have
\begin{equation}\label{eq:cj41z53x7f}
\lvert \zeta(\sigma + it) \rvert \geq \zeta(1 + \sigma)^{-1}.
\end{equation}
\end{lemma}
Here we say that an infinite product \emph{converges absolutely} if it converges to the same nonzero value for any rearrangement of the factors.
\begin{proof}
  We first observe that for $S$ any finite set of primes, we have
  \begin{equation*}
    \prod_{p \in S}
    \frac{1}{1 - p^{- s}}
    = \prod_{p \in S}
    \sum_{k \geq 0} \frac{1}{ p^{k s}}
    =
    \sum_{n \in \mathbb{N}(S)}
    \frac{1}{n^s},
  \end{equation*}
  where $\mathbb{N}(S)$ denotes the set of natural numbers each of whose prime factors lie in $S$.  This identity follows from the fundamental theorem of arithmetic, which implies that every element of $\mathbb{N}(S)$ may be written uniquely as a product of powers of elements of $S$.  We now take the limit as $S \rightarrow \mathbb{N}$.

  To see that the product converges to a nonzero limit, we can apply the following general fact: if $\sum_p |a_p| < \infty$, then the product $\prod_{p} (1 + a_p)$ converges to a limit, and that limit is nonzero provided that each factor $1 + a_p$ is nonzero.  Applying this facts with $a_p = p^{-s}$ gives what was claimed.


  For the quantitative estimate \eqref{eq:cj41z53x7f}, we observe that
  \begin{equation*}
     \left\lvert \frac{1}{\zeta(\sigma + it)}  \right\rvert
     = \prod_p \left\lvert 1 - p^{-s} \right\rvert
     \leq \prod_p \left(  1 + p^{-\sigma} \right)
     \leq \sum_n n^{-\sigma} = \zeta(1 + \sigma).
  \end{equation*}
\end{proof}


As a basic illustration of the resulting link between the zeta function and the primes, we verify the following:
\begin{lemma}
  Let $s > 1$ be real, tending to $1$.  Then
  \begin{equation*}
\sum_p \frac{1}{ p^s } = \log \frac{1}{1 - s} + \O(1).
\end{equation*}
\end{lemma}
\begin{proof}
  Using the Taylor series for the logarithm, we compute
  \begin{equation}\label{eq:cj402w3a5w}
    \log \zeta (s) = \sum_p \log \left( \frac{1}{1 - p^{- s}} \right)
    =
    \sum_p \sum_{k \geq 1} \frac{p^{- k s }}{ k}.
  \end{equation}
  We leave it to the reader to check that
  \begin{equation}
\sum_{k \geq 2} \frac{p^{- k s}}{ k }  = \O(1)
  \end{equation}
  in the indicated range.
\end{proof}
\begin{corollary}
$\sum_p 1 / p = \infty$.
\end{corollary}

\subsection{Logarithmic derivatives}
While the logarithm $\log \zeta(s)$ is multi-valued, the logarithmic derivative
\begin{equation*}
  \left( \log \zeta(s) \right) '
  = \frac{\zeta '}{\zeta } (s) 
\end{equation*}
is single-valued.  It may be computed by differentiating \eqref{eq:cj402w3a5w}:
\begin{equation*}
- \frac{\zeta '}{ \zeta } (s) = \sum _p (\log p) \sum_{k \geq 1} p^{- k s} = \sum_n \frac{\Lambda(n) }{n^s},
\end{equation*}
where the \emph{von Mangoldt function} $\Lambda(n)$ is defined by
\begin{equation*}
\Lambda(n) :=
\begin{cases}
\log p & \text{ if } n = p ^k, \\
0 & \text{ otherwise.}
\end{cases}
\end{equation*}


The general theme of this course suggests a relationship between the asymptotics of $\Lambda(n)$ and the meromorphic behavior of $- \zeta ' / \zeta$.  The poles of the latter may be described as follows.  Suppose that $\rho \in \mathbb{C}$ is a point at which $\zeta$ has a zero of order $m \in \mathbb{Z}$, i.e., $\zeta(s) \sim c (s - \rho)^m$ for some $c \in \mathbb{C}^\times$ as $s \rightarrow \rho$.  Then
\begin{equation*}
- \frac{\zeta'}{\zeta}(s) \sim \frac{- m }{s - \rho}.
\end{equation*}
Therefore the poles of $- \zeta ' / \zeta $ are all simple, and occur at
\begin{itemize}
\item the (unique simple) pole $\rho = 1$ of $\zeta$, with residue $1$, and
\item at the zeros $\rho$ of $\zeta$, with residue $-m$, where $m \geq 1$ denotes the order of the zero.
\end{itemize}

Suppose, for instance, that we wish to understand the asymptotics of a sum like
\begin{equation*}
S := \sum_n \Lambda (n) f \left( \frac{n}{x} \right).
\end{equation*}
Here $f \in C_c^\infty (\mathbb{R}^+ )$ is fixed, while we think of $x > 1$ as a parameter tending off to infinity.  To do this, we expand $f$ using its Mellin transform, which gives
\begin{equation*}
f \left( \frac{n}{x} \right) = \int_{(c)} F (s) \left( \frac{x}{n} \right)^s \, \frac{d s}{2 \pi i}.
\end{equation*}
Here $c \in \mathbb{R}$ is any real number.  The double sum/integral obtained by inserting this expansion into the definition of $S$ converges absolutely for $c > 1$, where we may rearrange it to obtain
\begin{equation*}
S = \int_{(c)} F (s) x^s \underbrace
{
  \sum_n 
\frac{\Lambda (n)}{ n^s } 
}_{
  - \frac{\zeta ' }{\zeta }(s)
}\, \frac{d s}{2 \pi i}.
\end{equation*}
We would now like to shift the contour to the left.

Working formally for the moment, we pick up contributiones from the poles of $\zeta$ at $1$ and its zeros $\rho$, giving
\begin{equation*}
  S =
  F(1) x
  -
  \sum_{\rho}
  F(\rho) x^\rho
  +
  \dotsb,
\end{equation*}
where $\dotsb$ denotes the ``remainder'' term given by an integral like $S$, but with $c$ large and negative, while $\rho$ ranges over the zeros of $\zeta$ with $\Re(\rho) > c$.  To implement such an argument rigorously requires some control over the growth of zeta and its number of zeros in vertical directions, which we will address shortly.  Ignoring such details for the moment, one can see from such expansions why the zeros of $\zeta$ play such a large role in describing the asymptotics of sums over primes such as $S$.  For instance, if we know that each $\rho$ has real part strictly less than $1$, then the corresponding term $F(\rho) x^\rho$ has lesser magnitude than $x$ as $x \rightarrow \infty$.

To make arguments like the above precise, we need some control over the growth of $ \zeta ' / \zeta $ in vertical strips, and also over the number of zeros.

\subsection{Crude control over zeros}\label{sec:cj410a1fzy}
For the following, we closely follow the presentation of part 2 of \href{these notes}{https://terrytao.wordpress.com/2014/12/09/254a-notes-2-complex-analytic-multiplicative-number-theory/}.  In particular, the following results can all be sharpened a bit using the functional equation, but we won't pause to do so here.

We begin by giving a simple formula for $\zeta$ valid in the critical strip $0 < \Re(s) < 1$.
\begin{lemma}
  For $\Re(s) > 0$, we have
  \begin{equation}\label{eq:cj41j93u44}
    \zeta(s) - \frac{1}{s - 1}
    = \sum_{n = 1}^\infty \int_n^{n+1}  \left( \frac{1}{n^s }  - \frac{1}{y^s} \right)   \, d y.
\end{equation}
  
\end{lemma}
\begin{proof}
  We have
  \begin{equation*}
    \int_1^\infty \frac{1}{y^s } \, d y = \frac{1}{s - 1},
  \end{equation*}
  thus
  \begin{align*}
    \zeta(s) - \frac{1}{s - 1}
    &=
      \sum_{n = 1}^\infty \frac{1}{n^s } - \int_1^\infty \frac{1}{y^s } \, d y \\
    &=
    \sum_{n = 1}^\infty \left( \frac{1}{n^s } - \int_n^{n+1} \frac{1}{y^s} \, d y \right) \\
 &=
    \sum_{n = 1}^\infty \int_n^{n+1}  \left( \frac{1}{n^s }  - \frac{1}{y^s} \right)   \, d y.
  \end{align*}
\end{proof}

From this, we conclude a basic but useful upper bound for $\zeta$ in the critical strip.  (Much sharper forms are available, but we won't need them immediately.)
\begin{lemma}
  For $s = \sigma + it $ with $\sigma \geq \eps > 0$ and $\lvert t \rvert \geq 2$, we have
  \begin{equation}\label{eq:cj41kdcbj7}
\log \lvert \zeta(s) \rvert \leq \log (\lvert t \rvert) + \O_{\eps}(1).
  \end{equation}
\end{lemma}
\begin{proof}
  We use the mean value theorem to estimate the integrand in \eqref{eq:cj41j93u44}:
  \begin{equation*}
    \frac{1}{n^s }  - \frac{1}{y^s}
    = - s \int_{t = n}^y t^{-s - 1} \, d t
    \ll s n^{-s - 1}.
  \end{equation*}
  It follows that for $\Re(s) \geq \eps > 0$, we have
  \begin{equation*}
    \zeta(s) - \frac{1}{s - 1}
    \ll s \sum_{n = 1}^\infty n^{-\eps - 1}
    \ll_\eps s.
  \end{equation*}
  We conclude by taking the logarithm of this estimate.  We can discard the contribution of the pole because we have assumed that $\lvert t \rvert \geq 2$.
\end{proof}

With these results in hand, we can appeal to general properties of holomorphic functions to deduce the following pair of propositions:
\begin{proposition}\label{proposition:cj41z9de5r}
  Fix $\eps > 0$.
  \begin{enumerate}[(i)]
  \item\label{enumerate:cj41z9jrqm} For $t_0 \in \mathbb{R}$, the number of zeros of $\zeta$ in the rectangle $\{\sigma + i t: \eps \leq \sigma \leq 1, \lvert t - T \rvert \leq 1\}$ is $\O_\eps (\log (3 + \lvert t_0 \rvert))$.
  \item\label{enumerate:cj41z9j28y} For $s \in \mathbb{C}$ with $\Re(s) \geq \eps$, we have
    \begin{equation}\label{eq:cj41z82pyp}
      \frac{\zeta ' }{\zeta }(s)
      =
      \frac{1}{s - 1} - \sum_{
        \lvert \rho - s \rvert \leq \eps/2
      }
      \frac{1}{s - \rho } + \O_{\eps } \left( \log (3 + \lvert t \rvert) \right).
    \end{equation}
  \end{enumerate}
\end{proposition}
\begin{proof}
  We apply Lemma \ref{lemma:cj41z5hd4a} of \href{20230907T143521--complex-analysis-preliminaries.pdf}{these notes} to the function $f(s) := (s - 1) \zeta(s)$, the basepoint $s_0 := 2 + i t_0$, the radius $r := 2 - \eps/4$ and the disc $D = \{s : \lvert s - s_0 \rvert < r\}$, where $\eps > 0$ is sufficiently small.  The Euler product estimate \eqref{eq:cj41z53x7f} gives $\zeta(s_0) \asymp 1$. For $s$ with $\lvert s - s_0\rvert = r$, we have by \eqref{eq:cj41kdcbj7} $\lvert f(s) \rvert \leq M \lvert f(s_0) \rvert$ for some $M \ll \log(3 + \lvert t_0 \rvert)$.  This gives \eqref{enumerate:cj41z9jrqm}.

  For \eqref{enumerate:cj41z9j28y}, we argue as in \eqref{enumerate:cj41z9jrqm} with $t_0 := \Im(s)$.  We use the bound for the number of zeros to see that \eqref{eq:cj41z82pyp} doesn't change if we modify the sum by including or excluding any zeros that are at least some fixed distance away from $s$.  This leads to the claimed estimate.
\end{proof}


\subsection{Contour shifting and an explicit formula}
We can now carry out the promised contour shifting arguments.
\begin{proposition}
  Let $f \in C_c^\infty(\mathbb{R})$, with Mellin transform $F(s) = \int_{\mathbb{R}^+} y^s f(y) \,d^\times y$.  Let $x > 1$ and $\eps_1 > 0$.  Then there exists $\eps \in (\eps_1/2,\eps_1)$ so that
  \begin{equation*}
    \sum_n \Lambda (n) f \left( \frac{n}{x} \right)
    = F(1) x + \sum_{\rho : \Re(x) > \eps } F(\rho) x^{\rho}
    +
    \O_{\eps} \left( x^\eps \mathcal{E}_\eps \right),
  \end{equation*}
  where
  \begin{equation*}
\mathcal{E}_{\eps} := \int_{t \in \mathbb{R} }
      F(\eps + i t) \log (3 + |t|) \, d t.
    \end{equation*}
\end{proposition}
Note that since $f$ is smooth, the quantity $\mathcal{E}_{\eps}$ is finite (see external Lemma \ref{lemma:cj4ungtm6r} of \href{20230907T143130--fourier-and-mellin-transforms.pdf}{these notes}).

\begin{proof}
The follows by a contour shifting argument from \eqref{eq:cj41z82pyp}, and will be presented ``next time''.  Remark that we can reduce formally to the case $x=1$, by incorporating dilation by $x$ into $f$.
\end{proof}


\bibliography{refs}{} \bibliographystyle{plain}
\end{document}
