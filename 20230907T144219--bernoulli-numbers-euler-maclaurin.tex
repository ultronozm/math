\documentclass[reqno]{amsart} \input{common.tex}

\usepackage{xr-hyper}
\externaldocument{2023-introduction-to-zeta-and-l-functions}
\externaldocument{20230907T142550--generating-functions-asymptotics}
\externaldocument{20230907T161723--poisson-summation}

\begin{document}

\title{Bernoulli numbers and Euler--Maclaurin summation}
\begin{abstract}
  Part of the course notes for \href{2023-introduction-to-zeta-and-l-functions.pdf}{this course}.
\end{abstract}



Here I'm recording my own notes, closely following the presentation of \cite[\S4]{zagier-mellin} but filling in the details.  Only a small part of this material was covered in lecture.


\section{Definitions and statements}

The Bernoulli numbers $B_n$ may be defined (up to conventions) by the generating function
\begin{equation*}
  \frac{1}{e^t - 1}
  =
  \frac{1}{t + t^2/2 + \dotsb }
  = \sum_{n \geq 0} \frac{B_n }{ n !} t^{n - 1}.
\end{equation*}
\begin{exercise}
Use the methods of \href{20230907T142550--generating-functions-asymptotics.pdf}{these previous notes} to determine the leading order asymptotics of $B_n$ as $n \rightarrow \infty$.
\end{exercise}
The Bernoulli polynomials $B_n(x)$ may be defined by the generating function
\begin{equation}\label{eq:cj4sr7qmvd}
\frac{e^{x t} }{ e^t - 1 } = \sum_{n \geq 0} \frac{B_n (x) }{n !} t^{n-1},
\end{equation}
so that $B_n = B_n(0)$.


\begin{example}
  By direct calculation, we have
  \begin{equation}\label{eq:cj4sr7wpkw}
    B_0(x) = 1.
  \end{equation}
  \begin{equation*}
    B_1(x) = x - \frac{1}{2},
  \end{equation*}
  \begin{equation*}
    B_2(x) = x^2 - x + \frac{1}{6}.
  \end{equation*}
\end{example}

\begin{notation}
  For $x \in \mathbb{R}$, we denote by $\lfloor x \rfloor \in \mathbb{Z}$ its integer part and $\{x\} \in [0,1)$ its fractional part, which are characterized by the identity
  \begin{equation*}
x = \lfloor x \rfloor + \{x\}.
\end{equation*}
\end{notation}


\begin{theorem}[Euler--Maclaurin formula]\label{theorem:cj4vked1rh}
  For integers $a < b$, and a smooth function $f$ on the real line,
\begin{align*}
  \int_{a}^b f (x) \, d x = \frac{f (a) }{2} + \sum_{n = a + 1}^{b - 1} f (n)
  + \frac{f (b)}{2} 
  + \sum_{n = 1}^{N - 1 } \frac{(- 1 )^n B_{n + 1 }}{ (n + 1 )!}
  \left( f^{(n) } (b ) - f^{(n )} (a) \right) \\
  + (- 1)^N \int_{a}^b f^{(N)} (x) \frac{B_N (\{x\})}{N!} \, d x.
\end{align*}
\end{theorem}

\begin{example}
  In the case $N=1$, this specializes to the ``trapezoid rule'':
  \begin{equation*}
    \int_a^b f (x) \, d x
    =
    \frac{f (a) }{2} + \sum_{n = a + 1}^{b - 1} f (n)
  + \frac{f (b)}{2} 
  - \int_{a}^b \{ x - \tfrac{1}{2} \} f' (x)\, d x.
\end{equation*}
\end{example}
\begin{exercise}
Prove \ref{sec:cj4vkej9k1}.  [Reduce to the case $(a,b) = (0,1)$, then integrate by parts.]
\end{exercise}

\section{Applications}

Theorem \ref{theorem:cj4vked1rh} can be useful for asymptotic analysis.  The reference \cite[\S4]{zagier-mellin} contains many interesting examples.  Here is a less interesting example, which can be studied in other ways (e.g., using Poisson summation).

\subsection{Asymptotics of Riemann sums}\label{sec:cj4vkfaant}
(Compare with external \S\ref{sec:cj4vkkp26q}.)

Let $f : \mathbb{R} \rightarrow \mathbb{C}$ be smooth and compactly-supported, or more generally, an element of the \href{https://en.wikipedia.org/wiki/Schwartz_space}{Schwartz space} $\mathcal{S}(\mathbb{R})$.  Define the function $g : \mathbb{R}^+ \rightarrow \mathbb{C}$ by
\begin{equation*}
g(y) := \sum_{n \in \mathbb{Z} } f (n y ).
\end{equation*}
Then as $y \rightarrow 0$, the quantity $y g(y)$ is a Riemann sum, hence converges to $\int f$.  We will verify that this convergence is quite rapid:
\begin{lemma}
  For each fixed $N \geq 0$, we have for $y \in (0,1)$ the estimate
  \begin{equation*}
g(y) = y^{-1}  \int f + \O(y^N).
  \end{equation*}
\end{lemma}
\begin{proof}
  We take the limit of Theorem \ref{theorem:cj4vked1rh} as $(a,b) \rightarrow (-\infty, +\infty)$, applied to $f_y(x) := f(x y)$; we can do this because $f$ and all its derivatives vanish at $\infty$.  Since $f_y^{(n)}(x) = y^n f^{(n)}(x)$, we obtain for each $N \geq 1$,
  \begin{equation*}
\int_{\mathbb{R} } f (x y ) \, d x = \sum_{n \in \mathbb{Z} } f (n y) + (- y)^N \int_{\mathbb{R} } f^{(N)} (x y) \frac{B_N (\{x\})}{N!} \, d x.
  \end{equation*}
  We see that the left hand side is $y^{-1} \int f$ via the substitution $x \mapsto x/y$.  We estimate the integral on the right hand side using that $f^{(N)}$ has finite $L^1$-norm and $B_N (\{x\})$ is bounded, due to the periodicity of $\{x\}$.  The required conclusion follows.
\end{proof}

\begin{remark}
It's more interesting to analyze $\sum_{n \geq 0 } f(n y)$ this way.  We refer again to \cite[\S4]{zagier-mellin}.
\end{remark}


\section{Proofs}\label{sec:cj4vkej9k1}

Here we record the proof of Theorem \ref{theorem:cj4vked1rh} (again, following the presentation of \cite[\S4]{zagier-mellin}), after some preliminaries.


\begin{lemma}\label{lemma:cj4sr702cf}
  For $n \geq 1$, we have
  \begin{equation*}
B_n'(x) = n B_{n-1}(x).
\end{equation*}
\end{lemma}
\begin{proof}
  It suffices to show that
  \begin{equation*}
    \sum_{n \geq 0} \frac{B_n'(x)}{n!} t^{n-1}
    =
\sum_{n \geq 1} \frac{B_{n-1}(x)}{(n-1)!} t^{n-1},
\end{equation*}
or equivalently, that
\begin{equation*}
\frac{d}{d x } \sum_{n \geq 0 } \frac{B_n(x)}{n!} t^{n-1} = t \sum_{n \geq 0 } \frac{B_n (x)}{ n !} t^{n-1},
\end{equation*}
which is clear from the definition \eqref{eq:cj4sr7qmvd}.
\end{proof}

\begin{lemma}\label{lemma:cj4vkas8qw}
The Bernoulli polynomials are given by the formula
\begin{equation*}
B_n (x) = \sum_{r = 0}^n \binom{n}{r} B_r x^{n - r}.
\end{equation*}
\end{lemma}
\begin{proof}
  The claimed formula holds when $x=0$ because the right hand side simplifies to $B_n$.  It also holds when $n = 0$ in view of \eqref{eq:cj4sr7wpkw}.  We may thus suppose by induction on $n \geq 1$ that the claimed formula holds for smaller values of $n$.

  Since we have checked that the claimed formula holds when $x=0$, we reduce to verifying that the derivatives of the two sides coincide.  The derivative of the right hand side is
  \begin{equation*}
\sum_{r = 0 }^{n - 1 } \frac{n!}{r! (n-r)!} (n-r) B_r x^{n-1-r},
\end{equation*}
which simplifies to
\begin{equation*}
 n \sum_{r = 0 }^{n - 1 } \frac{(n-1)!}{r! (n-1-r)!} B_r x^{n-1-r}.
\end{equation*}
By our inductive hypothesis, this simplifies to $n B_{n-1}(x)$, which, in view of Lemma \ref{lemma:cj4sr702cf}, coincides with the derivative of the left hand side, $B_n'(x)$.
\end{proof}

\begin{lemma}\label{lemma:cj4ssc4a2b}
  For each $a \geq 0$, we have
  \begin{equation*}
    \int_a^{a+1} B_n(x) \, d x = a^n.
  \end{equation*}
  Here we interpret $0^0 = 1$.
\end{lemma}
\begin{proof}
  It is enough to check that
  \begin{equation*}
    \int_{a}^{a+1}
    \sum_{n \geq 0} \frac{B_n (x) }{ n ! } t^n
    \,d x
    = \sum_{n \geq 0} \frac{a^n}{n!} t^n,
  \end{equation*}
  or equivalently, in view of \eqref{eq:cj4sr7qmvd}, that
  \begin{equation}\label{eq:cj4ssco825}
    \int_{a}^{a + 1 }\frac{t e^{x t} }{ e^t - 1 } \, d x = e^{a t}.
  \end{equation}
  Indeed, we have
  \begin{align*}
    \int_a^{a + 1 } e^{x t } \, d x
    &=
    \frac{1}{t} e^{x t }|_{x=a}^{a+1} \\
    &=
    \frac{e^{(a+1) t} - e^{a t}}{t} \\
    &=
      e^{a t}
    \frac{e^t - 1}{t},
  \end{align*}
  from which the formula \eqref{eq:cj4ssco825} follows.
\end{proof}


\begin{lemma}\label{lemma:cj4vkapsln}
  We have
  \begin{equation*}
B_{n} (x + 1 ) - B_n (x) = n x^{n - 1}.
\end{equation*}
\end{lemma}
\begin{proof}
  Combine Lemmas \ref{lemma:cj4sr702cf} and \ref{lemma:cj4ssc4a2b}.
\end{proof}

\begin{lemma}
  For $n \geq 2$, we have
  \begin{equation*}
B_n (1) = B_n.
\end{equation*}
\end{lemma}
\begin{proof}
  We need to check that
  \begin{equation*}
    B_n(1) - B_n(0) = 0.
  \end{equation*}
  By the fundamental theorem of calculus, it is enough to check that
  \begin{equation*}
    \int_0^1 B_n'(x) \, d x = 0.
  \end{equation*}
  By Lemma \ref{lemma:cj4sr702cf}, it is equivalent to check that
  \begin{equation*}
    \int_0^1 B_{n - 1 } (x) \, d x = 0.
  \end{equation*}
  This follows from Lemma \ref{lemma:cj4ssc4a2b}.  We use here that $n-1 \geq 1$, which ensures that $0^{n-1} = 0$.
\end{proof}

\begin{lemma}
  For a smooth function $f$ on $[0,1]$, we have
  \begin{multline*}
    \int_0^1 f^{(n)} (x) \frac{B_n (x) }{n !} \, d x
    = - \int_0^1 f^{(n + 1 )} (x)
    \frac{B_{n + 1 } (x) }{ (n + 1 )! } \, d x \\
    +
    \begin{cases}
      \frac{f (0) + f (1) }{2} & \text{ if } n = 0, \\
      \frac{B_{n + 1}}{(n + 1 )!} \left( f^{(n)} (1) - f^{(n)} (0)  \right) &  \text{ if } n \geq 1.
    \end{cases}
  \end{multline*}
\end{lemma}
\begin{proof}
  We integrate by parts.  Take
  \begin{equation*}
u = f^{(n)}(x), \qquad v = \frac{B_{n+1}(x)}{(n+1)!}.
  \end{equation*}
  Then, by Lemma \ref{lemma:cj4sr702cf},
  \begin{equation*}
u \, d v  = f^{(n )} (x)  \frac{B_n (x) }{n !}.
  \end{equation*}
  The first term on the right hand side of the claimed formula is the integral of $v \, d u$.  The second term is $u v |_{0}^1$, using here that $B_m(1) = B_m(0) = B_m$ for $m \geq 2$, while $B_1(0) = 1/2 = - B_1(0)$.
\end{proof}

By inducting on $N$ and using that $B_0(x) = 1$, we obtain:
\begin{multline*}
  \int_0^1 f(x) \, d x
  = (-1)^N \int_0^1 f^{(N + 1 )} (x)
  \frac{B_{N + 1 } (x) }{ (N + 1 )! } \, d x \\
  +
  \frac{f (0) + f (1) }{2}
  + 
  \sum_{n = 1}^{N - 1} \frac{B_{n + 1}}{(n + 1 )!} \left( f^{(n)} (1) - f^{(n)} (0)  \right).
\end{multline*}
By replacing $f$ with its shifts by integer and summing, we obtain Theorem \ref{theorem:cj4vked1rh}.




\bibliography{refs}{} \bibliographystyle{plain}
\end{document}
