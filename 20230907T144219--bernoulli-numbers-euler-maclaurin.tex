\documentclass[reqno]{amsart} \usepackage{graphicx, amsmath, amssymb, amsfonts, amsthm, stmaryrd, amscd}
\usepackage[usenames, dvipsnames]{xcolor}
\usepackage{tikz}
% \usepackage{tikzcd}
% \usepackage{comment}

% \let\counterwithout\relax
% \let\counterwithin\relax
% \usepackage{chngcntr}

\usepackage{enumerate}
% \usepackage{enumitem}
% \usepackage{times}
\usepackage[normalem]{ulem}
% \usepackage{minted}
% \usepackage{xypic}
% \usepackage{color}


% \usepackage{silence}
% \WarningFilter{latex}{Label `tocindent-1' multiply defined}
% \WarningFilter{latex}{Label `tocindent0' multiply defined}
% \WarningFilter{latex}{Label `tocindent1' multiply defined}
% \WarningFilter{latex}{Label `tocindent2' multiply defined}
% \WarningFilter{latex}{Label `tocindent3' multiply defined}
\usepackage{hyperref}
% \usepackage{navigator}


% \usepackage{pdfsync}
\usepackage{xparse}


\usepackage[all]{xy}
\usepackage{enumerate}
\usetikzlibrary{matrix,arrows,decorations.pathmorphing}



\makeatletter
\newcommand*{\transpose}{%
  {\mathpalette\@transpose{}}%
}
\newcommand*{\@transpose}[2]{%
  % #1: math style
  % #2: unused
  \raisebox{\depth}{$\m@th#1\intercal$}%
}
\makeatother


\makeatletter
\newcommand*{\da@rightarrow}{\mathchar"0\hexnumber@\symAMSa 4B }
\newcommand*{\da@leftarrow}{\mathchar"0\hexnumber@\symAMSa 4C }
\newcommand*{\xdashrightarrow}[2][]{%
  \mathrel{%
    \mathpalette{\da@xarrow{#1}{#2}{}\da@rightarrow{\,}{}}{}%
  }%
}
\newcommand{\xdashleftarrow}[2][]{%
  \mathrel{%
    \mathpalette{\da@xarrow{#1}{#2}\da@leftarrow{}{}{\,}}{}%
  }%
}
\newcommand*{\da@xarrow}[7]{%
  % #1: below
  % #2: above
  % #3: arrow left
  % #4: arrow right
  % #5: space left 
  % #6: space right
  % #7: math style 
  \sbox0{$\ifx#7\scriptstyle\scriptscriptstyle\else\scriptstyle\fi#5#1#6\m@th$}%
  \sbox2{$\ifx#7\scriptstyle\scriptscriptstyle\else\scriptstyle\fi#5#2#6\m@th$}%
  \sbox4{$#7\dabar@\m@th$}%
  \dimen@=\wd0 %
  \ifdim\wd2 >\dimen@
    \dimen@=\wd2 %   
  \fi
  \count@=2 %
  \def\da@bars{\dabar@\dabar@}%
  \@whiledim\count@\wd4<\dimen@\do{%
    \advance\count@\@ne
    \expandafter\def\expandafter\da@bars\expandafter{%
      \da@bars
      \dabar@ 
    }%
  }%  
  \mathrel{#3}%
  \mathrel{%   
    \mathop{\da@bars}\limits
    \ifx\\#1\\%
    \else
      _{\copy0}%
    \fi
    \ifx\\#2\\%
    \else
      ^{\copy2}%
    \fi
  }%   
  \mathrel{#4}%
}
\makeatother
% \DeclareMathOperator{\rg}{rg}

\usepackage{mathtools}
\DeclarePairedDelimiter{\paren}{(}{)}
\DeclarePairedDelimiter{\abs}{\lvert}{\rvert}
\DeclarePairedDelimiter{\norm}{\lVert}{\rVert}
\DeclarePairedDelimiter{\innerproduct}{\langle}{\rangle}
\newcommand{\Of}[2]{{\operatorname{#1}} {\paren*{#2}}}
\newcommand{\of}[2]{{{{#1}} {\paren*{#2}}}}

\DeclareMathOperator{\Shim}{Shim}
\DeclareMathOperator{\sgn}{sgn}
\DeclareMathOperator{\fdeg}{fdeg}
\DeclareMathOperator{\SL}{SL}
\DeclareMathOperator{\slLie}{\mathfrak{s}\mathfrak{l}}
\DeclareMathOperator{\soLie}{\mathfrak{s}\mathfrak{o}}
\DeclareMathOperator{\spLie}{\mathfrak{s}\mathfrak{p}}
\DeclareMathOperator{\glLie}{\mathfrak{g}\mathfrak{l}}
\newcommand{\pn}[1]{{\color{ForestGreen} \sf PN: [#1]}}
\DeclareMathOperator{\Mp}{Mp}
\DeclareMathOperator{\Mat}{Mat}
\DeclareMathOperator{\GL}{GL}
\DeclareMathOperator{\Gr}{Gr}
\DeclareMathOperator{\GU}{GU}
\def\gl{\mathfrak{g}\mathfrak{l}}
\DeclareMathOperator{\odd}{odd}
\DeclareMathOperator{\even}{even}
\DeclareMathOperator{\GO}{GO}
\DeclareMathOperator{\good}{good}
\DeclareMathOperator{\bad}{bad}
\DeclareMathOperator{\PGO}{PGO}
\DeclareMathOperator{\htt}{ht}
\DeclareMathOperator{\height}{height}
\DeclareMathOperator{\Ass}{Ass}
\DeclareMathOperator{\coheight}{coheight}
\DeclareMathOperator{\GSO}{GSO}
\DeclareMathOperator{\SO}{SO}
\DeclareMathOperator{\so}{\mathfrak{s}\mathfrak{o}}
\DeclareMathOperator{\su}{\mathfrak{s}\mathfrak{u}}
\DeclareMathOperator{\ad}{ad}
% \DeclareMathOperator{\sc}{sc}
\DeclareMathOperator{\Ad}{Ad}
\DeclareMathOperator{\disc}{disc}
\DeclareMathOperator{\inv}{inv}
\DeclareMathOperator{\Pic}{Pic}
\DeclareMathOperator{\uc}{uc}
\DeclareMathOperator{\Cl}{Cl}
\DeclareMathOperator{\Clf}{Clf}
\DeclareMathOperator{\Hom}{Hom}
\DeclareMathOperator{\hol}{hol}
\DeclareMathOperator{\Heis}{Heis}
\DeclareMathOperator{\Haar}{Haar}
\DeclareMathOperator{\h}{h}
\def\sp{\mathfrak{s}\mathfrak{p}}
\DeclareMathOperator{\heis}{\mathfrak{h}\mathfrak{e}\mathfrak{i}\mathfrak{s}}
\DeclareMathOperator{\End}{End}
\DeclareMathOperator{\JL}{JL}
\DeclareMathOperator{\image}{image}
\DeclareMathOperator{\red}{red}
\def\div{\operatorname{div}}
\def\eps{\varepsilon}
\def\cHom{\mathcal{H}\operatorname{om}}
\DeclareMathOperator{\Ops}{Ops}
\DeclareMathOperator{\Symb}{Symb}
\def\boldGL{\mathbf{G}\mathbf{L}}
\def\boldSO{\mathbf{S}\mathbf{O}}
\def\boldU{\mathbf{U}}
\DeclareMathOperator{\hull}{hull}
\DeclareMathOperator{\LL}{LL}
\DeclareMathOperator{\PGL}{PGL}
\DeclareMathOperator{\class}{class}
\DeclareMathOperator{\lcm}{lcm}
\DeclareMathOperator{\spann}{span}
\DeclareMathOperator{\Exp}{Exp}
\DeclareMathOperator{\ext}{ext}
\DeclareMathOperator{\Ext}{Ext}
\DeclareMathOperator{\Tor}{Tor}
\DeclareMathOperator{\et}{et}
\DeclareMathOperator{\tor}{tor}
\DeclareMathOperator{\loc}{loc}
\DeclareMathOperator{\tors}{tors}
\DeclareMathOperator{\pf}{pf}
\DeclareMathOperator{\smooth}{smooth}
\DeclareMathOperator{\prin}{prin}
\DeclareMathOperator{\Kl}{Kl}
\newcommand{\kbar}{\mathchar'26\mkern-9mu k}
\DeclareMathOperator{\der}{der}
% \DeclareMathOperator{\abs}{abs}
\DeclareMathOperator{\Sub}{Sub}
\DeclareMathOperator{\Comp}{Comp}
\DeclareMathOperator{\Err}{Err}
\DeclareMathOperator{\dom}{dom}
\DeclareMathOperator{\radius}{radius}
\DeclareMathOperator{\Fitt}{Fitt}
\DeclareMathOperator{\Sel}{Sel}
\DeclareMathOperator{\rad}{rad}
\DeclareMathOperator{\id}{id}
\DeclareMathOperator{\Center}{Center}
\DeclareMathOperator{\Der}{Der}
\DeclareMathOperator{\U}{U}
% \DeclareMathOperator{\norm}{norm}
\DeclareMathOperator{\trace}{trace}
\DeclareMathOperator{\Equid}{Equid}
\DeclareMathOperator{\Feas}{Feas}
\DeclareMathOperator{\bulk}{bulk}
\DeclareMathOperator{\tail}{tail}
\DeclareMathOperator{\sys}{sys}
\DeclareMathOperator{\atan}{atan}
\DeclareMathOperator{\temp}{temp}
\DeclareMathOperator{\Asai}{Asai}
\DeclareMathOperator{\glob}{glob}
\DeclareMathOperator{\Kuz}{Kuz}
\DeclareMathOperator{\Irr}{Irr}
\newcommand{\rsL}{ \frac{ L^{(R)}(\Pi \times \Sigma, \std, \frac{1}{2})}{L^{(R)}(\Pi \times \Sigma, \Ad, 1)}  }
\DeclareMathOperator{\GSp}{GSp}
\DeclareMathOperator{\PGSp}{PGSp}
\DeclareMathOperator{\BC}{BC}
\DeclareMathOperator{\Ann}{Ann}
\DeclareMathOperator{\Gen}{Gen}
\DeclareMathOperator{\SU}{SU}
\DeclareMathOperator{\PGSU}{PGSU}
% \DeclareMathOperator{\gen}{gen}
\DeclareMathOperator{\PMp}{PMp}
\DeclareMathOperator{\PGMp}{PGMp}
\DeclareMathOperator{\PB}{PB}
\DeclareMathOperator{\ind}{ind}
\DeclareMathOperator{\Jac}{Jac}
\DeclareMathOperator{\jac}{jac}
\DeclareMathOperator{\im}{im}
\DeclareMathOperator{\Aut}{Aut}
\DeclareMathOperator{\Int}{Int}
\DeclareMathOperator{\PSL}{PSL}
\DeclareMathOperator{\co}{co}
\DeclareMathOperator{\irr}{irr}
\DeclareMathOperator{\prim}{prim}
\DeclareMathOperator{\bal}{bal}
\DeclareMathOperator{\baln}{bal}
\DeclareMathOperator{\dist}{dist}
\DeclareMathOperator{\RS}{RS}
\DeclareMathOperator{\Ram}{Ram}
\DeclareMathOperator{\Sob}{Sob}
\DeclareMathOperator{\Sol}{Sol}
\DeclareMathOperator{\soc}{soc}
\DeclareMathOperator{\nt}{nt}
\DeclareMathOperator{\mic}{mic}
\DeclareMathOperator{\Gal}{Gal}
\DeclareMathOperator{\st}{st}
\DeclareMathOperator{\std}{std}
\DeclareMathOperator{\diag}{diag}
\DeclareMathOperator{\Sym}{Sym}
\DeclareMathOperator{\gr}{gr}
\DeclareMathOperator{\aff}{aff}
\DeclareMathOperator{\Dil}{Dil}
\DeclareMathOperator{\Lie}{Lie}
\DeclareMathOperator{\Symp}{Symp}
\DeclareMathOperator{\Stab}{Stab}
\DeclareMathOperator{\St}{St}
\DeclareMathOperator{\stab}{stab}
\DeclareMathOperator{\codim}{codim}
\DeclareMathOperator{\linear}{linear}
\newcommand{\git}{/\!\!/}
\DeclareMathOperator{\geom}{geom}
\DeclareMathOperator{\spec}{spec}
\def\O{\operatorname{O}}
\DeclareMathOperator{\Au}{Aut}
\DeclareMathOperator{\Fix}{Fix}
\DeclareMathOperator{\Opp}{Op}
\DeclareMathOperator{\opp}{op}
\DeclareMathOperator{\Size}{Size}
\DeclareMathOperator{\Save}{Save}
% \DeclareMathOperator{\ker}{ker}
\DeclareMathOperator{\coker}{coker}
\DeclareMathOperator{\sym}{sym}
\DeclareMathOperator{\mean}{mean}
\DeclareMathOperator{\elliptic}{ell}
\DeclareMathOperator{\nilpotent}{nil}
\DeclareMathOperator{\hyperbolic}{hyp}
\DeclareMathOperator{\newvector}{new}
\DeclareMathOperator{\new}{new}
\DeclareMathOperator{\full}{full}
\newcommand{\qr}[2]{\left( \frac{#1}{#2} \right)}
\DeclareMathOperator{\unr}{u}
\DeclareMathOperator{\ram}{ram}
% \DeclareMathOperator{\len}{len}
\DeclareMathOperator{\fin}{fin}
\DeclareMathOperator{\cusp}{cusp}
\DeclareMathOperator{\curv}{curv}
\DeclareMathOperator{\rank}{rank}
\DeclareMathOperator{\rk}{rk}
\DeclareMathOperator{\pr}{pr}
\DeclareMathOperator{\Transform}{Transform}
\DeclareMathOperator{\mult}{mult}
\DeclareMathOperator{\Eis}{Eis}
\DeclareMathOperator{\reg}{reg}
\DeclareMathOperator{\sing}{sing}
\DeclareMathOperator{\alt}{alt}
\DeclareMathOperator{\irreg}{irreg}
\DeclareMathOperator{\sreg}{sreg}
\DeclareMathOperator{\Wd}{Wd}
\DeclareMathOperator{\Weil}{Weil}
\DeclareMathOperator{\Th}{Th}
\DeclareMathOperator{\Sp}{Sp}
\DeclareMathOperator{\Ind}{Ind}
\DeclareMathOperator{\Res}{Res}
\DeclareMathOperator{\ini}{in}
\DeclareMathOperator{\ord}{ord}
\DeclareMathOperator{\osc}{osc}
\DeclareMathOperator{\fluc}{fluc}
\DeclareMathOperator{\size}{size}
\DeclareMathOperator{\ann}{ann}
\DeclareMathOperator{\equ}{eq}
\DeclareMathOperator{\res}{res}
\DeclareMathOperator{\pt}{pt}
\DeclareMathOperator{\src}{source}
\DeclareMathOperator{\Zcl}{Zcl}
\DeclareMathOperator{\Func}{Func}
\DeclareMathOperator{\Map}{Map}
\DeclareMathOperator{\Frac}{Frac}
\DeclareMathOperator{\Frob}{Frob}
\DeclareMathOperator{\ev}{eval}
\DeclareMathOperator{\pv}{pv}
\DeclareMathOperator{\eval}{eval}
\DeclareMathOperator{\Spec}{Spec}
\DeclareMathOperator{\Speh}{Speh}
\DeclareMathOperator{\Spin}{Spin}
\DeclareMathOperator{\GSpin}{GSpin}
\DeclareMathOperator{\Specm}{Specm}
\DeclareMathOperator{\Sphere}{Sphere}
\DeclareMathOperator{\Sqq}{Sq}
\DeclareMathOperator{\Ball}{Ball}
\DeclareMathOperator\Cond{\operatorname{Cond}}
\DeclareMathOperator\proj{\operatorname{proj}}
\DeclareMathOperator\Swan{\operatorname{Swan}}
\DeclareMathOperator{\Proj}{Proj}
\DeclareMathOperator{\bPB}{{\mathbf P}{\mathbf B}}
\DeclareMathOperator{\Projm}{Projm}
\DeclareMathOperator{\Tr}{Tr}
\DeclareMathOperator{\Type}{Type}
\DeclareMathOperator{\Prop}{Prop}
\DeclareMathOperator{\vol}{vol}
\DeclareMathOperator{\covol}{covol}
\DeclareMathOperator{\Rep}{Rep}
\DeclareMathOperator{\Cent}{Cent}
\DeclareMathOperator{\val}{val}
\DeclareMathOperator{\area}{area}
\DeclareMathOperator{\nr}{nr}
\DeclareMathOperator{\CM}{CM}
\DeclareMathOperator{\CH}{CH}
\DeclareMathOperator{\tr}{tr}
\DeclareMathOperator{\characteristic}{char}
\DeclareMathOperator{\supp}{supp}


\theoremstyle{plain} \newtheorem{theorem} {Theorem} \newtheorem{conjecture} [theorem] {Conjecture} \newtheorem{corollary} [theorem] {Corollary} \newtheorem{proposition} [theorem] {Proposition} \newtheorem{fact} [theorem] {Fact}
\theoremstyle{definition} \newtheorem{definition} [theorem] {Definition} \newtheorem{hypothesis} [theorem] {Hypothesis} \newtheorem{assumptions} [theorem] {Assumptions}
\newtheorem{example} [theorem] {Example}
\newtheorem{assertion}[theorem] {Assertion}
\newtheorem{note}[theorem] {Note}
\newtheorem{conclusion}[theorem] {Conclusion}
\newtheorem{claim}            {Claim}
\newtheorem{homework} {Homework}
\newtheorem{exercise} {Exercise}  \newtheorem{question}[theorem] {Question}    \newtheorem{answer} {Answer}  \newtheorem{problem} {Problem}    \newtheorem{remark} [theorem] {Remark}
\newtheorem{notation} [theorem]           {Notation}
\newtheorem{terminology}[theorem]            {Terminology}
\newtheorem{convention}[theorem]            {Convention}
\newtheorem{motivation}[theorem]            {Motivation}


\newtheoremstyle{itplain} % name
{6pt}                    % Space above
{5pt\topsep}                    % Space below
{\itshape}                   % Body font
{}                           % Indent amount
{\itshape}                   % Theorem head font
{.}                          % Punctuation after theorem head
{5pt plus 1pt minus 1pt}                       % Space after theorem head
% {.5em}                       % Space after theorem head
{}  % Theorem head spec (can be left empty, meaning ‘normal’)

% \theoremstyle{mytheoremstyle}


\theoremstyle{itplain} %--default
% \theoremheaderfont{\itshape}
% \newtheorem{lemma}{Lemma}
\newtheorem{lemma}[theorem]{Lemma}
% \newtheorem{lemma}{Lemma}[subsubsection]

\newtheorem*{lemma*}{Lemma}
\newtheorem*{proposition*}{Proposition}
\newtheorem*{definition*}{Definition}
\newtheorem*{example*}{Example}

\newtheorem*{results*}{Results}
\newtheorem{results} [theorem] {Results}


\usepackage[displaymath,textmath,sections,graphics]{preview}
\PreviewEnvironment{align*}
\PreviewEnvironment{multline*}
\PreviewEnvironment{tabular}
\PreviewEnvironment{verbatim}
\PreviewEnvironment{lstlisting}
\PreviewEnvironment*{frame}
\PreviewEnvironment*{alert}
\PreviewEnvironment*{emph}
\PreviewEnvironment*{textbf}



\usepackage{xr-hyper}
\externaldocument{2023-introduction-to-zeta-and-l-functions}
\externaldocument{20230907T142550--generating-functions-asymptotics}
\externaldocument{20230907T161723--poisson-summation}

\begin{document}

\title{Bernoulli numbers and Euler--Maclaurin summation}
\begin{abstract}
  Part of the course notes for \href{2023-introduction-to-zeta-and-l-functions.pdf}{this course}.
\end{abstract}



Here I'm recording my own notes, closely following the presentation of \cite[\S4]{zagier-mellin} but filling in the details.  Only a small part of this material was covered in lecture.


\section{Definitions and statements}

The Bernoulli numbers $B_n$ may be defined (up to conventions) by the generating function
\begin{equation*}
  \frac{1}{e^t - 1}
  =
  \frac{1}{t + t^2/2 + \dotsb }
  = \sum_{n \geq 0} \frac{B_n }{ n !} t^{n - 1}.
\end{equation*}
\begin{exercise}
Use the methods of \href{20230907T142550--generating-functions-asymptotics.pdf}{these previous notes} to determine the leading order asymptotics of $B_n$ as $n \rightarrow \infty$.
\end{exercise}
The Bernoulli polynomials $B_n(x)$ may be defined by the generating function
\begin{equation}\label{eq:cj4sr7qmvd}
\frac{e^{x t} }{ e^t - 1 } = \sum_{n \geq 0} \frac{B_n (x) }{n !} t^{n-1},
\end{equation}
so that $B_n = B_n(0)$.


\begin{example}
  By direct calculation, we have
  \begin{equation}\label{eq:cj4sr7wpkw}
    B_0(x) = 1.
  \end{equation}
  \begin{equation*}
    B_1(x) = x - \frac{1}{2},
  \end{equation*}
  \begin{equation*}
    B_2(x) = x^2 - x + \frac{1}{6}.
  \end{equation*}
\end{example}

\begin{notation}
  For $x \in \mathbb{R}$, we denote by $\lfloor x \rfloor \in \mathbb{Z}$ its integer part and $\{x\} \in [0,1)$ its fractional part, which are characterized by the identity
  \begin{equation*}
x = \lfloor x \rfloor + \{x\}.
\end{equation*}
\end{notation}


\begin{theorem}[Euler--Maclaurin formula]\label{theorem:cj4vked1rh}
  For integers $a < b$, and a smooth function $f$ on the real line,
\begin{align*}
  \int_{a}^b f (x) \, d x = \frac{f (a) }{2} + \sum_{n = a + 1}^{b - 1} f (n)
  + \frac{f (b)}{2} 
  + \sum_{n = 1}^{N - 1 } \frac{(- 1 )^n B_{n + 1 }}{ (n + 1 )!}
  \left( f^{(n) } (b ) - f^{(n )} (a) \right) \\
  + (- 1)^N \int_{a}^b f^{(N)} (x) \frac{B_N (\{x\})}{N!} \, d x.
\end{align*}
\end{theorem}

\begin{example}
  In the case $N=1$, this specializes to the ``trapezoid rule'':
  \begin{equation*}
    \int_a^b f (x) \, d x
    =
    \frac{f (a) }{2} + \sum_{n = a + 1}^{b - 1} f (n)
  + \frac{f (b)}{2} 
  - \int_{a}^b \{ x - \tfrac{1}{2} \} f' (x)\, d x.
\end{equation*}
\end{example}
\begin{exercise}
Prove \ref{sec:cj4vkej9k1}.  [Reduce to the case $(a,b) = (0,1)$, then integrate by parts.]
\end{exercise}

\section{Applications}

Theorem \ref{theorem:cj4vked1rh} can be useful for asymptotic analysis.  The reference \cite[\S4]{zagier-mellin} contains many interesting examples.  Here is a less interesting example, which can be studied in other ways (e.g., using Poisson summation).

\subsection{Asymptotics of Riemann sums}\label{sec:cj4vkfaant}
(Compare with external \S\ref{sec:cj4vkkp26q}.)

Let $f : \mathbb{R} \rightarrow \mathbb{C}$ be smooth and compactly-supported, or more generally, an element of the \href{https://en.wikipedia.org/wiki/Schwartz_space}{Schwartz space} $\mathcal{S}(\mathbb{R})$.  Define the function $g : \mathbb{R}^+ \rightarrow \mathbb{C}$ by
\begin{equation*}
g(y) := \sum_{n \in \mathbb{Z} } f (n y ).
\end{equation*}
Then as $y \rightarrow 0$, the quantity $y g(y)$ is a Riemann sum, hence converges to $\int f$.  We will verify that this convergence is quite rapid:
\begin{lemma}
  For each fixed $N \geq 0$, we have for $y \in (0,1)$ the estimate
  \begin{equation*}
g(y) = y^{-1}  \int f + \O(y^N).
  \end{equation*}
\end{lemma}
\begin{proof}
  We take the limit of Theorem \ref{theorem:cj4vked1rh} as $(a,b) \rightarrow (-\infty, +\infty)$, applied to $f_y(x) := f(x y)$; we can do this because $f$ and all its derivatives vanish at $\infty$.  Since $f_y^{(n)}(x) = y^n f^{(n)}(x)$, we obtain for each $N \geq 1$,
  \begin{equation*}
\int_{\mathbb{R} } f (x y ) \, d x = \sum_{n \in \mathbb{Z} } f (n y) + (- y)^N \int_{\mathbb{R} } f^{(N)} (x y) \frac{B_N (\{x\})}{N!} \, d x.
  \end{equation*}
  We see that the left hand side is $y^{-1} \int f$ via the substitution $x \mapsto x/y$.  We estimate the integral on the right hand side using that $f^{(N)}$ has finite $L^1$-norm and $B_N (\{x\})$ is bounded, due to the periodicity of $\{x\}$.  The required conclusion follows.
\end{proof}

\begin{remark}
It's more interesting to analyze $\sum_{n \geq 0 } f(n y)$ this way.  We refer again to \cite[\S4]{zagier-mellin}.
\end{remark}


\section{Proofs}\label{sec:cj4vkej9k1}

Here we record the proof of Theorem \ref{theorem:cj4vked1rh} (again, following the presentation of \cite[\S4]{zagier-mellin}), after some preliminaries.


\begin{lemma}\label{lemma:cj4sr702cf}
  For $n \geq 1$, we have
  \begin{equation*}
B_n'(x) = n B_{n-1}(x).
\end{equation*}
\end{lemma}
\begin{proof}
  It suffices to show that
  \begin{equation*}
    \sum_{n \geq 0} \frac{B_n'(x)}{n!} t^{n-1}
    =
\sum_{n \geq 1} \frac{B_{n-1}(x)}{(n-1)!} t^{n-1},
\end{equation*}
or equivalently, that
\begin{equation*}
\frac{d}{d x } \sum_{n \geq 0 } \frac{B_n(x)}{n!} t^{n-1} = t \sum_{n \geq 0 } \frac{B_n (x)}{ n !} t^{n-1},
\end{equation*}
which is clear from the definition \eqref{eq:cj4sr7qmvd}.
\end{proof}

\begin{lemma}\label{lemma:cj4vkas8qw}
The Bernoulli polynomials are given by the formula
\begin{equation*}
B_n (x) = \sum_{r = 0}^n \binom{n}{r} B_r x^{n - r}.
\end{equation*}
\end{lemma}
\begin{proof}
  The claimed formula holds when $x=0$ because the right hand side simplifies to $B_n$.  It also holds when $n = 0$ in view of \eqref{eq:cj4sr7wpkw}.  We may thus suppose by induction on $n \geq 1$ that the claimed formula holds for smaller values of $n$.

  Since we have checked that the claimed formula holds when $x=0$, we reduce to verifying that the derivatives of the two sides coincide.  The derivative of the right hand side is
  \begin{equation*}
\sum_{r = 0 }^{n - 1 } \frac{n!}{r! (n-r)!} (n-r) B_r x^{n-1-r},
\end{equation*}
which simplifies to
\begin{equation*}
 n \sum_{r = 0 }^{n - 1 } \frac{(n-1)!}{r! (n-1-r)!} B_r x^{n-1-r}.
\end{equation*}
By our inductive hypothesis, this simplifies to $n B_{n-1}(x)$, which, in view of Lemma \ref{lemma:cj4sr702cf}, coincides with the derivative of the left hand side, $B_n'(x)$.
\end{proof}

\begin{lemma}\label{lemma:cj4ssc4a2b}
  For each $a \geq 0$, we have
  \begin{equation*}
    \int_a^{a+1} B_n(x) \, d x = a^n.
  \end{equation*}
  Here we interpret $0^0 = 1$.
\end{lemma}
\begin{proof}
  It is enough to check that
  \begin{equation*}
    \int_{a}^{a+1}
    \sum_{n \geq 0} \frac{B_n (x) }{ n ! } t^n
    \,d x
    = \sum_{n \geq 0} \frac{a^n}{n!} t^n,
  \end{equation*}
  or equivalently, in view of \eqref{eq:cj4sr7qmvd}, that
  \begin{equation}\label{eq:cj4ssco825}
    \int_{a}^{a + 1 }\frac{t e^{x t} }{ e^t - 1 } \, d x = e^{a t}.
  \end{equation}
  Indeed, we have
  \begin{align*}
    \int_a^{a + 1 } e^{x t } \, d x
    &=
    \frac{1}{t} e^{x t }|_{x=a}^{a+1} \\
    &=
    \frac{e^{(a+1) t} - e^{a t}}{t} \\
    &=
      e^{a t}
    \frac{e^t - 1}{t},
  \end{align*}
  from which the formula \eqref{eq:cj4ssco825} follows.
\end{proof}


\begin{lemma}\label{lemma:cj4vkapsln}
  We have
  \begin{equation*}
B_{n} (x + 1 ) - B_n (x) = n x^{n - 1}.
\end{equation*}
\end{lemma}
\begin{proof}
  Combine Lemmas \ref{lemma:cj4sr702cf} and \ref{lemma:cj4ssc4a2b}.
\end{proof}

\begin{lemma}
  For $n \geq 2$, we have
  \begin{equation*}
B_n (1) = B_n.
\end{equation*}
\end{lemma}
\begin{proof}
  We need to check that
  \begin{equation*}
    B_n(1) - B_n(0) = 0.
  \end{equation*}
  By the fundamental theorem of calculus, it is enough to check that
  \begin{equation*}
    \int_0^1 B_n'(x) \, d x = 0.
  \end{equation*}
  By Lemma \ref{lemma:cj4sr702cf}, it is equivalent to check that
  \begin{equation*}
    \int_0^1 B_{n - 1 } (x) \, d x = 0.
  \end{equation*}
  This follows from Lemma \ref{lemma:cj4ssc4a2b}.  We use here that $n-1 \geq 1$, which ensures that $0^{n-1} = 0$.
\end{proof}

\begin{lemma}
  For a smooth function $f$ on $[0,1]$, we have
  \begin{multline*}
    \int_0^1 f^{(n)} (x) \frac{B_n (x) }{n !} \, d x
    = - \int_0^1 f^{(n + 1 )} (x)
    \frac{B_{n + 1 } (x) }{ (n + 1 )! } \, d x \\
    +
    \begin{cases}
      \frac{f (0) + f (1) }{2} & \text{ if } n = 0, \\
      \frac{B_{n + 1}}{(n + 1 )!} \left( f^{(n)} (1) - f^{(n)} (0)  \right) &  \text{ if } n \geq 1.
    \end{cases}
  \end{multline*}
\end{lemma}
\begin{proof}
  We integrate by parts.  Take
  \begin{equation*}
u = f^{(n)}(x), \qquad v = \frac{B_{n+1}(x)}{(n+1)!}.
  \end{equation*}
  Then, by Lemma \ref{lemma:cj4sr702cf},
  \begin{equation*}
u \, d v  = f^{(n )} (x)  \frac{B_n (x) }{n !}.
  \end{equation*}
  The first term on the right hand side of the claimed formula is the integral of $v \, d u$.  The second term is $u v |_{0}^1$, using here that $B_m(1) = B_m(0) = B_m$ for $m \geq 2$, while $B_1(0) = 1/2 = - B_1(0)$.
\end{proof}

By inducting on $N$ and using that $B_0(x) = 1$, we obtain:
\begin{multline*}
  \int_0^1 f(x) \, d x
  = (-1)^N \int_0^1 f^{(N + 1 )} (x)
  \frac{B_{N + 1 } (x) }{ (N + 1 )! } \, d x \\
  +
  \frac{f (0) + f (1) }{2}
  + 
  \sum_{n = 1}^{N - 1} \frac{B_{n + 1}}{(n + 1 )!} \left( f^{(n)} (1) - f^{(n)} (0)  \right).
\end{multline*}
By replacing $f$ with its shifts by integer and summing, we obtain Theorem \ref{theorem:cj4vked1rh}.




\bibliography{refs}{} \bibliographystyle{plain}
\end{document}
