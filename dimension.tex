% Created 2021-02-27 Sat 22:17
% Intended LaTeX compiler: pdflatex
\documentclass[10pt]{article}
\usepackage[utf8]{inputenc}
\usepackage[T1]{fontenc}
\usepackage{graphicx}
\usepackage{grffile}
\usepackage{longtable}
\usepackage{wrapfig}
\usepackage{rotating}
\usepackage[normalem]{ulem}
\usepackage{amsmath}
\usepackage{textcomp}
\usepackage{amssymb}
\usepackage{capt-of}
\usepackage{hyperref}
\usepackage{graphicx, amsmath, amssymb, amsfonts, amsthm, stmaryrd, amscd}
\usepackage[usenames, dvipsnames]{xcolor}
\usepackage{tikz}
% \usepackage{tikzcd}
% \usepackage{comment}

% \let\counterwithout\relax
% \let\counterwithin\relax
% \usepackage{chngcntr}

\usepackage{enumerate}
% \usepackage{enumitem}
% \usepackage{times}
\usepackage[normalem]{ulem}
% \usepackage{minted}
% \usepackage{xypic}
% \usepackage{color}


% \usepackage{silence}
% \WarningFilter{latex}{Label `tocindent-1' multiply defined}
% \WarningFilter{latex}{Label `tocindent0' multiply defined}
% \WarningFilter{latex}{Label `tocindent1' multiply defined}
% \WarningFilter{latex}{Label `tocindent2' multiply defined}
% \WarningFilter{latex}{Label `tocindent3' multiply defined}
\usepackage{hyperref}
% \usepackage{navigator}


% \usepackage{pdfsync}
\usepackage{xparse}


\usepackage[all]{xy}
\usepackage{enumerate}
\usetikzlibrary{matrix,arrows,decorations.pathmorphing}



\makeatletter
\newcommand*{\transpose}{%
  {\mathpalette\@transpose{}}%
}
\newcommand*{\@transpose}[2]{%
  % #1: math style
  % #2: unused
  \raisebox{\depth}{$\m@th#1\intercal$}%
}
\makeatother


\makeatletter
\newcommand*{\da@rightarrow}{\mathchar"0\hexnumber@\symAMSa 4B }
\newcommand*{\da@leftarrow}{\mathchar"0\hexnumber@\symAMSa 4C }
\newcommand*{\xdashrightarrow}[2][]{%
  \mathrel{%
    \mathpalette{\da@xarrow{#1}{#2}{}\da@rightarrow{\,}{}}{}%
  }%
}
\newcommand{\xdashleftarrow}[2][]{%
  \mathrel{%
    \mathpalette{\da@xarrow{#1}{#2}\da@leftarrow{}{}{\,}}{}%
  }%
}
\newcommand*{\da@xarrow}[7]{%
  % #1: below
  % #2: above
  % #3: arrow left
  % #4: arrow right
  % #5: space left 
  % #6: space right
  % #7: math style 
  \sbox0{$\ifx#7\scriptstyle\scriptscriptstyle\else\scriptstyle\fi#5#1#6\m@th$}%
  \sbox2{$\ifx#7\scriptstyle\scriptscriptstyle\else\scriptstyle\fi#5#2#6\m@th$}%
  \sbox4{$#7\dabar@\m@th$}%
  \dimen@=\wd0 %
  \ifdim\wd2 >\dimen@
    \dimen@=\wd2 %   
  \fi
  \count@=2 %
  \def\da@bars{\dabar@\dabar@}%
  \@whiledim\count@\wd4<\dimen@\do{%
    \advance\count@\@ne
    \expandafter\def\expandafter\da@bars\expandafter{%
      \da@bars
      \dabar@ 
    }%
  }%  
  \mathrel{#3}%
  \mathrel{%   
    \mathop{\da@bars}\limits
    \ifx\\#1\\%
    \else
      _{\copy0}%
    \fi
    \ifx\\#2\\%
    \else
      ^{\copy2}%
    \fi
  }%   
  \mathrel{#4}%
}
\makeatother
% \DeclareMathOperator{\rg}{rg}

\usepackage{mathtools}
\DeclarePairedDelimiter{\paren}{(}{)}
\DeclarePairedDelimiter{\abs}{\lvert}{\rvert}
\DeclarePairedDelimiter{\norm}{\lVert}{\rVert}
\DeclarePairedDelimiter{\innerproduct}{\langle}{\rangle}
\newcommand{\Of}[2]{{\operatorname{#1}} {\paren*{#2}}}
\newcommand{\of}[2]{{{{#1}} {\paren*{#2}}}}

\DeclareMathOperator{\Shim}{Shim}
\DeclareMathOperator{\sgn}{sgn}
\DeclareMathOperator{\fdeg}{fdeg}
\DeclareMathOperator{\SL}{SL}
\DeclareMathOperator{\slLie}{\mathfrak{s}\mathfrak{l}}
\DeclareMathOperator{\soLie}{\mathfrak{s}\mathfrak{o}}
\DeclareMathOperator{\spLie}{\mathfrak{s}\mathfrak{p}}
\DeclareMathOperator{\glLie}{\mathfrak{g}\mathfrak{l}}
\newcommand{\pn}[1]{{\color{ForestGreen} \sf PN: [#1]}}
\DeclareMathOperator{\Mp}{Mp}
\DeclareMathOperator{\Mat}{Mat}
\DeclareMathOperator{\GL}{GL}
\DeclareMathOperator{\Gr}{Gr}
\DeclareMathOperator{\GU}{GU}
\def\gl{\mathfrak{g}\mathfrak{l}}
\DeclareMathOperator{\odd}{odd}
\DeclareMathOperator{\even}{even}
\DeclareMathOperator{\GO}{GO}
\DeclareMathOperator{\good}{good}
\DeclareMathOperator{\bad}{bad}
\DeclareMathOperator{\PGO}{PGO}
\DeclareMathOperator{\htt}{ht}
\DeclareMathOperator{\height}{height}
\DeclareMathOperator{\Ass}{Ass}
\DeclareMathOperator{\coheight}{coheight}
\DeclareMathOperator{\GSO}{GSO}
\DeclareMathOperator{\SO}{SO}
\DeclareMathOperator{\so}{\mathfrak{s}\mathfrak{o}}
\DeclareMathOperator{\su}{\mathfrak{s}\mathfrak{u}}
\DeclareMathOperator{\ad}{ad}
% \DeclareMathOperator{\sc}{sc}
\DeclareMathOperator{\Ad}{Ad}
\DeclareMathOperator{\disc}{disc}
\DeclareMathOperator{\inv}{inv}
\DeclareMathOperator{\Pic}{Pic}
\DeclareMathOperator{\uc}{uc}
\DeclareMathOperator{\Cl}{Cl}
\DeclareMathOperator{\Clf}{Clf}
\DeclareMathOperator{\Hom}{Hom}
\DeclareMathOperator{\hol}{hol}
\DeclareMathOperator{\Heis}{Heis}
\DeclareMathOperator{\Haar}{Haar}
\DeclareMathOperator{\h}{h}
\def\sp{\mathfrak{s}\mathfrak{p}}
\DeclareMathOperator{\heis}{\mathfrak{h}\mathfrak{e}\mathfrak{i}\mathfrak{s}}
\DeclareMathOperator{\End}{End}
\DeclareMathOperator{\JL}{JL}
\DeclareMathOperator{\image}{image}
\DeclareMathOperator{\red}{red}
\def\div{\operatorname{div}}
\def\eps{\varepsilon}
\def\cHom{\mathcal{H}\operatorname{om}}
\DeclareMathOperator{\Ops}{Ops}
\DeclareMathOperator{\Symb}{Symb}
\def\boldGL{\mathbf{G}\mathbf{L}}
\def\boldSO{\mathbf{S}\mathbf{O}}
\def\boldU{\mathbf{U}}
\DeclareMathOperator{\hull}{hull}
\DeclareMathOperator{\LL}{LL}
\DeclareMathOperator{\PGL}{PGL}
\DeclareMathOperator{\class}{class}
\DeclareMathOperator{\lcm}{lcm}
\DeclareMathOperator{\spann}{span}
\DeclareMathOperator{\Exp}{Exp}
\DeclareMathOperator{\ext}{ext}
\DeclareMathOperator{\Ext}{Ext}
\DeclareMathOperator{\Tor}{Tor}
\DeclareMathOperator{\et}{et}
\DeclareMathOperator{\tor}{tor}
\DeclareMathOperator{\loc}{loc}
\DeclareMathOperator{\tors}{tors}
\DeclareMathOperator{\pf}{pf}
\DeclareMathOperator{\smooth}{smooth}
\DeclareMathOperator{\prin}{prin}
\DeclareMathOperator{\Kl}{Kl}
\newcommand{\kbar}{\mathchar'26\mkern-9mu k}
\DeclareMathOperator{\der}{der}
% \DeclareMathOperator{\abs}{abs}
\DeclareMathOperator{\Sub}{Sub}
\DeclareMathOperator{\Comp}{Comp}
\DeclareMathOperator{\Err}{Err}
\DeclareMathOperator{\dom}{dom}
\DeclareMathOperator{\radius}{radius}
\DeclareMathOperator{\Fitt}{Fitt}
\DeclareMathOperator{\Sel}{Sel}
\DeclareMathOperator{\rad}{rad}
\DeclareMathOperator{\id}{id}
\DeclareMathOperator{\Center}{Center}
\DeclareMathOperator{\Der}{Der}
\DeclareMathOperator{\U}{U}
% \DeclareMathOperator{\norm}{norm}
\DeclareMathOperator{\trace}{trace}
\DeclareMathOperator{\Equid}{Equid}
\DeclareMathOperator{\Feas}{Feas}
\DeclareMathOperator{\bulk}{bulk}
\DeclareMathOperator{\tail}{tail}
\DeclareMathOperator{\sys}{sys}
\DeclareMathOperator{\atan}{atan}
\DeclareMathOperator{\temp}{temp}
\DeclareMathOperator{\Asai}{Asai}
\DeclareMathOperator{\glob}{glob}
\DeclareMathOperator{\Kuz}{Kuz}
\DeclareMathOperator{\Irr}{Irr}
\newcommand{\rsL}{ \frac{ L^{(R)}(\Pi \times \Sigma, \std, \frac{1}{2})}{L^{(R)}(\Pi \times \Sigma, \Ad, 1)}  }
\DeclareMathOperator{\GSp}{GSp}
\DeclareMathOperator{\PGSp}{PGSp}
\DeclareMathOperator{\BC}{BC}
\DeclareMathOperator{\Ann}{Ann}
\DeclareMathOperator{\Gen}{Gen}
\DeclareMathOperator{\SU}{SU}
\DeclareMathOperator{\PGSU}{PGSU}
% \DeclareMathOperator{\gen}{gen}
\DeclareMathOperator{\PMp}{PMp}
\DeclareMathOperator{\PGMp}{PGMp}
\DeclareMathOperator{\PB}{PB}
\DeclareMathOperator{\ind}{ind}
\DeclareMathOperator{\Jac}{Jac}
\DeclareMathOperator{\jac}{jac}
\DeclareMathOperator{\im}{im}
\DeclareMathOperator{\Aut}{Aut}
\DeclareMathOperator{\Int}{Int}
\DeclareMathOperator{\PSL}{PSL}
\DeclareMathOperator{\co}{co}
\DeclareMathOperator{\irr}{irr}
\DeclareMathOperator{\prim}{prim}
\DeclareMathOperator{\bal}{bal}
\DeclareMathOperator{\baln}{bal}
\DeclareMathOperator{\dist}{dist}
\DeclareMathOperator{\RS}{RS}
\DeclareMathOperator{\Ram}{Ram}
\DeclareMathOperator{\Sob}{Sob}
\DeclareMathOperator{\Sol}{Sol}
\DeclareMathOperator{\soc}{soc}
\DeclareMathOperator{\nt}{nt}
\DeclareMathOperator{\mic}{mic}
\DeclareMathOperator{\Gal}{Gal}
\DeclareMathOperator{\st}{st}
\DeclareMathOperator{\std}{std}
\DeclareMathOperator{\diag}{diag}
\DeclareMathOperator{\Sym}{Sym}
\DeclareMathOperator{\gr}{gr}
\DeclareMathOperator{\aff}{aff}
\DeclareMathOperator{\Dil}{Dil}
\DeclareMathOperator{\Lie}{Lie}
\DeclareMathOperator{\Symp}{Symp}
\DeclareMathOperator{\Stab}{Stab}
\DeclareMathOperator{\St}{St}
\DeclareMathOperator{\stab}{stab}
\DeclareMathOperator{\codim}{codim}
\DeclareMathOperator{\linear}{linear}
\newcommand{\git}{/\!\!/}
\DeclareMathOperator{\geom}{geom}
\DeclareMathOperator{\spec}{spec}
\def\O{\operatorname{O}}
\DeclareMathOperator{\Au}{Aut}
\DeclareMathOperator{\Fix}{Fix}
\DeclareMathOperator{\Opp}{Op}
\DeclareMathOperator{\opp}{op}
\DeclareMathOperator{\Size}{Size}
\DeclareMathOperator{\Save}{Save}
% \DeclareMathOperator{\ker}{ker}
\DeclareMathOperator{\coker}{coker}
\DeclareMathOperator{\sym}{sym}
\DeclareMathOperator{\mean}{mean}
\DeclareMathOperator{\elliptic}{ell}
\DeclareMathOperator{\nilpotent}{nil}
\DeclareMathOperator{\hyperbolic}{hyp}
\DeclareMathOperator{\newvector}{new}
\DeclareMathOperator{\new}{new}
\DeclareMathOperator{\full}{full}
\newcommand{\qr}[2]{\left( \frac{#1}{#2} \right)}
\DeclareMathOperator{\unr}{u}
\DeclareMathOperator{\ram}{ram}
% \DeclareMathOperator{\len}{len}
\DeclareMathOperator{\fin}{fin}
\DeclareMathOperator{\cusp}{cusp}
\DeclareMathOperator{\curv}{curv}
\DeclareMathOperator{\rank}{rank}
\DeclareMathOperator{\rk}{rk}
\DeclareMathOperator{\pr}{pr}
\DeclareMathOperator{\Transform}{Transform}
\DeclareMathOperator{\mult}{mult}
\DeclareMathOperator{\Eis}{Eis}
\DeclareMathOperator{\reg}{reg}
\DeclareMathOperator{\sing}{sing}
\DeclareMathOperator{\alt}{alt}
\DeclareMathOperator{\irreg}{irreg}
\DeclareMathOperator{\sreg}{sreg}
\DeclareMathOperator{\Wd}{Wd}
\DeclareMathOperator{\Weil}{Weil}
\DeclareMathOperator{\Th}{Th}
\DeclareMathOperator{\Sp}{Sp}
\DeclareMathOperator{\Ind}{Ind}
\DeclareMathOperator{\Res}{Res}
\DeclareMathOperator{\ini}{in}
\DeclareMathOperator{\ord}{ord}
\DeclareMathOperator{\osc}{osc}
\DeclareMathOperator{\fluc}{fluc}
\DeclareMathOperator{\size}{size}
\DeclareMathOperator{\ann}{ann}
\DeclareMathOperator{\equ}{eq}
\DeclareMathOperator{\res}{res}
\DeclareMathOperator{\pt}{pt}
\DeclareMathOperator{\src}{source}
\DeclareMathOperator{\Zcl}{Zcl}
\DeclareMathOperator{\Func}{Func}
\DeclareMathOperator{\Map}{Map}
\DeclareMathOperator{\Frac}{Frac}
\DeclareMathOperator{\Frob}{Frob}
\DeclareMathOperator{\ev}{eval}
\DeclareMathOperator{\pv}{pv}
\DeclareMathOperator{\eval}{eval}
\DeclareMathOperator{\Spec}{Spec}
\DeclareMathOperator{\Speh}{Speh}
\DeclareMathOperator{\Spin}{Spin}
\DeclareMathOperator{\GSpin}{GSpin}
\DeclareMathOperator{\Specm}{Specm}
\DeclareMathOperator{\Sphere}{Sphere}
\DeclareMathOperator{\Sqq}{Sq}
\DeclareMathOperator{\Ball}{Ball}
\DeclareMathOperator\Cond{\operatorname{Cond}}
\DeclareMathOperator\proj{\operatorname{proj}}
\DeclareMathOperator\Swan{\operatorname{Swan}}
\DeclareMathOperator{\Proj}{Proj}
\DeclareMathOperator{\bPB}{{\mathbf P}{\mathbf B}}
\DeclareMathOperator{\Projm}{Projm}
\DeclareMathOperator{\Tr}{Tr}
\DeclareMathOperator{\Type}{Type}
\DeclareMathOperator{\Prop}{Prop}
\DeclareMathOperator{\vol}{vol}
\DeclareMathOperator{\covol}{covol}
\DeclareMathOperator{\Rep}{Rep}
\DeclareMathOperator{\Cent}{Cent}
\DeclareMathOperator{\val}{val}
\DeclareMathOperator{\area}{area}
\DeclareMathOperator{\nr}{nr}
\DeclareMathOperator{\CM}{CM}
\DeclareMathOperator{\CH}{CH}
\DeclareMathOperator{\tr}{tr}
\DeclareMathOperator{\characteristic}{char}
\DeclareMathOperator{\supp}{supp}


\theoremstyle{plain} \newtheorem{theorem} {Theorem} \newtheorem{conjecture} [theorem] {Conjecture} \newtheorem{corollary} [theorem] {Corollary} \newtheorem{proposition} [theorem] {Proposition} \newtheorem{fact} [theorem] {Fact}
\theoremstyle{definition} \newtheorem{definition} [theorem] {Definition} \newtheorem{hypothesis} [theorem] {Hypothesis} \newtheorem{assumptions} [theorem] {Assumptions}
\newtheorem{example} [theorem] {Example}
\newtheorem{assertion}[theorem] {Assertion}
\newtheorem{note}[theorem] {Note}
\newtheorem{conclusion}[theorem] {Conclusion}
\newtheorem{claim}            {Claim}
\newtheorem{homework} {Homework}
\newtheorem{exercise} {Exercise}  \newtheorem{question}[theorem] {Question}    \newtheorem{answer} {Answer}  \newtheorem{problem} {Problem}    \newtheorem{remark} [theorem] {Remark}
\newtheorem{notation} [theorem]           {Notation}
\newtheorem{terminology}[theorem]            {Terminology}
\newtheorem{convention}[theorem]            {Convention}
\newtheorem{motivation}[theorem]            {Motivation}


\newtheoremstyle{itplain} % name
{6pt}                    % Space above
{5pt\topsep}                    % Space below
{\itshape}                   % Body font
{}                           % Indent amount
{\itshape}                   % Theorem head font
{.}                          % Punctuation after theorem head
{5pt plus 1pt minus 1pt}                       % Space after theorem head
% {.5em}                       % Space after theorem head
{}  % Theorem head spec (can be left empty, meaning ‘normal’)

% \theoremstyle{mytheoremstyle}


\theoremstyle{itplain} %--default
% \theoremheaderfont{\itshape}
% \newtheorem{lemma}{Lemma}
\newtheorem{lemma}[theorem]{Lemma}
% \newtheorem{lemma}{Lemma}[subsubsection]

\newtheorem*{lemma*}{Lemma}
\newtheorem*{proposition*}{Proposition}
\newtheorem*{definition*}{Definition}
\newtheorem*{example*}{Example}

\newtheorem*{results*}{Results}
\newtheorem{results} [theorem] {Results}


\usepackage[displaymath,textmath,sections,graphics]{preview}
\PreviewEnvironment{align*}
\PreviewEnvironment{multline*}
\PreviewEnvironment{tabular}
\PreviewEnvironment{verbatim}
\PreviewEnvironment{lstlisting}
\PreviewEnvironment*{frame}
\PreviewEnvironment*{alert}
\PreviewEnvironment*{emph}
\PreviewEnvironment*{textbf}


\usepackage{fancyhdr}
\pagestyle{fancy}
\makeatother
\fancyhead[R]{\thepage}
\author{Paul Nelson}
\date{\today}
\title{Commutative algebra: some basics on Krull dimension}
\hypersetup{
 pdfauthor={Paul Nelson},
 pdftitle={Commutative algebra: some basics on Krull dimension},
 pdfkeywords={},
 pdfsubject={},
 pdfcreator={Emacs 25.3.50.1 (Org mode )}, 
 pdflang={English}}

\DeclareMathOperator{\trdeg}{trdeg}

\begin{document}

\maketitle
\tableofcontents


\section{Introduction}
\label{sec:org2573665}

We recall some definitions and background, record proofs of some
of the main theorems regarding Krull dimension, and give some of
their geometric interpretations.
We mainly follow the course reference by Bosch.

\section{Basic definitions}
\label{sec:orgdb2d3a0}

Let \(A\) be a ring (always commutative and with identity).
In what follows,
the symbols \(\mathfrak{p}\) or \(\mathfrak{p}_i\)
always denote prime ideals.
We set
\[
    \dim(A) :=
    \sup \{n \geq 0 : \exists \mathfrak{p}_0 \subsetneq \dotsb \subsetneq \mathfrak{p}_n \}.
  \]

For a prime ideal \(\mathfrak{p}\) of \(A\),
we set
\[
    \height(\mathfrak{p}) :=
    \sup \{n \geq 0 : \exists \mathfrak{p}_0 \subsetneq \dotsb \subsetneq \mathfrak{p}_n \subseteq \mathfrak{p}\},
  \]
\[
    \coheight(\mathfrak{p}) :=
    \sup \{n \geq 0 : \exists \mathfrak{p} \subseteq \mathfrak{p}_0 \subsetneq \dotsb \subsetneq \mathfrak{p}_n \}.
  \]
For a general ideal \(\mathfrak{a}\),
we set
\[
    \height(\mathfrak{a}) := \inf_{\mathfrak{p} \supseteq
      \mathfrak{a}} \height(\mathfrak{p}),
  \]
\[
    \coheight(\mathfrak{a}) := \sup_{\mathfrak{p} \supseteq
      \mathfrak{a}} \coheight(\mathfrak{p})
    = \sup \{n \geq 0 : \exists \mathfrak{a} \subseteq \mathfrak{p}_0 \subsetneq \dotsb \subsetneq \mathfrak{p}_n \}.
  \]
Since prime ideals in the localization \(A_\mathfrak{p}\)
correspond to the primes in \(A\) contained in \(\mathfrak{p}\),
we have
\[
    \height(\mathfrak{p}) = \dim(A_\mathfrak{p}).
  \]
Since prime ideals in the quotient \(A/\mathfrak{a}\)
correspond to the primes in \(A\) containing \(\mathfrak{a}\),
we have
\[
    \coheight(\mathfrak{a}) = \dim(A/\mathfrak{a}).
  \]
We note the following easy inequality:
\begin{lemma}\label{lem:easy-dimension-inequality}
  $\height(\mathfrak{a}) + \dim(A/\mathfrak{a}) \leq \dim(A)$.
\end{lemma}
\begin{proof}
  It suffices to show that
  if $\height(\mathfrak{a}) \geq r$
  and $\dim(A/\mathfrak{a}) \geq s$,
  then $\dim(A) \geq r + s$.
  By hypothesis,
  we may find primes
  $\mathfrak{a} \subseteq \mathfrak{q}_0 \subsetneq \dotsb
  \subsetneq \mathfrak{q}_s$.
  Then $\height(\mathfrak{q}_0) \geq \height(\mathfrak{a}) \geq
  r$,
  so we may find primes
  $\mathfrak{p}_0 \subsetneq \dotsb \subsetneq \mathfrak{p}_r =
  \mathfrak{q}_0$.
  Then
  \[
    \mathfrak{p}_0 \subsetneq \dotsb \subsetneq \mathfrak{p}_r =
    \mathfrak{q}_0
    \subsetneq \mathfrak{q}_1 \subsetneq \dotsb \subsetneq
    \mathfrak{q}_s
  \]
  is a chain of primes in  $A$ of length $r+s$.
\end{proof}
We also note:
\begin{lemma}\label{lem:local-ring-height-equals-dim}
  Let $(A,\mathfrak{m})$ be a local ring.
  Then $\dim(A) = \height(\mathfrak{m})$.
\end{lemma}
\begin{proof}
  Let $\mathfrak{p}_0 \subsetneq \dotsb \subsetneq
  \mathfrak{p}_r$ be a chain of primes in $A$.
  By enlarging this chain if necessary, we may assume that
  $\mathfrak{p}_r = \mathfrak{m}$.
  Thus the suprema in the definitions of
  $\dim(A)$ and $\height(\mathfrak{m})$
  may be taken over the same chains of primes.
\end{proof}


\section{Geometric interpretations}
\label{sec:org60bbe56}

Reference for this section: exercises in Chapter 1 of Atiyah--Macdonald.

Let \(A\) be a ring.  Recall that \(\Spec(A)\) denotes the set of
prime ideals \(\mathfrak{p}\) in \(A\).
Each \(f \in A\) defines a function
\[
  f|_{\Spec(A)} : \Spec(A) \rightarrow
  \bigsqcup_{\mathfrak{p} \in \Spec(A)} A/\mathfrak{p}
  \]
sending \(\mathfrak{p}\) to the class of \(f\) in the quotient ring \(A/\mathfrak{p}\).
For \(f \in A\) and any subset \(X\) of \(\Spec(A)\),
we may form the restriction
\(f|_X\) of \(f\) to \(X\).
For the sake of illustration,
note that \(f|_{\Spec(A)} = 0\) (i.e., \(f|_{\Spec(A)}\)
maps each \(\mathfrak{p}\) to the zero class in \(A / \mathfrak{p}\))
if and only if \(f\) belongs
to the nilradical of \(A\).

For example, we have seen (using the Nullstellensatz) that if
\(A = \mathbb{C}[X_1,\dotsc,X_n]/I\) for some ideal
\(I \subseteq \mathbb{C}[X_1,\dotsc,X_n]\), then the set
\(\Specm(A)\) of maximal ideals in \(A\) is in natural bijection
with
\(V := \{(x_1,\dotsc,x_n) \in \mathbb{C}^n : f(x_1,\dotsc,x_n) = 0
  \text{ for all } f \in I \}\).
For each such maximal ideal \(\mathfrak{m}\) we may identify
\(A/\mathfrak{m}\) with \(\mathbb{C}\).
For \(f \in A\),
the function
\(f|_{\Specm(A)}\) then identifies with the obvious map
\(V \ni (x_1,\dotsc,x_n) \mapsto f(x_1,\dotsc,x_n) \in
  \mathbb{C}\).

For a subset \(S\) of \(A\), we set
\[
  V(S) := \{\mathfrak{p} \in \Spec(A) : \mathfrak{p} \supseteq
  S\}
  = \{\mathfrak{p} \in \Spec(A) : f(\mathfrak{p}) = 0 \text{ for
    each }
  f \in S
  \}.
  \]
For finite sets \(S = \{f_1,\dotsc,f_n\}\)
we write simply \(V(f_1,\dotsc,f_n) := V(S)\).
Note that if \(S\) generates an ideal \(\mathfrak{a}\), then
\(V(S) = V(\mathfrak{a})\).
Given any subset \(X\) of \(\Spec(A)\),
we set
\[
  I(X)
  := \cap_{\mathfrak{p} \in X} \mathfrak{p}
  = \{f \in A : f|_X = 0\}.
  \]
Recall that a subset of \(\Spec(A)\) is
called \emph{closed} if it is of the form \(V(S)\) for some \(S\);
this defines
a topology on \(\Spec(A)\).
Recall that an ideal \(\mathfrak{a}\) is \emph{radical}
if \(\rad(\mathfrak{a}) = \mathfrak{a}\).
\begin{lemma}~
  \begin{enumerate}[(i)]
  \item For each ideal $\mathfrak{a}$ of $A$,
    we have $I(V(\mathfrak{a})) = \rad(\mathfrak{a})$.
  \item For each subset $X$ of $\Spec(A)$,
    we have $V(I(X)) = \overline{X}$ (the closure of $X$).
  \item The maps $V$ and $I$ define mutually-inverse
    inclusion-reversing
    bijections between the set of radical ideals of $A$ and the
    set of closed subsets of $\Spec(A)$.
  \end{enumerate}
\end{lemma}
\begin{proof}
  The maps $I$ and $V$ are readily seen to be inclusion-reversing
  (cf. Exercise Sheet \#1).
  \begin{enumerate}[(i)]
  \item By definition, $I(V(\mathfrak{a})) = \cap_{\mathfrak{p} \in
      V(\mathfrak{a})} \mathfrak{p}
    = \cap_{\mathfrak{p} \supseteq \mathfrak{a}} \mathfrak{p} =
    \rad(\mathfrak{a})$.
  \item The set $V(I(X))$ is closed and contains $X$, so it will suffice to
    verify for each closed set $V(\mathfrak{a})$ containing $X$
    that $V(\mathfrak{a}) \supseteq V(I(X))$.  From
    $V(\mathfrak{a}) \supseteq X$ we see that $f|_X = 0$ for all
    $f \in \mathfrak{a}$, thus $\mathfrak{a} \subseteq I(X)$.
    Applying the inclusion-reversing map $V$, we obtain
    $V(\mathfrak{a}) \supseteq V(I(X))$, as required.
  \item Immediate by the above.
  \end{enumerate}
\end{proof}
\begin{lemma}
  Let $X$ be a closed subset of $\Spec(A)$.
  The following are equivalent:
  \begin{enumerate}[(i)]
  \item 
    $X = V(\mathfrak{p})$ for some prime ideal
    $\mathfrak{p}$ of $A$.
  \item $I(X)$ is a prime ideal of $A$.
  \item $X$ is nonempty
    and may not be written as $X = X_1 \cup X_2$ for closed subsets
    $X_1, X_2$ of $\Spec(A)$ except in the trivial case that either
    $X \subseteq X_1$ or $X \subseteq X_2$.
  \end{enumerate}
\end{lemma}
We say that a closed
subset \(X\) of \(\Spec(A)\)
is \emph{irreducible}
if it satisfies the equivalent conditions of the preceeding
lemma.
The irreducible closed subsets of \(\Spec(A)\)
correspond bijectively to the prime ideals of \(A\).

We note that for any ideal \(\mathfrak{a}\),
we may identify
\[
  V(\mathfrak{a}) = \Spec(A/\mathfrak{a}).
  \]
We note also that if \(\mathfrak{p}\) is a prime of \(A\),
then the primes of the localization \(A_\mathfrak{p}\)
correspond to the primes of \(A\) contained in \(\mathfrak{p}\),
hence the spectrum of \(A_\mathfrak{p}\)
identifies with the set of closed irreducible subsets
of \(\Spec(A)\) that \emph{contain} \(\mathfrak{p}\):
\[
  \Spec(A_\mathfrak{p})
  =
  \{\mathfrak{q} \in \Spec(A) : \mathfrak{q} \subseteq \mathfrak{p} \}
  =
  \{\mathfrak{q} \in \Spec(A) : \mathfrak{p} \in V(\mathfrak{q}) \}.
  \]

By an \emph{irreducible component} of a closed subset \(X\) of
\(\Spec(A)\), we shall mean a maximal closed irreducible subset of
\(X\), i.e., a closed irreducible subset \(Z \subseteq X\) with the
property that if \(Z' \subseteq X\) is any closed irreducible
subset with \(Z' \supseteq Z\), then \(Z' = Z\).  Using the inclusion-reversing bijections
noted above, we verify readily
that for any ideal \(\mathfrak{a}\), the irreducible components of
\(X = V(\mathfrak{a})\) correspond bijectively to the set (denoted
\(\Ass'(\mathfrak{a})\) in lecture) of minimal prime ideals
\(\mathfrak{p} \supseteq \mathfrak{a}\).  

We assume henceforth that \(A\) is Noetherian.
Then the set of minimal primes of any ideal is finite, and any prime containing an ideal contains a minimal prime of that ideal.
It follows that the set
of irreducible components of any closed subset \(X\) of \(\Spec(A)\) is
a finite set \(\{Z_1,\dotsc,Z_n\}\) for which \(X = Z_1 \cup \dotsb \cup Z_n\).

We define the \emph{dimension}
of a closed subset \(X\) of \(\Spec(A)\)
to be
\[
  \dim(X) = \sup \{n \geq 0 : \exists \text{ closed irreducible
    subsets }
  Z_n \subsetneq \dotsb \subsetneq Z_0 \subseteq X
  \}
  \]
and the \emph{codimension}
in the special case
that \(Z\) is closed irreducible
to be
\[
  \codim(Z) :=
  \sup \{n \geq 0 : \exists \text{ closed irreducible
    subsets }
  Z_0 \supsetneq \dotsb \supsetneq Z_n \supset Z \}
  \]
and then in general by
\[
  \codim(X) :=
  \inf_{\substack{Z \subseteq X : \text{closed irreducible}}}
  \codim(Z).
  \]
Equivalently, \(\codim(X)\) is the smallest codimension
of any irreducible component of \(X\).
We note also that \(\dim(X)\) coincides with the largest
dimension of any irreducible component of \(X\).
We might write \(\codim(X)\) as \(\codim_{\Spec(A)}(X)\)
when we wish to emphasize the reference space \(\Spec(A)\).

Using the inclusion-reversing bijections noted above,
we see that
\[
  \dim \Spec A = \dim A
  \]
and more generally
that
\[
  \dim V(\mathfrak{a}) = \coheight \mathfrak{a} = \dim A/\mathfrak{a},
  \quad
  \dim X = \coheight I(X) = \dim A/I(X),
  \]
\[
  \codim V(\mathfrak{a}) = \height \mathfrak{a},
  \quad
  \codim X = \height I(X)
  \]
for any ideal \(\mathfrak{a}\) and any closed \(X \subseteq
  \Spec(A)\).
Lemma \ref{lem:easy-dimension-inequality}
says that \(\dim X + \codim X \leq \dim \Spec A\).


\section{Prime avoidance lemma\label{sec:prime-avoidance}}
\label{sec:org9d9ed7c}

\begin{lemma}
  Let $A$ be a ring, let $\mathfrak{p}_1,\dotsc,\mathfrak{p}_n$
  be prime ideals, and let $\mathfrak{a}$ be an ideal contained
  in the union $\cup \mathfrak{p}_j$.  Then there exists an index $j$
  for which $\mathfrak{a} \subseteq \mathfrak{p}_j$.
  Equivalently, if $\mathfrak{a} \not\subseteq \mathfrak{p}_j$
  for each $j$, then
  $\mathfrak{a} \not\subseteq \cup \mathfrak{p}_j$.

  In ``geometric'' terms,
  let $Z_1,\dotsc,Z_n \subseteq \Spec(A)$
  be closed irreducible subsets,
  and
  let $X = V(\mathfrak{a})$
  be a closed irreducible subset
  of $\Spec(A)$,
  defined by an ideal $\mathfrak{a}$,
  with the property that $X \not\supseteq Z_j$ for all $j$.
  Then there exists $f \in \mathfrak{a}$
  with $f|_{Z_j} \neq 0$ for all $j$.
  In particular, we may find $f \in A$ with $f|_{X} = 0$
  but $f|_{Z_j} \neq 0$ for all $j$.
\end{lemma}
\begin{proof}
  We verify 
  that if $\mathfrak{a}$ is not contained in any of the
  $\mathfrak{p}_j$,
  then it is not contained in their union.
  For this we may induct on $n$.
  The case $n = 1$ is trivial,
  so suppose $n > 2$.
  By our inductive hypothesis,
  we may find for each $i=1..n$ an element $a_i \in
  \mathfrak{a}$
  with $a_i \notin \mathfrak{p}_j$
  whenever $j \neq i$.
  If moreover $a_i \notin \mathfrak{p}_i$
  for some $i$,
  then we are done,
  so suppose otherwise that $a_i \in \mathfrak{p}_i$ for all
  $i$.
  Set $b_i := \prod_{j : j \neq i} a_j$.
  Then $b_i \notin \mathfrak{p}_i$ (using that $\mathfrak{p}_i$
  is prime)
  but $b_i \in \mathfrak{p}_j$ for all $j \neq i$.
  It follows that $x := b_1 + \dotsb + b_n$
  belongs to $\mathfrak{a}$
  but not to $\mathfrak{p}_i$ for any $i$,
  hence $\mathfrak{a}$ is not contained in the union of the $\mathfrak{p}_i$.
\end{proof}

\section{Artin rings}
\label{sec:org49d017a}

\begin{theorem}\label{thm:artin-vs-noeth-dim0}
  Let $A$ be a ring.
  The following are equivalent:
  \begin{enumerate}[(i)]
  \item $A$ is an Artin ring.
  \item $A$ is a Noetherian ring of dimension zero.
  \end{enumerate}
\end{theorem}


\section{Krull intersection theorem}
\label{sec:orgd90b421}

\begin{theorem}\label{thm:krull-intersection}
  Let $\mathfrak{a}$ be an ideal contained in the Jacobson
  radical $\Jac(A)$ of a Noetherian ring
  $A$.
  Then
  \[
    \cap_{n \geq 0} \mathfrak{a}^n = 0.
  \]
\end{theorem}
\begin{corollary}\label{cor:krull-intersection1}
  With $A, \mathfrak{a}$ as before, let $M$ be a finitely-generated module.
  Then $\cap_{n \geq 0} \mathfrak{a}^n M = 0$.
\end{corollary}
\begin{corollary}\label{cor:krull-intersection2}
  Let $(A,\mathfrak{m})$ be a  Noetherian local ring.
  Then $\cap_{n \geq 0} \mathfrak{m}^n =0$.
\end{corollary}
For the proof of Theorem \ref{thm:krull-intersection}, the fact that \(\mathfrak{a}\) is contained in the Jacobson
radical
suggests an application of Nakayama's lemma to the ideal
\(M' := \cap_{n \geq 0} \mathfrak{a}^n\),
for which it
is clear
that \(\mathfrak{a} M' \subseteq M'\)
and
plausible but non-obvious that \(\mathfrak{a} M' = M'\).
The key tool in establishing  the latter is the following:
\begin{lemma}[Artin--Rees lemma]
  Let $A$ be Noetherian,
  let $\mathfrak{a}$ be an ideal,
  let $M$ be a finitely-generated module,
  and let $M' \leq M$ be a submodule.
  There exists $n \geq 0$ so that for all $k \geq 0$,
  \[
    \mathfrak{a}^k (\mathfrak{a}^n M \cap M')
    = \mathfrak{a}^{n+k} M \cap M'.
  \]
\end{lemma}
Taking
\(M := A, M' := \cap_{n \geq 0} \mathfrak{a}^n\),
\(k := 1\)
in the Artin--Rees lemma
gives \(\mathfrak{a}^n M \cap M' = \mathfrak{a}^{n+k} M \cap M' =
  M'\)
and hence \(\mathfrak{a} M' = M'\);
we then conclude  the proof of Theorem
\ref{thm:krull-intersection}
by Nakayama, as indicated above.

The proof of Artin--Rees reduces formally
to the case \(k = 1\),
and the containment
\[
    \mathfrak{a} (\mathfrak{a}^n M \cap M')
    \subseteq \mathfrak{a}^{n+1} M \cap M'
  \]
is clear.
The proof of the trickier reverse containment
is expressed
most transparently using the graded ring
\[\tilde{A}
    := \boxplus_{i \geq 0} A_i = \{a = (a_i)_{i \geq 0} : a_i \in
    A_i\}, \quad A_i := \mathfrak{a}^i,
  \]
where the multiplication law extends
the bilinear maps \(\mathfrak{a}^i \times \mathfrak{a}^j \rightarrow \mathfrak{a}^{i+j}\):
\[
    (a \cdot b)_k = \sum_{i+j=k} a_i b_j.
  \]
This graded ring acts by the rule
\((a \cdot m)_k := \sum_{i+j=k} a_i m_j\)
on the graded module
\[\tilde{M}
    := \boxplus_{i \geq 0} M_i,
    \quad M_i := \mathfrak{a}^i M,
  \]
and its graded submodule
\[\tilde{M'}
    := \boxplus_{i \geq 0} M'_i,
    \quad M'_i := \mathfrak{a}^i M \cap M'.
  \]
Since \(\mathfrak{a}\) is finitely-generated as a module over \(A\),
\(\tilde{A}\) is finitely-generated as an algebra over \(A_0 = A\);
by the Hilbert basis theorem,
it follows that \(\tilde{A}\) is Noetherian.
The module \(M\) is finitely-generated over \(A\),
from which it follows readily that the graded module \(\tilde{M}\)
is finitely-generated over \(\tilde{A}\);
since the ring \(\tilde{A}\) is Noetherian,
so is the module \(\tilde{M}\),
hence its submodule \(\tilde{M'}\) is finitely-generated.
Choose \(n\) large enough that
the module \(\tilde{M'}\) is generated by \(\boxplus_{0 \leq i \leq
    n} M'_i\),
thus
\[
    \tilde{M'}
    =
    \tilde{A} \boxplus_{0 \leq i \leq
      n} M'_i.
  \]
By taking the degree \(n+1\) homogeneous  component of this identity,
we see that
\begin{align*}
  \mathfrak{a}^{n+1} M \cap M'
  &=
  \tilde{M'_{n+1}}
  =
  \sum_{0 \leq i \leq n}
  A_{n+1-i} M'_i
  =
  \sum_{0 \leq i \leq n}
  \mathfrak{a}^{n+1-i} (\mathfrak{a}^i M \cap M')
  \\
  &\subseteq
    \sum_{0 \leq i \leq n}
    \mathfrak{a} (\mathfrak{a}^n M \cap \mathfrak{a}^{n-i} M')
    \subseteq \mathfrak{a} (\mathfrak{a}^n M \cap M'),
\end{align*}
giving the required reverse containment.
The proof of Artin--Rees and hence of the Krull intersection
theorem is then complete.


\section{Kernel of localization with respect to a prime\label{sec:kernel-localize}}
\label{sec:org762c52a}

Let \(\mathfrak{p}\) be a prime ideal in a Noetherian ring \(A\).
Let \(\mathfrak{p}^{(n)}\) denote the \$n\$th symbolic power;
it is the $\mathfrak{p}$-primary ideal
given by \(A \cap \mathfrak{p}^n A_\mathfrak{p} := \iota^*(
  (\iota_* \mathfrak{p} )^n)\),
where \(\iota : A \rightarrow A_\mathfrak{p}\) denotes the
localization map.
\begin{theorem}
  $\ker(\iota) = \cap_{n \geq 0} \mathfrak{p}^{(n)}$.
\end{theorem}
\begin{proof}
  Set $\mathfrak{m} := \iota_* \mathfrak{p}$.  We have
  $\ker(\iota) = \iota^{(-1)}(0)$ and
  $\iota^{-1}(\cap_{n \geq 0} \mathfrak{m}^n) = \cap_{n \geq 0}
  \mathfrak{p}^{(n)}$, so it suffices to show that
  $\cap_{n \geq 0} \mathfrak{m}^n = 0$, which is the content of
  Corollary \ref{cor:krull-intersection2} of the Krull
  intersection theorem applied to the Noetherian local ring
  $(A_\mathfrak{p},\mathfrak{m})$.
\end{proof}


\section{Krull's theorems on heights and dimensions}
\label{sec:orgd3770e5}
\subsection{Principal ideal theorem}
\label{sec:orgb8870ab}

We start with the special case to which the general one will eventually be reduced:
\begin{lemma}
  Let $(A,\mathfrak{m})$ be a local Noetherian integral domain.
  Suppose that $\mathfrak{m}$ is a minimal prime of
  some principal ideal $(f)$, with $f \in \mathfrak{m}$.
  Then $\mathfrak{m}$ and $(0)$ are the only primes
  of $A$.

  In ``geometric'' terms: suppose
  that $\{\mathfrak{m}\} = V(f)$ for some $f \in \mathfrak{m}$.
  Then $\Spec(A) = \{\mathfrak{m}, (0)\}$.
\end{lemma}
\begin{proof}
  Let $\mathfrak{p}$ be any prime in $A$ other than
  $\mathfrak{m}$.  Necessarily
  $\mathfrak{p} \subsetneq \mathfrak{m}$; our task is to show
  that $\mathfrak{p} = (0)$.
  Since $A$ is a domain, it will suffice to show
  for some $n$ that $\mathfrak{p}^n = (0)$.
  Recall that $\mathfrak{p}^{(n)}$
  denotes the $n$th symbolic power of $\mathfrak{p}$,
  given here with respect
  to the injective localization map
  $A \hookrightarrow  A_\mathfrak{p}$
  by $\mathfrak{p}^{(n)} = A \cap \mathfrak{p}^n
  A_\mathfrak{p}$;
  it is a $\mathfrak{p}$-primary ideal which contains
  $\mathfrak{p}^n$.
  It will then suffice to verify that
  $\mathfrak{p}^{(n)} = (0)$ for some $n$.
  By \S\ref{sec:kernel-localize},
  we have $\cap_{n \geq 0} \mathfrak{p}^{(n)} = \ker(A
  \rightarrow A_\mathfrak{p}) = (0)$,
  so it will suffice to verify that
  the chain of ideals $\mathfrak{p}^{(n)}$
  stabilizes,
  i.e.,
  that
  $\mathfrak{p}^{(n)} = \mathfrak{p}^{(n+1)}$ for large $n$.


  Set $\overline{A} := A/(f)$,
  $\overline{\mathfrak{m} } := \mathfrak{m}/(f)$.  Our
  hypotheses imply that $\overline{\mathfrak{m}}$ is the only
  prime ideal of $\overline{A}$.  Thus $\overline{A}$ is a Noetherian ring of dimension $0$.
  By Theorem \ref{thm:artin-vs-noeth-dim0}, it follows
  that $\overline{A}$ is an Artin
  ring.
  Thus the descending chain of ideals
  $\mathfrak{p}^{(n)} + (f)$
  must stabilize;
  in particular,
  \[
  \mathfrak{p}^{(n)} \subseteq \mathfrak{p}^{(n+1)} + (f)
  \]
  for
  large $n$.
  This says that any $x \in \mathfrak{p}^{(n)}$
  may be written
  $x = y + z f$
  for some $y \in \mathfrak{p}^{(n+1)}$ and $z \in A$.
  In that case, $x-y \in \mathfrak{p}^{(n)}$,
  and so
  $z \in (\mathfrak{p}^{(n)} : f)$.
  Since $\mathfrak{p}^{(n)}$ is $\mathfrak{p}$-primary
  and $f \notin \mathfrak{p}$,
  we have $(\mathfrak{p}^{(n)} : f) = \mathfrak{p}^{(n)}$,
  and so in fact $z \in \mathfrak{p}^{(n)}$.
  Thus
  \[
  \mathfrak{p}^{(n)} \subseteq \mathfrak{p}^{(n+1)} + \mathfrak{p}^{(n)} f,
  \]
  and in fact equality holds,
  with the reverse containment being clear.
  This says that $f M = M$
  for the finitely-generated module
  $M := \mathfrak{p}^{(n)} / \mathfrak{p}^{(n+1)}$.
  Since $f \in \mathfrak{m} = \Jac(A)$,
  it follows from Nakayama's lemma that $M = 0$.
  Thus $\mathfrak{p}^{(n)} = \mathfrak{p}^{(n+1)}$ for large
  $n$,
  as was to be shown.
\end{proof}

\begin{theorem}\label{thm:krull-principal-ideal}
  Let $A$ be a Noetherian ring, and let $f \in A$.  
  \begin{enumerate}[(i)]
  \item Every
    minimal prime $\mathfrak{p}$ of $(f)$ satisfies
    $\height(\mathfrak{p}) \leq 1$.
  \item If $f$ is a non-zerodivisor,
    then every minimal prime $\mathfrak{p}$ of $(f)$
    satisfies $\height(\mathfrak{p}) = 1$.
  \end{enumerate}

  In ``geometric'' terms,
  $\codim(Z) \leq 1$ for each irreducible component $Z$
  of $V(f) \subseteq \Spec(A)$;
  if $f$ is a non-zerodivisor,
  then $\codim(Z) = 1$ for each such $Z$.
  (This ``generalizes''
  the fact from linear algebra
  that the kernel of a linear functional has codimension $\leq 1$,
  with equality whenever the functional is nonzero.)
\end{theorem}
\begin{proof}
  To deduce (ii) from (i), suppose that some minimal prime $\mathfrak{p}$
  of $(f)$ has $\height(\mathfrak{p}) = 0$.  Then $\mathfrak{p}$
  is a minimal prime of $(0)$, hence consists of zero-divisors,
  and so $f$ is a zerodivisor.

  Our main task is thus to establish (i).  We must verify that
  if $\mathfrak{p}_2$ is a minimal prime of $(f)$ and if
  $\mathfrak{p}_0 \subseteq \mathfrak{p}_1 \subsetneq
  \mathfrak{p}_2$
  are inclusions of prime ideals, then
  $\mathfrak{p}_0 = \mathfrak{p}_1$.  After replacing $A$ by its
  quotient $A/\mathfrak{p}_0$, we may reduce to the case that
  $\mathfrak{p}_0 = (0)$; in particular, $A$ is a local
  Noetherian domain.  After then replacing $A$ by its
  localization $A_{\mathfrak{p}_2}$, we reduce further to the
  case that $A$ is a local Noetherian domain whose maximal ideal
  $\mathfrak{p}_2$ is a minimal prime of $(f)$.
  We now appeal to the previous lemma.
\end{proof}

We will often apply the above result in a local
context:
\begin{corollary}\label{cor:krull-pith-local}
  Let $(A,\mathfrak{m})$ be a Noetherian local ring.
  Suppose there exists $f \in A$ for which
  $\mathfrak{m}$ is the unique prime containing $f$,
  thus
  $V(f) = \{\mathfrak{m} \}$.
  Then
  $\dim(A) = \height(\mathfrak{m}) \leq 1$.
\end{corollary}
\begin{proof}
  Given that $\mathfrak{m}$ is maximal,
  our assumption is equivalent
  to requiring that $\mathfrak{m}$ be a minimal
  prime of $(f)$.
\end{proof}

For the sake of illustration,
let's reformulate Theorem \ref{thm:krull-principal-ideal} in the contrapositive.
Let \(A\) be a Noetherian ring.
Let \(\mathfrak{p}_0 \subsetneq \mathfrak{p}_2\)
be an inclusion of primes in \(A\).
By an \emph{intermediary prime}
we will mean a prime \(\mathfrak{p}_1\)
for which
\(\mathfrak{p}_0 \subsetneq \mathfrak{p}_1 \subsetneq
  \mathfrak{p}_2\).
\begin{corollary}
  \label{cor:}
  The following are equivalent:
  \begin{enumerate}[(i)]
  \item There exists an intermediary prime.
  \item For each $f \in \mathfrak{p}_2$ there exists an
    intermediary
    prime containing $f$.
  \end{enumerate}

  In ``geometric'' terms,
  let $Y_2 \subsetneq Y_0$ be irreducible closed subsets
  of $\Spec(A)$.
  Then either there are no irreducible closed subsets
  $Y_1$ contained strictly between $Y_2$ and $Y_0$,
  or for each $f \in I(Y_2)$
  there exists an irreducible closed
  subset $Y_2 \subsetneq Y_1 \subsetneq Y_0$
  with $Y_1 \subseteq Z(f)$.
\end{corollary}
\begin{proof}
  We need only show that (i) implies (ii).
  If (ii) fails,
  then we may find $f \in \mathfrak{p}_2$
  not contained in any intermediary primes.
  In other words, after replacing $A$ with $A/\mathfrak{p}_0$ as
  necessary
  to reduce to the case that $\mathfrak{p}_0$ is a minimal prime
  of $A$,
  we are given that $\mathfrak{p}_2$ is a minimal prime of
  $(f)$.
  By Krull's principal ideal theorem,
  it follows that
  $\height(\mathfrak{p}_2) \leq 1$;
  thus there exist no intermediary primes,
  and so (i) fails.
\end{proof}

\subsection{Dimension theorem}
\label{sec:orgbe18c9e}

\begin{theorem}\label{thm:krull-dimn-thm}
  Let $A$ be a Noetherian ring,
  and let $f_1,\dotsc,f_n \in A$.
  Then each minimal prime $\mathfrak{p}$ of $(f_1,\dotsc,f_r)$
  satisfies $\height(\mathfrak{p}) \leq r$.
  In particular,
  $\height(f_1,\dotsc,f_r) \leq r$.

  In ``geometric'' terms,
  $\codim(Z) \leq r$
  for each irreducible component $Z$ of $V(f_1,\dotsc,f_r)
  \subseteq \Spec(A)$.
  (This ``generalizes'' the fact from linear algebra
  that the solution set to a system of $r$ linear equations
  has codimension $\leq r$.)
\end{theorem}

Here's a lemma that I think clarifies the key step in the proof.
\begin{lemma}\label{lem:key-step-krull-dimn}
  Let $(A,\mathfrak{m})$ be a Noetherian local ring, and
  let $f_1,\dotsc,f_r \in \mathfrak{m}$
  with $V(f_1,\dotsc,f_r) = \{\mathfrak{m}\}$.
  Let
  $\mathfrak{p} \subsetneq \mathfrak{m}$ be
  a prime with no prime strictly
  contained between $\mathfrak{p}$ and $\mathfrak{m}$.
  Then there exist $g_1,\dotsc,g_r \in \mathfrak{m}$
  for which
  \begin{enumerate}
  \item $V(g_1,\dotsc,g_r) = \{\mathfrak{m}\}$ and
  \item $\mathfrak{p}$ contains and is a minimal prime of $(g_1,\dotsc,g_{r-1})$.
  \end{enumerate}

  In ``geometric'' terms,
  let $Z$ be a closed irreducible subset of $\Spec(A)$
  that is minimal among the closed irreducible sets that properly contain $\{\mathfrak{m}\}$.
  Then we may find $g_1,\dotsc,g_r$
  for which $V(g_1,\dotsc,g_r) = \{\mathfrak{m}\}$
  and for which $Z$ is an
  irreducible component of $V(g_1,\dotsc,g_{r-1})$.
\end{lemma}
\begin{proof}
  Since $\mathfrak{m}$ is the unique prime ideal containing
  $(f_1,\dotsc,f_r)$,
  we may assume after reindexing $f_1,\dotsc,f_r$ as necessary
  that $f_r \notin \mathfrak{p}$.
  Then the ideal $\mathfrak{p} + (f_r)$ strictly contains
  $\mathfrak{p}$
  and is contained in $\mathfrak{m}$;
  our hypotheses on $\mathfrak{p}$ imply that
  $\mathfrak{m}$ is the only prime ideal
  containing $\mathfrak{p}  + (f_r)$,
  i.e., that
  $V(\mathfrak{p} + (f_r)) = \{\mathfrak{m}\}$,
  or that $\rad(\mathfrak{p} + (f_r)) = \mathfrak{m}$.
  In particular,
  for each $1 \leq i \leq r-1$
  we may find $n_i$
  for which $f_i^{n_i} \in \mathfrak{p} + (f_r)$,
  say
  \[
    f_i^{n_i} = g_i + z_i f_r
    \text{ with } g_i \in \mathfrak{p}, z_i \in A.
  \]
  We claim that the conclusion of the lemma is now satisfied
  with $g_1,\dotsc,g_{r-1}$ as above and $g_r := f_r$:
  \begin{enumerate}
  \item The above equation
    shows that any prime $\mathfrak{q}$ that contains
    $g_1,\dotsc,g_{r-1},f_r$
    also contains $f_i^{n_i}$ and hence $f_i$ for $1 \leq i \leq r$,
    hence $\mathfrak{q} = \mathfrak{m}$.
    Thus $V(g_1,\dotsc,g_r) = \{\mathfrak{m}\}$.
  \item It's clear by construction that $\mathfrak{p}$ contains
    $(g_1,\dotsc,g_{r-1})$.  There is thus a minimal prime
    $\mathfrak{p}'$ of $(g_1,\dotsc,g_{r-1})$ contained in
    $\mathfrak{p}$; we must verify that
    $\mathfrak{p} = \mathfrak{p} '$.  (Geometrically,
    $\mathfrak{p}'$ corresponds to an irreducible component $Z'$
    of $V(g_1,\dotsc,g_{r-1})$ containing $Z$.)  To see this,
    consider the quotient ring $\overline{A} := A/(g_1,\dotsc,g_{r-1})$.
    Let
    \begin{equation}\label{eq:prime-chain-in-quotient-for-key-lemma-of-dimn-thm}
            \overline{\mathfrak{m}} \supsetneq
      \overline{\mathfrak{p}} \supseteq \overline{\mathfrak{p}'}
    \end{equation}
    denote the chain of primes in $\overline{A}$
    given by the image of
    $\mathfrak{m} \supsetneq \mathfrak{p} \supseteq
    \mathfrak{p}' \supseteq (g_1,\dotsc,g_{r-1})$.
    Then $(\overline{A},\overline{\mathfrak{m}})$ is a
    Noetherian local ring,
    and our task is equivalent to showing that
    $\overline{\mathfrak{p} } = \overline{\mathfrak{p} '}$.
    Let
    $f \in \overline{A}$ denote the image of $f_r$.  The primes
    of $\overline{A}$ containing $f$ are in bijection with the
    primes of $A$ containing $g_1,\dotsc,g_{r-1},f_r$, so
    $V_{\overline{A}}(f)
    = \{\overline{\mathfrak{m}}\}$.    
    By Krull's principal
    ideal theorem (in the form of Corollary \ref{cor:krull-pith-local}),
    it follows that $\height(\overline{\mathfrak{m} }) \leq 1$.
    From
    \eqref{eq:prime-chain-in-quotient-for-key-lemma-of-dimn-thm}
    we then deduce that $\overline{\mathfrak{p} } =
    \overline{\mathfrak{p}'}$,
    as required.
    (Intuitively, by choosing $f_r$ not to vanish on any irreducible component
    of $V(f_1,\dotsc,f_{r-1})$,
    we guarantee that appending it to the set of generators has the effect
    of knocking down the dimension of each such component by $1$.)
  \end{enumerate}
\end{proof}

We now deduce Theorem \ref{thm:krull-dimn-thm}.  We must show
that if \(\mathfrak{p}\) is a minimal prime of \((f_1,\dotsc,f_r)\),
then \(\height(\mathfrak{p}) \leq r\).  We may assume without loss
of generality (replacing \(A\) with \(A_\mathfrak{p}\) and
\(\mathfrak{p}\) with \(\mathfrak{p}_\mathfrak{p}\), which doesn't
change the height of or minimality assumption on the latter)
that \((A,\mathfrak{p})\) is a Noetherian local ring with
\(V(f_1,\dotsc,f_r) = \{\mathfrak{p}\}\); we must show then that
\(\height(\mathfrak{p}) \leq r\).  We do this by induction on \(r\).
The case \(r = 1\) is given by Krull's principal ideal theorem, so
suppose \(r > 1\).  Let \(\mathfrak{q} \subsetneq \mathfrak{p}\) be
a maximal element of the set of primes strictly contained in
\(\mathfrak{p}\); it will suffice then to show that
\(\height(\mathfrak{q}) \leq r-1\).  By Lemma
\ref{lem:key-step-krull-dimn}, we may assume without loss of
generality that \(\mathfrak{q}\) is a minimal prime of
\((f_1,\dotsc,f_{r-1})\); the required inequality then follows from our
inductive hypothesis.

\begin{corollary}
  Let $\mathfrak{a}$ be an ideal in a Noetherian ring $A$.
  Then $\height(\mathfrak{a}) < \infty$.
\end{corollary}
\begin{proof}
  Write $\mathfrak{a} = (f_1,\dotsc,f_r)$.
  Then $\height(\mathfrak{a}) \leq r$.
\end{proof}
\begin{corollary}
  Let $(A,\mathfrak{m})$ be a Noetherian local ring.
  Then $\dim(A) = \height(\mathfrak{m}) < \infty$.
\end{corollary}
\begin{proof}
Use Lemma \ref{lem:local-ring-height-equals-dim}.
\end{proof}
\begin{remark}
  Dimension theory works best for \emph{local} Noetherian rings:
  there exist non-local Noetherian rings $A$ with
  $\dim(A) = \infty$.  On the other hand, the height of an ideal
  in a Noetherian ring $A$ is always finite, regardless of
  whether $A$ is local.
\end{remark}

\subsection{Converse to the dimension theorem}
\label{sec:org858899e}

\begin{theorem}\label{thm:conv-to-krull-1}
Let $A$ be a Noetherian ring.
Let $r,s$ be nonnegative integers
with $s \leq r$.
Let $\mathfrak{a}$ be an ideal with $\height(\mathfrak{a})
\geq r$,
and let
$f_1,\dotsc,f_s \in \mathfrak{a}$
satisfy
$\height(f_1,\dotsc,f_s) = s$.
Then there exist $f_{s+1},\dotsc,f_r \in \mathfrak{a}$
so that
$\height(f_1,\dotsc,f_i) = i$
for all $s \leq i \leq r$.
\end{theorem}
\begin{proof}
  It suffices (after finitely many iterations) to consider the
  case $s = r-1$.  For each minimal prime $\mathfrak{q}$ of
  $(f_1,\dotsc,f_{r-1})$, we have $\height(\mathfrak{q}) = r-1$
  (here the inequality ``$\geq$'' follows from our assumption
  $\height(f_1,\dotsc,f_{r-1}) = r-1$, while ``$\leq$'' follows
  from the Krull dimension theorem);
  it follows from this
  and the inequality
  $\height(\mathfrak{a}) \geq r$
  that $\mathfrak{a} \not\subseteq \mathfrak{q}$.
  By the prime avoidance lemma (\S\ref{sec:prime-avoidance}),
  we may find an element $f_r \in \mathfrak{a}$ not contained in
  any minimal prime
  of $(f_1,\dotsc,f_{r-1})$.
  We claim then that $\height(f_1,\dotsc,f_r) = r$.
  Consider any minimal prime $\mathfrak{q}$ of
  $(f_1,\dotsc,f_r)$;
  we must verify that $\height(\mathfrak{q}) = r$.
  The upper bound ``$\leq$'' follows
  as before from the Krull dimension theorem.
  For the lower bound,
  note that $\mathfrak{q}$ contains $(f_1,\dotsc,f_{r-1})$,
  and so contains some minimal prime $\mathfrak{q} '$ of
  $(f_1,\dotsc,f_{r-1})$.
  By construction, we have $f_r \in \mathfrak{q}$
  but $f_r \notin \mathfrak{q} '$,
  hence $\mathfrak{q} \supsetneq \mathfrak{q} '$,
  and so $\height(\mathfrak{q}) > \height(\mathfrak{q} ')= r-1$.
  This forces $\height(\mathfrak{q}) = r$, as required.
\end{proof}
\begin{remark}
  It may be instructive
  to recast in geometric terms
  some parts of the proof given above.
  Our hypothesis is that each irreducible
  component of $V(f_1,\dotsc,f_{r-1})$ has codimension $r-1$,
  while each irreducible component of $V(\mathfrak{a})$ has
  codimension
  $\geq r$.
  It follows readily that $V(\mathfrak{a})$ contains
  no irreducible component of $V(f_1,\dotsc,f_{r-1})$.
  By the prime avoidance lemma,
  we may thus find an element $f_r \in \mathfrak{a}$
  which does not vanish on any irreducible component
  of $V(f_1,\dotsc,f_{r-1})$.
  Now, each irreducible component $Z$ of $V(f_1,\dotsc,f_r)$
  is contained in some irreducible component $Z'$ of
  $V(f_1,\dotsc,f_{r-1})$.
  Since $f|_{Z'} \neq 0$,
  this containment must be strict: $Z \subsetneq Z'$.
  Therefore $\codim(Z) \geq r$;
  Krull then gives $\codim(Z) \leq r$,
  hence $\codim(Z) = r$,
  hence $\codim(V(f_1,\dotsc,f_r)) = r$, as required.
\end{remark}

\begin{corollary}\label{cor:existence-parameters-for-prime}
  Let $\mathfrak{p}$ be a prime ideal of height $r$ in a
  Noetherian ring $A$.  Then there exist
  $f_1,\dotsc,f_r \in \mathfrak{p}$ so that $\mathfrak{p}$ is a
  minimal prime of $(f_1,\dotsc,f_r)$.  

  In ``geometric'' terms, every closed irreducible subset $Z$ of
  $\Spec(A)$ with $\codim(Z) = r$ arises as an irreducible
  component of $V(f_1,\dotsc,f_r) \subseteq \Spec(A)$ for some
  $f_1,\dotsc,f_r \in A$.
\end{corollary}
\begin{proof}
  We apply the previous result
  with $s = 0$.
\end{proof}

Here's a slightly sharper variant:
\begin{theorem}
  Let $\mathfrak{p}$ be a prime ideal in a Noetherian ring
  with $\height(\mathfrak{p}) = r$.
  Let
  $\mathfrak{p}_0 \subsetneq \dotsb \subsetneq \mathfrak{p}_r =
  \mathfrak{p}$ be a chain of primes
  realizing the height of $\mathfrak{p}$.
  (Note that this forces $\height(\mathfrak{p}_i) = i$ for all $i$.)
  There exist
  $f_1,\dotsc,f_r$ so that
  for each $0 \leq i \leq r$,
  \begin{itemize}
  \item $\height(f_1,\dotsc,f_i) = i$, and
  \item $\mathfrak{p}_i$ is a
    minimal prime of $(f_1,\dotsc,f_i)$ for each $0 \leq i \leq
    r$.
  \end{itemize}

  In ``geometric'' terms,
  let $Z_0 \supsetneq \dotsb \supsetneq Z_r$
  be closed irreducible subsets of $\Spec(A)$
  with $\codim(Z_r) = r$.
  (Note that this forces $\codim(Z_i) = i$ for all $i$.)
  Then we may find $f_1,\dotsc,f_r \in A$
  so that for each $0 \leq i \leq r$,
  \begin{itemize}
  \item every irreducible component of $V(f_1,\dotsc,f_i)$
    has codimension $i$, and
  \item $Z_i$ is an irreducible component of
    $V(f_1,\dotsc,f_i)$.
  \end{itemize}
\end{theorem}
\begin{proof}
  We argue by induction as above,
  choosing $f_{i+1}$ to belong to $\mathfrak{p}_{i+1}$
  but not to any minimal prime of $(f_1,\dotsc,f_i)$.
\end{proof}

\section{Systems of parameters}
\label{sec:orgfa54a11}
\subsection{A characterization of dimension}
\label{sec:org815b8ba}

\begin{lemma}\label{lem:easy-equivalences-before-system-of-params}
  Let $(A,\mathfrak{m})$ be a Noetherian local ring
  and $x_1,\dotsc,x_n \in \mathfrak{m}$.
  The following conditions are equivalent:
  \begin{enumerate}[(i)]
  \item $\mathfrak{m}$ is the only prime containing
    $(x_1,\dotsc,x_n)$,
    i.e.:
    \[
    V((x_1,\dotsc,x_n)) = \{\mathfrak{m}\}.
    \]
  \item $\mathfrak{m}$ is a minimal prime of $(x_1,\dotsc,x_n)$.
  \item $\rad((x_1,\dotsc,x_n)) = \mathfrak{m}$.
  \item The ideal $(x_1,\dotsc,x_n)$ is $\mathfrak{m}$-primary.
  \end{enumerate}
\end{lemma}
\begin{proof}
  The equivalence of (i),(ii) and (iii) follows from the
  assumption that $\mathfrak{m}$ is maximal.  The equivalence of
  (iii) and (iv) follows from the fact that
  an ideal is primary whenever its radical is a maximal ideal.
\end{proof}

\begin{theorem}\label{thm:dimension-in-terms-of-params}
  Let $(A,\mathfrak{m})$ be a Noetherian local ring.
  Then
  $\dim(A)$ is the smallest integer $n$ for which the equivalent
  conditions of
  Lemma
  \ref{lem:easy-equivalences-before-system-of-params}
  are satisfied,
  i.e.,
  \[
  \dim(A) = \min \{n \geq 0 : \exists
  x_1,\dotsc,x_n \in \mathfrak{m}
  \text{ with }
  V((x_1,\dotsc,x_n)) = \{\mathfrak{m} \}
  \}.
  \]
\end{theorem}
\begin{proof}
  If there exist
  $x_1,\dotsc,x_n$ with $V((x_1,\dotsc,x_n)) =
  \{\mathfrak{m} \}$
  then
  Krull's dimension theorem implies that
  $\dim(A) = \height(\mathfrak{m}) \leq n$.
  If $n = \dim(A)$,
  then
  the ``converse to Krull'' (Corollary \ref{cor:existence-parameters-for-prime})
  implies that there exist
  $x_1,\dotsc,x_n$ with $V((x_1,\dotsc,x_n)) =
  \{\mathfrak{m} \}$.
\end{proof}
Theorem \ref{thm:dimension-in-terms-of-params} will be very useful
as a tool for giving \emph{upper bounds} on the dimension
of a Noetherian local ring \((A,\mathfrak{m})\):
to show that \(\dim(A) \leq n\), it suffices
to construct elements \(x_1,\dotsc,x_n\)
with \(V((x_1,\dotsc,x_n)) = \mathfrak{m}\).

\subsection{Definition}
\label{sec:org9e31ea4}

\begin{definition}
  Let $(A,\mathfrak{m})$ be a Noetherian local ring.
  We say that $x_1,\dotsc,x_n \in \mathfrak{m}$
  form a \emph{system of parameters}
  for $\mathfrak{m}$
  if
  \begin{enumerate}[(i)]
  \item $n = \dim(A) = \height(\mathfrak{m})$,
    and
  \item   the equivalent conditions of
    Lemma
    \ref{lem:easy-equivalences-before-system-of-params}
    are
    satisfied,
    e.g., if $V((x_1,\dotsc,x_n)) = \{\mathfrak{m}\}$.
  \end{enumerate}
\end{definition}
Theorem \ref{thm:dimension-in-terms-of-params} implies
that systems of parameters exist.

\subsection{Extensions of partial systems of parameters}
\label{sec:orgdc48040}

Let \((A,\mathfrak{m})\) be a Noetherian local ring.
Given a collection of \(x_1,\dotsc,x_r \in \mathfrak{m}\) of elements of its maximal ideal,
we aim to understand when this collection may be extended
to a system of parameters.
To that end, define the quotient ring
\(\overline{A} := A / (x_1,\dotsc,x_r)\);
it is a Noetherian local ring with maximal ideal
\(\overline{\mathfrak{m}}\) given by the image of \(\mathfrak{m}\),
and satisfies the following general dimension lower-bound:
\begin{lemma}\label{lem:lower-bound-dimension-quotient-by-r-gens}
  $\dim(\overline{A}) \geq \dim(A) - r$.
\end{lemma}
\begin{proof}
  Write $s = \dim(\overline{A})$.
  Choose elements $y_1,\dotsc,y_s \in A$
  whose images $\overline{y_1},\dotsc,\overline{y_s} \in
  \overline{A}$
  form a system of parameters for $\overline{\mathfrak{m}}$.
  In particular,
  $\overline{\mathfrak{m}}$
  is the only prime containing
  $(\overline{y_1},\dotsc,\overline{y_s})$.
  It follows that
  $\mathfrak{m}$ is the only prime containing
  $(x_1,\dotsc,x_r,y_1,\dotsc,y_s)$.
  From this we deduce the upper bound $\dim(A) \leq r + s$,
  which rearranges to the required inequality.
\end{proof}

\begin{theorem}\label{thm:extensions-systems-params}
  Among the following assertions,
  (i) implies (ii) and (iii),
  while (ii) and (iii) are equivalent.
  \begin{enumerate}[(i)]
  \item $\height((x_1,\dotsc,x_r)) = r$.
  \item We may extend $\{x_1,\dotsc,x_r\}$ to a system of
    parameters
    for $\mathfrak{m}$.
  \item $\dim(\overline{A}) = \dim(A) - r$.
  \end{enumerate}
\end{theorem}
\begin{proof}~
  \begin{itemize}
  \item (i) implies (ii): Set
    $n := \height(\mathfrak{m}) = \dim(A)$.  Then every prime in
    $A$ has height $\leq n$, so $r \leq n$.  By Theorem \ref{thm:conv-to-krull-1},
     we may find
    $x_{r+1},\dotsc,x_n$ for which
    $\height((x_1,\dotsc,x_n)) = n$, i.e., so that $n$ is the
    minimal height among primes containing $(x_1,\dotsc,x_n)$.
    Since $\mathfrak{m}$ is the unique prime in $A$ of height
    $n$, we deduce that it is the only prime containing
    $(x_1,\dotsc,x_n)$.
    Thus $x_1,\dotsc,x_n$ is a system of parameters.
  \item
    (i) implies (iii):
    we combine Lemma
    \ref{lem:lower-bound-dimension-quotient-by-r-gens}
    with the easy inequality
    $\dim(A) \geq \dim(\overline{A}) +
    \height((x_1,\dotsc,x_r))$ (cf. Lemma \ref{lem:easy-dimension-inequality}).
    (This implication has been included redundantly for the sake of illustration.)
  \item (ii) implies (iii):
    Suppose we can extend $x_1,\dotsc,x_r$
    to a system of parameters $x_1,\dotsc,x_r,y_1,\dotsc,y_s$
    for $\mathfrak{m}$.
    Then $r + s = \dim(A)$
    and $V((\overline{y_1},\dotsc,\overline{y_s})) =
    \overline{\mathfrak{m}}$,
    whence $s \geq \dim(\overline{A})$;
    by Lemma \ref{lem:lower-bound-dimension-quotient-by-r-gens},
    we deduce that
    $s \geq \dim(\overline{A}) \geq \dim(A) - r = s$,
    so equality holds and $\dim(\overline{A}) = s$, as required.
  \item (iii) implies (ii): Suppose that
    $s := \dim(\overline{A}) = \dim(A) - r$.  Let
    $y_1,\dotsc,y_s \in \mathfrak{m}$ be such that their images
    $\overline{y_1},\dotsc,\overline{y_s}$ form a system of
    parameters for $\overline{\mathfrak{m}}$.  Then
    $V((x_1,\dotsc,x_r,y_1,\dotsc,y_s)) = \{\mathfrak{m}\}$ and
    $r + s = \dim(A)$, so $x_1,\dotsc,x_r,y_1,\dotsc,y_s$ gives
    the required extension of $x_1,\dotsc,x_r$ to a system of
    parameters for $\mathfrak{m}$.
\end{itemize}
\end{proof}

\begin{corollary}
  Let $(A,\mathfrak{m})$ be a Noetherian local ring,
  and let $f \in \mathfrak{m}$ be a non-zerodivisor.
  Then
  \[
  \dim(A/(f)) = \dim(A) - 1.
  \]
\end{corollary}
\begin{proof}
  Since $f$ is a non-zerodivisor,
  Krull's principal ideal theorem
  implies that $\height((f)) = 1$.
  Theorem \ref{thm:extensions-systems-params}
  applies with $r := 1$ and $x_1 := f$
  to produce an extension of $\{f\}$
  to a system of parameters
  $f,y_1,\dotsc,y_s$ for $A$,
  with $s := \dim(A/(f))$.
  In particular, $\dim(A) = s + 1$, as required.
\end{proof}



\section{Dimensions of polynomial rings}
\label{sec:org80bf51b}

\begin{theorem}
Let $A$ be a Noetherian ring,
and $n \in \mathbb{Z}_{\geq 0}$.
Then \[\dim A[X_1,\dotsc,X_n] = \dim A + n.\]
\end{theorem}
\begin{proof}
  By iterating, it suffices to consider the case $n=1$.
  Set $r := \dim(A)$.
  We must verify that $\dim A[X] = r + 1$.
  Let $\mathfrak{p}_0 \subsetneq \dotsb \subsetneq
  \mathfrak{p}_r$ be a chain of primes in $A$ of length realizing the dimension of $A$.
  Then
  \[
    \mathfrak{p}_0 A[x] \subsetneq \dotsb \subsetneq
    \mathfrak{p}_r A[X]
    \subsetneq \mathfrak{p}_r A[X] + X A[X]
  \]
  is readily seen to give a chain of primes in $A[X]$ of length
  $r+1$,
  hence $\dim A[X] \geq r + 1$.
  The upper bound is trickier.
  It will suffice to show for each maximal ideal $\mathfrak{m}
  \subseteq A[X]$ 
  that $\height(\mathfrak{m}) \leq r + 1$.
  Set $\mathfrak{p} := \mathfrak{m} \cap A$;
  it is a prime ideal.
  We may replace $A$ with its localization $A_\mathfrak{p}$
  and $A[X]$ with 
  $(A[X])_\mathfrak{p} = A_\mathfrak{p}[X]$
  to reduce to the case that $(A,\mathfrak{p})$ is a Noetherian
  local ring.
  The quotient $A/\mathfrak{p}$ is then a field
  and so the ring $A[X]/\mathfrak{p} A[X] = A/\mathfrak{p}[X]$ is then
  a PID.
  The image of $\mathfrak{m}$ in the latter ring is thus
  principal.
  We may thus write $\mathfrak{m} = \mathfrak{p} A[X] + f A[X]$
  for some $f \in A[X]$.
  Let $x_1,\dotsc,x_r \in \mathfrak{p}$ be a system  of
  parameters for $\mathfrak{p}$.
  Then $\mathfrak{m}$ is the only prime containing
  $(x_1,\dotsc,x_r,f)$:
  any such prime $\mathfrak{q}$ contains $x_1,\dotsc,x_r$
  and hence contains $\mathfrak{p}$,
  and so identifies with a prime ideal in the quotient
  $A/\mathfrak{p}[X]$ that contains
  the image of $f$,
  whence $\mathfrak{q} = \mathfrak{m}$.
  It follows from Theorem
  \ref{thm:dimension-in-terms-of-params} that $\dim(A) = \height(\mathfrak{m}) \leq r + 1$, as required.
\end{proof}
For example:
\begin{proposition}
  Let $A := \mathbb{C}[X_1,\dotsc,X_n]$,
  and let $\mathfrak{m}$ be a maximal ideal,
  thus $\mathfrak{m} = (X_1-x_1,\dotsc,X_n-x_n)$
  for some $(x_1,\dotsc,x_n) \in \mathbb{C}^n$.
  Then
  $\height(\mathfrak{m}) = n$.
  The localization $A_\mathfrak{m}$ is a local ring of
  dimension $n$,
  whose maximal ideal is generated by a system of parameters.
\end{proposition}
\begin{proof}
  The  ideals $\mathfrak{p}_i := (X_1-x_1, \dotsc, X_i - x_i)$
  ($i=0..n$) are prime, distinct and increasing
  to $\mathfrak{p}_n = \mathfrak{m}$,
  so $\height(\mathfrak{m}) \geq n$.
  Conversely,
  it's clear that $\height(\mathfrak{m}) \leq \dim(A) = n$.
  Therefore $\height(\mathfrak{m}) = n$.
  The assertion concerning $A_\mathfrak{m}$
  then follows from the identity
  $\dim(A_\mathfrak{m}) = \height(\mathfrak{m})$
  and the fact that $\mathfrak{m}$
  is generated by $X_1-x_1,\dotsc,X_n-x_n$.
\end{proof}

\section{Preliminaries on regular local rings}
\label{sec:org3fde1d4}

Let \((A,\mathfrak{m})\)  be a Noetherian local ring of dimension
\(d := \dim(A) = \height(\mathfrak{m})\).
Denote by \(k := A/\mathfrak{m}\) the residue field.
For any module \(M\),
the quotient \(M/\mathfrak{m} M\) is then naturally a \$k\$-vector
space.
This consideration applies in particular
when \(M = \mathfrak{m}\),
so that \(M / \mathfrak{m} M = \mathfrak{m} / \mathfrak{m}^2\).
\begin{lemma}
  In general,
  $\dim_k \mathfrak{m} / \mathfrak{m}^2 \geq d$.
  The following are equivalent:
  \begin{enumerate}[(i)]
  \item $\mathfrak{m}$ is generated by $d$ elements,
    necessarily a system of parameters.
  \item $\dim_{k} \mathfrak{m}/\mathfrak{m}^2 = d$.
  \end{enumerate}
\end{lemma}
\begin{proof}
  Set $r := \dim_k \mathfrak{m} / \mathfrak{m}^2$.

  For the first inequality,
  suppose $x_1,\dotsc,x_{r} \in \mathfrak{m}$
  have the property that their images give a $k$-basis of
  $\mathfrak{m} / \mathfrak{m}^2$.
  By Nakayama's lemma,
  it follows that $x_1,\dotsc,x_r$ generate $\mathfrak{m}$.
  By Krull's dimension theorem,
  it follows that $d = \height(\mathfrak{m}) \leq r$,
  as required.

  (i) implies (ii):
  If $x_1,\dotsc,x_d$ generate $\mathfrak{m}$,
  then their images span $\mathfrak{m}/\mathfrak{m}^2$,
  whence $d \geq r$.
  Comparing with the reverse inequality which holds in general,
  we deduce that $d = r$.

  (ii) implies (i):
  Assuming (ii),
  we may find $x_1,\dotsc,x_d \in \mathfrak{m}$ which generate
  $\mathfrak{m}/\mathfrak{m}^2$,
  hence (by Nakayama) generate $\mathfrak{m}$,
  giving (i).
\end{proof}

\section{Basics on integral extensions}
\label{sec:org4afff50}

Let \(A \subseteq B\) be rings.
The discussion that follows
applies with minor modifications to any \$A\$-algebra
\(\phi : A \rightarrow B\),
by replacing \(A\) with \(\phi(A)\).

We say that \(x \in B\) is \emph{integral over $A$}
if it is a root of a monic polynomial with coefficients in \(a\),
i.e., if there exist \(a_1,\dotsc,a_n \in A\)
so that
\[
  x^n + a_1 x^{n-1} + \dotsb + a_n = 0.
  \]
We say that \(B\) is \emph{integral over $A$}
if each \(x \in B\) is integral over \(A\).

For example, if \(K \subset L\) are fields, then \(x \in L\)
(resp. \(L\) itself)
is integral
over \(K\) precisely when it is algebraic over \(K\).


We say that the extension \(A \subseteq B\) is \emph{finite} if \(B\) is a finite \$A\$-module.
(We might also say more generally that a morphism \(\phi : A \rightarrow B\) is finite
if \(\phi(A) \subseteq B\) is finite.)

The key point is the following:
\begin{theorem}\label{thm:characterizing-integrality}
  Let $x \in B$.
  The following are equivalent:
  \begin{enumerate}[(i)]
  \item $x$ is integral over $A$.
  \item The subring $A[x] \subseteq B$
    generated by $x$ and $\phi(A)$
    is a finitely-generated $A$-module.
  \item $A[x]$ is contained in a subring $C$ of $B$
    which is finitely-generated as an $A$-module.
  \item There is a faithful $A[x]$-module $M$
    which is finitely-generated over $A$.
  \end{enumerate}
\end{theorem}

\begin{proof}[Proof of Theorem \ref{thm:characterizing-integrality}]
  (i) implies (ii):
  If $x$ is integral over $A$,
  say $x^n = \dotsb$,
  then the subring $A[x]$ is generated
  as an $A$-module
  by $1, x, \dotsc, x^{n-1}$.

  (ii) implies (iii):
  take $C := A[x]$.

  (iii) implies (iv):
  take $M := C$, and note $C \ni 1$.

  (iv) implies (i):
  Write $M = \sum_{i=1}^n A e_i$,
  then $x M \subseteq M$ gives
  equations
  $x e_i = \sum_{j=1}^n a_{i j} e_j$,
  which we may rewrite in the form
  $\Delta e = 0$,
  where
  $\Delta_{i j} := \delta_{i j} x -  a_{i j} \in A[x]$
  and $e$ is the column vector with $j$th entry $e_j \in A$.
  By Cramer's rule, we have
  $\Delta^{\ad} \Delta = \det(\Delta) 1_n$
  for some matrix $\Delta^{\ad}$ with entries in $A[x]$,
  whose $(i,j)$ entry
  may be expressed as $(-1)^{i+j} \det (\Delta^{i,j})$,
  where $\Delta^{i,j}$ denotes the matrix obtained
  by striking out from $\Delta$ the $i$th row and $j$th column.
  Thus $\det(\Delta) e_j = 0$ for all $j$.
  Since $M$ is faithful,
  it follows that
  $\det(\Delta) = 0$, which gives a monic polynomial equation for $x$ over $A$.
\end{proof}

\begin{lemma}
  Let $A \subseteq B \subseteq C$ be rings.
  Let $x \in C$ be integral over $A$.
  Then $x$ is integral over $B$.
\end{lemma}
\begin{proof}
  Clear: a monic equation satisfied by $x$ with coefficients in $A$
  also has coefficients in $B$.
\end{proof}
\begin{lemma}
  Let $A \subseteq B \subseteq C$ be rings,
  with $B$ finite over $A$ and $C$ finite over $B$.
  Then $C$ is finite over $A$.
\end{lemma}
\begin{proof}
  By hypothesis,
  we have $B = \sum_i A x_i$ and $C = \sum_j B y_j$
  for some finite collections $\{x_i\} \subseteq B$,
  $\{y_j\} \subseteq C$.
  Then
  \[
  C = \sum_j (\sum_i A x_i) y_j
  = \sum_{i,j} A x_i y_j.
  \]
\end{proof}

\begin{lemma}
Let $x_1,\dotsc,x_n \in B$ be integral over $A$.
Then $A[x_1,\dotsc,x_n]$ is a finite over $A$.
\end{lemma}
\begin{proof}
  We induct on $n$.
  The case $n = 1$ follows from the theorem.
  For $n > 1$,
  we see that $A[x_1,\dotsc,x_n]$ is integral and hence finite
  over $A[x_1,\dotsc,x_{n-1}]$, which is in turn finite over
  $A$; we then conclude by the previous lemma.  
\end{proof}

\begin{corollary}
  The following are equivalent
  for an extension of rings $A \subset B$:
  \begin{enumerate}
  \item $B$ is finite over $A$.
  \item $B$ is integral and finitely-generated over $A$.
  \end{enumerate}
\end{corollary}
\begin{proof}
  If $B$ is finite over $A$,
  then the equivalence ``(iii) implies (i)''
  proved above
  shows that it is integral over $A$;
  on the other hand, it is clearly finitely-generated.
  The converse is given by the preceeding lemma.
\end{proof}

\begin{corollary}
  The set of elements $x \in B$ that are integral over $A$ forms
  an $A$-subalgebra of $B$.
\end{corollary}
\begin{proof}
  If $x,y \in B$ are integral over $A$,
  then the previous results show that $A[x,y]$
  is finite over $A$,
  hence integral over $A$.
  In particular, $a_1 x + a_2 y$ (for $a_1,a_2 \in A$)
  and $x y$ are integral over $A$.
\end{proof}

\begin{definition}
  Let $A \subseteq B$ be rings.
  The \emph{integral closure $\overline{A}$ of $A$ in $B$}
  is the set of all $x \in B$ that are integral over $A$.
  By the previous result, $\overline{A}$ is an $A$-subalgebra of $B$.
  We say that $A$ is \emph{integrally closed in $B$}
  if $A = \overline{A}$,
  i.e., if every element of $B$ that is integral over $A$ already belongs to $A$.
\end{definition}

\begin{lemma}
  Let $A \subset B \subset C$ be rings.
  If $B$ is integral over $A$ and
  $C$ is integral over $B$,
  then $C$ is integral over $A$.
\end{lemma}
\begin{proof}
  Let $x \in C$;
  we must check that $x$ is integral over $A$.
  Since $C$ is integral over $B$, we have
  \[
  x^n + b_1 x^{n-1} + \dotsb + b_n = 0
  \]
  for some $b_1,\dotsc,b_n \in B$.  Since each $b_i$ is integral
  over $A$, the ring $B_0 := A[b_1,\dotsc,b_n]$ is finite over
  $A$.  The element $x$ is integral over $B_0$, and so
  $B_0[x]= A[b_1,\dotsc,b_n,x]$ is finite over $B_0$.  Thus
  $A[b_1,\dotsc,b_n,x]$ is finite over $A$.  By Theorem
  \ref{thm:characterizing-integrality}, we conclude that $x$ is
  integral over $A$.
\end{proof}

\begin{corollary}
  Let $\overline{A}$ denote the integral
  closure of $A \subset B$.
  Then $\overline{A}$ is integrally closed in $B$.
\end{corollary}


\begin{lemma}
  Let $A \subset B$ be an integral extension of rings.
  \begin{enumerate}
  \item Let $\mathfrak{b} \subset B$ be an ideal, and set
    $\mathfrak{a} := A/\mathfrak{a}$.
    Then
    the extension of rings
    $A / \mathfrak{a} \subset B / \mathfrak{b}$ is integral.
  \item Let $S \subset A$ be a multiplicative subset.
    Then $S^{-1} A \subset S^{-1} B$ is integral.
  \end{enumerate}
\end{lemma}
\begin{proof}
  Just do it.
\end{proof}

\begin{definition}
  We say that an integral domain $A$ is \emph{normal}
  if it is integrally closed in its field of fractions $K := \Frac(A)$.
\end{definition}

\begin{lemma}
Let $A$ be a UFD.  Then $A$ is normal.
\end{lemma}
\begin{proof}
  Suppose $r/s \in K := \Frac(A)$ is integral over $A$,
  where $r,s \in A$ are coprime.
  Then $(r/s)^n + a_1 (r/s)^{n-1} + \dotsb + a_n = 0$ for some
  $a_1,\dotsc,a_n \in A$.
  Thus
  \[
  -r^n = a_1 r^{n-1} s + a_2 r^{n-2} s^2 + \dotsb + a_n s^n.
  \]
  The RHS and hence the LHS is then divisible by $s$,
  so $s$ divides $r^n$;
  since $r,s$ are coprime, we must have $s \in A^\times$,
  and so $r/s \in A$.
\end{proof}

\begin{example}
  In particular, $\mathbb{Z}$ and $k[x]$ (for a field $k$)
  are both normal.
  On the other hand, the ring $A := k[x^2,x^3]$ is not integrally
  closed in its field of fractions $K := k(x)$ (because $x \in K - A$ is integral over $A$),
  hence is not normal.

  The integral closure of $\mathbb{Z}$ in $\mathbb{C}$ is the
  ring of \emph{algebraic integers}.  Given a finite field
  extension $K/\mathbb{Q}$, the integral closure $\mathcal{O}_K$
  of $\mathbb{Z}$ in $K$ is called the \emph{ring of integers}
  of $K$.  These rings are all normal.
  One can show that if
  $K = \mathbb{Q}(\sqrt{-3})$,
  then $\mathcal{O}_K = \mathbb{Z} [\frac{1 + \sqrt{-3}}{2}]$,
  which properly contains the subring
  $\mathbb{Z}[\sqrt{-3}]$,
  which is thus \emph{not} a normal ring.

  More generally, if $A$ is a normal domain with field of
  fractions $K$, if $L/K$ is a field extension, and if $B$
  denotes the integral closure of $A$ in $L$, then $B$ is
  normal.
  This situation arises often both in algebraic number theory
  (where $K = \mathbb{Q}$, for instance)
  and in the study of curves in algebraic geometry (where,
  e.g., $K = k[t]$ for some field $k$).

\end{example}

The \emph{normalization} \(A^{\mathrm{norm}}\) of an integral domain \(A\)
is the integral closure of \(A\) inside \(\Frac(A)\).
The preceeding result shows that \(A^{\mathrm{norm}}\) is normal.

\begin{lemma}
  Normality is a local property,
  i.e., the following are equivalent
  for an integral domain $A$:
  \begin{enumerate}
  \item $A$ is normal.
  \item $S^{-1} A$ is normal for each multiplicative subset $S \subset A$.
  \item $A_\mathfrak{m}$ is normal for each maximal ideal $\mathfrak{m}$ of $A$.
  \end{enumerate}
\end{lemma}
\begin{proof}
  (i) implies (ii):
  Welp, let $x \in K$ be integral over $S^{-1} A$.
  Write down a monic equation.
  Clearing denominators, we deduce that $s x$ is integral over
  $A$ for some
  $s \in S$.
  Since $A$ is normal,
  we have $s x = a$ for some $a \in A$,
  hence $x = a/s \in S^{-1} A$.

  (ii) implies (iii): immediate.

  (iii) implies (i): Let $x \in K$ be integral over $A$.  For
  each $\mathfrak{m}$, we may then find some
  $b_\mathfrak{m} \in \mathfrak{m}$ so that $b_\mathfrak{m} x$
  is integral over $A$.  The ideal generated by the
  $b_\mathfrak{m}$ is not contained in any maximal ideal, hence
  equals the unit ideal, so we may write
  $1 = \sum a_\mathfrak{m} b_\mathfrak{m}$ as a finite sum with
  $a_\mathfrak{m} \in A$, almost all zero.  Then
  $x = \sum a_\mathfrak{m} b_\mathfrak{m} x$ is integral over
  $A$.
\end{proof}

\section{Lying over}
\label{sec:org5a401b5}

\begin{lemma}
  Let $A \subseteq B$ be an integral extension of integral
  domains.
  Then $A$ is a field if and only if $B$ is a field.
  Equivalently, if $A \subseteq B$  is an integral extension of rings
  and $\mathfrak{q} \subset B$ is prime,
  then $\mathfrak{p} := \mathfrak{q} \cap A$ is maximal if and only if
  $\mathfrak{q}$ is maximal.
\end{lemma}
\begin{proof}
  Suppose $A$ is a field.  Let $x \in B$ is nonzero.  Write
  $x^n + a_1 x^{n-1} + \dotsb + a_n = 0$, with $n$ minimal.
  Since $B$ is a domain, we then have $a_n \neq 0$, and so
  $x^{-1} = -(x^{n-1} + a_1 x^{n-2} + \dotsb + a_{n-1})/a_n$.
  Thus $B$ is a field.

  Conversely, suppose $B$ is a field Let $x \in A$ be nonzero.
  Then $1/x \in B$.  Write
  $(1/x)^n + a_1 (1/x)^{n-1} + \dotsb + a_n = 0$.  Then
  \[
  1/x = - (a_1 + a_2 x + \dotsb + a_n x^{n-1}) \in A.
  \]
\end{proof}
\begin{corollary}
  Let $(A,\mathfrak{p})$ be a local ring
  and $A \subset B$ an integral extension.
  Then
  \[
  \{\text{maximal ideals $\mathfrak{m} \subset B$}\}
  =
  \{\text{primes $\mathfrak{q} \subset B$} : \mathfrak{q} \cap A = \mathfrak{p} \}.
  \]
\end{corollary}
\begin{proof}
  We apply the lemma, taking into account that $\mathfrak{p}$ is the only maximal ideal of $A$.
\end{proof}


\begin{theorem}
  Let $A \subseteq B$ be integral.
  Then the natural map $\Spec(B) \rightarrow \Spec(A)$ is
  surjective,
  and its fibers have no inclusion relations, i.e.:
  \begin{enumerate}
  \item
    We say that a prime $\mathfrak{q}$ of $B$ \emph{lies over}
    a prime $\mathfrak{p}$ of $A$ if $\mathfrak{q} \cap A =
    \mathfrak{p}$.
    Then each prime $\mathfrak{p}$ of $A$ has some prime
    $\mathfrak{q}$ of $B$ lying over it.
  \item If $\mathfrak{q}, \mathfrak{q} '$ are primes
    of $B$ that lie over the same prime of $A$,
    then $\mathfrak{q} \subseteq \mathfrak{q} ' \implies \mathfrak{q} = \mathfrak{q} '$.
  \end{enumerate}
\end{theorem}
\begin{proof}
  The extension $A_\mathfrak{p} \subset B_\mathfrak{p}$ remains
  integral.
  The primes of $B$ lying over $\mathfrak{p}$ correspond
  (in inclusion-preserving manner)
  to the primes of $B_\mathfrak{p}$ lying over $\mathfrak{m} :=
  \mathfrak{p} A_\mathfrak{p}$
  hence (by the previous lemma) to the maximal ideals of
  $B_\mathfrak{p}$.
  We have $B_\mathfrak{p} \neq 0$,
  so maximal ideals exist, and it is clear that no inclusion
  relations
  exist between maximal ideals.
\end{proof}


\section{Going up}
\label{sec:orgf2a76c1}

We say that an extension of rings \(A \subseteq B\) satisfies
\emph{going up} if whenever
\(\mathfrak{p} \subset \mathfrak{p} ' \subset A\) and
\(\mathfrak{q} \subset B\) are primes with \(\mathfrak{q}\) lying
over \(\mathfrak{p}\), there exists a prime
\(\mathfrak{q}' \supseteq \mathfrak{q}\) lying over
\(\mathfrak{p} '\).

Equivalently,
given \(m < n\) and chain of primes
\(\mathfrak{p}_0 \supset \dotsb \subset \mathfrak{p}_n \subset A\)
and
\(\mathfrak{q}_0 \supset \dotsb \subset \mathfrak{q}_m \subset A\)
with \(\mathfrak{q}_i\) lying over \(\mathfrak{p}_i\) for \(i \leq m\),
we can
extend the latter chain
to
\(\mathfrak{q}_0 \supset \dotsb \subset \mathfrak{q}_n \subset A\)
with \(\mathfrak{q}_i\) lying over \(\mathfrak{p}_i\) for \(i \leq
  n\).

\begin{theorem}
  Any integral extension $A \subset B$
  satisfies ``going up''.
\end{theorem}
\begin{proof}
  Apply ``lying over'' to the prime
  $\mathfrak{p} ' / \mathfrak{p}$ of $A/\mathfrak{p}$ inside
  $B/\mathfrak{q}$.
\end{proof}

As an application:
\begin{proposition}
  Let $A \subseteq B$ be an integral ring extension.
  Let $\mathfrak{b} \subset B$ be an ideal,
  and $\mathfrak{a} := \mathfrak{b} \cap A$.
  Then
  \begin{enumerate}[(i)]
  \item $\dim(A) = \dim(B)$.
  \item $\dim(A/\mathfrak{a}) = \dim(B/\mathfrak{b})$.
  \item $\height(\mathfrak{b}) \leq \height(\mathfrak{a})$.
  \end{enumerate}
\end{proposition}
\begin{proof}
  (i):
  A chain of primes in $B$ lies over a chain of primes in $A$.
  The constituents of the latter are distinct
  by the ``non-inclusions'' theorem.
  Thus $\dim(A) \geq \dim(B)$.

  Conversely, a chain of primes in $A$ lifts to a chain in $B$,
  by the ``going up'' theorem.
  Thus $\dim(A) \leq \dim(B)$.

  (ii) is a consequence of (i).

  (iii): Let $\mathfrak{p}$ be a minimal prime of
  $\mathfrak{a}$.  By applying ``lying over'' to
  $A/\mathfrak{a} \subseteq B/\mathfrak{b}$, there is a minimal
  $\mathfrak{q}$ of $\mathfrak{b}$ lying over $\mathfrak{p}$.  A
  chain of primes ending at $\mathfrak{q}$ lies over a chain of
  primes ending at $\mathfrak{p}$.  Thus
  $\height(\mathfrak{b}) \leq \height(\mathfrak{q}) \leq
  \height(\mathfrak{p})$; since $\mathfrak{p}$ was arbitrary, we
  conclude upon taking infima that
  $\height(\mathfrak{b}) \leq \height(\mathfrak{a})$.
\end{proof}


\section{Galois action on primes}
\label{sec:org5aca683}

\begin{lemma}
  Let $A$ be a normal integral domain.  Let $K := \Frac(A)$, and
  let $L/K$ be a normal extension.  Let $B$ denote the integral
  closure in $L$ of $A$.  Then
  \begin{enumerate}
  \item each $\sigma \in G := \Aut(L/K)$ induces a ring
    automorphism of $B$ that fixes $A$;
  \item $G$ acts on the set of primes of $B$ lying above a given
    prime $\mathfrak{p}$ of $A$, i.e.: if
    $\mathfrak{q} \subset B$ is prime, then so is
    $\sigma(\mathfrak{q}) \subset B$, and one has
    $\mathfrak{q} \cap A = \sigma(\mathfrak{q}) \cap A$.
  \item This action is transitive: given primes
    $\mathfrak{q}, \mathfrak{q} '$ of $B$ lying over the same
    prime $\mathfrak{p}$ of $A$, one can find $\sigma \in G$ for
    which $\sigma(\mathfrak{q}) = \mathfrak{q} '$.
  \end{enumerate}
\end{lemma}
We treat first the case of a finite extension.
We reduce in that case, using prime avoidance and the absence of inclusions
among fibers of \(\Spec(B) \rightarrow \Spec(A)\),
to showing that
\[
    \mathfrak{q} ' \subset \cup_{\sigma \in
      G} \sigma(\mathfrak{q}).
  \]
Recall that for \(L/K\)
normal, one has \(L/L^G\) Galois with \(\Gal(L/L^G) = G\) and
\(L^G/K\) purely inseparable: for each \(x \in L^G\) the exists
\(n \geq 1\) (a prime power) with \(x^n \in K\).
Let \(x \in \mathfrak{q} '\).
Since \(L/K\) is finite,
we may set \(y := \prod_{\sigma \in G}
  \sigma(x)\).
Then \(y \in L^G \cap B\), so some \(y^n \in K \cap B\).
Since \(A\) is normal and \(B\) is integral over \(A\), we have \(K
  \cap B = A\),
thus \(y^n \in \mathfrak{q} ' \cap A = \mathfrak{p} \subseteq \mathfrak{q}\).
Thus there exists \(\sigma\) with \(\sigma(x) \in \mathfrak{q}\),
so \(\sigma^{-1}(y) \in \mathfrak{q} '\).

In general,
let 

Then it takes a bit of Zorn's lemma to finish it all off.
TODO: write that down.

Give the example of a Galois extension of number fields.


\section{Going down}
\label{sec:org4bb7b6b}

For ``going down'' theorem, you use a bit of Galois theory.
Hard to beat Eisenbud or Matsumura's treatment; Bosch is also fine.

It says that if \(A \subseteq B\) is an integral extension of
integral domains, with \(A\) normal, then the ``going down''
property holds: for a prime \(\mathfrak{q} \subset B\) lying over
\(\mathfrak{p} \subset A\) and another prime
\(\mathfrak{p} ' \subset \mathfrak{p}\), we can find a prime
\(\mathfrak{q} ' \subset \mathfrak{q}\) lying over
\(\mathfrak{p}'\).

This is somehow equivalent to: Let \(\mathfrak{p} \subset A\) be
prime.  Then any minimal prime of \(\mathfrak{p} B\) lies over
\(\mathfrak{p}\).
\begin{itemize}
\item Suppose $\mathfrak{p}$ satisfies what we called ``going
  down'' above.  Let $\mathfrak{q}'$ be a minimal prime of
  $\mathfrak{p} B$.  Set
  $\mathfrak{p} ' := \mathfrak{q} \cap A$.  Then
  $\mathfrak{p} \subset \mathfrak{p}'$, so by going down, we can
  find a prime $\mathfrak{q} \subset \mathfrak{q} '$ lying over
  $\mathfrak{p}$.  But then
  $\mathfrak{q} \supset \mathfrak{p} B$, so
  $\mathfrak{q} = \mathfrak{q} '$ by minimality.  But this
  contradicts the fact that there are no inclusion relations in
  the fibers of $\Spec(B) \rightarrow \Spec(A)$ for integral
  extensions $A \subset B$.
\item Suppose any minimal prime of $\mathfrak{p} B$ lies over
  $\mathfrak{p}$, for all primes $\mathfrak{p}$ of $A$, and let
  the inclusion $\mathfrak{p} \subset \mathfrak{p}'$ of primes
  be given, together with a prime $\mathfrak{q}'$ over
  $\mathfrak{p}'$.  Then $\mathfrak{q}' \supset \mathfrak{p} B$.
  I think we can just use Zorn's lemma to produce a minimal
  prime
  $\mathfrak{q} ' \supset \mathfrak{q} \supset \mathfrak{p} B$.
\end{itemize}

It'd be nice to give an example showing how this can fail.

The proof is fairly easy, using the Galois trick.


\section{Applications of theorems on integral ring extensions to dimension}
\label{sec:org04b0a0a}

If \(A \subseteq B\) is integral,
\(\mathfrak{b} \subseteq B\) is an ideal,
and \(A := \mathfrak{b} \cap A\),
then
\begin{enumerate}
\item $\dim(A) = \dim(B)$,
\item $\height(\mathfrak{b}) \leq \height(\mathfrak{a})$,
  with equality if $A,B$ are integral domains
  with $A$ normal (or more generally, if ``going down'' holds), and
\item $\coheight(\mathfrak{a}) = \coheight(\mathfrak{b})$.
\end{enumerate}


\section{Noether normalization?}
\label{sec:orgae6adf2}

Let \(k\) be a field.
The following result is very useful:
\begin{theorem}[Noether normalization]
  Let $A$ be a finitely-generated $k$-algebra.
  There exist
  $x_1,\dotsc,x_n \in A$ with the following properties.
  \begin{enumerate}
  \item $x_1,\dotsc,x_n$ are algebraically independent,
    i.e.,
    the map
    \[
      k[X_1,\dotsc,X_n] \rightarrow A
    \]
    \[
      X_j \mapsto x_j
    \]
    is injective.
  \item
    $A$ is integral, hence finite, over the subring
    $k[x_1,\dotsc,x_n]$.
  \end{enumerate}
\end{theorem}
Since integral extensions preserve
dimension,
the conclusion implies in particular
that \(\dim(A) = n\),
and also that if \(A\) is a domain,
then \(\dim(A) = n = \trdeg_k(\Frac(A))\)
(noting that \(\Frac(A)\) is an algebraic extension of \(k(x_1,\dotsc,x_n)\)).

The key lemma for the proof is the following.
\begin{lemma}
  Let $A$ be a $k$-algebra
  and $x_1,\dotsc,x_n \in A$.
  Let $f$ be a nonzero element
  of the polynomial ring
  $k[X_1,\dotsc,X_n]$.
  Set $w := f(x_1,\dotsc,x_n) \in A$.
  Then
  there exist $z_1,\dotsc,z_{n-1} \in k[x_1,\dotsc,x_n]$
  such that
  $k[x_1,\dotsc,x_n]$ is integral over $k[w,z_1,\dotsc,z_{n-1}]$.
\end{lemma}

To deduce Noether normalization from this,
suppose that the finitely-generated \$k\$-algebra \(A\) is
generated by elements \(x_1,\dotsc,x_n\).
If these elements are algebraically independent,
then we are done.
Otherwise we may find a nonzero element \(f\) of the polynomial
ring \(k[X_1,\dotsc,X_n]\)
for which \(0 = f(x_1,\dotsc,x_n)\).
The key lemma (applied with \(w = 0\)) produces elements \(z_1,\dotsc,z_{n-1} \in A\)
for which \(k[x_1,\dotsc,x_n]\) is integral over
\(k[z_1,\dotsc,z_{n-1}]\).
By iterating this procedure finitely many times, we conclude.

We note that ``finite'' and ``integral'' are the same in these contexts,
because everything is finitely-generated.

For the proof of the key lemma,
we warm up by considering rings of the form
\(A := k[X,Y]/(f)\)
for some nonzero \(f \in k[X,Y]\).
We denote by \(x,y \in A\) the images of \(X,Y\).
Thus \(A\) has two generators \(x,y\)
satisfying the relation \(f(x,y) = 0\).
We aim to produce an element \(z \in A\) so that
\(A\) is integral over \(k[z]\).
This is a special case of the key lemma that illustrates
the basic idea.

Let's start with the example \(f = X Y - 1\).
Then we may think
\(A \cong k[x,1/x] \subseteq k(x) = \Frac(A)\).
Observe in this case that \(A\) is \emph{not} integral over \(k[x]\)
or \(k[y]\).
For instance, \(k[x]\) is a UFD,
hence normal
(i.e., integrally closed in \(k(x)\)),
hence \(1/x \in A \notin k[x]\) is not integral over \(k[x]\).

On the other hand,
\(A\) is integral over \(k[z]\)
if \(z = y - a x\) with \(a \neq 0\):
We have \(A = k[x,z]\),
so it suffices
to show that \(x\) is integral over \(k[z]\).
We have \(z x = x y - a x^2 = 1 - a x^2\),
so \(a x^2 + z x - 1 = 0\);
we may
divide by \(a\) to get the required monic equation
for \(x\) over \(k[z]\).

This suggests a fairly general strategy.  Write
\(f = f_n + \dotsb + f_0\), where \(f_j\) denotes the homogeneous
component of degree \(j\) (thus \(f_j\) is a linear combination of
terms \(x^i y^{j-i}\)).
We may assume that \(f_n \neq 0\).

\begin{lemma}
  Suppose $Y$ does not divide $f_n$.  Then $A$ is integral over
  $k[y]$.
\end{lemma}
\begin{proof}
  Our hypothesis implies that $f = a X^n + \dotsb$, where
  $a \neq 0$ and $\dotsb$ consists of lower order terms in $X$,
  with coefficients in $k[Y]$.  This relation taken in the
  quotient ring $A$ shows that $x$ is integral over $y$.
\end{proof}
\begin{corollary}
  Let $a \in k$.  Suppose $Y - a X$ does not divide $f_n$.  Then
  $A$ is integral over $k[y - a x]$.
\end{corollary}
\begin{proof}
  Apply the lemma with modified coordinates
  $X' := X, Y ' := Y - a X$.
\end{proof}
\begin{corollary}
  Suppose the field $k$ is infinite.  Then there exists
  $a \in k$ so that $A$ is integral over $k[y-ax]$.
\end{corollary}
\begin{proof}
  Over an algebraic closure $\overline{k}$, we may factor
  $f_n = \prod_{i=1..n} (a_i x - b_i y)$ for some
  $a_i,b_i \in \overline{k}$ with $(a_i,b_i) \neq (0,0)$.  We
  then choose $a \in k$ not equal to any of the ratios $b_i/a_i$
  and apply the previous corollary.
\end{proof}

This completes the proof of Noether normalization for
\(A = k[X,Y]/(f)\) in the special case that \(k\) is infinite.  What
if \(k\) is finite?
Consider for instance the case that \(k = \mathbb{F}_2\)
and
\(f = Y(Y+ X)- 1\),
so that \(y (y+x) = 1\).
Thus \(A = k[y, y - y^{-1}] \subseteq k(y)\).
The above proof obviously doesn't work, and
in fact \(A\) is not integral over \(k[y]\) or \(k[y-1]\).
So we
need another trick to address this special case.
We might try, instead of \(y - a x\),
something like \(z := y - x^2\).
Then \(y = x^2 + z\).
We expand out:
\[0 =
    f(x,y)
    = f(x,x^2 + z)
    = (x^2 + z)(x^2 + z + x) - 1
    = x^4 + \dotsb,
  \]
where \(\dotsb\) denotes terms
of lower order in \(x\) with coefficients in \(k[z]\).
Thus \(x\) is integral over \(k[z]\);
since \(A = k[x,z]\),
we conclude that \(A\) is integral over \(k[z]\).
There's nothing special  about the number \(2\):
we could have just as well taken \(z := y - x^s\) for any \(s \geq 2\)
and then expanded out \(0 = f(x, x^s + z)\).
The same trick obviously works for any polynomial \(f\):
if \(s\) is large enough, the same argument will give a monic relation satisfied
by \(x\) over \(k[z]\).
\begin{lemma}
  Let $s \in \mathbb{Z}_{\geq 1}$ be sufficiently large in terms
  of $0 \neq f  \in k[X,Y]$,
  and take
  $A = k[X,Y]/(f)\ni x,y$ as above.
  Then $A$ is integral over $k[z]$ with $z := y - x^s$.
\end{lemma}
\begin{proof}
  (Not clear whether there's any point in writing this all out; might already be clear enough.)
  We expand out $0 = f(x,y) = f(x,x^s+z)$ as above.
  Say $f(X,Y) = \sum_{\alpha \in I} c_\alpha X^{\alpha_1}
  Y^{\alpha_2}$,
  where $I \subseteq \mathbb{Z}_{\geq 0}^2$ is finite and
  $c_\alpha \in k^\times$.
  We have
  \[
    x^{\alpha_1} y^{\alpha_2} = x^{\alpha_1} (x^s +
    z)^{\alpha_2}
    = x^{\alpha_1 + s \alpha_2} + \dotsb,
  \]
  where $\dotsb$ denotes terms of lower order in $X$ with
  coefficients in $k[z]$.
  We may choose $s$ so that for all $\alpha, \beta \in I$,
  we have
  \[
    \alpha \neq \beta \implies \alpha_1 + s \alpha_2 \neq \beta_1 + s \beta _2.
  \]
  (Just take $s$ larger than the maximum of $\alpha_j$ over all
  $\alpha \in I$ and $j = 1,2$.)
  Take $\alpha \in I$
  with $\alpha_1 + s \alpha_2$ maximal.
  Then
  \[
    0 = f(x,x^s + z)
    = c_\alpha x^{\alpha_1 + s \alpha_2} + \dotsb,
  \]
  with $\dotsb$ as above.
  Thus $x$ is integral over $k[z]$.
\end{proof}

More generally,
let's suppose now that we have
a ring \(A\) of dimension \(d\)
with \(n\) generators \(x_1,\dotsc,x_n\).
As indicated above,
we may assume that \(n > d\)
and that there is some nontrivial polynomial relation
\(f\) satisfied by the \(x_i\),
i.e.,
\(0 \notin f \in k[X_1,\dotsc,X_n]\)
with \(f(x_1,\dotsc,x_n) = 0\).
We choose integers \(s_j\) and introduce the variables
\(z_j := x_j - x_n^{s_j}\) for \(1 \leq j < n\).
Then
\[
    0 = f(x_1,\dotsc,x_n)
    =
    f(x_n^{s_1} + z_1,
    \dotsc
    x_n^{s_{n-1}} + z_{n-1},
    x_n).
  \]
By choosing the \(s_j\) suitably
as in the case \(n = 2\) (e.g.,
\(s_j := s^j\) for \(s\) large enough),
we may arrange that this proves
that
\(x_n\) is integral
over \(A_0 := k[z_1,\dotsc,z_{n-1}]\).
But then clearly \(A\) is integral over \(A_0\).
We can now apply our induction hypothesis to \(A_0\) to conclude.

Actually, you like Bosch's treatment much better.
He shows that if \(A = k[x_1,\dotsc,x_n]\) is a finitely-generated
\$k\$-algebra
and \(f = \sum c_\alpha x^\alpha\)
with some \(c_\alpha \neq 0\),
then there exist \(z_1,\dotsc,z_{n-1}\)
so that \(k[x_1,\dotsc,x_n]\)
is finite over \(k[f,z_1,\dotsc,z_{n-1}]\).
We can then deduce Noether normalization
by recursively applying this step 
with \(f = 0\)
until we arrive at the case
that the \(x_i\) are algebraically independent.
That's much less clunky than what you had done.


How about the refined version?


After this, you can re-prove Nullstellensatz.
You can then give applications
to heights of primes in \$k\$-domains.


Here we follow Matsumura's treatment (p89)
of Nagata's approach.

\begin{theorem}
  Let $k$ be a field, let $A$ be a finitely-generated
  $k$-algebra with $\dim(A) = n$,
  and let $\mathfrak{a} \subseteq A$ be an ideal
  with $\dim(A/\mathfrak{a}) = n-r$.
  There exist $y_1,\dotsc,y_n \in k[x]$
  so that
  \begin{enumerate}
  \item $A$ is finite over the subring $k[y] := k[y_1,\dotsc,y_n]$, and
  \item $\mathfrak{a} \cap k[y] = (y_1,\dotsc,y_r)$.
  \end{enumerate}
\end{theorem}
\begin{proof}
  We induct on $r$.

  If $r = 0$,
  then $A/\mathfrak{a}$
\end{proof}<++>

For Noether normalization, I think 13.1 in Eisenbud looks like a pretty good treatment.
It also leads readily to the dimension formulas you're after.
11.2.4 in Vakil is also a nice exposition of Nagata's proof.
Maybe try to combine that w/ what Eisenbud does, and also read Matsumura?


\begin{theorem}[Noether normalization]
  Let $A$ be an integral domain that is finitely-generated over
  a subfield $k$.  Set $d = \dim(A)$.  Then there is a finite
  injective map $k[X_1,\dotsc,X_d] \hookrightarrow A$.  (Okay,
  need to know that finite implies integral for the application
  below.)
\end{theorem}
\begin{proof}
  We write $A = k[Y_1,\dotsc,Y_n]/\mathfrak{p}$, and induct on
  $n$.  Then $d \leq n$.  If $d = n$, then $\mathfrak{p} = (0)$,
  so we're done.  We assume now that $n > d$ and induct on $n$.
  We try to find elements
  $Z_1, \dotsc, Z_{n-1} \in k[Y_1,\dotsc,Y_n]$ so that $A$ is
  finite over $k[Z_1,\dotsc,Z_{n-1}] / \mathfrak{q}$, where
  $\mathfrak{q} := k[Z_1,\dotsc,Z_{n-1}] \cap \mathfrak{p}$.
  Then $\dim k[Z]/\mathfrak{q} = d$,
  so by
  our inductive hypothesis,
  $k[Z]/\mathfrak{q}$
  is finite over some $k[X_1,\dotsc,X_d]$.

  To find the required elements,
  we choose $0 \neq f \in \mathfrak{p}$.
  We then choose a large natural number $e$
  and introduce the variables
  $Z_j := Y_j - Y_n^{e^j}$ for $j=1..n-1$,
  so that $Y_j = Z_j + Y_n^{e^j}$
  and thus if $f = \sum c_\alpha Y^\alpha$,
  then
  \[
    f = \sum c_\alpha Y^\alpha
    = \sum c_\alpha (Z_1 + Y_n^{e})^{\alpha_1}
    \dotsb (Z_{n-1} + Y_n^{e^{n-1}})^{\alpha_{n-1}}
    Y_n^{\alpha_n}.
  \]
  Define the weight of $\alpha$ to be
  $\sum_{j=1..n} e^j \alpha_j$.  We choose
  $e > \sup \{\alpha_j : c_\alpha \neq 0, j \in \{1..n\}\}$.
  Then each monomial occurring in the expansion of $f$ has a
  different weight.
  Let $\beta$ be the multi-index of highest weight
  for which $c_\beta \neq 0$,
  and set $\ell := \sum_{j=1..n} e^j \beta_j$.
  Then we may write
  \[
    f = Y_n^{\ell} + a_1(Z_1,\dotsc,Z_{n-1}) Y_n^{\ell-1} + \dotsb
    + a_\ell(Z_1,\dotsc,Z_{n-1})
  \]
  for some $a_j \in k[Z_1,\dotsc,Z_{n-1}]$.
  This shows that $Y_n$ is integral
  over $k[f,Z_1,\dotsc,Z_{n-1}]$.
  Reducing this equation modulo $\mathfrak{p}$,
  we deduce that the image $y_n \in A$
  of $Y_n$ is integral over $k[z_1,\dotsc,z_{n-1}]$.

  Maybe it's cleaner to start by saying that you'll induct on
  the number
  of generators of $A$.
  If $A$ admits $d$ generators,
  then it's a polynomial ring, so you're done.
  Then you can avoid some futzing.
\end{proof}


\section{Dimension vs. transcendence degree}
\label{sec:org12a4f19}



\begin{theorem}
Let $k$ be a field,
and let $A$ be an integral domain
that is a finitely-generated $k$-algebra.
Then $\dim( A) = \trdeg_k (\Frac(A))$.
\end{theorem}
\begin{proof}
  Consider first the case
  that $A = k[X_1,\dotsc,X_n]$ is a polynomial ring
  in $n$ indeterminates.
  We've seen then that $\dim(A) = n$,
  while it's clear that $\trdeg_k (\Frac(A)) = n$,
  so the required identity holds.

  We now reduce the general case to this one.  By Noether
  normalization, there is a finite map
  $k[X_1,\dotsc,X_d] \hookrightarrow A$ for some $d$.  Now Now
  we should have seen by now that integral ring morphisms
  preserve dimensions/heights/coheights (this is an application
  of lying over and going up/down theorems).  Thus
  $\dim(A) = d$.  Moreover, the field extension
  $\Frac(A) / \Frac(k[X_1,\dotsc,X_d])$ is algebraic, so the two
  fields have the same transcendence degree.
\end{proof}

\begin{theorem}
  Let $A$ be as above.
  Then $\height(\mathfrak{p}) + \dim(A/\mathfrak{p}) = \dim(A)$
  for every prime $\mathfrak{p}$.
  In particular,
  $\height(\mathfrak{m}) = \dim(A)$ for every maximal ideal $\mathfrak{m}$.
\end{theorem}
\begin{proof}
  DFD
\end{proof}


\section{Valuation rings}
\label{sec:orgd85e6aa}

Let \(K\) be a field,
and let \((G, +, 0, \leq)\) be a totally ordered abelian group written additively
(e.g., \(G = \mathbb{Z}\));
thus for \(x,y,z \in G\) with \(x \leq y\), we have \(x + z \leq y +
  z\).

A \emph{valuation} of \(K\) with values in \(G\) is a map
\(v : K^\times \rightarrow G\)
such that for all \(x,y \in K^\times\),
we have
\begin{enumerate}
\item $v(x y) = v(x) + v(y)$, and
\item $v(x + y) \geq \min(v(x), v(y))$ (called the \emph{ultrametric inequality}).
\end{enumerate}
In that case, the set \(A := \{x \in K^\times : v(x) \geq 0\} \cup
  \{0\}\) is a subring
of \(K\), called the \emph{valuation ring} of \(v\).
We note that by replacing \(G\) with its image, we may assume that
\(v\) is surjective; this will often be convenient.
\begin{lemma}
  The subset $\mathfrak{m} := \{x \in K^\times : v(x) > 0\} \cup
  \{0\}$
  is an ideal, and $(A,\mathfrak{m})$ is a local ring
  with unit group $A^\times = \{x \in K^\times : v(x) = 0\}$.
\end{lemma}
\begin{proof}~
  \begin{itemize}
  \item $\mathfrak{m}$ is an abelian group:
    if $v(x) > 0$ and $v(y) > 0$, then
    $v(x+y) \geq \min(v(x),v(y)) > 0$.
  \item $\mathfrak{m}$ is closed under multiplication by $A$:
    if $v(x) \geq 0$ and $v(y) > 0$,
    then
    $v(x y) = v(x) + v(y) > 0$.
  \item 
    If $x \in A - \mathfrak{m}$,
    then $v(x) = 0$,
    hence $v(1/x) = -v(x) = 0$,
    hence $1/x \in A$,
    and so $x \in A^\times$.
    Conversely,
    if $x \in A^\times$,
    then $v(x), v(1/x) \geq 0$,
    so $v(x) = 0$ and $x \in A - \mathfrak{m}$.
    Thus $(A,\mathfrak{m})$ is local
    and $A^\times$ is as described.
  \end{itemize}
\end{proof}
It is customary
extend such a valuation
to a map \(v : K \rightarrow G \cup \{\infty \}\),
where \(n \leq \infty\) for all \(n \in G\),
by setting \(v(0) := 0\),
so that these identities read more concisely
as
\[
    A = \{x \in K : v(x) \geq 0\},
    \quad
    \mathfrak{m} = \{x \in K : v(x) > 0\},
    \quad
    A^\times  = \{x \in K : v(x) = 0\}.
  \]

\begin{definition}
  Let $A$ be an integral domain with field of fractions
  $K := \Frac(A)$.  We say that $A$ is a \emph{valuation ring}
  (or VR for short) if it is the valuation ring of some
  valuation $v$ of $K$.

  A valuation $v$ is \emph{discrete} if its value group $G$ is
  isomorphic to $\mathbb{Z}$ and if $v$ is nontrivial; the
  corresponding valuation ring is then called a \emph{discrete
    valuation ring} (or DVR for short).  Identifying $G$ with
  $\mathbb{Z}$, the image of $v$ is then a nontrivial subgroup
  of $\mathbb{Z}$, hence of the form $n \mathbb{Z}$; replacing
  $v$ with $v/n$, we may thus assume without loss of generality
  that $v : K^\times \rightarrow \mathbb{Z}$ is surjective.  If
  $\mathfrak{a}$ is a nonzero ideal in $A$, then there exists a
  smallest integer $n \geq 0$ for which $v(x) = n$ for some
  $x \in \mathfrak{a}$; then $\mathfrak{a}$ contains and hence
  equals $\{x \in K : v(x) \geq n\}$, so all ideals are of the
  latter form.  It follows easily that $A$ is regular Noetherian
  local PID in which every ideal is a power of the maximal
  ideal.
\end{definition}

\begin{example}
  The ring $k[[X]]$ of formal power series in a variable $X$ is
  a DVR: take
  $v(\sum_{i} c_i X^i) := \inf \{i : c_i \neq 0\} \in \mathbb{Z}
  \cup \{\infty\}$.  For example, $v(X + X^2) = 1$,
  $v(X^{-3} + X^{10}) = -3$, $v(1) = 0$, $v(0) = \infty$.

  The ring $\cup_{n \geq 1} k[[X^{1/n}]]$ of formal power series
  with rational exponents is a valuation ring with value group
  $\mathbb{Q}$.  (We will see below that it is \emph{not} a
  DVR.)

  If $K = \mathbb{Q}$
  and
  $p$ is a prime number,
  then we may write any $x \in \mathbb{Q}^\times$
  as $p^n u/v$ with $u,v$ coprime to $p$.
  We then set $v(x) := n$.
  This defines a discrete valuation on $\mathbb{Q}$
  with valuation ring $\mathbb{Z}_{(p)}$.
\end{example}


\begin{lemma}
  Let $A$ be an integral domain, $K := \Frac(A)$.
  The following are equivalent:
  \begin{enumerate}
  \item $A$ is a valuation ring.
  \item For each $x \in K^\times$,
    one has $x \in A$ or $x^{-1} \in A$ (or both).
  \end{enumerate}
\end{lemma}
\begin{proof}
  (i) implies (ii):
  Suppose $v : K \rightarrow G \cup \{\infty\}$ is a valuation
  with valuation ring $A$,
  so that $A = \{x \in K : v(x) \geq 0 \}$,
  and let $x \in K^\times$,
  so that $v(x) \in G$.
  If $v(x) \geq 0$, then $x \in A$.
  If $v(x) \leq 0$, then $v(x^{-1}) \geq 0$,
  and so $x^{-1} \in A$.

  (ii) implies (i):
  Let $G := K^\times / A^\times$;
  it is an abelian group.
  Let $v : K^\times \rightarrow G$
  denote the natural quotient map.
  We denote the group law on $G$ by $+$,
  and write $0 := v(1)$
  for its identity element.
  Clearly $v(x y) = v(x) + v(y)$.

  We equip $G$ with a partial order
  $\leq$ given by $A$-divisibility:
  $x A^\times  \leq y A^\times$ iff $y \in A x$.
  We check that this defines a total order:  if
  $x,y \in K^\times$, then either $x/y \in A$ (in which case
  $y K^\times \leq x K^\times$) or $y/x \in A$ (in which case
  $x K^\times \leq y K^\times$).

  We check the ultrametric inequality,
  that $v(x+y) \geq \min(v(x), v(y))$.
  Suppose for instance that $v(x) \leq v(y)$,
  so that $y \in A x$.
  Then $x + y \in A x$,
  so $v(x) \leq v(x+y)$, as required.

  Thus $v$ is a valuation.
  We have $v(x) \geq 0$ iff
  $x \in A \cdot 1$,
  so $A$ is the valuation ring of $v$.
\end{proof}

\begin{example}
  $A := k[x^2,x^3] \subseteq K = k(x)$ is not a valuation ring:
  the element $x \in K^\times$ satisfies
  $x \notin A, x^{-1} \notin A$.
\end{example}


The ultrametric inequality
can be strengthened to an equality
if \(v(x) \neq v(y)\):
\begin{lemma}
  If $v(x) \neq v(y)$,
  then $v(x+ y) = \min(v(x),v(y))$.
  More generally,
  if $v(x_1) < v(x_2) \leq v(x_3) \leq \dotsb \leq v(x_n)$,
  then $v(x_1 + \dotsb + x_n) = v(x_1)$.
\end{lemma}
\begin{proof}
  Suppose for instance that $v(x) < v(y)$.
  Then $y/x \in \mathfrak{m}$,
  hence $1 + y/x \in A^\times$,
  hence $v(1+y/x) = 0$,
  hence $v(x+y) = v(x) v(1+y/x) = v(x)$.
  The second assertion follows by induction.
\end{proof}

\begin{lemma}
Valuation rings are normal.
\end{lemma}
\begin{proof}
  Let $v : K^\times  \rightarrow G$ be a valuation
  with valuation ring $A$.
  Let $x \in K$ be integral over $A$,
  thus
  \[
    x^n + a_1 x^{n-1} + \dotsb + a_n = 0
  \]
  for some $a_i \in A$.
  Suppose that $v(x) < 0$.
  Then for $i=1..n$,
  we have $v(a_i) \geq 0$,
  hence $v(x^n) < v(x^{i}) \leq v(a_i x^i)$.
  By the strengthened ultrametric inequality,
  it follows that
  \[
    \infty = v(0) = v(x^n + \dotsb + a_n) = v(x^n) < 0,
  \]
  a contradiction.
\end{proof}

\begin{theorem}
  Let $A$ be an integral domain with field of fractions $K$
  and integral closure $\overline{A} \subseteq K$.
  Then $\overline{A}$ is the intersection of all valuation rings
  $B$ of $K$ containing $A$.
\end{theorem}
\begin{proof}
  If $x \in K$ is integral over $A$,
  then it is also integral over each valuation ring $B$ of $K$
  containing $A$;
  by the previous lemma, it follows that $x$ belongs to $B$.
  This gives the ``easy'' inclusion.
  Conversely,
  we must verify that if $x \in K$ is \emph{not} integral over
  $A$ (so that in particular, $x \neq 0$),
  then there \emph{exists} a valuation ring
  $B$ of $K$ containing $A$ for which $x \notin B$:

  Observe first that $x$ is not integral over $A[1/x]$:
  if it were,
  then we could clear denominators
  to get a monic equation for $x$ over $A$, contrary to
  hypothesis.
  In particular,
  $x \notin A[1/x]$.
  Thus $1/x$ does not belong to the unit group of $A[1/x]$,
  and there is a prime (or even maximal) ideal
  $\mathfrak{p}$ of $A[1/x]$
  for which $1/x \in \mathfrak{p}$.

  Observe that if $(B_i)_{i \in I}$
  are subrings of $K$ containing $A[1/x]$
  for which
  \begin{enumerate}
  \item $i \leq j \implies B_i \subseteq B_j$, and
  \item $\mathfrak{p} B_i \neq B_i$,
  \end{enumerate}
  then the (increasing) union $B := \cup_{i \in I} B_i$
  also has the property that $\mathfrak{p} B \neq B$,
  for else
  we may write $1 = p_1 b_1 + \dotsb + p_n b_n$
  with $p_1,\dotsc,p_n \in \mathfrak{p}$
  and $b_1,\dotsc,b_n \in B_i$ for some $i$,
  contrary to hypothesis.

  By Zorn's lemma, there is thus a subring $B$ of $K$ containing
  $A[1/x]$ and maximal with respect to the property that
  $\mathfrak{p} B \neq B$.
  \emph{We claim that $B$ is a valuation ring.}
  Assume the claim for the moment.
  Let $v$ be a defining valuation for $B$,
  and
  let $\mathfrak{m} = \{x \in B : v(x) > 0\} \subseteq B$ denote the maximal ideal of
  $B$.
  Then $1/x \in \mathfrak{p} \subseteq \mathfrak{p} B \subseteq
  \mathfrak{m}$,
  so $v(x) = - v(1/x) < 0$,
  so $x \notin B$.
  We have thus produced the required valuation ring
  of $K$, containing $A$,
  which does not contain $x$.
  It remains only to establish the claim.

  To that end, we verify first that \emph{$B$ is local}.
  Choose any maximal ideal $\mathfrak{m}$ of $B$
  containing $\mathfrak{p} B \neq B$.
  Then $B_\mathfrak{m} \neq 0$,
  so
  $\mathfrak{m} B_\mathfrak{m} \neq B_\mathfrak{m}$.
  In particular,
  $\mathfrak{p} B_\mathfrak{m} \neq B_\mathfrak{m}$.
  By the maximality of $B$,
  it follows that $B = B_\mathfrak{m}$.
  Thus $(B,\mathfrak{m})$ is local.

  We verify next that \emph{$B$ is normal}.
  Suppose otherwise $y \in K$ is integral over $B$,
  but that $y \notin B$.
  Then $B \subsetneq B[y]$ is an integral extension,
  so $B[y]$ contains a prime $\mathfrak{n}$ lying over
  $\mathfrak{m}$.
  (Here we use lying over!)
  In particular, $\mathfrak{p} B[y] \subseteq \mathfrak{n} \subsetneq B[y]$.
  But this contradicts the maximality of $B$.

  We verify next that $B \subseteq A_\mathfrak{p}$.
  We must verify that if $y \in A - \mathfrak{p}$,
  then $y \in B^\times$,
  i.e., that $y \notin \mathfrak{m}$.
  In the contrapositive,
  we must verify that
  $\mathfrak{m} \cap A \subseteq \mathfrak{p}$.
  TODO.  You forget how to finish this off; the important thing is just that
  you need, at the very least, the lying over property, and so this lecture
  will have to come a bit later.

  Okay, but even then, why is $B$ a valuation ring?
  If $y \in K - B$
  then you need to show that $B[y] = B$,
  for instance, by verifying that
  $\mathfrak{p} B[y] \neq B[y]$.
  Why should that be?
\end{proof}
The key step is that if \(\mathfrak{p}\) is a prime in \(A\) and \(B\)
is a subring of \(K\) that is maximal with respect to the property
\(\mathfrak{p} B \neq B\), then \(B\) is local, normal, contains
\(A_\mathfrak{p}\), and satisfies
\(\mathfrak{m}_B \cap A = \mathfrak{p}\).

\begin{enumerate}
\item Local: let $\mathfrak{m} \supseteq \mathfrak{p} B$.
  Then $\mathfrak{p} B_\mathfrak{m} \neq B_\mathfrak{m}$.
\item
  Normal: if $x \in K$ is integral over $B$, then $B[x]$
  contains a prime lying over the maximal ideal of $B$.
\item $B \supseteq A_\mathfrak{p}$ : invert
\item
  $\mathfrak{m}_B \cap A_\mathfrak{p} = \mathfrak{p}
  A_\mathfrak{p}$, etc.
\end{enumerate}
I guess the point here is to give some insight regarding normal rings, and also
to introduce DVR's and to prove that awesome theorem
characterizing them.

TODO: determine what facts from above are really needed
for this discussion.
Do you need the ``lying over'' and ``going down'' theorems,
for instance?
If not, it might be worth introducing
this stuff a bit earlier,
so as to give more examples to play with.


\section{Dedekind domains}
\label{sec:org0694603}

Let \((A,\mathfrak{p})\) be a one-dimensional Noetherian local
domain.  Let \(K\) denote the fraction field of \(A\).

\begin{lemma}
  For each $x \in K$ there exists $n \geq 0$ so that
  $x \mathfrak{p}^n \subseteq A$.
\end{lemma}
\begin{proof}
  It suffices to show for each $0 \neq y \in A$ that one has
  $\mathfrak{p}^n \subseteq (y)$ for some $n$.  This follows
  from the fact that $A/(y)$ is an Artin local ring with maximal
  ideal $\mathfrak{p}/(y)$, which is thus its nilradical.
\end{proof}

\begin{lemma}
  Suppose $y \in K$ satisfies
  $\mathfrak{p} y \subseteq A$.
  Then either
  \begin{enumerate}
  \item $1/y \in \mathfrak{p}$, or
  \item $y$ is integral over $A$.
  \end{enumerate}
\end{lemma}
\begin{proof}
  If $\mathfrak{p} y = A$,
  then we can write $t y = 1$ for some $t \in \mathfrak{p}$.
  Otherwise $\mathfrak{p} y \subseteq \mathfrak{p}$;
  since $\mathfrak{p}$ is a
  finitely-generated $A$-module, it follows that $y$ is integral over $A$.
\end{proof}
In other words, every element of \(\mathfrak{p}^{-1}\) is either (..)

\begin{theorem}
  If $A$ is normal, then $A$ is a DVR.
\end{theorem}
\begin{proof}
  Let $(B,\mathfrak{m})$ be a valuation ring in $K$ that
  contains $A$ and for which $\mathfrak{m}$ lies over
  $\mathfrak{p}$.  Let $v$ be a defining valuation for $B$.
  Then $v(A) \geq 0$ and $v(\mathfrak{p}) > 0$.  We aim to show
  that $A = B$.  (Then we're done, since a Noetherian VR is a
  DVR.)

  Let $x \in B$.  Thus $x \in K$ and $v(x) \geq 0$.  We have
  seen that $\mathfrak{p}^n x \subseteq A$ for some $n$.  (This
  used the one-dimensionality of $A$.)  Choose $n$ minimal with
  this property.  If $n = 0$, then we're done:
  $A x \subseteq A$, so $x \in A$.  Suppose otherwise that
  $n \geq 1$, so that $\mathfrak{p}^{n-1} x \not\subseteq A$.
  Choose $y \in \mathfrak{p}^{n-1} x$ with $y \notin A$.  Then
  $v(y) \geq v(x) \geq 0$ and
  $\mathfrak{p} y \subseteq \mathfrak{p}^n x \subseteq A$.  We
  have $v(1/y) \leq 0$, so $1/y \notin \mathfrak{p}$.  By the
  lemma, we deduce that $y$ is integral over $A$, whence by the
  normality of $A$ that $y \in A$, contradiction.
\end{proof}


\section{The future}
\label{sec:org75bf553}

\subsection{Some exercises}
\label{sec:orge6fe150}

\subsubsection{}
\label{sec:org29f0766}

Let \(A\) be a ring, not necessarily Noetherian.
We are asked to show that
\[
    \dim A + 1 \leq \dim A[X] \leq 2 \dim A + 1.
  \]
We may establish the first inequality
as in the Noetherian case.
For the second,
it suffices to verify that
for any prime chain
\(\mathfrak{p}_1 \subsetneq \mathfrak{p}_2 \subseteq A[X]\)
with \(\mathfrak{p}_1 \cap A = \mathfrak{p}_2 \cap A =:
  \mathfrak{p}\),
one has \(\mathfrak{p}_1 = \mathfrak{p} A[X]\),
I think we just reduce to case in which \((A,\mathfrak{p})\)
is local,
and then use that \(\dim (A/\mathfrak{p})[X] = 1\).


\subsubsection{}
\label{sec:org0e0b82b}

Let \(A = k[X,Y]\)
for a field \(k\).
Let \(f \in A\) be non-constant.
We want to show that \(\dim A/(f) = 1\).
Well, \(A\) is Noetherian,
and \(\dim(A) = 2\),
and \(f\) is non-constant,
hence a non-unit.
It suffices to show for each maximal ideal \(\mathfrak{m}\)
containing \((f)\)
that \(\dim A_\mathfrak{m}/(f) = 1\).
For this we can reduce to the local case?
We need to know that \(f\) is not a zero-divisor.
Everything in sight is an integral domain,
so there's no issue there.


\subsubsection{}
\label{sec:org6c497dd}

Let \((A,\mathfrak{m})\) be a regular Noetherian local ring with residue field \(k = A/\mathfrak{m}\).

Suppose first that \(\dim(A) = 0\).  We are asked then to verify
that \(A\) is a field.  We are given that
\(\dim(A) = 0 = \dim_k(\mathfrak{m}/\mathfrak{m}^2)\).
Thus \(\mathfrak{m} = \mathfrak{m}^2\).
By Nakayama, it follows that \(\mathfrak{m} = (0)\).
Thus \(A\) is a field.

Suppose next that \(\dim(A) = 1\).
We are asked to verify that
\(A\) is a local principal ideal domain.
Welp,
let \(f \in \mathfrak{m} - \mathfrak{m}^2\),
so that the image of \(f\) spans \(\mathfrak{m}/\mathfrak{m}^2\).
By Nakayama, \(f\) generates \(\mathfrak{m}\).
So \(\mathfrak{m} = (f)\) is a principal ideal.
Now consider any ideal \(\mathfrak{a} \subseteq A\).
We must verify that \(\mathfrak{a}\) is principal.
We may suppose that \(\mathfrak{a}\) is nonzero.
We have \(\mathfrak{m} \supseteq \mathfrak{a}\).
Since \(\mathfrak{a}\) is nonzero and \(\cap \mathfrak{m}^n = 0\),
there exists \(n \geq 1\) so that \(\mathfrak{m}^n \supseteq
  \mathfrak{a}\)
but \(\mathfrak{m}^{n+1} \not\supseteq \mathfrak{a}\).
Thus \(\mathfrak{a}\) is contained in \((f^n)\),
but \(\mathfrak{a}\) is not contained in \((f^{n+1})\).
Choose \(x \in \mathfrak{a}\)
with \(x \notin (f^{n+1})\).
We may write \(x = f^n y\).
If \(y \in \mathfrak{m}\),
then \(x \in (f^{n+1})\), contradiction.
Thus \(y\) is a unit.
Thus \((x) = (f^n)\).
Thus \(\mathfrak{a}\) contains \(\mathfrak{m}^n\).
Thus \(\mathfrak{a} = \mathfrak{m}^n\), as required.



\subsubsection{}
\label{sec:org6abe4ac}

Let \((A,\mathfrak{m})\) be a regular Noetherian local ring of
dimension \(d\) with residue field \(k := A/\mathfrak{m}\).
This means,
we recall,
that \(\dim_k(\mathfrak{m}/\mathfrak{m}^2) = \dim(A)\),
or equivalently,
that \(\mathfrak{m}\) is generated by a system of parameters.

Let \(f_1,\dotsc,f_r \in \mathfrak{m}\).
We are asked to verify that the following are equivalent:
\begin{enumerate}
\item The quotient $\overline{A} := A / (f_1,\dotsc,f_r)$
  is regular of dimension $d-r$.
\item The residue classes $\overline{f_1}, \dotsc,
  \overline{f_r}
  \in \mathfrak{m} / \mathfrak{m}^2$
  are linearly independent over $k := A/\mathfrak{m}$.
\end{enumerate}
Well, assume (i), i.e.,
that the quotient is regular of dimension \(d-r\).
Then what?
Well, the dimensional equality
implies (by an earlier result) that
we may extend \(f_1,\dotsc,f_r\) to a system of parameters
\(f_1,\dotsc,f_d\) of \(\mathfrak{m}\).
Now, we are given
that \(\dim_{k}(\overline{\mathfrak{m} } / \overline{\mathfrak{m}
  }^2) = d - r\).

Now, hmm.  It's surely not the case that \emph{any} system of
parameters generates \(\mathfrak{m}/\mathfrak{m}^2\); after all,
we can take the squares of the elements of a given system to
obtain a new system that is contained in \(\mathfrak{m}^2\).

A simple example by which to understand
what's going on might be when
\(A\) is a DVR, \(\mathfrak{m} = (f)\),
and we take \(f_1 := f^2\).
Then (ii) fails, and indeed, \(A/(f^2)\) is not regular.
But why not, exactly?
Okay, well, we can compute pretty easily
the dimension of \(\overline{\mathfrak{m} } /
  \overline{\mathfrak{m} }^2\), perhaps?
I think it should be \(d - \dim \operatorname{span}
  \{\overline{f_1},\dotsc,\overline{f_r}\}\).


\subsubsection{}
\label{sec:orgdd66e74}
\end{document}
