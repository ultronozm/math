\documentclass[reqno]{amsart} \usepackage{graphicx, amsmath, amssymb, amsfonts, amsthm, stmaryrd, amscd}
\usepackage[usenames, dvipsnames]{xcolor}
\usepackage{tikz}
% \usepackage{tikzcd}
% \usepackage{comment}

% \let\counterwithout\relax
% \let\counterwithin\relax
% \usepackage{chngcntr}

\usepackage{enumerate}
% \usepackage{enumitem}
% \usepackage{times}
\usepackage[normalem]{ulem}
% \usepackage{minted}
% \usepackage{xypic}
% \usepackage{color}


% \usepackage{silence}
% \WarningFilter{latex}{Label `tocindent-1' multiply defined}
% \WarningFilter{latex}{Label `tocindent0' multiply defined}
% \WarningFilter{latex}{Label `tocindent1' multiply defined}
% \WarningFilter{latex}{Label `tocindent2' multiply defined}
% \WarningFilter{latex}{Label `tocindent3' multiply defined}
\usepackage{hyperref}
% \usepackage{navigator}


% \usepackage{pdfsync}
\usepackage{xparse}


\usepackage[all]{xy}
\usepackage{enumerate}
\usetikzlibrary{matrix,arrows,decorations.pathmorphing}



\makeatletter
\newcommand*{\transpose}{%
  {\mathpalette\@transpose{}}%
}
\newcommand*{\@transpose}[2]{%
  % #1: math style
  % #2: unused
  \raisebox{\depth}{$\m@th#1\intercal$}%
}
\makeatother


\makeatletter
\newcommand*{\da@rightarrow}{\mathchar"0\hexnumber@\symAMSa 4B }
\newcommand*{\da@leftarrow}{\mathchar"0\hexnumber@\symAMSa 4C }
\newcommand*{\xdashrightarrow}[2][]{%
  \mathrel{%
    \mathpalette{\da@xarrow{#1}{#2}{}\da@rightarrow{\,}{}}{}%
  }%
}
\newcommand{\xdashleftarrow}[2][]{%
  \mathrel{%
    \mathpalette{\da@xarrow{#1}{#2}\da@leftarrow{}{}{\,}}{}%
  }%
}
\newcommand*{\da@xarrow}[7]{%
  % #1: below
  % #2: above
  % #3: arrow left
  % #4: arrow right
  % #5: space left 
  % #6: space right
  % #7: math style 
  \sbox0{$\ifx#7\scriptstyle\scriptscriptstyle\else\scriptstyle\fi#5#1#6\m@th$}%
  \sbox2{$\ifx#7\scriptstyle\scriptscriptstyle\else\scriptstyle\fi#5#2#6\m@th$}%
  \sbox4{$#7\dabar@\m@th$}%
  \dimen@=\wd0 %
  \ifdim\wd2 >\dimen@
    \dimen@=\wd2 %   
  \fi
  \count@=2 %
  \def\da@bars{\dabar@\dabar@}%
  \@whiledim\count@\wd4<\dimen@\do{%
    \advance\count@\@ne
    \expandafter\def\expandafter\da@bars\expandafter{%
      \da@bars
      \dabar@ 
    }%
  }%  
  \mathrel{#3}%
  \mathrel{%   
    \mathop{\da@bars}\limits
    \ifx\\#1\\%
    \else
      _{\copy0}%
    \fi
    \ifx\\#2\\%
    \else
      ^{\copy2}%
    \fi
  }%   
  \mathrel{#4}%
}
\makeatother
% \DeclareMathOperator{\rg}{rg}

\usepackage{mathtools}
\DeclarePairedDelimiter{\paren}{(}{)}
\DeclarePairedDelimiter{\abs}{\lvert}{\rvert}
\DeclarePairedDelimiter{\norm}{\lVert}{\rVert}
\DeclarePairedDelimiter{\innerproduct}{\langle}{\rangle}
\newcommand{\Of}[2]{{\operatorname{#1}} {\paren*{#2}}}
\newcommand{\of}[2]{{{{#1}} {\paren*{#2}}}}

\DeclareMathOperator{\Shim}{Shim}
\DeclareMathOperator{\sgn}{sgn}
\DeclareMathOperator{\fdeg}{fdeg}
\DeclareMathOperator{\SL}{SL}
\DeclareMathOperator{\slLie}{\mathfrak{s}\mathfrak{l}}
\DeclareMathOperator{\soLie}{\mathfrak{s}\mathfrak{o}}
\DeclareMathOperator{\spLie}{\mathfrak{s}\mathfrak{p}}
\DeclareMathOperator{\glLie}{\mathfrak{g}\mathfrak{l}}
\newcommand{\pn}[1]{{\color{ForestGreen} \sf PN: [#1]}}
\DeclareMathOperator{\Mp}{Mp}
\DeclareMathOperator{\Mat}{Mat}
\DeclareMathOperator{\GL}{GL}
\DeclareMathOperator{\Gr}{Gr}
\DeclareMathOperator{\GU}{GU}
\def\gl{\mathfrak{g}\mathfrak{l}}
\DeclareMathOperator{\odd}{odd}
\DeclareMathOperator{\even}{even}
\DeclareMathOperator{\GO}{GO}
\DeclareMathOperator{\good}{good}
\DeclareMathOperator{\bad}{bad}
\DeclareMathOperator{\PGO}{PGO}
\DeclareMathOperator{\htt}{ht}
\DeclareMathOperator{\height}{height}
\DeclareMathOperator{\Ass}{Ass}
\DeclareMathOperator{\coheight}{coheight}
\DeclareMathOperator{\GSO}{GSO}
\DeclareMathOperator{\SO}{SO}
\DeclareMathOperator{\so}{\mathfrak{s}\mathfrak{o}}
\DeclareMathOperator{\su}{\mathfrak{s}\mathfrak{u}}
\DeclareMathOperator{\ad}{ad}
% \DeclareMathOperator{\sc}{sc}
\DeclareMathOperator{\Ad}{Ad}
\DeclareMathOperator{\disc}{disc}
\DeclareMathOperator{\inv}{inv}
\DeclareMathOperator{\Pic}{Pic}
\DeclareMathOperator{\uc}{uc}
\DeclareMathOperator{\Cl}{Cl}
\DeclareMathOperator{\Clf}{Clf}
\DeclareMathOperator{\Hom}{Hom}
\DeclareMathOperator{\hol}{hol}
\DeclareMathOperator{\Heis}{Heis}
\DeclareMathOperator{\Haar}{Haar}
\DeclareMathOperator{\h}{h}
\def\sp{\mathfrak{s}\mathfrak{p}}
\DeclareMathOperator{\heis}{\mathfrak{h}\mathfrak{e}\mathfrak{i}\mathfrak{s}}
\DeclareMathOperator{\End}{End}
\DeclareMathOperator{\JL}{JL}
\DeclareMathOperator{\image}{image}
\DeclareMathOperator{\red}{red}
\def\div{\operatorname{div}}
\def\eps{\varepsilon}
\def\cHom{\mathcal{H}\operatorname{om}}
\DeclareMathOperator{\Ops}{Ops}
\DeclareMathOperator{\Symb}{Symb}
\def\boldGL{\mathbf{G}\mathbf{L}}
\def\boldSO{\mathbf{S}\mathbf{O}}
\def\boldU{\mathbf{U}}
\DeclareMathOperator{\hull}{hull}
\DeclareMathOperator{\LL}{LL}
\DeclareMathOperator{\PGL}{PGL}
\DeclareMathOperator{\class}{class}
\DeclareMathOperator{\lcm}{lcm}
\DeclareMathOperator{\spann}{span}
\DeclareMathOperator{\Exp}{Exp}
\DeclareMathOperator{\ext}{ext}
\DeclareMathOperator{\Ext}{Ext}
\DeclareMathOperator{\Tor}{Tor}
\DeclareMathOperator{\et}{et}
\DeclareMathOperator{\tor}{tor}
\DeclareMathOperator{\loc}{loc}
\DeclareMathOperator{\tors}{tors}
\DeclareMathOperator{\pf}{pf}
\DeclareMathOperator{\smooth}{smooth}
\DeclareMathOperator{\prin}{prin}
\DeclareMathOperator{\Kl}{Kl}
\newcommand{\kbar}{\mathchar'26\mkern-9mu k}
\DeclareMathOperator{\der}{der}
% \DeclareMathOperator{\abs}{abs}
\DeclareMathOperator{\Sub}{Sub}
\DeclareMathOperator{\Comp}{Comp}
\DeclareMathOperator{\Err}{Err}
\DeclareMathOperator{\dom}{dom}
\DeclareMathOperator{\radius}{radius}
\DeclareMathOperator{\Fitt}{Fitt}
\DeclareMathOperator{\Sel}{Sel}
\DeclareMathOperator{\rad}{rad}
\DeclareMathOperator{\id}{id}
\DeclareMathOperator{\Center}{Center}
\DeclareMathOperator{\Der}{Der}
\DeclareMathOperator{\U}{U}
% \DeclareMathOperator{\norm}{norm}
\DeclareMathOperator{\trace}{trace}
\DeclareMathOperator{\Equid}{Equid}
\DeclareMathOperator{\Feas}{Feas}
\DeclareMathOperator{\bulk}{bulk}
\DeclareMathOperator{\tail}{tail}
\DeclareMathOperator{\sys}{sys}
\DeclareMathOperator{\atan}{atan}
\DeclareMathOperator{\temp}{temp}
\DeclareMathOperator{\Asai}{Asai}
\DeclareMathOperator{\glob}{glob}
\DeclareMathOperator{\Kuz}{Kuz}
\DeclareMathOperator{\Irr}{Irr}
\newcommand{\rsL}{ \frac{ L^{(R)}(\Pi \times \Sigma, \std, \frac{1}{2})}{L^{(R)}(\Pi \times \Sigma, \Ad, 1)}  }
\DeclareMathOperator{\GSp}{GSp}
\DeclareMathOperator{\PGSp}{PGSp}
\DeclareMathOperator{\BC}{BC}
\DeclareMathOperator{\Ann}{Ann}
\DeclareMathOperator{\Gen}{Gen}
\DeclareMathOperator{\SU}{SU}
\DeclareMathOperator{\PGSU}{PGSU}
% \DeclareMathOperator{\gen}{gen}
\DeclareMathOperator{\PMp}{PMp}
\DeclareMathOperator{\PGMp}{PGMp}
\DeclareMathOperator{\PB}{PB}
\DeclareMathOperator{\ind}{ind}
\DeclareMathOperator{\Jac}{Jac}
\DeclareMathOperator{\jac}{jac}
\DeclareMathOperator{\im}{im}
\DeclareMathOperator{\Aut}{Aut}
\DeclareMathOperator{\Int}{Int}
\DeclareMathOperator{\PSL}{PSL}
\DeclareMathOperator{\co}{co}
\DeclareMathOperator{\irr}{irr}
\DeclareMathOperator{\prim}{prim}
\DeclareMathOperator{\bal}{bal}
\DeclareMathOperator{\baln}{bal}
\DeclareMathOperator{\dist}{dist}
\DeclareMathOperator{\RS}{RS}
\DeclareMathOperator{\Ram}{Ram}
\DeclareMathOperator{\Sob}{Sob}
\DeclareMathOperator{\Sol}{Sol}
\DeclareMathOperator{\soc}{soc}
\DeclareMathOperator{\nt}{nt}
\DeclareMathOperator{\mic}{mic}
\DeclareMathOperator{\Gal}{Gal}
\DeclareMathOperator{\st}{st}
\DeclareMathOperator{\std}{std}
\DeclareMathOperator{\diag}{diag}
\DeclareMathOperator{\Sym}{Sym}
\DeclareMathOperator{\gr}{gr}
\DeclareMathOperator{\aff}{aff}
\DeclareMathOperator{\Dil}{Dil}
\DeclareMathOperator{\Lie}{Lie}
\DeclareMathOperator{\Symp}{Symp}
\DeclareMathOperator{\Stab}{Stab}
\DeclareMathOperator{\St}{St}
\DeclareMathOperator{\stab}{stab}
\DeclareMathOperator{\codim}{codim}
\DeclareMathOperator{\linear}{linear}
\newcommand{\git}{/\!\!/}
\DeclareMathOperator{\geom}{geom}
\DeclareMathOperator{\spec}{spec}
\def\O{\operatorname{O}}
\DeclareMathOperator{\Au}{Aut}
\DeclareMathOperator{\Fix}{Fix}
\DeclareMathOperator{\Opp}{Op}
\DeclareMathOperator{\opp}{op}
\DeclareMathOperator{\Size}{Size}
\DeclareMathOperator{\Save}{Save}
% \DeclareMathOperator{\ker}{ker}
\DeclareMathOperator{\coker}{coker}
\DeclareMathOperator{\sym}{sym}
\DeclareMathOperator{\mean}{mean}
\DeclareMathOperator{\elliptic}{ell}
\DeclareMathOperator{\nilpotent}{nil}
\DeclareMathOperator{\hyperbolic}{hyp}
\DeclareMathOperator{\newvector}{new}
\DeclareMathOperator{\new}{new}
\DeclareMathOperator{\full}{full}
\newcommand{\qr}[2]{\left( \frac{#1}{#2} \right)}
\DeclareMathOperator{\unr}{u}
\DeclareMathOperator{\ram}{ram}
% \DeclareMathOperator{\len}{len}
\DeclareMathOperator{\fin}{fin}
\DeclareMathOperator{\cusp}{cusp}
\DeclareMathOperator{\curv}{curv}
\DeclareMathOperator{\rank}{rank}
\DeclareMathOperator{\rk}{rk}
\DeclareMathOperator{\pr}{pr}
\DeclareMathOperator{\Transform}{Transform}
\DeclareMathOperator{\mult}{mult}
\DeclareMathOperator{\Eis}{Eis}
\DeclareMathOperator{\reg}{reg}
\DeclareMathOperator{\sing}{sing}
\DeclareMathOperator{\alt}{alt}
\DeclareMathOperator{\irreg}{irreg}
\DeclareMathOperator{\sreg}{sreg}
\DeclareMathOperator{\Wd}{Wd}
\DeclareMathOperator{\Weil}{Weil}
\DeclareMathOperator{\Th}{Th}
\DeclareMathOperator{\Sp}{Sp}
\DeclareMathOperator{\Ind}{Ind}
\DeclareMathOperator{\Res}{Res}
\DeclareMathOperator{\ini}{in}
\DeclareMathOperator{\ord}{ord}
\DeclareMathOperator{\osc}{osc}
\DeclareMathOperator{\fluc}{fluc}
\DeclareMathOperator{\size}{size}
\DeclareMathOperator{\ann}{ann}
\DeclareMathOperator{\equ}{eq}
\DeclareMathOperator{\res}{res}
\DeclareMathOperator{\pt}{pt}
\DeclareMathOperator{\src}{source}
\DeclareMathOperator{\Zcl}{Zcl}
\DeclareMathOperator{\Func}{Func}
\DeclareMathOperator{\Map}{Map}
\DeclareMathOperator{\Frac}{Frac}
\DeclareMathOperator{\Frob}{Frob}
\DeclareMathOperator{\ev}{eval}
\DeclareMathOperator{\pv}{pv}
\DeclareMathOperator{\eval}{eval}
\DeclareMathOperator{\Spec}{Spec}
\DeclareMathOperator{\Speh}{Speh}
\DeclareMathOperator{\Spin}{Spin}
\DeclareMathOperator{\GSpin}{GSpin}
\DeclareMathOperator{\Specm}{Specm}
\DeclareMathOperator{\Sphere}{Sphere}
\DeclareMathOperator{\Sqq}{Sq}
\DeclareMathOperator{\Ball}{Ball}
\DeclareMathOperator\Cond{\operatorname{Cond}}
\DeclareMathOperator\proj{\operatorname{proj}}
\DeclareMathOperator\Swan{\operatorname{Swan}}
\DeclareMathOperator{\Proj}{Proj}
\DeclareMathOperator{\bPB}{{\mathbf P}{\mathbf B}}
\DeclareMathOperator{\Projm}{Projm}
\DeclareMathOperator{\Tr}{Tr}
\DeclareMathOperator{\Type}{Type}
\DeclareMathOperator{\Prop}{Prop}
\DeclareMathOperator{\vol}{vol}
\DeclareMathOperator{\covol}{covol}
\DeclareMathOperator{\Rep}{Rep}
\DeclareMathOperator{\Cent}{Cent}
\DeclareMathOperator{\val}{val}
\DeclareMathOperator{\area}{area}
\DeclareMathOperator{\nr}{nr}
\DeclareMathOperator{\CM}{CM}
\DeclareMathOperator{\CH}{CH}
\DeclareMathOperator{\tr}{tr}
\DeclareMathOperator{\characteristic}{char}
\DeclareMathOperator{\supp}{supp}


\theoremstyle{plain} \newtheorem{theorem} {Theorem} \newtheorem{conjecture} [theorem] {Conjecture} \newtheorem{corollary} [theorem] {Corollary} \newtheorem{proposition} [theorem] {Proposition} \newtheorem{fact} [theorem] {Fact}
\theoremstyle{definition} \newtheorem{definition} [theorem] {Definition} \newtheorem{hypothesis} [theorem] {Hypothesis} \newtheorem{assumptions} [theorem] {Assumptions}
\newtheorem{example} [theorem] {Example}
\newtheorem{assertion}[theorem] {Assertion}
\newtheorem{note}[theorem] {Note}
\newtheorem{conclusion}[theorem] {Conclusion}
\newtheorem{claim}            {Claim}
\newtheorem{homework} {Homework}
\newtheorem{exercise} {Exercise}  \newtheorem{question}[theorem] {Question}    \newtheorem{answer} {Answer}  \newtheorem{problem} {Problem}    \newtheorem{remark} [theorem] {Remark}
\newtheorem{notation} [theorem]           {Notation}
\newtheorem{terminology}[theorem]            {Terminology}
\newtheorem{convention}[theorem]            {Convention}
\newtheorem{motivation}[theorem]            {Motivation}


\newtheoremstyle{itplain} % name
{6pt}                    % Space above
{5pt\topsep}                    % Space below
{\itshape}                   % Body font
{}                           % Indent amount
{\itshape}                   % Theorem head font
{.}                          % Punctuation after theorem head
{5pt plus 1pt minus 1pt}                       % Space after theorem head
% {.5em}                       % Space after theorem head
{}  % Theorem head spec (can be left empty, meaning ‘normal’)

% \theoremstyle{mytheoremstyle}


\theoremstyle{itplain} %--default
% \theoremheaderfont{\itshape}
% \newtheorem{lemma}{Lemma}
\newtheorem{lemma}[theorem]{Lemma}
% \newtheorem{lemma}{Lemma}[subsubsection]

\newtheorem*{lemma*}{Lemma}
\newtheorem*{proposition*}{Proposition}
\newtheorem*{definition*}{Definition}
\newtheorem*{example*}{Example}

\newtheorem*{results*}{Results}
\newtheorem{results} [theorem] {Results}


\usepackage[displaymath,textmath,sections,graphics]{preview}
\PreviewEnvironment{align*}
\PreviewEnvironment{multline*}
\PreviewEnvironment{tabular}
\PreviewEnvironment{verbatim}
\PreviewEnvironment{lstlisting}
\PreviewEnvironment*{frame}
\PreviewEnvironment*{alert}
\PreviewEnvironment*{emph}
\PreviewEnvironment*{textbf}



\begin{document}

In this note, we define some character sums relevant for subconvexity on $\GL_{n+1} \times \GL_n$ at ``prime depth.''

Let $F$ be a finite field of order $q$.  Set
\begin{equation*}
  (G,H) := (\GL_{n+1}(F), \GL_n(F)),
\end{equation*}
with $H$ embedded as the upper-left block.  Let $(B, B_H)$ denote the upper-triangular Borel subgroups.

Quasi-invariants for the two-sided action of $B \times B_H$ on $G$ (by left and right translation) are given by the following determinental minors:
\begin{itemize}
\item For $m = 1, \dotsc, n+1$, the ``left-and-bottom-anchored'' minors
  \begin{equation*}
    A_j(g) := \det
    (g_{i j}    )_{
      n+2-m \leq i \leq n+1
    }^{1 \leq j \leq i}.
  \end{equation*}
\item For $m = 1, \dotsc, n$, the ``left-and-second-to-bottom-anchored'' minors
  \begin{equation*}
    B_j(g) := \det
    (g_{i j}    )_{
      n+1-m \leq i \leq n
    }^{1 \leq j \leq i}.
  \end{equation*}
\end{itemize}
We also adopt the convention $A_0(g) := B_0(g) := 1$.

The action of $B \times B_H$ on $G$ has an open orbit, given by the nonvanishing of each of the invariants $A_1,\dotsc,A_{n+1},B_1,\dotsc,B_n$.  Let $\alpha$ be a representative for this orbit.
\begin{example}\label{example:cj59m8hvjh}
We could take, e.g., for $n+1 = 4$,
\begin{equation}\label{eq:cj59nalcll}
\alpha =
\begin{pmatrix}
0 & 0 & 1 & 1 \\
0 & 1 & 1 & 0 \\
1 & 1 & 0 & 0 \\
1 & 0 & 0 & 0 \\
\end{pmatrix}.
\end{equation}
Indeed, we then have $A_i(\alpha) = B_j(\alpha) = 1$ for all $i$ and $j$.
\end{example}
\begin{example}
  One could instead take
  \begin{equation*}
    \alpha =
    \begin{pmatrix}
      1 & 0 & 0 & 1 \\
      1 & 0 & 1 & 0 \\
      1 & 1 & 0 & 0 \\
      1 & 0 & 0 & 0 \\
    \end{pmatrix}.
  \end{equation*}
\end{example}

Let $\chi : B \rightarrow \U(1)$ and $\eta : B_H \rightarrow \U(1)$ be characters.  We define the character sum
\begin{equation*}
  S(\gamma) := \sum_{
    \substack{
      x,y \in B_H : 
      \\
       \alpha^{-1} y^{-1} \gamma x \alpha  \in B
    }
  }
  \chi (\alpha^{-1} y^{-1} \gamma x \alpha ) \eta (x^{-1} y).
\end{equation*}
We would like to understand the magnitude of this sum, together with averaged variants such as
\begin{equation*}
\max_{x \in B_H} \sum_{y \in B_H} \lvert S(x \gamma y) \rvert.
\end{equation*}
\begin{remark}
  The sum $S(\gamma)$ is $\eta$-equivariant under the action of $B_H \times B_H$, so in studying that sum, we can assume that $\gamma$ lies in a set of representatives for that action.  Generic representatives are given by, e.g., when $n+1=4$,
  \begin{equation}\label{eq:cj59nab7hw}
    \gamma = \begin{pmatrix}
               0 & 0 & 1 & \gamma_1  \\
               0 & 1 & 1 & \gamma_2  \\
               1 & 1 & 0 &  \gamma_3  \\
               1 & 0 & 0 & \gamma_4 \\
             \end{pmatrix}.
  \end{equation}
\end{remark}
Let $\Theta : G \rightarrow \mathbb{C}$ denote the function supported on $B_H \alpha B$ and given there by
\begin{equation*}
  \Theta(y \alpha z) = \eta(y) \chi(z).
\end{equation*}
Then
\begin{equation*}
  S (\gamma) = \sum_{x \in B_H}
  \eta^{-1}(x)
  \Theta (\gamma x \alpha).
\end{equation*}


\begin{lemma}
  Define the components $\chi_j$ and $\eta_j$ of $\chi$ and $\eta$ by writing
  \begin{equation}\label{eq:cj4y8xe8vt}
    y = \diag(y_n,\dotsc,y_1,1), \qquad z = \diag(z_1,\dotsc,z_{n+1}),
  \end{equation}
  \begin{equation*}
    \eta(y) = \prod _j \eta_j (y_j),
    \qquad
    \chi (z) = \prod_j \chi_j (z_j ).
  \end{equation*}
  Let $\alpha$ be as in Example \ref{example:cj59m8hvjh}.  Then on $B_H \alpha B$, we have
  \begin{equation*}
    \Theta
    =
    \chi_1 \left( \frac{A_1}{B_0} \right)
  \dotsb 
  \chi_{n+1} \left( \frac{A_{n+1}}{B_n} \right)
  \eta_1 \left( x^{-1} \frac{B_1}{A_1} \right)
  \dotsb 
  \eta_{n} \left( x^{-1} \frac{B_{n}}{A_n} \right).
  \end{equation*}
\end{lemma}
\begin{example}
  Suppose that the $\eta_j$ are trivial and the $\chi_j$ are all equal.  Call their common value $\chi$.  Then
  \begin{equation*}
    \Theta = \chi \left( \frac{A_1 \dotsb A_{n + 1}}{ B_1 \dotsb B_n } \right).
  \end{equation*}
\end{example}
\begin{example}
For $n+1=2$ and $\gamma =
\begin{pmatrix}
a & b \\
c & d \\
\end{pmatrix}$ and $x =
\begin{pmatrix}
x & 0 \\
0 & 1 \\
\end{pmatrix}$ and $\alpha =
\begin{pmatrix}
1 & 1 \\
1 & 0 \\
\end{pmatrix}$, the invariants of
\begin{equation*}
  \gamma x \alpha =
\begin{pmatrix}a x + b & a x\\c x + d & c x\end{pmatrix}
\end{equation*}
are given by
\begin{equation*}
A_1 = c x + d, \quad B_1 = a x + b, \quad A_2 = (a d - b c)x,
\end{equation*}
so the character sum in question is a unit scalar multiple of
\begin{equation*}
  \sum_{x \in F^\times }^*
  \chi_1 (c x + d ) \chi_2 \left( \frac{x}{a x + b} \right) \eta \left( x^{-1} \frac{a x + b}{c x + d} \right).
\end{equation*}
By Weil, this exhibits square-root cancellation except when
\begin{itemize}
\item $\gamma$ is diagonal,
\item $\gamma$ is lower-triangular and $\chi_1 = \eta$, or
\item $\gamma$ is upper-triangular and $\chi_2 = \eta$.
\end{itemize}
In those cases, we don't get any cancellation.  For example, suppose that $\chi_2 = \eta = 1$.  Then the character sum is
\begin{equation*}
\sum_{x \in F^\times } \chi_1 (c x + d).
\end{equation*}
  If $\gamma$ is upper-triangular, i.e., $c = 0$, then we get no cancellation.

Thus we see that ``conductor-dropping'' manifests in an expanded degeneracy locus for the character sum.
\end{example}

\begin{example}\label{example:cj59ndds8r}
  One can check that for $n+1=3$, $\eta$ trivial, and $\gamma$ ``generic'' as in \eqref{eq:cj59nab7hw}, we have, with the notation
  \begin{equation*}
x =
\begin{pmatrix}
x_1 & x_3 & 0 \\
0 & x_2 & 0 \\
0 & 0 & 1 \\
\end{pmatrix},
\end{equation*}
  \begin{multline*}
    \Theta(\gamma x \alpha) =
    \chi_1 \left(
      \gamma_{3} + x_{3}
    \right)
    \\ \cdot 
    \chi_2 \left(
      \frac{\gamma_{2} x_{1} + \gamma_{2} x_{3} - \gamma_{3} x_{1} - \gamma_{3} x_{2} - \gamma_{3} x_{3} + x_{1} x_{2}}{\gamma_{2} + x_{2} + x_{3}}
    \right)
    \\ \cdot 
    \chi_3
    \left(
      \frac{x_{1} x_{2} (\gamma_{1} - \gamma_{2} + \gamma_{3})}{\gamma_{1} x_{1} + \gamma_{1} x_{2} + \gamma_{1} x_{3} - \gamma_{2} x_{2} + x_{1} x_{2}}
    \right).
  \end{multline*}
  Here's a more explicit example, obtained by specializing the $\gamma_j$ at random:
  \begin{equation*}
\chi_1 \left(
    x_{3} - 5
  \right)
  \chi_2 \left(
    \frac{x_{1} x_{2} + 7 x_{1} + 5 x_{2} + 7 x_{3}}{x_{2} + x_{3} + 2}\right)
  \chi_3 \left(
  - \frac{11 x_{1} x_{2}}{x_{1} x_{2} - 4 x_{1} - 6 x_{2} - 4 x_{3}}\right).
\end{equation*}
\end{example}

\begin{question}
  In the special case $\chi_1 = \chi_2 = \chi_3$, is the character sum obtained in \ref{example:cj59ndds8r} equivalent to that considered by Sharma, namely
  \begin{equation*}
    \sum_{t_1, t_2, t_3 }
    \chi \left( \frac{t_{1} t_{3} (t_{2} + 1) (c_{2} k_{1} t_{3} + c_{2} k_{2} + t_{2})}{t_{2} (t_{1} + 1) (c_{1} t_{3} + t_{1}) (k_{1} t_{3} + k_{2})} \right)?
  \end{equation*}
  Maybe there's some clever change of variables relating $(t_1, t_2 , t_3 )$ to $(x_1 , x_2 , x_3 )$.
\end{question}




\bibliography{refs}{} \bibliographystyle{plain}
\end{document}
