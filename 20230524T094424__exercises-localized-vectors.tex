\documentclass[reqno]{amsart} \usepackage{graphicx, amsmath, amssymb, amsfonts, amsthm, stmaryrd, amscd}
\usepackage[usenames, dvipsnames]{xcolor}
\usepackage{tikz}
% \usepackage{tikzcd}
% \usepackage{comment}

% \let\counterwithout\relax
% \let\counterwithin\relax
% \usepackage{chngcntr}

\usepackage{enumerate}
% \usepackage{enumitem}
% \usepackage{times}
\usepackage[normalem]{ulem}
% \usepackage{minted}
% \usepackage{xypic}
% \usepackage{color}


% \usepackage{silence}
% \WarningFilter{latex}{Label `tocindent-1' multiply defined}
% \WarningFilter{latex}{Label `tocindent0' multiply defined}
% \WarningFilter{latex}{Label `tocindent1' multiply defined}
% \WarningFilter{latex}{Label `tocindent2' multiply defined}
% \WarningFilter{latex}{Label `tocindent3' multiply defined}
\usepackage{hyperref}
% \usepackage{navigator}


% \usepackage{pdfsync}
\usepackage{xparse}


\usepackage[all]{xy}
\usepackage{enumerate}
\usetikzlibrary{matrix,arrows,decorations.pathmorphing}



\makeatletter
\newcommand*{\transpose}{%
  {\mathpalette\@transpose{}}%
}
\newcommand*{\@transpose}[2]{%
  % #1: math style
  % #2: unused
  \raisebox{\depth}{$\m@th#1\intercal$}%
}
\makeatother


\makeatletter
\newcommand*{\da@rightarrow}{\mathchar"0\hexnumber@\symAMSa 4B }
\newcommand*{\da@leftarrow}{\mathchar"0\hexnumber@\symAMSa 4C }
\newcommand*{\xdashrightarrow}[2][]{%
  \mathrel{%
    \mathpalette{\da@xarrow{#1}{#2}{}\da@rightarrow{\,}{}}{}%
  }%
}
\newcommand{\xdashleftarrow}[2][]{%
  \mathrel{%
    \mathpalette{\da@xarrow{#1}{#2}\da@leftarrow{}{}{\,}}{}%
  }%
}
\newcommand*{\da@xarrow}[7]{%
  % #1: below
  % #2: above
  % #3: arrow left
  % #4: arrow right
  % #5: space left 
  % #6: space right
  % #7: math style 
  \sbox0{$\ifx#7\scriptstyle\scriptscriptstyle\else\scriptstyle\fi#5#1#6\m@th$}%
  \sbox2{$\ifx#7\scriptstyle\scriptscriptstyle\else\scriptstyle\fi#5#2#6\m@th$}%
  \sbox4{$#7\dabar@\m@th$}%
  \dimen@=\wd0 %
  \ifdim\wd2 >\dimen@
    \dimen@=\wd2 %   
  \fi
  \count@=2 %
  \def\da@bars{\dabar@\dabar@}%
  \@whiledim\count@\wd4<\dimen@\do{%
    \advance\count@\@ne
    \expandafter\def\expandafter\da@bars\expandafter{%
      \da@bars
      \dabar@ 
    }%
  }%  
  \mathrel{#3}%
  \mathrel{%   
    \mathop{\da@bars}\limits
    \ifx\\#1\\%
    \else
      _{\copy0}%
    \fi
    \ifx\\#2\\%
    \else
      ^{\copy2}%
    \fi
  }%   
  \mathrel{#4}%
}
\makeatother
% \DeclareMathOperator{\rg}{rg}

\usepackage{mathtools}
\DeclarePairedDelimiter{\paren}{(}{)}
\DeclarePairedDelimiter{\abs}{\lvert}{\rvert}
\DeclarePairedDelimiter{\norm}{\lVert}{\rVert}
\DeclarePairedDelimiter{\innerproduct}{\langle}{\rangle}
\newcommand{\Of}[2]{{\operatorname{#1}} {\paren*{#2}}}
\newcommand{\of}[2]{{{{#1}} {\paren*{#2}}}}

\DeclareMathOperator{\Shim}{Shim}
\DeclareMathOperator{\sgn}{sgn}
\DeclareMathOperator{\fdeg}{fdeg}
\DeclareMathOperator{\SL}{SL}
\DeclareMathOperator{\slLie}{\mathfrak{s}\mathfrak{l}}
\DeclareMathOperator{\soLie}{\mathfrak{s}\mathfrak{o}}
\DeclareMathOperator{\spLie}{\mathfrak{s}\mathfrak{p}}
\DeclareMathOperator{\glLie}{\mathfrak{g}\mathfrak{l}}
\newcommand{\pn}[1]{{\color{ForestGreen} \sf PN: [#1]}}
\DeclareMathOperator{\Mp}{Mp}
\DeclareMathOperator{\Mat}{Mat}
\DeclareMathOperator{\GL}{GL}
\DeclareMathOperator{\Gr}{Gr}
\DeclareMathOperator{\GU}{GU}
\def\gl{\mathfrak{g}\mathfrak{l}}
\DeclareMathOperator{\odd}{odd}
\DeclareMathOperator{\even}{even}
\DeclareMathOperator{\GO}{GO}
\DeclareMathOperator{\good}{good}
\DeclareMathOperator{\bad}{bad}
\DeclareMathOperator{\PGO}{PGO}
\DeclareMathOperator{\htt}{ht}
\DeclareMathOperator{\height}{height}
\DeclareMathOperator{\Ass}{Ass}
\DeclareMathOperator{\coheight}{coheight}
\DeclareMathOperator{\GSO}{GSO}
\DeclareMathOperator{\SO}{SO}
\DeclareMathOperator{\so}{\mathfrak{s}\mathfrak{o}}
\DeclareMathOperator{\su}{\mathfrak{s}\mathfrak{u}}
\DeclareMathOperator{\ad}{ad}
% \DeclareMathOperator{\sc}{sc}
\DeclareMathOperator{\Ad}{Ad}
\DeclareMathOperator{\disc}{disc}
\DeclareMathOperator{\inv}{inv}
\DeclareMathOperator{\Pic}{Pic}
\DeclareMathOperator{\uc}{uc}
\DeclareMathOperator{\Cl}{Cl}
\DeclareMathOperator{\Clf}{Clf}
\DeclareMathOperator{\Hom}{Hom}
\DeclareMathOperator{\hol}{hol}
\DeclareMathOperator{\Heis}{Heis}
\DeclareMathOperator{\Haar}{Haar}
\DeclareMathOperator{\h}{h}
\def\sp{\mathfrak{s}\mathfrak{p}}
\DeclareMathOperator{\heis}{\mathfrak{h}\mathfrak{e}\mathfrak{i}\mathfrak{s}}
\DeclareMathOperator{\End}{End}
\DeclareMathOperator{\JL}{JL}
\DeclareMathOperator{\image}{image}
\DeclareMathOperator{\red}{red}
\def\div{\operatorname{div}}
\def\eps{\varepsilon}
\def\cHom{\mathcal{H}\operatorname{om}}
\DeclareMathOperator{\Ops}{Ops}
\DeclareMathOperator{\Symb}{Symb}
\def\boldGL{\mathbf{G}\mathbf{L}}
\def\boldSO{\mathbf{S}\mathbf{O}}
\def\boldU{\mathbf{U}}
\DeclareMathOperator{\hull}{hull}
\DeclareMathOperator{\LL}{LL}
\DeclareMathOperator{\PGL}{PGL}
\DeclareMathOperator{\class}{class}
\DeclareMathOperator{\lcm}{lcm}
\DeclareMathOperator{\spann}{span}
\DeclareMathOperator{\Exp}{Exp}
\DeclareMathOperator{\ext}{ext}
\DeclareMathOperator{\Ext}{Ext}
\DeclareMathOperator{\Tor}{Tor}
\DeclareMathOperator{\et}{et}
\DeclareMathOperator{\tor}{tor}
\DeclareMathOperator{\loc}{loc}
\DeclareMathOperator{\tors}{tors}
\DeclareMathOperator{\pf}{pf}
\DeclareMathOperator{\smooth}{smooth}
\DeclareMathOperator{\prin}{prin}
\DeclareMathOperator{\Kl}{Kl}
\newcommand{\kbar}{\mathchar'26\mkern-9mu k}
\DeclareMathOperator{\der}{der}
% \DeclareMathOperator{\abs}{abs}
\DeclareMathOperator{\Sub}{Sub}
\DeclareMathOperator{\Comp}{Comp}
\DeclareMathOperator{\Err}{Err}
\DeclareMathOperator{\dom}{dom}
\DeclareMathOperator{\radius}{radius}
\DeclareMathOperator{\Fitt}{Fitt}
\DeclareMathOperator{\Sel}{Sel}
\DeclareMathOperator{\rad}{rad}
\DeclareMathOperator{\id}{id}
\DeclareMathOperator{\Center}{Center}
\DeclareMathOperator{\Der}{Der}
\DeclareMathOperator{\U}{U}
% \DeclareMathOperator{\norm}{norm}
\DeclareMathOperator{\trace}{trace}
\DeclareMathOperator{\Equid}{Equid}
\DeclareMathOperator{\Feas}{Feas}
\DeclareMathOperator{\bulk}{bulk}
\DeclareMathOperator{\tail}{tail}
\DeclareMathOperator{\sys}{sys}
\DeclareMathOperator{\atan}{atan}
\DeclareMathOperator{\temp}{temp}
\DeclareMathOperator{\Asai}{Asai}
\DeclareMathOperator{\glob}{glob}
\DeclareMathOperator{\Kuz}{Kuz}
\DeclareMathOperator{\Irr}{Irr}
\newcommand{\rsL}{ \frac{ L^{(R)}(\Pi \times \Sigma, \std, \frac{1}{2})}{L^{(R)}(\Pi \times \Sigma, \Ad, 1)}  }
\DeclareMathOperator{\GSp}{GSp}
\DeclareMathOperator{\PGSp}{PGSp}
\DeclareMathOperator{\BC}{BC}
\DeclareMathOperator{\Ann}{Ann}
\DeclareMathOperator{\Gen}{Gen}
\DeclareMathOperator{\SU}{SU}
\DeclareMathOperator{\PGSU}{PGSU}
% \DeclareMathOperator{\gen}{gen}
\DeclareMathOperator{\PMp}{PMp}
\DeclareMathOperator{\PGMp}{PGMp}
\DeclareMathOperator{\PB}{PB}
\DeclareMathOperator{\ind}{ind}
\DeclareMathOperator{\Jac}{Jac}
\DeclareMathOperator{\jac}{jac}
\DeclareMathOperator{\im}{im}
\DeclareMathOperator{\Aut}{Aut}
\DeclareMathOperator{\Int}{Int}
\DeclareMathOperator{\PSL}{PSL}
\DeclareMathOperator{\co}{co}
\DeclareMathOperator{\irr}{irr}
\DeclareMathOperator{\prim}{prim}
\DeclareMathOperator{\bal}{bal}
\DeclareMathOperator{\baln}{bal}
\DeclareMathOperator{\dist}{dist}
\DeclareMathOperator{\RS}{RS}
\DeclareMathOperator{\Ram}{Ram}
\DeclareMathOperator{\Sob}{Sob}
\DeclareMathOperator{\Sol}{Sol}
\DeclareMathOperator{\soc}{soc}
\DeclareMathOperator{\nt}{nt}
\DeclareMathOperator{\mic}{mic}
\DeclareMathOperator{\Gal}{Gal}
\DeclareMathOperator{\st}{st}
\DeclareMathOperator{\std}{std}
\DeclareMathOperator{\diag}{diag}
\DeclareMathOperator{\Sym}{Sym}
\DeclareMathOperator{\gr}{gr}
\DeclareMathOperator{\aff}{aff}
\DeclareMathOperator{\Dil}{Dil}
\DeclareMathOperator{\Lie}{Lie}
\DeclareMathOperator{\Symp}{Symp}
\DeclareMathOperator{\Stab}{Stab}
\DeclareMathOperator{\St}{St}
\DeclareMathOperator{\stab}{stab}
\DeclareMathOperator{\codim}{codim}
\DeclareMathOperator{\linear}{linear}
\newcommand{\git}{/\!\!/}
\DeclareMathOperator{\geom}{geom}
\DeclareMathOperator{\spec}{spec}
\def\O{\operatorname{O}}
\DeclareMathOperator{\Au}{Aut}
\DeclareMathOperator{\Fix}{Fix}
\DeclareMathOperator{\Opp}{Op}
\DeclareMathOperator{\opp}{op}
\DeclareMathOperator{\Size}{Size}
\DeclareMathOperator{\Save}{Save}
% \DeclareMathOperator{\ker}{ker}
\DeclareMathOperator{\coker}{coker}
\DeclareMathOperator{\sym}{sym}
\DeclareMathOperator{\mean}{mean}
\DeclareMathOperator{\elliptic}{ell}
\DeclareMathOperator{\nilpotent}{nil}
\DeclareMathOperator{\hyperbolic}{hyp}
\DeclareMathOperator{\newvector}{new}
\DeclareMathOperator{\new}{new}
\DeclareMathOperator{\full}{full}
\newcommand{\qr}[2]{\left( \frac{#1}{#2} \right)}
\DeclareMathOperator{\unr}{u}
\DeclareMathOperator{\ram}{ram}
% \DeclareMathOperator{\len}{len}
\DeclareMathOperator{\fin}{fin}
\DeclareMathOperator{\cusp}{cusp}
\DeclareMathOperator{\curv}{curv}
\DeclareMathOperator{\rank}{rank}
\DeclareMathOperator{\rk}{rk}
\DeclareMathOperator{\pr}{pr}
\DeclareMathOperator{\Transform}{Transform}
\DeclareMathOperator{\mult}{mult}
\DeclareMathOperator{\Eis}{Eis}
\DeclareMathOperator{\reg}{reg}
\DeclareMathOperator{\sing}{sing}
\DeclareMathOperator{\alt}{alt}
\DeclareMathOperator{\irreg}{irreg}
\DeclareMathOperator{\sreg}{sreg}
\DeclareMathOperator{\Wd}{Wd}
\DeclareMathOperator{\Weil}{Weil}
\DeclareMathOperator{\Th}{Th}
\DeclareMathOperator{\Sp}{Sp}
\DeclareMathOperator{\Ind}{Ind}
\DeclareMathOperator{\Res}{Res}
\DeclareMathOperator{\ini}{in}
\DeclareMathOperator{\ord}{ord}
\DeclareMathOperator{\osc}{osc}
\DeclareMathOperator{\fluc}{fluc}
\DeclareMathOperator{\size}{size}
\DeclareMathOperator{\ann}{ann}
\DeclareMathOperator{\equ}{eq}
\DeclareMathOperator{\res}{res}
\DeclareMathOperator{\pt}{pt}
\DeclareMathOperator{\src}{source}
\DeclareMathOperator{\Zcl}{Zcl}
\DeclareMathOperator{\Func}{Func}
\DeclareMathOperator{\Map}{Map}
\DeclareMathOperator{\Frac}{Frac}
\DeclareMathOperator{\Frob}{Frob}
\DeclareMathOperator{\ev}{eval}
\DeclareMathOperator{\pv}{pv}
\DeclareMathOperator{\eval}{eval}
\DeclareMathOperator{\Spec}{Spec}
\DeclareMathOperator{\Speh}{Speh}
\DeclareMathOperator{\Spin}{Spin}
\DeclareMathOperator{\GSpin}{GSpin}
\DeclareMathOperator{\Specm}{Specm}
\DeclareMathOperator{\Sphere}{Sphere}
\DeclareMathOperator{\Sqq}{Sq}
\DeclareMathOperator{\Ball}{Ball}
\DeclareMathOperator\Cond{\operatorname{Cond}}
\DeclareMathOperator\proj{\operatorname{proj}}
\DeclareMathOperator\Swan{\operatorname{Swan}}
\DeclareMathOperator{\Proj}{Proj}
\DeclareMathOperator{\bPB}{{\mathbf P}{\mathbf B}}
\DeclareMathOperator{\Projm}{Projm}
\DeclareMathOperator{\Tr}{Tr}
\DeclareMathOperator{\Type}{Type}
\DeclareMathOperator{\Prop}{Prop}
\DeclareMathOperator{\vol}{vol}
\DeclareMathOperator{\covol}{covol}
\DeclareMathOperator{\Rep}{Rep}
\DeclareMathOperator{\Cent}{Cent}
\DeclareMathOperator{\val}{val}
\DeclareMathOperator{\area}{area}
\DeclareMathOperator{\nr}{nr}
\DeclareMathOperator{\CM}{CM}
\DeclareMathOperator{\CH}{CH}
\DeclareMathOperator{\tr}{tr}
\DeclareMathOperator{\characteristic}{char}
\DeclareMathOperator{\supp}{supp}


\theoremstyle{plain} \newtheorem{theorem} {Theorem} \newtheorem{conjecture} [theorem] {Conjecture} \newtheorem{corollary} [theorem] {Corollary} \newtheorem{proposition} [theorem] {Proposition} \newtheorem{fact} [theorem] {Fact}
\theoremstyle{definition} \newtheorem{definition} [theorem] {Definition} \newtheorem{hypothesis} [theorem] {Hypothesis} \newtheorem{assumptions} [theorem] {Assumptions}
\newtheorem{example} [theorem] {Example}
\newtheorem{assertion}[theorem] {Assertion}
\newtheorem{note}[theorem] {Note}
\newtheorem{conclusion}[theorem] {Conclusion}
\newtheorem{claim}            {Claim}
\newtheorem{homework} {Homework}
\newtheorem{exercise} {Exercise}  \newtheorem{question}[theorem] {Question}    \newtheorem{answer} {Answer}  \newtheorem{problem} {Problem}    \newtheorem{remark} [theorem] {Remark}
\newtheorem{notation} [theorem]           {Notation}
\newtheorem{terminology}[theorem]            {Terminology}
\newtheorem{convention}[theorem]            {Convention}
\newtheorem{motivation}[theorem]            {Motivation}


\newtheoremstyle{itplain} % name
{6pt}                    % Space above
{5pt\topsep}                    % Space below
{\itshape}                   % Body font
{}                           % Indent amount
{\itshape}                   % Theorem head font
{.}                          % Punctuation after theorem head
{5pt plus 1pt minus 1pt}                       % Space after theorem head
% {.5em}                       % Space after theorem head
{}  % Theorem head spec (can be left empty, meaning ‘normal’)

% \theoremstyle{mytheoremstyle}


\theoremstyle{itplain} %--default
% \theoremheaderfont{\itshape}
% \newtheorem{lemma}{Lemma}
\newtheorem{lemma}[theorem]{Lemma}
% \newtheorem{lemma}{Lemma}[subsubsection]

\newtheorem*{lemma*}{Lemma}
\newtheorem*{proposition*}{Proposition}
\newtheorem*{definition*}{Definition}
\newtheorem*{example*}{Example}

\newtheorem*{results*}{Results}
\newtheorem{results} [theorem] {Results}


\usepackage[displaymath,textmath,sections,graphics]{preview}
\PreviewEnvironment{align*}
\PreviewEnvironment{multline*}
\PreviewEnvironment{tabular}
\PreviewEnvironment{verbatim}
\PreviewEnvironment{lstlisting}
\PreviewEnvironment*{frame}
\PreviewEnvironment*{alert}
\PreviewEnvironment*{emph}
\PreviewEnvironment*{textbf}



\usepackage{xr-hyper}
\externaldocument{20230522T150333__microlocal-localized-vectors}
\externaldocument{standard}
\externaldocument{20230522T190340__pgl2-spectral-aspect-subconvexity_expository}
\externaldocument{var-quat-3-submitted}
% \externaldocument{20230528T113309__microlocal-irreducibility-pgl2}

\numberwithin{equation}{section}

\title{Some exercises concerning localized vectors in low rank}

\begin{document}

\maketitle
\tableofcontents


\begin{abstract}
  We record some exercises whose purpose is to verify that certain classes of vectors in representations of $\SO(3)$ and $\PGL_2(\mathbb{R})$ are ``localized'' in a strong sense under the action of those groups.
\end{abstract}

\section{Overview}\label{sec:d1a9162ede03}

In this note, we give exercises that aim to convey some computational feeling for ``localized vectors'' in the precise sense defined in \S\ref{sec:d1a8de614cc9} of \href{20230522T150333__microlocal-localized-vectors.pdf}{this note}, focusing on low-rank examples.  Along the way, we recall the basic representation theory for such examples.

\section{Setup}\label{sec:d1a94d0b364d}
We let $T \rightarrow \infty$ be an asymptotic parameter, and retain the asymptotic notation and conventions of \S\ref{sec:d1a8dda545b5} of \href{20230522T150333__microlocal-localized-vectors.pdf}{this note} concerning ``$T$-dependent elements'', ``fixed'' (equivalently, ``$T$-independent'') and ``classes''.  In particular, we recall that ``class'' means ``collection of $T$-dependent sets''.  A typical example is the class $\O(1)$ inside $\mathbb{C}$, consisting of all $T$-dependent subsets $S = S_T \subseteq \mathbb{C}$ for which there is a fixed $C \geq 0$ so that for all $T$, we have $\lVert c_T \rVert \leq C$ for all $c_T \in S_T$.

Let $G$ be a fixed real Lie group, and let $\pi = \pi_T$ be a $T$-dependent unitary representation.  We recall Theorem \ref{theorem:d1a913c8091b} from \href{20230522T150333__microlocal-localized-vectors.pdf}{this note}:
\begin{theorem}\label{theorem:d1a94d0ddd25}
  Let $M$ be a class of $T$-dependent vectors $v = v_T$ in $\pi = \pi_T$ with the following properties:
  \begin{enumerate}[(i)]
  \item For each $v \in M$, we have $\lVert v \rVert \leq T^{\O(1)}$.
  \item For all $u, v \in M$, we have $u + v \in M$.
  \item For all $v \in M$ and $c \in \mathbb{C}$ with $c = \O(1)$, we have $c v \in M$.
  \item For each fixed $\eps > 0$, fixed $x \in \mathfrak{g}$ and each $v \in M$, we have
    \begin{equation}\label{eqn:d1a94c11ed29}
      x v - \langle x, \tau  \rangle v \in T^{1/2+\eps} M.
    \end{equation}
    That is to say, the $T$-dependent vector on the left hand side may be written $T^{1/2+\eps} u$, where $u$ belongs to the class $M$.
  \end{enumerate}
  Then each $v \in M$ is localized at $\tau$ in the sense of Definition \ref{definition:d1a913bed772} of \href{20230522T150333__microlocal-localized-vectors.pdf}{this note}.
\end{theorem}

The purpose of the present note is to give some examples of classes $M$ satisfying the above conditions, hence, in particular, examples of localized vectors.  In each case, the first three properties will clearly hold, so we do not mention them; the main point is to verify the approximate eigenvector property \eqref{eqn:d1a94c11ed29} for elements $x$ of a fixed basis of $\mathfrak{g}$.


\section{The group $\SO(3)$ via weight vectors}\label{sec:d1a9162ed4bc}

\subsection{Lie algebra}\label{sec:d1a9162ece67}
We consider the Lie group $\SO(3)$.  Its Lie algebra $\so(3)$ admits a basis $\{R_1, R_2, R_3\}$, where for any angle $\theta$, the element $\exp(\theta R_j)$ defines rotation by $\theta$ about the $j$th axis.  These satisfy the commutation relations
\begin{equation*}
  ~ [R_1,R_2] = R_3, \quad [R_2,R_3] = R_1, \quad [R_3,R_1] = R_2.
\end{equation*}
The center of the universal enveloping algebra is generated by the Casimir element
\begin{equation*}
  \Omega = -(R_1^2 + R_2^2 + R_3^2).
\end{equation*}

Define the following elements $X,Y$ of the complexified Lie algebra $\so(3)_{\mathbb{C}}$:
\begin{equation*}
  X := R_1 + i R_2,
  \quad
  Y := -R_1 + i R_2.
\end{equation*}
Then $X,Y,R_3$ is a basis for $\so(3)_{\mathbb{C}}$ satisfying the commutation relations
\begin{equation}\label{eqn:20230524102817} [X,Y] = 2 i R_3,
  \quad
  [i R_3, X] = X,
  \quad
  [i R_3, Y] = - Y.
\end{equation}

We observe also that
\begin{equation*}
  R_1 = \frac{X - Y}{2},
  \quad
  R_2 = \frac{X + Y}{2 i},
\end{equation*}
\begin{equation*}
  \Omega =
  \frac{X Y + Y X}{2} - R_3^2.
\end{equation*}
By writing $X Y = [X,Y] + Y X$ and appealing to the formula \eqref{eqn:20230524102817} for $[X,Y]$, we see that
\begin{equation}\label{eqn:20230524102839}
  \Omega = Y X +  i R_3(i R_3 + 1).
\end{equation}
Similarly,
\begin{equation}\label{eqn:20230524104145}
  \Omega = X Y  + i R_3 (i R_3 - 1).
\end{equation}

The imaginary dual of the Lie algebra identifies with the space of triples of imaginary numbers:
\begin{equation}\label{eqn:d1a914f48fd7}
  \so(3)^\wedge \cong  i \mathbb{R}^3.
\end{equation}
Here $\xi \in i \mathbb{R}^3$ corresponds to the linear map $\so(3) \rightarrow i \mathbb{R}$ given on basis elements by $R_j \mapsto \xi_j$.

\subsection{Representations}\label{sec:d1a9162ec67d}
Let $\pi$ be a (complex) representation of $\SO(3)$.  It may be decomposed into eigenspaces for $R_3$.  Since $\exp(2 \pi R_3) = 1$, the eigenvalues of $i R_3$ are integers:
\begin{equation*}
  \pi = \oplus_{m \in \mathbb{Z}} \pi(m),
  \quad
  \pi(m) := \{ v \in \pi : i R_3 v = m v \}.
\end{equation*}
The $m$ for which $\pi(m) \neq 0$ are called the \emph{weights} of $\pi$, and the dimensions $\dim \pi(m)$ the corresponding \emph{weight multiplicities}.  From the commutation relations \eqref{eqn:20230524102817}, we see that
\begin{equation}\label{eqn:20230524104326}
  X : \pi(m) \rightarrow \pi(m+1),
  \quad
  Y : \pi(m) \rightarrow \pi(m-1).
\end{equation}

\begin{proposition}\label{proposition:d1a913c915d4}
  Let $\pi$ be an irreducible unitary representation of $\SO(3)$.  Then $\Omega$ acts on $\pi$ by a scalar of the form
  \begin{equation}\label{eqn:d1a9136775d2}
    \Omega_\pi = \ell (\ell + 1)
  \end{equation}
  for some nonnegative integer $\ell$.  This scalar determines the isomorphism class of $\pi$.  In fact, there is a basis
  \begin{equation*}
    e_{- \ell}, \qquad e_{- \ell + 1}, \qquad \dotsc, \qquad e_{\ell }
  \end{equation*}
  for $\pi$ on which the Lie algebra acts by the formulas
  \begin{equation}\label{eqn:d1a913647caa}
    X e_m = \left( \Omega_\pi  - m (m + 1) \right)^{1/2} e_{m + 1}
  \end{equation}
  \begin{equation}\label{eqn:d1a9136592ce}
    Y e_{m+1} = \left( \Omega_\pi - m (m + 1) \right)^{1/2} e_{m},
  \end{equation}
  \begin{equation}\label{eqn:d1a91367ba12}
    i R_3 e_m = m e_m.
  \end{equation}
  If $\pi$ is unitary, then this basis is orthonormal.
\end{proposition}
\begin{proof}
  Since $\SO(3)$ is compact, we know by the Peter--Weyl theorem that $\pi$ is finite-dimensional.  There is thus a largest element $\ell \in \mathbb{Z}_{\geq 0}$ with $\pi(\ell) \neq 0$.  For any $v \in \pi(\ell)$, we have $X v \in \pi(\ell+1) = \{0\}$, hence $X v = 0$.  By \eqref{eqn:20230524102839}, it follows that
  \begin{equation}\label{eqn:20230524102429}
    0 = Y X v = \Omega v - \ell (\ell + 1) v.
  \end{equation}
  Since $\pi$ is irreducible and $\Omega$ commutes with the action of $\SO(3)$, we know by Schur's lemma that $\Omega$ acts on $\pi$ by a scalar.  Taking $v$ to be a nonzero element of $\pi(\ell)$, we see from \eqref{eqn:20230524102429} that this scalar must be given by \eqref{eqn:d1a9136775d2}.  We now choose a nonzero vector $e_{\ell} \in V(\ell)$ and define $e_m$ by reverse induction for integers $m$ with $- \ell \leq m < \ell$ by requiring that \eqref{eqn:d1a9136592ce} hold, noting that the square root is positive in the stated range.  Using \eqref{eqn:20230524104145}, we see that $Y e_{- \ell} = 0$.  By the mapping property \eqref{eqn:20230524104326}, we have $e_m \in \pi(m)$, hence \eqref{eqn:d1a91367ba12} holds.  By the formula \eqref{eqn:20230524104145} for $\Omega$, we have
  \begin{equation}\label{eqn:d1a94c62c57d}
    X Y e_{m + 1} = \Omega e_{m+1} - m(m+1) e_{m+1}
    =
    (\Omega - m (m + 1)) e_{m+1},
  \end{equation}
  and so \eqref{eqn:d1a913647caa} holds.  From the formulas established thus far, we see that the $e_m$ ($- \ell \leq m \leq \ell$) span an invariant subspace of $\pi$, which by the irreducibility hypothesis must be $\pi$ itself.

  We have established all assertions except that the basis may be taken orthonormal when $\pi$ is unitary.  We may assume that $e_{\ell}$ was normalized to be a unit vector, and will verify then by reverse inductive on $m < \ell$ that $e_m$ is then likewise a unit vector.  To that end, observe first by \eqref{eqn:d1a9136592ce} that
  \begin{equation*}
    (\Omega_\pi - m(m+1))
    \lVert e_{m} \rVert^2
    = \left\langle Y e_{m + 1}, Y e_{m+1} \right\rangle,
  \end{equation*}
  then use that the adjoint of $Y$ is $- \bar{Y} = X$ to see that
  \begin{equation*}
    \left\langle Y e_{m + 1}, Y e_{m+1} \right\rangle
    =
    \left\langle X Y e_{m + 1}, e_{m+1} \right\rangle.
  \end{equation*}
  By \eqref{eqn:d1a94c62c57d}, we deduce that $e_m$ and $e_{m + 1}$ have the same norm, so the induction follows as claimed.
\end{proof}

The integer $\ell$ as in the conclusion of Proposition \ref{proposition:d1a913c915d4} is called the \emph{highest weight} of $\pi$.  The coadjoint orbit for $\pi$ turns out to be given in the optic \eqref{eqn:d1a914f48fd7} by the sphere of radius $\ell + 1/2$:
\begin{equation*}
  \mathcal{O}_\pi = \left\{ (a,b,c) : a^2 + b^2 + c^2 = (T + \tfrac{1}{2} )^2 \right\}.
\end{equation*}

\subsection{Localized vectors}\label{sec:d1a9162ebd8a}
In the following exercises, we assume that the asymptotic parameter $T \rightarrow \infty$ is valued in the nonnegative integers, and let $\pi$ denote the $T$-dependent representation having highest weight $T$.

\begin{exercise}\label{exercise:d1a913e1e9ec}
  Let $M$ denote the class of $T$-dependent vectors $v$ in $\pi$ given in terms of a basis as in Proposition \ref{proposition:d1a913c915d4} by $v = \sum_m a_m e_m$, where the coefficients have the following properties:
  \begin{enumerate}
  \item $a_m = 0$ unless $m = T + \O(1)$.
  \item Each $a_m = \O(1)$.
  \end{enumerate}
  Verify that for all $v \in M$, we have
  \begin{equation*}
    X v \in T^{1/2} M,
  \end{equation*}
  \begin{equation*}
    Y v \in T^{1/2} M,
  \end{equation*}
  \begin{equation*}
    i R_3 v - i T v \in T^{1/2} M.
  \end{equation*}
  Deduce from Theorem \ref{theorem:d1a94d0ddd25} that every element of $M$ is localized at the $T$-dependent element $\tau \in \mathfrak{g}^\wedge$ given in the optic \eqref{eqn:d1a914f48fd7} by
  \begin{equation*}
    \tau = (0,0,i T).
  \end{equation*}
\end{exercise}

\begin{exercise}\label{exercise:d1a94c55524c}
  Let $M$ be the class of $T$-dependent vectors in $\pi$ of the form $\sum a_m e_m$, where the coefficients have the following properties:
  \begin{enumerate}
  \item $a_m = 0$ unless $m = \O(T^{1/2})$.
  \item The function of $\theta \in \mathbb{R} / \mathbb{Z}$ defined by
    \begin{equation*}
      a(\theta) := \sum_n a(n) e (n \theta), \qquad
      e(\theta) := e^{2 \pi i \theta }      
    \end{equation*}
    is an $L^2$-normalized bump of width $T^{-1/2}$, in the following sense: for fixed $k , \ell \in \mathbb{Z}_{\geq 0}$,
    \begin{equation}\label{eqn:d1a9c4e68198}
      a^{(\ell)}(\theta) \ll
      T^{1/4 + \ell/2} \left( 1+ \frac{\lVert \theta  \rVert}{T^{1/2}} \right)^{-k},
    \end{equation}
    where $a^{(\ell)}$ denotes the $\ell$th derivative and $\lVert \theta \rVert$ the distance to the nearest integer.
  \end{enumerate}
  We note that, by Parseval, the second condition implies that $\sum_m \lvert a_m \rvert^2 = \O(1)$.
  \begin{enumerate}[(i)]
  \item Show that if $f \in C_c^\infty(\mathbb{R})$ is fixed, then the $T$-dependent vector $\sum_m a_m e_m$ with coefficients
    \begin{equation}\label{eqn:d1a9c4e9c308}
      a_n := T^{-1/4} f\left( \frac{n}{T^{1/2}} \right)
    \end{equation}
    belongs to $M$.
  \item Show that for all $v \in M$,
    \begin{equation*}
      X v - T v \in T^{1/2} M,
    \end{equation*}
    \begin{equation*}
      Y v - T v \in T^{1/2} M,
    \end{equation*}
    \begin{equation*}
      R_3 v \in T^{1/2} M.
    \end{equation*}
  \end{enumerate}
  Deduce that every element of $M$, and in particular, the element defined by \eqref{eqn:d1a9c4e9c308}, is localized at the $T$-dependent element $\tau \in \mathfrak{g}^*$ given in the optic \eqref{eqn:d1a914f48fd7} by
  \begin{equation*}
    \tau = (0,-iT,0).
  \end{equation*}
\end{exercise}



\section{The group $\PGL_2(\mathbb{R})$ via weight vectors}\label{sec:d1a9162eb3c7}
\subsection{Preliminaries}\label{sec:d1a915ba8ec6}
We now turn to the group
\begin{equation*}
  G := \PGL_2(\mathbb{R}).
\end{equation*}
We will use the following notation for a basis of its complexified Lie algebra $\mathfrak{sl}_2(\mathbb{R})_{\mathbb{C}} = \slLie_2(\mathbb{C})$:
\begin{align*}
  X &:= \frac{1}{2 i}
      \begin{pmatrix}
        1 & i  \\
        i & -1
      \end{pmatrix},
  \\
  Y &:= \frac{1}{2 i}
      \begin{pmatrix}
        1  & -i \\
        -i & -1
      \end{pmatrix},
  \\
  H &:= \frac{1}{2 i}
      \begin{pmatrix}
        0  & 1 \\
        -1 & 0
      \end{pmatrix}.
\end{align*}
The standard maximal compact connected subgroup $K$ of $G$, namely the image of $\SO(2)$, is then
\begin{equation*}
  K = \left\{ \exp(\theta H) : \theta \in \mathbb{R} / 2 \pi \mathbb{Z}  \right\}.
\end{equation*}
The commutation relations are
\begin{equation*}
  ~
  [X,Y] = - 2 H,  \quad
  [H,X] = X,
  \quad
  [H,Y] = -Y.
\end{equation*}
The center of the universal enveloping algebra is generated by the Casimir element
\begin{align*}
  \Omega &:= H^2 - \frac{X Y + Y X}{2} \\
         &= H(H-1) - X Y \\
         &= H(H+1) - Y X.
\end{align*}
The imaginary dual of the Lie algebra identifies with the space of imaginary traceless $2 \times 2$ matrices:
\begin{equation}\label{eqn:d1a9150ab652}
  \slLie_2(\mathbb{R})^\wedge \cong i \slLie_2(\mathbb{R}).
\end{equation}
Here $\xi \in i \slLie_2(\mathbb{R})$ corresponds to the linear map $\slLie_2(\mathbb{R}) \rightarrow i \mathbb{R} $ given by $x \mapsto \trace(x \xi)$.


\subsection{Representations}\label{sec:d1a915c57e2f}

\begin{proposition}
  Let $\pi$ be an irreducible unitary representation of $\PGL_2(\mathbb{R})$.  Then $\Omega$ acts on $\pi$ by a scalar, say $\Omega_\pi$.  Then, either:
  \begin{enumerate}[(i)]
  \item $\pi$ is a one-dimensional representation, either trivial or the sign representation, in which case $\Omega_\pi = 0$.
  \item $\pi$ is a discrete series representation $\pi(k)$ for some $k \in \mathbb{Z}_{\geq 1}$, with $\Omega_\pi = k(k-1)$.  (We have chosen the numbering so that such representations correspond to holomorphic modular forms of weight $2 k$ in the traditional sense.)
  \item $\pi$ is a unitary principal series representation $\pi(t,\eps)$, with
    \begin{itemize}
    \item$t \in \mathbb{R}$ and $\eps \in \{\pm 1\}$, or
    \item $t \in i (\tfrac{1}{2}, \tfrac{1}{2} ) - \{0\}$ and $\eps = 1$,
    \end{itemize}
    with $\Omega_\pi = - \tfrac{1}{4} - t^2$.
  \end{enumerate}
  The only equivalences are that $\pi(t,\eps) \cong \pi(-t,\eps)$.

  The representation $\pi = \pi(t,\eps)$ admits a basis $e_m$, indexed by $m \in \mathbb{Z}$, on which the Lie algebra elements act by the formulas
  \begin{equation*}
    X e_m = ( m(m+1) - \Omega_\pi)^{1/2} e_{m+1},
  \end{equation*}
  \begin{equation*}
    Y e_{m+1} = (m(m+1) - \Omega_\pi)^{1/2} e_{m},
  \end{equation*}
  \begin{equation*}
    H e_m = m e_m,
  \end{equation*}
  \begin{equation*}
    \diag(-1,1) e_m = (-1)^{\eps} e_{-m}.
  \end{equation*}
  The representation $\pi = \pi(k)$ admits a basis $e_m$, indexed by $\{m \in \mathbb{Z} : |m| \geq k\}$, on which the Lie algebra elements act by the same formulas as above, but with $\eps = 1$.
\end{proposition}
\begin{proof}
  Similar to that of Proposition \ref{proposition:d1a913c915d4}.
\end{proof}

The tempered irreducible representations are the $\pi(k)$ and the $\pi(t,\eps)$ with $t \in \mathbb{R}$.  For either of these, the coadjoint orbit $\mathcal{O}_\pi$ is given in the optic \eqref{eqn:d1a9150ab652} by
\begin{equation}\label{eqn:d1a91628a316}
  \mathcal{O}_\pi = \left\{ 0 \neq \xi \in i \slLie_2(\mathbb{R}) :
    \det(\xi/i) =   
    \tfrac{1}{4} + \Omega_\pi     
  \right\}.
\end{equation}

\subsection{Localized vectors}\label{sec:d1a9162ea6e6}

\begin{exercise}\label{exercise:d1aa01c52923}
  Let $\pi$ be the $T$-dependent representation of $\PGL_2(\mathbb{R})$ given by the discrete series representation $\pi_T = \pi(k)$ of lowest weight
  \begin{equation*}
    k = k_T :=T.
  \end{equation*}
  Let $M$ denote the class of $T$-dependent vectors $v$ in $\pi$ of the form $v = \sum_m a_m e_m$, where
  \begin{enumerate}
  \item $a_m = 0$ unless $m = T + \O(1)$, and
  \item each $a_m = \O(1)$.
  \end{enumerate}
  Verify that for all $v \in M$, we have
  \begin{equation*}
    X v \in T^{1/2} M,
  \end{equation*}
  \begin{equation*}
    Y v \in T^{1/2} M,
  \end{equation*}
  \begin{equation*}
    H v - T v \in T^{1/2} M.
  \end{equation*}
  Deduce that every element of $M$ is localized at the $T$-dependent element $\tau \in \mathfrak{g}^*$ given in the optic \eqref{eqn:d1a9150ab652} by
  \begin{equation*}
    \tau =
    i T
    \begin{pmatrix}
      0 & 1 \\
      -1 & 0 \\
    \end{pmatrix}.
  \end{equation*}
\end{exercise}


\begin{exercise}\label{exercise:d1aa01c53f48}
  Let $\pi$ be the $T$-dependent representation of $\PGL_2(\mathbb{R})$ given by the tempered principal series representation $\pi_T = \pi(t,\eps)$, where
  \begin{equation*}
    t = t_T := T
  \end{equation*}
  while $\eps \in \{\pm 1\}$ is fixed.  Let $M$ denote the class of $T$-dependent vectors $v$ in $\pi$ of the form $v = \sum_m a_m e_m$, where the coefficients satisfy the support condition
  \begin{equation*}
    a_m \neq 0 \implies m = \O(T^{1/2})
  \end{equation*}
  as well as the Fourier series condition \eqref{eqn:d1a9c4e68198} enunciated in Exercise \ref{exercise:d1a94c55524c}.  Verify that for all $v \in M$, we have
  \begin{equation*}
    X v - T v \in T^{1/2} M,
  \end{equation*}
  \begin{equation*}
    Y v - T v \in T^{1/2}  M,
  \end{equation*}
  \begin{equation*}
    H v \in T^{1/2} M.
  \end{equation*}
  Deduce that every element of $M$ is localized at
  \begin{equation*}
    \tau = i T
    \begin{pmatrix}
      1 &  0 \\
      0 & -1 \\
    \end{pmatrix}.
  \end{equation*}
\end{exercise}

\begin{remark}
  Suppose that $\pi$ as in Exercise \ref{exercise:d1aa01c53f48} comes equipped with a equivariant isometric embedding
  \begin{equation*}
    \iota : \pi \hookrightarrow L^2(\Gamma \backslash G)
  \end{equation*}
  for some finite volume quotient $\Gamma \backslash G$ by a discrete subgroup $\Gamma < G = \PGL_2(\mathbb{R})$, such as $\Gamma = \PGL_2(\mathbb{Z})$.  Under such an embedding, the spherical vector $e_0$ maps to an $L^2$-normalized Maass form
  \begin{equation*}
    \varphi_0 := \iota (e_0) \in \pi^K \subseteq C^\infty(\Gamma \backslash G / K)
  \end{equation*}
  of eigenvalue $1/4 + t^2$.  The image of the class $M$ is closely related to the microlocal lift of Zelditch et al (see \cite{MR916129, MR1859345, MR2346281, MR2314452}).  More precisely, for each unit vector $v \in M$, its image
  \begin{equation*}
    \varphi := \iota(v) \in  \pi \subseteq C^\infty(\Gamma \backslash G)
  \end{equation*}
  may be shown to have the following properties
  \begin{itemize}
  \item For each fixed $\Psi \in C_c^\infty(\Gamma \backslash G / K)$, we have
    \begin{equation*}
      \int_{\Gamma \backslash G} \lvert \varphi  \rvert^2 \Psi
      = 
      \int_{\Gamma \backslash G} \lvert \varphi_0  \rvert^2 \Psi
      + \O(T^{1/2+\eps}).
    \end{equation*}
  \item Let $G_\tau \leq G$ denote the centralizer of $\tau$, thus $G_\tau$ is the diagonal subgroup
    \begin{equation*}
      G_\tau =
      \begin{pmatrix}
        \ast & 0 \\
        0 & \ast \\
      \end{pmatrix}.
    \end{equation*}
    For each fixed $\Psi \in C_c^\infty(\Gamma \backslash G)$ and fixed $g \in G_\tau$, we have
    \begin{equation*}
      \int_{x \in \Gamma \backslash G}
      \lvert \varphi(x g)  \rvert^2 \Psi(x) \, d \mu(x)
      =
      \int_{x \in \Gamma \backslash G}
      \lvert \varphi(x)  \rvert^2 \Psi(x) \, d \mu(x)
      + \O(T^{1/2+\eps}).
    \end{equation*}
  \end{itemize}
  For some more precise assertions, see \cite[\S7.1, Lemma 2]{nelson-variance-3} (direct link: \ref{thm:characterize-mu-pi}) for further discussion.
  % For some orbit method perspective on these results, see \href{20230528T113309__microlocal-irreducibility-pgl2.pdf}{this note}.
\end{remark}

\begin{exercise}\label{exercise:d1aa01c4e580}
  Let $\pi$ be any fixed infinite-dimensional irreducible unitary representation of $\PGL_2(\mathbb{R})$ (e.g., the tempered principal series representation $\pi(0,1)$).  Let $M$ denote the class of $T$-dependent vectors $v$ in $\pi$ of the form $v = \sum_m a_m e_m$, where the coefficients satisfy the support condition
  \begin{equation*}
    a_m \neq 0 \implies m = T+ \O(T^{1/2})
  \end{equation*}
  as well as the Fourier series condition \eqref{eqn:d1a9c4e68198} enunciated in Exercise \ref{exercise:d1a94c55524c}.  Verify that for all $v \in M$, we have
  \begin{equation*}
    X v - T v \in T^{1/2} M,
  \end{equation*}
  \begin{equation*}
    Y v - T v \in T^{1/2} M,
  \end{equation*}
  \begin{equation*}
    H v - T v \in T^{1/2} M.
  \end{equation*}
  Deduce that every element of $M$ is localized at
  \begin{equation*}
    \tau =  i T
    \begin{pmatrix}
      1 & -1 \\
      1 & -1 \\
    \end{pmatrix}.
  \end{equation*}
\end{exercise}

\section{The group $\PGL_2(\mathbb{R})$ via the Kirillov model}\label{sec:d1a9162e9d89}

\subsection{Preliminaries}\label{sec:d1a9162e9385}
Set $G := \PGL_2(\mathbb{R})$.  We will work with the subgroups
\begin{equation*}
  N := \left\{ n(x) :=
    \begin{pmatrix}
      1 & x \\
      0 & 1 \\
    \end{pmatrix} : x \in \mathbb{R}  \right\},
\end{equation*}
\begin{equation*}
  A :=
  \left\{   a(y) := 
    \begin{pmatrix}
      y  & 0 \\
      0 & 1 \\
    \end{pmatrix} : y \in \mathbb{R}^\times  \right\},
\end{equation*}
\begin{equation*}
  B := N A
  =
  \begin{pmatrix}
    \ast & \ast \\
    0 & \ast \\
  \end{pmatrix}.  
\end{equation*}

Let $\psi : \mathbb{R} \rightarrow \U(1)$ be a nontrivial unitary character.  We may identify $\psi$ with a character of the subgroup Set
\begin{equation*}
  C^\infty((N,\psi) \backslash G) := \left\{ W \in C^\infty(G): W(n(x) g) = \psi(x) W(g) \text{ for all } (x,g) \in \mathbb{R} \times G \right\}
\end{equation*}
Let $\pi$ be an irreducible representation of $G$.  More precisely, we denote here by $\pi$ the subspace of smooth vectors.  We recall that $\pi$ is \emph{generic} if there is an equivariant embedding $\pi \hookrightarrow C^\infty((N,\psi) \backslash G)$.  The space of such embeddings is then one-dimensional, so the image, call it $\mathcal{W}(\pi,\psi)$, is well-defined.  Moreover, the restriction map
\begin{equation*}
  \mathcal{W}(\pi,\psi) \rightarrow \{\text{functions } A \rightarrow \mathbb{C} \}
\end{equation*}
is injective, and its image contains $C_c^\infty(A)$.  Consequently, each $W \in \mathcal{W}(\pi,\psi)$ is determined by the function $W : \mathbb{R}^\times \rightarrow \mathbb{C}$ given by
\begin{equation*}
  W(y) := W(a(y)),
\end{equation*}
and every smooth compactly-supported function arises in this way.  We obtain in this way a realization of $\pi$ as a space of functions on $\mathbb{R}^\times$, called the \emph{Kirillov model}.  When $\pi$ is unitary, an invariant inner product may be given in the Kirillov model by
\begin{equation}\label{eqn:d1a915c9af68}
  \langle W_1, W_2 \rangle := \int_{y \in \mathbb{R}^\times } W_1(y) \overline{W_2(y)} \, d^\times y, \quad
  d^\times y := \frac{d y}{\lvert y \rvert}.
\end{equation}
Standard references for these facts include \cite[\S6]{MR748505}, \cite[\S10.2]{MR1999922}, \cite{MR2733072}.

The action of $B$ on the Kirillov model is completely explicit: we have
\begin{equation}\label{eqn:d1a915ae9c3b}
  n(x) W(y) = \psi(y x) W(y),
\end{equation}
\begin{equation}\label{eqn:d1a915b0643b}
  a(u) W(y) = W(y u).
\end{equation}
Indeed \eqref{eqn:d1a915ae9c3b} follows from the commutation property $a(y) n(x) = n(y x) a(y)$ and the left $N$-equivariance of $W$, while \eqref{eqn:d1a915b0643b} is obvious.

The infinitesimal generators of $N$ and $A$ are the matrices
\begin{equation*}
  e :=
  \begin{pmatrix}
    0 & 1 \\
    0 & 0 \\
  \end{pmatrix},
  \quad
  h :=
  \frac{1}{2} 
  \begin{pmatrix}
    1 & 0 \\
    0 & -1 \\
  \end{pmatrix}.
\end{equation*}
These act on $\pi$ by differential operators.  The formulas for their action is simplest when
\begin{equation}\label{eqn:d1a915c27da9}
  \psi(x) := e^{i x},
\end{equation}
so let's specialize to that case.  By differentiating \eqref{eqn:d1a915ae9c3b} and \eqref{eqn:d1a915b0643b}, we see that
\begin{equation}\label{eqn:d1a915a9f0b7}
  e W(y) = i y W(y),
\end{equation}
\begin{equation}\label{eqn:d1a915aa06ba}
  h W (y) = y \partial_y W(y).
\end{equation}
The other standard Lie algebra basis element is
\begin{equation*}
  f :=
  \begin{pmatrix}
    0 & 0 \\
    1 & 0 \\
  \end{pmatrix}.
\end{equation*}
These elements satisfy
\begin{equation*}
  ~ [e,f] = 2 h, \quad
  [h, e] = e, \quad
  [h,f] = - f.
\end{equation*}
The Casimir element $\Omega$ is given (with the same normalization as in \S\ref{sec:d1a915ba8ec6}) by
\begin{equation*}
  \Omega =
  h^2 + \frac{e f + f e}{2}
  =
  h^2 - h + e f.
\end{equation*}
Writing $\Omega_\pi$ as before for the eigenvalue by which $\Omega$ acts on $\pi$, we can solve for the action of $f$ on $\pi$:
\begin{equation}\label{eqn:d1a915c0e3d2}
  f W(y)
  =
  \frac{1}{i y} \left(\Omega_\pi - (y \partial_y)^2 + y \partial_y \right) W(y).
\end{equation}
This formula is the key to verifying the fact recorded above, that any smooth compactly-supported function on $\mathbb{R}^\times$ arises from some (smooth!) vector in $\pi$.  We refer to \cite{MR2733072} and \cite[\S12]{2021arXiv210915230N} (direct link: \S\ref{sec:overview-asymp-kirillov}) for further discussion.

\subsection{Localized vectors}\label{sec:d1a9162d5b8b}
We just give one representative example of the sort of analysis that can be achieved in this way.

\begin{exercise}\label{exercise:d1a9162d527e}
  Let $\pi$ be a $T$-dependent generic irreducible unitary representation of $\PGL_2(\mathbb{R})$, realized in its Kirillov model with respect to the character \eqref{eqn:d1a915c27da9} as above and with unitary structure given by \eqref{eqn:d1a915c9af68}.  Assume that
  \begin{equation*}
    \Omega_\pi = \O(T^2).
  \end{equation*}
  In other words, in the notation of \S\ref{sec:d1a915c57e2f}, either
  \begin{enumerate}
  \item $\pi = \pi(t,\eps)$ with $t = \O(T)$, or
  \item $\pi = \pi(k)$ with $k = \O(T)$.
  \end{enumerate}

  Let $\alpha$ and $\rho$ be $T$-dependent real numbers with $\rho = \O(T)$ and $\alpha \asymp T$ (see \S\ref{sec:d1a8dda545b5} of \href{20230522T150333__microlocal-localized-vectors.pdf}{this note} for notation).  Let $M$ denote the class of all $T$-dependent elements $W \in \pi$ that are given in the Kirillov model for large enough $T$ by the formula
  \begin{equation}\label{eqn:d1aa012dc1aa}
    W(y) =
    T^{1/4}
    \lvert y \rvert^{i \rho }
    \phi \left( \frac{y - \alpha}{T^{1/2} } \right),
  \end{equation}
  where $\phi$ belongs to some fixed bounded subset of the space $C_c^\infty(\mathbb{R}^\times)$.  (We recall that a subset $\mathfrak{B}$ of this space is bounded if there are constants $C_n \geq 0$ and a compact set $E \subseteq \mathbb{R}^\times$ such that each $\phi \in \mathfrak{B}$ is supported in $E$ and has $n$th derivative bounded in $L^\infty$-norm by $C_n$.)
  \begin{enumerate}[(i)]
  \item Verify that $\lVert W \rVert = \O(1)$ for each $W \in M$.
  \item Verify that for each $W \in M$, we have
    \begin{equation*}
      e W - i \alpha  W \in T^{1/2} M,
    \end{equation*}
    \begin{equation*}
      h W
      - i \rho W
      \in T^{1/2} M.
    \end{equation*}
  \item Use \eqref{eqn:d1a915c0e3d2} to show that
    \begin{equation*}
      f W - i \beta W \in T^{1/2} M,
    \end{equation*}
    where
    \begin{equation*}
      \beta := \frac{\rho^2 - \Omega_\pi}{\alpha}.
    \end{equation*}    
  \item Deduce that every element of $M$ is localized at the $T$-dependent element of $\mathfrak{g}^\wedge$ given by
    \begin{equation*}
      \tau = i 
      \begin{pmatrix}
        \rho  & \beta \\
        \alpha & -\rho \\
      \end{pmatrix},
    \end{equation*}
    for which
    \begin{equation*}
      \det (\tau/i) = \Omega_\pi.
    \end{equation*}
    (Compare with \eqref{eqn:d1a91628a316}.)
  \item Define the Weyl group element
    \begin{equation*}
      w :=
      \begin{pmatrix}
        0 & 1 \\
        1 & 0 \\
      \end{pmatrix}
      \in G,
    \end{equation*}
    which satisfies
    \begin{equation*}
      \Ad(w) e = f, \qquad
      \Ad(w) h = -h,
      \qquad
      \Ad(w) f = e.
    \end{equation*}
    Show that for each $W \in M$, the $w$-translate $w W$ of $W$, given in the Kirillov model by
    \begin{equation}\label{eqn:d1aa0102b8bb}
      w W(y) := W(a(y) w),
    \end{equation}
    is localized at
    \begin{equation*}
      w \cdot \tau = i
      \begin{pmatrix}
        -\rho  & \alpha  \\
        \beta  & \rho \\
      \end{pmatrix}.
    \end{equation*}
  \item Show that fixed $k, \ell \in \mathbb{Z}_{\geq 0}$, we have
    \begin{equation*}
      \int_{y \in \mathbb{R}^\times }
      \left\lvert \frac{(y \partial_y)^k |y|^{i \rho} w W (y)}{T^{k/2}}  \right\rvert^2
      \left\lvert \frac{y - \beta }{ T^{1/2} } \right\rvert^{\ell} \,d^\times y \ll 1.
    \end{equation*}
    Informally, and at least when $\beta \asymp T$, this says that $w W(y)$ looks roughly like $T^{1/4} |y|^{-i \rho}$ times a bump function on $\beta + \O(T^{1/2})$.

    [Hint: consider the identity
    \begin{equation*}
      (e - i \beta)^k (h + i \rho)^\ell w W
      =
      w (f - i \beta)^k (-h + i \rho)^\ell W
    \end{equation*}
    and use that the norm defined by \eqref{eqn:d1a915c9af68} is $G$-invariant and that each element of $M$ has norm $\O(1)$.]
  \end{enumerate}
\end{exercise}

\begin{remark}
  Set $\Gamma := \PGL_{2}(\mathbb{Z}) < G$.  Suppose that the representation $\pi$ of $G$ considered in Exercise \ref{exercise:d1a9162d527e} comes with an embedding $\pi \hookrightarrow L^2_{\cusp}(\Gamma \backslash G)$ as a Hecke eigenspace, with Hecke eigenvalues $\lambda : \mathbb{N} \rightarrow \mathbb{C}$.  Then, as explained in \S\ref{sec:35ac3e587a} of \href{20230522T190340__pgl2-spectral-aspect-subconvexity_expository.pdf}{this note}, each $\varphi \in \pi$ (regarded as a function $\varphi : \Gamma \backslash G \rightarrow \mathbb{C}$) admits a Whittaker expansion
  \begin{equation}\label{eqn:d1aa0111162c}
    \varphi(g) = \sum_{n \neq 0} \frac{\lambda (\lvert n \rvert)}{ \lvert n \rvert^{1/2} } W (a (2 \pi n) g)
  \end{equation}
  where $W \in \mathcal{W}(\pi,\psi)$ denotes the image of $\varphi$ under the equivariant map defined by integrating over $(\Gamma \cap N) \backslash N$ against $\psi^{-1}$.  (The factor $2 \pi$ appears because we are using the unconventional choice \eqref{eqn:d1a915c27da9} for $\psi$.)  The left-invariance of $\varphi$ under $w$ implies a case of the Voronoi summation formula (see \href{https://people.math.osu.edu/cogdell.1/bessel-www.pdf}{this note of Cogdell} for a more general discussion): by applying the expansion \eqref{eqn:d1aa0111162c} to both sides of the identity $\varphi(1) = \varphi(w)$, and with the same abbreviations
  \begin{equation*}
    W(y) := W(a(y)),
    \qquad
    w W(y) := W(a(y) w)
  \end{equation*}
  as before, we obtain

\begin{equation*}
  \sum_{n \neq 0 } \frac{\lambda (\lvert n \rvert)}{ \lvert n \rvert^{1/2} }
  W(2 \pi n)
  =
  \sum_{n \neq 0 } \frac{\lambda (\lvert n \rvert)}{ \lvert n \rvert^{1/2} }
  w W(2 \pi n).
\end{equation*}
The values $W(y)$ and $w W(y)$ are related by an integral transform involving the Bessel function attached to $\pi$ (see \S2.2 of \href{https://people.math.osu.edu/cogdell.1/bessel-www.pdf}{Cogdell's note}, or \cite[Appendix A]{KMV02}).  When $W(y)$ is given by \eqref{eqn:d1aa012dc1aa}, one can derive asymptotics for $w W(y)$ using stationary phase analysis; derivations like this may be found in many analytic number theory papers.  One consequence of Exercise \ref{exercise:d1a9162d527e} is that in many cases, the shapes of $W(y)$ and $w W(y)$ may be related by ``pure thought''.
\end{remark}

\section{TODO}

\begin{enumerate}
\item Something for $\SO(3)$ using the Borel--Weil model over $\mathbb{C} \mathbb{P}^1$.
\item Something for induced models for $\PGL_2(\mathbb{R})$.
\item \sout{Some follow-up to exercise \ref{sec:d1a9162d5b8b} explaining what it says about $w W(y)$, for $w$ the nontrivial Weyl element.}
\end{enumerate}



\bibliography{refs}{} \bibliographystyle{plain}
\end{document}
