\documentclass[reqno]{amsart} \input{common.tex}

\begin{document}


Here's a lemma that I think clarifies the key step in the proof.
\begin{lemma}\label{lem:key-step-krull-dimn}
  Let $(A,\mathfrak{m})$ be a Noetherian local ring, and let $f_1,\dotsc,f_r \in \mathfrak{m}$ with $V(f_1,\dotsc,f_r) = \{\mathfrak{m}\}$.  Let $\mathfrak{p} \subsetneq \mathfrak{m}$ be a prime with no prime strictly contained between $\mathfrak{p}$ and $\mathfrak{m}$.  Then there exist $g_1,\dotsc,g_r \in \mathfrak{m}$ for which
  \begin{enumerate}
  \item $V(g_1,\dotsc,g_r) = \{\mathfrak{m}\}$ and
  \item $\mathfrak{p}$ contains and is a minimal prime of $(g_1,\dotsc,g_{r-1})$.
  \end{enumerate}

  In ``geometric'' terms, let $Z$ be a closed irreducible subset of $\Spec(A)$ that is minimal among the closed irreducible sets that properly contain $\{\mathfrak{m}\}$.  Then we may find $g_1,\dotsc,g_r$ for which $V(g_1,\dotsc,g_r) = \{\mathfrak{m}\}$ and for which $Z$ is an irreducible component of $V(g_1,\dotsc,g_{r-1})$.
\end{lemma}
\begin{proof}
  Since $\mathfrak{m}$ is the unique prime ideal containing $(f_1,\dotsc,f_r)$, we may assume after reindexing $f_1,\dotsc,f_r$ as necessary that $f_r \notin \mathfrak{p}$.  Then the ideal $\mathfrak{p} + (f_r)$ strictly contains $\mathfrak{p}$ and is contained in $\mathfrak{m}$; our hypotheses on $\mathfrak{p}$ imply that $\mathfrak{m}$ is the only prime ideal containing $\mathfrak{p} + (f_r)$, i.e., that $V(\mathfrak{p} + (f_r)) = \{\mathfrak{m}\}$, or that $\rad(\mathfrak{p} + (f_r)) = \mathfrak{m}$.  In particular, for each $1 \leq i \leq r-1$ we may find $n_i$ for which $f_i^{n_i} \in \mathfrak{p} + (f_r)$, say
  \[
    f_i^{n_i} = g_i + z_i f_r
    \text{ with } g_i \in \mathfrak{p}, z_i \in A.
  \]
  We claim that the conclusion of the lemma is now satisfied
  with $g_1,\dotsc,g_{r-1}$ as above and $g_r := f_r$:
  \begin{enumerate}
  \item The above equation
    shows that any prime $\mathfrak{q}$ that contains
    $g_1,\dotsc,g_{r-1},f_r$
    also contains $f_i^{n_i}$ and hence $f_i$ for $1 \leq i \leq r$,
    hence $\mathfrak{q} = \mathfrak{m}$.
    Thus $V(g_1,\dotsc,g_r) = \{\mathfrak{m}\}$.
  \item It's clear by construction that $\mathfrak{p}$ contains
    $(g_1,\dotsc,g_{r-1})$.  There is thus a minimal prime
    $\mathfrak{p}'$ of $(g_1,\dotsc,g_{r-1})$ contained in
    $\mathfrak{p}$; we must verify that
    $\mathfrak{p} = \mathfrak{p} '$.  (Geometrically,
    $\mathfrak{p}'$ corresponds to an irreducible component $Z'$
    of $V(g_1,\dotsc,g_{r-1})$ containing $Z$.)  To see this,
    consider the quotient ring $\overline{A} := A/(g_1,\dotsc,g_{r-1})$.
    Let
    \begin{equation}\label{eq:prime-chain-in-quotient-for-key-lemma-of-dimn-thm}
            \overline{\mathfrak{m}} \supsetneq
      \overline{\mathfrak{p}} \supseteq \overline{\mathfrak{p}'}
    \end{equation}
    denote the chain of primes in $\overline{A}$
    given by the image of
    $\mathfrak{m} \supsetneq \mathfrak{p} \supseteq
    \mathfrak{p}' \supseteq (g_1,\dotsc,g_{r-1})$.
    Then $(\overline{A},\overline{\mathfrak{m}})$ is a
    Noetherian local ring,
    and our task is equivalent to showing that
    $\overline{\mathfrak{p} } = \overline{\mathfrak{p} '}$.
    Let
    $f \in \overline{A}$ denote the image of $f_r$.  The primes
    of $\overline{A}$ containing $f$ are in bijection with the
    primes of $A$ containing $g_1,\dotsc,g_{r-1},f_r$, so
    $V_{\overline{A}}(f)
    = \{\overline{\mathfrak{m}}\}$.    
    By Krull's principal
    ideal theorem (in the form of Corollary \ref{cor:krull-pith-local}),
    it follows that $\height(\overline{\mathfrak{m} }) \leq 1$.
    From
    \eqref{eq:prime-chain-in-quotient-for-key-lemma-of-dimn-thm}
    we then deduce that $\overline{\mathfrak{p} } =
    \overline{\mathfrak{p}'}$,
    as required.
    (Intuitively, by choosing $f_r$ not to vanish on any irreducible component
    of $V(f_1,\dotsc,f_{r-1})$,
    we guarantee that appending it to the set of generators has the effect
    of knocking down the dimension of each such component by $1$.)
  \end{enumerate}
\end{proof}

We now deduce Theorem \ref{thm:krull-dimn-thm}.  We must show that if \(\mathfrak{p}\) is a minimal prime of \((f_1,\dotsc,f_r)\), then \(\height(\mathfrak{p}) \leq r\).  We may assume without loss of generality (replacing \(A\) with \(A_\mathfrak{p}\) and \(\mathfrak{p}\) with \(\mathfrak{p}_\mathfrak{p}\), which doesn't change the height of or minimality assumption on the latter) that \((A,\mathfrak{p})\) is a Noetherian local ring with \(V(f_1,\dotsc,f_r) = \{\mathfrak{p}\}\); we must show then that \(\height(\mathfrak{p}) \leq r\).  We do this by induction on \(r\).  The case \(r = 1\) is given by Krull's principal ideal theorem, so suppose \(r > 1\).  Let \(\mathfrak{q} \subsetneq \mathfrak{p}\) be a maximal element of the set of primes strictly contained in \(\mathfrak{p}\); it will suffice then to show that \(\height(\mathfrak{q}) \leq r-1\).  By Lemma \ref{lem:key-step-krull-dimn}, we may assume without loss of generality that \(\mathfrak{q}\) is a minimal prime of \((f_1,\dotsc,f_{r-1})\); the required inequality then follows from our inductive hypothesis.

\bibliography{refs}{} \bibliographystyle{plain}
\end{document}
