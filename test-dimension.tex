\documentclass[reqno]{amsart} \input{common.tex}

\begin{document}

\section{Introduction}
\label{sec:org2573665}

We recall some definitions and background, record proofs of some
of the main theorems regarding Krull dimension, and give some of
their geometric interpretations.
We mainly follow the course reference by Bosch.

\section{Basic definitions}
\label{sec:orgdb2d3a0}

Let \(A\) be a ring (always commutative and with identity).
In what follows,
the symbols \(\mathfrak{p}\) or \(\mathfrak{p}_i\)
always denote prime ideals.
We set
\[
    \dim(A) :=
    \sup \{n \geq 0 : \exists \mathfrak{p}_0 \subsetneq \dotsb \subsetneq \mathfrak{p}_n \}.
  \]

For a prime ideal \(\mathfrak{p}\) of \(A\),
we set
\[
    \height(\mathfrak{p}) :=
    \sup \{n \geq 0 : \exists \mathfrak{p}_0 \subsetneq \dotsb \subsetneq \mathfrak{p}_n \subseteq \mathfrak{p}\},
  \]
\[
    \coheight(\mathfrak{p}) :=
    \sup \{n \geq 0 : \exists \mathfrak{p} \subseteq \mathfrak{p}_0 \subsetneq \dotsb \subsetneq \mathfrak{p}_n \}.
  \]
For a general ideal \(\mathfrak{a}\),
we set
\[
    \height(\mathfrak{a}) := \inf_{\mathfrak{p} \supseteq
      \mathfrak{a}} \height(\mathfrak{p}),
  \]
\[
    \coheight(\mathfrak{a}) := \sup_{\mathfrak{p} \supseteq
      \mathfrak{a}} \coheight(\mathfrak{p})
    = \sup \{n \geq 0 : \exists \mathfrak{a} \subseteq \mathfrak{p}_0 \subsetneq \dotsb \subsetneq \mathfrak{p}_n \}.
  \]
Since prime ideals in the localization \(A_\mathfrak{p}\)
correspond to the primes in \(A\) contained in \(\mathfrak{p}\),
we have
\[
    \height(\mathfrak{p}) = \dim(A_\mathfrak{p}).
  \]
Since prime ideals in the quotient \(A/\mathfrak{a}\)
correspond to the primes in \(A\) containing \(\mathfrak{a}\),
we have
\[
    \coheight(\mathfrak{a}) = \dim(A/\mathfrak{a}).
  \]
We note the following easy inequality:
\begin{lemma}\label{lem:easy-dimension-inequality}
  $\height(\mathfrak{a}) + \dim(A/\mathfrak{a}) \leq \dim(A)$.
\end{lemma}
\begin{proof}
  It suffices to show that
  if $\height(\mathfrak{a}) \geq r$
  and $\dim(A/\mathfrak{a}) \geq s$,
  then $\dim(A) \geq r + s$.
  By hypothesis,
  we may find primes
  $\mathfrak{a} \subseteq \mathfrak{q}_0 \subsetneq \dotsb
  \subsetneq \mathfrak{q}_s$.
  Then $\height(\mathfrak{q}_0) \geq \height(\mathfrak{a}) \geq
  r$,
  so we may find primes
  $\mathfrak{p}_0 \subsetneq \dotsb \subsetneq \mathfrak{p}_r =
  \mathfrak{q}_0$.
  Then
  \[
    \mathfrak{p}_0 \subsetneq \dotsb \subsetneq \mathfrak{p}_r =
    \mathfrak{q}_0
    \subsetneq \mathfrak{q}_1 \subsetneq \dotsb \subsetneq
    \mathfrak{q}_s
  \]
  is a chain of primes in  $A$ of length $r+s$.
\end{proof}
We also note:
\begin{lemma}\label{lem:local-ring-height-equals-dim}
  Let $(A,\mathfrak{m})$ be a local ring.
  Then $\dim(A) = \height(\mathfrak{m})$.
\end{lemma}
\begin{proof}
  Let $\mathfrak{p}_0 \subsetneq \dotsb \subsetneq
  \mathfrak{p}_r$ be a chain of primes in $A$.
  By enlarging this chain if necessary, we may assume that
  $\mathfrak{p}_r = \mathfrak{m}$.
  Thus the suprema in the definitions of
  $\dim(A)$ and $\height(\mathfrak{m})$
  may be taken over the same chains of primes.
\end{proof}


\section{Geometric interpretations}
\label{sec:org60bbe56}

Reference for this section: exercises in Chapter 1 of Atiyah--Macdonald.

Let \(A\) be a ring.  Recall that \(\Spec(A)\) denotes the set of
prime ideals \(\mathfrak{p}\) in \(A\).
Each \(f \in A\) defines a function
\[
  f|_{\Spec(A)} : \Spec(A) \rightarrow
  \bigsqcup_{\mathfrak{p} \in \Spec(A)} A/\mathfrak{p}
  \]
sending \(\mathfrak{p}\) to the class of \(f\) in the quotient ring \(A/\mathfrak{p}\).
For \(f \in A\) and any subset \(X\) of \(\Spec(A)\),
we may form the restriction
\(f|_X\) of \(f\) to \(X\).
For the sake of illustration,
note that \(f|_{\Spec(A)} = 0\) (i.e., \(f|_{\Spec(A)}\)
maps each \(\mathfrak{p}\) to the zero class in \(A / \mathfrak{p}\))
if and only if \(f\) belongs
to the nilradical of \(A\).

For example, we have seen (using the Nullstellensatz) that if
\(A = \mathbb{C}[X_1,\dotsc,X_n]/I\) for some ideal
\(I \subseteq \mathbb{C}[X_1,\dotsc,X_n]\), then the set
\(\Specm(A)\) of maximal ideals in \(A\) is in natural bijection
with
\(V := \{(x_1,\dotsc,x_n) \in \mathbb{C}^n : f(x_1,\dotsc,x_n) = 0
  \text{ for all } f \in I \}\).
For each such maximal ideal \(\mathfrak{m}\) we may identify
\(A/\mathfrak{m}\) with \(\mathbb{C}\).
For \(f \in A\),
the function
\(f|_{\Specm(A)}\) then identifies with the obvious map
\(V \ni (x_1,\dotsc,x_n) \mapsto f(x_1,\dotsc,x_n) \in
  \mathbb{C}\).

For a subset \(S\) of \(A\), we set
\[
  V(S) := \{\mathfrak{p} \in \Spec(A) : \mathfrak{p} \supseteq
  S\}
  = \{\mathfrak{p} \in \Spec(A) : f(\mathfrak{p}) = 0 \text{ for
    each }
  f \in S
  \}.
  \]
For finite sets \(S = \{f_1,\dotsc,f_n\}\)
we write simply \(V(f_1,\dotsc,f_n) := V(S)\).
Note that if \(S\) generates an ideal \(\mathfrak{a}\), then
\(V(S) = V(\mathfrak{a})\).
Given any subset \(X\) of \(\Spec(A)\),
we set
\[
  I(X)
  := \cap_{\mathfrak{p} \in X} \mathfrak{p}
  = \{f \in A : f|_X = 0\}.
  \]
Recall that a subset of \(\Spec(A)\) is
called \emph{closed} if it is of the form \(V(S)\) for some \(S\);
this defines
a topology on \(\Spec(A)\).
Recall that an ideal \(\mathfrak{a}\) is \emph{radical}
if \(\rad(\mathfrak{a}) = \mathfrak{a}\).
\begin{lemma}~
  \begin{enumerate}[(i)]
  \item For each ideal $\mathfrak{a}$ of $A$,
    we have $I(V(\mathfrak{a})) = \rad(\mathfrak{a})$.
  \item For each subset $X$ of $\Spec(A)$,
    we have $V(I(X)) = \overline{X}$ (the closure of $X$).
  \item The maps $V$ and $I$ define mutually-inverse
    inclusion-reversing
    bijections between the set of radical ideals of $A$ and the
    set of closed subsets of $\Spec(A)$.
  \end{enumerate}
\end{lemma}
\begin{proof}
  The maps $I$ and $V$ are readily seen to be inclusion-reversing
  (cf. Exercise Sheet \#1).
  \begin{enumerate}[(i)]
  \item By definition, $I(V(\mathfrak{a})) = \cap_{\mathfrak{p} \in
      V(\mathfrak{a})} \mathfrak{p}
    = \cap_{\mathfrak{p} \supseteq \mathfrak{a}} \mathfrak{p} =
    \rad(\mathfrak{a})$.
  \item The set $V(I(X))$ is closed and contains $X$, so it will suffice to
    verify for each closed set $V(\mathfrak{a})$ containing $X$
    that $V(\mathfrak{a}) \supseteq V(I(X))$.  From
    $V(\mathfrak{a}) \supseteq X$ we see that $f|_X = 0$ for all
    $f \in \mathfrak{a}$, thus $\mathfrak{a} \subseteq I(X)$.
    Applying the inclusion-reversing map $V$, we obtain
    $V(\mathfrak{a}) \supseteq V(I(X))$, as required.
  \item Immediate by the above.
  \end{enumerate}
\end{proof}
\begin{lemma}
  Let $X$ be a closed subset of $\Spec(A)$.
  The following are equivalent:
  \begin{enumerate}[(i)]
  \item 
    $X = V(\mathfrak{p})$ for some prime ideal
    $\mathfrak{p}$ of $A$.
  \item $I(X)$ is a prime ideal of $A$.
  \item $X$ is nonempty
    and may not be written as $X = X_1 \cup X_2$ for closed subsets
    $X_1, X_2$ of $\Spec(A)$ except in the trivial case that either
    $X \subseteq X_1$ or $X \subseteq X_2$.
  \end{enumerate}
\end{lemma}
We say that a closed
subset \(X\) of \(\Spec(A)\)
is \emph{irreducible}
if it satisfies the equivalent conditions of the preceeding
lemma.
The irreducible closed subsets of \(\Spec(A)\)
correspond bijectively to the prime ideals of \(A\).

We note that for any ideal \(\mathfrak{a}\),
we may identify
\[
  V(\mathfrak{a}) = \Spec(A/\mathfrak{a}).
  \]
We note also that if \(\mathfrak{p}\) is a prime of \(A\),
then the primes of the localization \(A_\mathfrak{p}\)
correspond to the primes of \(A\) contained in \(\mathfrak{p}\),
hence the spectrum of \(A_\mathfrak{p}\)
identifies with the set of closed irreducible subsets
of \(\Spec(A)\) that \emph{contain} \(\mathfrak{p}\):
\[
  \Spec(A_\mathfrak{p})
  =
  \{\mathfrak{q} \in \Spec(A) : \mathfrak{q} \subseteq \mathfrak{p} \}
  =
  \{\mathfrak{q} \in \Spec(A) : \mathfrak{p} \in V(\mathfrak{q}) \}.
  \]

By an \emph{irreducible component} of a closed subset \(X\) of
\(\Spec(A)\), we shall mean a maximal closed irreducible subset of
\(X\), i.e., a closed irreducible subset \(Z \subseteq X\) with the
property that if \(Z' \subseteq X\) is any closed irreducible
subset with \(Z' \supseteq Z\), then \(Z' = Z\).  Using the inclusion-reversing bijections
noted above, we verify readily
that for any ideal \(\mathfrak{a}\), the irreducible components of
\(X = V(\mathfrak{a})\) correspond bijectively to the set (denoted
\(\Ass'(\mathfrak{a})\) in lecture) of minimal prime ideals
\(\mathfrak{p} \supseteq \mathfrak{a}\).  

We assume henceforth that \(A\) is Noetherian.
Then the set of minimal primes of any ideal is finite, and any prime containing an ideal contains a minimal prime of that ideal.
It follows that the set
of irreducible components of any closed subset \(X\) of \(\Spec(A)\) is
a finite set \(\{Z_1,\dotsc,Z_n\}\) for which \(X = Z_1 \cup \dotsb \cup Z_n\).

We define the \emph{dimension}
of a closed subset \(X\) of \(\Spec(A)\)
to be
\[
  \dim(X) = \sup \{n \geq 0 : \exists \text{ closed irreducible
    subsets }
  Z_n \subsetneq \dotsb \subsetneq Z_0 \subseteq X
  \}
  \]
and the \emph{codimension}
in the special case
that \(Z\) is closed irreducible
to be
\[
  \codim(Z) :=
  \sup \{n \geq 0 : \exists \text{ closed irreducible
    subsets }
  Z_0 \supsetneq \dotsb \supsetneq Z_n \supset Z \}
  \]
and then in general by
\[
  \codim(X) :=
  \inf_{\substack{Z \subseteq X : \text{closed irreducible}}}
  \codim(Z).
  \]
Equivalently, \(\codim(X)\) is the smallest codimension
of any irreducible component of \(X\).
We note also that \(\dim(X)\) coincides with the largest
dimension of any irreducible component of \(X\).
We might write \(\codim(X)\) as \(\codim_{\Spec(A)}(X)\)
when we wish to emphasize the reference space \(\Spec(A)\).

Using the inclusion-reversing bijections noted above,
we see that
\[
  \dim \Spec A = \dim A
  \]
and more generally
that
\[
  \dim V(\mathfrak{a}) = \coheight \mathfrak{a} = \dim A/\mathfrak{a},
  \quad
  \dim X = \coheight I(X) = \dim A/I(X),
  \]
\[
  \codim V(\mathfrak{a}) = \height \mathfrak{a},
  \quad
  \codim X = \height I(X)
  \]
for any ideal \(\mathfrak{a}\) and any closed \(X \subseteq
  \Spec(A)\).
Lemma \ref{lem:easy-dimension-inequality}
says that \(\dim X + \codim X \leq \dim \Spec A\).


\section{Prime avoidance lemma\label{sec:prime-avoidance}}
\label{sec:org9d9ed7c}

\begin{lemma}
  Let $A$ be a ring, let $\mathfrak{p}_1,\dotsc,\mathfrak{p}_n$
  be prime ideals, and let $\mathfrak{a}$ be an ideal contained
  in the union $\cup \mathfrak{p}_j$.  Then there exists an index $j$
  for which $\mathfrak{a} \subseteq \mathfrak{p}_j$.
  Equivalently, if $\mathfrak{a} \not\subseteq \mathfrak{p}_j$
  for each $j$, then
  $\mathfrak{a} \not\subseteq \cup \mathfrak{p}_j$.

  In ``geometric'' terms,
  let $Z_1,\dotsc,Z_n \subseteq \Spec(A)$
  be closed irreducible subsets,
  and
  let $X = V(\mathfrak{a})$
  be a closed irreducible subset
  of $\Spec(A)$,
  defined by an ideal $\mathfrak{a}$,
  with the property that $X \not\supseteq Z_j$ for all $j$.
  Then there exists $f \in \mathfrak{a}$
  with $f|_{Z_j} \neq 0$ for all $j$.
  In particular, we may find $f \in A$ with $f|_{X} = 0$
  but $f|_{Z_j} \neq 0$ for all $j$.
\end{lemma}
\begin{proof}
  We verify 
  that if $\mathfrak{a}$ is not contained in any of the
  $\mathfrak{p}_j$,
  then it is not contained in their union.
  For this we may induct on $n$.
  The case $n = 1$ is trivial,
  so suppose $n > 2$.
  By our inductive hypothesis,
  we may find for each $i=1..n$ an element $a_i \in
  \mathfrak{a}$
  with $a_i \notin \mathfrak{p}_j$
  whenever $j \neq i$.
  If moreover $a_i \notin \mathfrak{p}_i$
  for some $i$,
  then we are done,
  so suppose otherwise that $a_i \in \mathfrak{p}_i$ for all
  $i$.
  Set $b_i := \prod_{j : j \neq i} a_j$.
  Then $b_i \notin \mathfrak{p}_i$ (using that $\mathfrak{p}_i$
  is prime)
  but $b_i \in \mathfrak{p}_j$ for all $j \neq i$.
  It follows that $x := b_1 + \dotsb + b_n$
  belongs to $\mathfrak{a}$
  but not to $\mathfrak{p}_i$ for any $i$,
  hence $\mathfrak{a}$ is not contained in the union of the $\mathfrak{p}_i$.
\end{proof}

\section{Artin rings}
\label{sec:org49d017a}

\begin{theorem}\label{thm:artin-vs-noeth-dim0}
  Let $A$ be a ring.
  The following are equivalent:
  \begin{enumerate}[(i)]
  \item $A$ is an Artin ring.
  \item $A$ is a Noetherian ring of dimension zero.
  \end{enumerate}
\end{theorem}


\section{Krull intersection theorem}
\label{sec:orgd90b421}

\begin{theorem}\label{thm:krull-intersection}
  Let $\mathfrak{a}$ be an ideal contained in the Jacobson
  radical $\Jac(A)$ of a Noetherian ring
  $A$.
  Then
  \[
    \cap_{n \geq 0} \mathfrak{a}^n = 0.
  \]
\end{theorem}
\begin{corollary}\label{cor:krull-intersection1}
  With $A, \mathfrak{a}$ as before, let $M$ be a finitely-generated module.
  Then $\cap_{n \geq 0} \mathfrak{a}^n M = 0$.
\end{corollary}
\begin{corollary}\label{cor:krull-intersection2}
  Let $(A,\mathfrak{m})$ be a  Noetherian local ring.
  Then $\cap_{n \geq 0} \mathfrak{m}^n =0$.
\end{corollary}
For the proof of Theorem \ref{thm:krull-intersection}, the fact that \(\mathfrak{a}\) is contained in the Jacobson
radical
suggests an application of Nakayama's lemma to the ideal
\(M' := \cap_{n \geq 0} \mathfrak{a}^n\),
for which it
is clear
that \(\mathfrak{a} M' \subseteq M'\)
and
plausible but non-obvious that \(\mathfrak{a} M' = M'\).
The key tool in establishing  the latter is the following:
\begin{lemma}[Artin--Rees lemma]
  Let $A$ be Noetherian,
  let $\mathfrak{a}$ be an ideal,
  let $M$ be a finitely-generated module,
  and let $M' \leq M$ be a submodule.
  There exists $n \geq 0$ so that for all $k \geq 0$,
  \[
    \mathfrak{a}^k (\mathfrak{a}^n M \cap M')
    = \mathfrak{a}^{n+k} M \cap M'.
  \]
\end{lemma}
Taking
\(M := A, M' := \cap_{n \geq 0} \mathfrak{a}^n\),
\(k := 1\)
in the Artin--Rees lemma
gives \(\mathfrak{a}^n M \cap M' = \mathfrak{a}^{n+k} M \cap M' =
  M'\)
and hence \(\mathfrak{a} M' = M'\);
we then conclude  the proof of Theorem
\ref{thm:krull-intersection}
by Nakayama, as indicated above.

The proof of Artin--Rees reduces formally
to the case \(k = 1\),
and the containment
\[
    \mathfrak{a} (\mathfrak{a}^n M \cap M')
    \subseteq \mathfrak{a}^{n+1} M \cap M'
  \]
is clear.
The proof of the trickier reverse containment
is expressed
most transparently using the graded ring
\[\tilde{A}
    := \boxplus_{i \geq 0} A_i = \{a = (a_i)_{i \geq 0} : a_i \in
    A_i\}, \quad A_i := \mathfrak{a}^i,
  \]
where the multiplication law extends
the bilinear maps \(\mathfrak{a}^i \times \mathfrak{a}^j \rightarrow \mathfrak{a}^{i+j}\):
\[
    (a \cdot b)_k = \sum_{i+j=k} a_i b_j.
  \]
This graded ring acts by the rule
\((a \cdot m)_k := \sum_{i+j=k} a_i m_j\)
on the graded module
\[\tilde{M}
    := \boxplus_{i \geq 0} M_i,
    \quad M_i := \mathfrak{a}^i M,
  \]
and its graded submodule
\[\tilde{M'}
    := \boxplus_{i \geq 0} M'_i,
    \quad M'_i := \mathfrak{a}^i M \cap M'.
  \]
Since \(\mathfrak{a}\) is finitely-generated as a module over \(A\),
\(\tilde{A}\) is finitely-generated as an algebra over \(A_0 = A\);
by the Hilbert basis theorem,
it follows that \(\tilde{A}\) is Noetherian.
The module \(M\) is finitely-generated over \(A\),
from which it follows readily that the graded module \(\tilde{M}\)
is finitely-generated over \(\tilde{A}\);
since the ring \(\tilde{A}\) is Noetherian,
so is the module \(\tilde{M}\),
hence its submodule \(\tilde{M'}\) is finitely-generated.
Choose \(n\) large enough that
the module \(\tilde{M'}\) is generated by \(\boxplus_{0 \leq i \leq
    n} M'_i\),
thus
\[
    \tilde{M'}
    =
    \tilde{A} \boxplus_{0 \leq i \leq
      n} M'_i.
  \]
By taking the degree \(n+1\) homogeneous  component of this identity,
we see that
\begin{align*}
  \mathfrak{a}^{n+1} M \cap M'
  &=
  \tilde{M'_{n+1}}
  =
  \sum_{0 \leq i \leq n}
  A_{n+1-i} M'_i
  =
  \sum_{0 \leq i \leq n}
  \mathfrak{a}^{n+1-i} (\mathfrak{a}^i M \cap M')
  \\
  &\subseteq
    \sum_{0 \leq i \leq n}
    \mathfrak{a} (\mathfrak{a}^n M \cap \mathfrak{a}^{n-i} M')
    \subseteq \mathfrak{a} (\mathfrak{a}^n M \cap M'),
\end{align*}
giving the required reverse containment.
The proof of Artin--Rees and hence of the Krull intersection
theorem is then complete.


\section{Kernel of localization with respect to a prime\label{sec:kernel-localize}}
\label{sec:org762c52a}

Let \(\mathfrak{p}\) be a prime ideal in a Noetherian ring \(A\).
Let \(\mathfrak{p}^{(n)}\) denote the $n$th symbolic power;
it is the $\mathfrak{p}$-primary ideal
given by \(A \cap \mathfrak{p}^n A_\mathfrak{p} := \iota^*(
  (\iota_* \mathfrak{p} )^n)\),
where \(\iota : A \rightarrow A_\mathfrak{p}\) denotes the
localization map.
\begin{theorem}
  $\ker(\iota) = \cap_{n \geq 0} \mathfrak{p}^{(n)}$.
\end{theorem}
\begin{proof}
  Set $\mathfrak{m} := \iota_* \mathfrak{p}$.  We have
  $\ker(\iota) = \iota^{(-1)}(0)$ and
  $\iota^{-1}(\cap_{n \geq 0} \mathfrak{m}^n) = \cap_{n \geq 0}
  \mathfrak{p}^{(n)}$, so it suffices to show that
  $\cap_{n \geq 0} \mathfrak{m}^n = 0$, which is the content of
  Corollary \ref{cor:krull-intersection2} of the Krull
  intersection theorem applied to the Noetherian local ring
  $(A_\mathfrak{p},\mathfrak{m})$.
\end{proof}


\section{Krull's theorems on heights and dimensions}
\label{sec:orgd3770e5}
\subsection{Principal ideal theorem}
\label{sec:orgb8870ab}

We start with the special case to which the general one will eventually be reduced:
\begin{lemma}
  Let $(A,\mathfrak{m})$ be a local Noetherian integral domain.
  Suppose that $\mathfrak{m}$ is a minimal prime of
  some principal ideal $(f)$, with $f \in \mathfrak{m}$.
  Then $\mathfrak{m}$ and $(0)$ are the only primes
  of $A$.

  In ``geometric'' terms: suppose
  that $\{\mathfrak{m}\} = V(f)$ for some $f \in \mathfrak{m}$.
  Then $\Spec(A) = \{\mathfrak{m}, (0)\}$.
\end{lemma}
\begin{proof}
  Let $\mathfrak{p}$ be any prime in $A$ other than
  $\mathfrak{m}$.  Necessarily
  $\mathfrak{p} \subsetneq \mathfrak{m}$; our task is to show
  that $\mathfrak{p} = (0)$.
  Since $A$ is a domain, it will suffice to show
  for some $n$ that $\mathfrak{p}^n = (0)$.
  Recall that $\mathfrak{p}^{(n)}$
  denotes the $n$th symbolic power of $\mathfrak{p}$,
  given here with respect
  to the injective localization map
  $A \hookrightarrow  A_\mathfrak{p}$
  by $\mathfrak{p}^{(n)} = A \cap \mathfrak{p}^n
  A_\mathfrak{p}$;
  it is a $\mathfrak{p}$-primary ideal which contains
  $\mathfrak{p}^n$.
  It will then suffice to verify that
  $\mathfrak{p}^{(n)} = (0)$ for some $n$.
  By \S\ref{sec:kernel-localize},
  we have $\cap_{n \geq 0} \mathfrak{p}^{(n)} = \ker(A
  \rightarrow A_\mathfrak{p}) = (0)$,
  so it will suffice to verify that
  the chain of ideals $\mathfrak{p}^{(n)}$
  stabilizes,
  i.e.,
  that
  $\mathfrak{p}^{(n)} = \mathfrak{p}^{(n+1)}$ for large $n$.


  Set $\overline{A} := A/(f)$,
  $\overline{\mathfrak{m} } := \mathfrak{m}/(f)$.  Our
  hypotheses imply that $\overline{\mathfrak{m}}$ is the only
  prime ideal of $\overline{A}$.  Thus $\overline{A}$ is a Noetherian ring of dimension $0$.
  By Theorem \ref{thm:artin-vs-noeth-dim0}, it follows
  that $\overline{A}$ is an Artin
  ring.
  Thus the descending chain of ideals
  $\mathfrak{p}^{(n)} + (f)$
  must stabilize;
  in particular,
  \[
  \mathfrak{p}^{(n)} \subseteq \mathfrak{p}^{(n+1)} + (f)
  \]
  for
  large $n$.
  This says that any $x \in \mathfrak{p}^{(n)}$
  may be written
  $x = y + z f$
  for some $y \in \mathfrak{p}^{(n+1)}$ and $z \in A$.
  In that case, $x-y \in \mathfrak{p}^{(n)}$,
  and so
  $z \in (\mathfrak{p}^{(n)} : f)$.
  Since $\mathfrak{p}^{(n)}$ is $\mathfrak{p}$-primary
  and $f \notin \mathfrak{p}$,
  we have $(\mathfrak{p}^{(n)} : f) = \mathfrak{p}^{(n)}$,
  and so in fact $z \in \mathfrak{p}^{(n)}$.
  Thus
  \[
  \mathfrak{p}^{(n)} \subseteq \mathfrak{p}^{(n+1)} + \mathfrak{p}^{(n)} f,
  \]
  and in fact equality holds,
  with the reverse containment being clear.
  This says that $f M = M$
  for the finitely-generated module
  $M := \mathfrak{p}^{(n)} / \mathfrak{p}^{(n+1)}$.
  Since $f \in \mathfrak{m} = \Jac(A)$,
  it follows from Nakayama's lemma that $M = 0$.
  Thus $\mathfrak{p}^{(n)} = \mathfrak{p}^{(n+1)}$ for large
  $n$,
  as was to be shown.
\end{proof}

\begin{theorem}\label{thm:krull-principal-ideal}
  Let $A$ be a Noetherian ring, and let $f \in A$.  
  \begin{enumerate}[(i)]
  \item Every
    minimal prime $\mathfrak{p}$ of $(f)$ satisfies
    $\height(\mathfrak{p}) \leq 1$.
  \item If $f$ is a non-zerodivisor,
    then every minimal prime $\mathfrak{p}$ of $(f)$
    satisfies $\height(\mathfrak{p}) = 1$.
  \end{enumerate}

  In ``geometric'' terms,
  $\codim(Z) \leq 1$ for each irreducible component $Z$
  of $V(f) \subseteq \Spec(A)$;
  if $f$ is a non-zerodivisor,
  then $\codim(Z) = 1$ for each such $Z$.
  (This ``generalizes''
  the fact from linear algebra
  that the kernel of a linear functional has codimension $\leq 1$,
  with equality whenever the functional is nonzero.)
\end{theorem}
\begin{proof}
  To deduce (ii) from (i), suppose that some minimal prime $\mathfrak{p}$
  of $(f)$ has $\height(\mathfrak{p}) = 0$.  Then $\mathfrak{p}$
  is a minimal prime of $(0)$, hence consists of zero-divisors,
  and so $f$ is a zerodivisor.

  Our main task is thus to establish (i).  We must verify that
  if $\mathfrak{p}_2$ is a minimal prime of $(f)$ and if
  $\mathfrak{p}_0 \subseteq \mathfrak{p}_1 \subsetneq
  \mathfrak{p}_2$
  are inclusions of prime ideals, then
  $\mathfrak{p}_0 = \mathfrak{p}_1$.  After replacing $A$ by its
  quotient $A/\mathfrak{p}_0$, we may reduce to the case that
  $\mathfrak{p}_0 = (0)$; in particular, $A$ is a local
  Noetherian domain.  After then replacing $A$ by its
  localization $A_{\mathfrak{p}_2}$, we reduce further to the
  case that $A$ is a local Noetherian domain whose maximal ideal
  $\mathfrak{p}_2$ is a minimal prime of $(f)$.
  We now appeal to the previous lemma.
\end{proof}

We will often apply the above result in a local
context:
\begin{corollary}\label{cor:krull-pith-local}
  Let $(A,\mathfrak{m})$ be a Noetherian local ring.
  Suppose there exists $f \in A$ for which
  $\mathfrak{m}$ is the unique prime containing $f$,
  thus
  $V(f) = \{\mathfrak{m} \}$.
  Then
  $\dim(A) = \height(\mathfrak{m}) \leq 1$.
\end{corollary}
\begin{proof}
  Given that $\mathfrak{m}$ is maximal,
  our assumption is equivalent
  to requiring that $\mathfrak{m}$ be a minimal
  prime of $(f)$.
\end{proof}

For the sake of illustration,
let's reformulate Theorem \ref{thm:krull-principal-ideal} in the contrapositive.
Let \(A\) be a Noetherian ring.
Let \(\mathfrak{p}_0 \subsetneq \mathfrak{p}_2\)
be an inclusion of primes in \(A\).
By an \emph{intermediary prime}
we will mean a prime \(\mathfrak{p}_1\)
for which
\(\mathfrak{p}_0 \subsetneq \mathfrak{p}_1 \subsetneq
  \mathfrak{p}_2\).
\begin{corollary}
  \label{cor:}
  The following are equivalent:
  \begin{enumerate}[(i)]
  \item There exists an intermediary prime.
  \item For each $f \in \mathfrak{p}_2$ there exists an
    intermediary
    prime containing $f$.
  \end{enumerate}

  In ``geometric'' terms,
  let $Y_2 \subsetneq Y_0$ be irreducible closed subsets
  of $\Spec(A)$.
  Then either there are no irreducible closed subsets
  $Y_1$ contained strictly between $Y_2$ and $Y_0$,
  or for each $f \in I(Y_2)$
  there exists an irreducible closed
  subset $Y_2 \subsetneq Y_1 \subsetneq Y_0$
  with $Y_1 \subseteq Z(f)$.
\end{corollary}
\begin{proof}
  We need only show that (i) implies (ii).
  If (ii) fails,
  then we may find $f \in \mathfrak{p}_2$
  not contained in any intermediary primes.
  In other words, after replacing $A$ with $A/\mathfrak{p}_0$ as
  necessary
  to reduce to the case that $\mathfrak{p}_0$ is a minimal prime
  of $A$,
  we are given that $\mathfrak{p}_2$ is a minimal prime of
  $(f)$.
  By Krull's principal ideal theorem,
  it follows that
  $\height(\mathfrak{p}_2) \leq 1$;
  thus there exist no intermediary primes,
  and so (i) fails.
\end{proof}

\subsection{Dimension theorem}
\label{sec:orgbe18c9e}

\begin{theorem}\label{thm:krull-dimn-thm}
  Let $A$ be a Noetherian ring,
  and let $f_1,\dotsc,f_n \in A$.
  Then each minimal prime $\mathfrak{p}$ of $(f_1,\dotsc,f_r)$
  satisfies $\height(\mathfrak{p}) \leq r$.
  In particular,
  $\height(f_1,\dotsc,f_r) \leq r$.

  In ``geometric'' terms,
  $\codim(Z) \leq r$
  for each irreducible component $Z$ of $V(f_1,\dotsc,f_r)
  \subseteq \Spec(A)$.
  (This ``generalizes'' the fact from linear algebra
  that the solution set to a system of $r$ linear equations
  has codimension $\leq r$.)
\end{theorem}

Here's a lemma that I think clarifies the key step in the proof.
\begin{lemma}\label{lem:key-step-krull-dimn}
  Let $(A,\mathfrak{m})$ be a Noetherian local ring, and let $f_1,\dotsc,f_r \in \mathfrak{m}$ with $V(f_1,\dotsc,f_r) = \{\mathfrak{m}\}$.  Let $\mathfrak{p} \subsetneq \mathfrak{m}$ be a prime with no prime strictly contained between $\mathfrak{p}$ and $\mathfrak{m}$.  Then there exist $g_1,\dotsc,g_r \in \mathfrak{m}$ for which
  \begin{enumerate}
  \item $V(g_1,\dotsc,g_r) = \{\mathfrak{m}\}$ and
  \item $\mathfrak{p}$ contains and is a minimal prime of $(g_1,\dotsc,g_{r-1})$.
  \end{enumerate}

  In ``geometric'' terms, let $Z$ be a closed irreducible subset of $\Spec(A)$ that is minimal among the closed irreducible sets that properly contain $\{\mathfrak{m}\}$.  Then we may find $g_1,\dotsc,g_r$ for which $V(g_1,\dotsc,g_r) = \{\mathfrak{m}\}$ and for which $Z$ is an irreducible component of $V(g_1,\dotsc,g_{r-1})$.
\end{lemma}
\begin{proof}
  Since $\mathfrak{m}$ is the unique prime ideal containing $(f_1,\dotsc,f_r)$, we may assume after reindexing $f_1,\dotsc,f_r$ as necessary that $f_r \notin \mathfrak{p}$.  Then the ideal $\mathfrak{p} + (f_r)$ strictly contains $\mathfrak{p}$ and is contained in $\mathfrak{m}$; our hypotheses on $\mathfrak{p}$ imply that $\mathfrak{m}$ is the only prime ideal containing $\mathfrak{p} + (f_r)$, i.e., that $V(\mathfrak{p} + (f_r)) = \{\mathfrak{m}\}$, or that $\rad(\mathfrak{p} + (f_r)) = \mathfrak{m}$.  In particular, for each $1 \leq i \leq r-1$ we may find $n_i$ for which $f_i^{n_i} \in \mathfrak{p} + (f_r)$, say
  \[
    f_i^{n_i} = g_i + z_i f_r
    \text{ with } g_i \in \mathfrak{p}, z_i \in A.
  \]
  We claim that the conclusion of the lemma is now satisfied
  with $g_1,\dotsc,g_{r-1}$ as above and $g_r := f_r$:
  \begin{enumerate}
  \item The above equation
    shows that any prime $\mathfrak{q}$ that contains
    $g_1,\dotsc,g_{r-1},f_r$
    also contains $f_i^{n_i}$ and hence $f_i$ for $1 \leq i \leq r$,
    hence $\mathfrak{q} = \mathfrak{m}$.
    Thus $V(g_1,\dotsc,g_r) = \{\mathfrak{m}\}$.
  \item It's clear by construction that $\mathfrak{p}$ contains
    $(g_1,\dotsc,g_{r-1})$.  There is thus a minimal prime
    $\mathfrak{p}'$ of $(g_1,\dotsc,g_{r-1})$ contained in
    $\mathfrak{p}$; we must verify that
    $\mathfrak{p} = \mathfrak{p} '$.  (Geometrically,
    $\mathfrak{p}'$ corresponds to an irreducible component $Z'$
    of $V(g_1,\dotsc,g_{r-1})$ containing $Z$.)  To see this,
    consider the quotient ring $\overline{A} := A/(g_1,\dotsc,g_{r-1})$.
    Let
    \begin{equation}\label{eq:prime-chain-in-quotient-for-key-lemma-of-dimn-thm}
            \overline{\mathfrak{m}} \supsetneq
      \overline{\mathfrak{p}} \supseteq \overline{\mathfrak{p}'}
    \end{equation}
    denote the chain of primes in $\overline{A}$
    given by the image of
    $\mathfrak{m} \supsetneq \mathfrak{p} \supseteq
    \mathfrak{p}' \supseteq (g_1,\dotsc,g_{r-1})$.
    Then $(\overline{A},\overline{\mathfrak{m}})$ is a
    Noetherian local ring,
    and our task is equivalent to showing that
    $\overline{\mathfrak{p} } = \overline{\mathfrak{p} '}$.
    Let
    $f \in \overline{A}$ denote the image of $f_r$.  The primes
    of $\overline{A}$ containing $f$ are in bijection with the
    primes of $A$ containing $g_1,\dotsc,g_{r-1},f_r$, so
    $V_{\overline{A}}(f)
    = \{\overline{\mathfrak{m}}\}$.    
    By Krull's principal
    ideal theorem (in the form of Corollary \ref{cor:krull-pith-local}),
    it follows that $\height(\overline{\mathfrak{m} }) \leq 1$.
    From
    \eqref{eq:prime-chain-in-quotient-for-key-lemma-of-dimn-thm}
    we then deduce that $\overline{\mathfrak{p} } =
    \overline{\mathfrak{p}'}$,
    as required.
    (Intuitively, by choosing $f_r$ not to vanish on any irreducible component
    of $V(f_1,\dotsc,f_{r-1})$,
    we guarantee that appending it to the set of generators has the effect
    of knocking down the dimension of each such component by $1$.)
  \end{enumerate}
\end{proof}

We now deduce Theorem \ref{thm:krull-dimn-thm}.  We must show that if \(\mathfrak{p}\) is a minimal prime of \((f_1,\dotsc,f_r)\), then \(\height(\mathfrak{p}) \leq r\).  We may assume without loss of generality (replacing \(A\) with \(A_\mathfrak{p}\) and \(\mathfrak{p}\) with \(\mathfrak{p}_\mathfrak{p}\), which doesn't change the height of or minimality assumption on the latter) that \((A,\mathfrak{p})\) is a Noetherian local ring with \(V(f_1,\dotsc,f_r) = \{\mathfrak{p}\}\); we must show then that \(\height(\mathfrak{p}) \leq r\).  We do this by induction on \(r\).  The case \(r = 1\) is given by Krull's principal ideal theorem, so suppose \(r > 1\).  Let \(\mathfrak{q} \subsetneq \mathfrak{p}\) be a maximal element of the set of primes strictly contained in \(\mathfrak{p}\); it will suffice then to show that \(\height(\mathfrak{q}) \leq r-1\).  By Lemma \ref{lem:key-step-krull-dimn}, we may assume without loss of generality that \(\mathfrak{q}\) is a minimal prime of \((f_1,\dotsc,f_{r-1})\); the required inequality then follows from our inductive hypothesis.

\bibliography{refs}{} \bibliographystyle{plain}
\end{document}
