\documentclass[reqno]{amsart} \usepackage{graphicx, amsmath, amssymb, amsfonts, amsthm, stmaryrd, amscd}
\usepackage[usenames, dvipsnames]{xcolor}
\usepackage{tikz}
% \usepackage{tikzcd}
% \usepackage{comment}

% \let\counterwithout\relax
% \let\counterwithin\relax
% \usepackage{chngcntr}

\usepackage{enumerate}
% \usepackage{enumitem}
% \usepackage{times}
\usepackage[normalem]{ulem}
% \usepackage{minted}
% \usepackage{xypic}
% \usepackage{color}


% \usepackage{silence}
% \WarningFilter{latex}{Label `tocindent-1' multiply defined}
% \WarningFilter{latex}{Label `tocindent0' multiply defined}
% \WarningFilter{latex}{Label `tocindent1' multiply defined}
% \WarningFilter{latex}{Label `tocindent2' multiply defined}
% \WarningFilter{latex}{Label `tocindent3' multiply defined}
\usepackage{hyperref}
% \usepackage{navigator}


% \usepackage{pdfsync}
\usepackage{xparse}


\usepackage[all]{xy}
\usepackage{enumerate}
\usetikzlibrary{matrix,arrows,decorations.pathmorphing}



\makeatletter
\newcommand*{\transpose}{%
  {\mathpalette\@transpose{}}%
}
\newcommand*{\@transpose}[2]{%
  % #1: math style
  % #2: unused
  \raisebox{\depth}{$\m@th#1\intercal$}%
}
\makeatother


\makeatletter
\newcommand*{\da@rightarrow}{\mathchar"0\hexnumber@\symAMSa 4B }
\newcommand*{\da@leftarrow}{\mathchar"0\hexnumber@\symAMSa 4C }
\newcommand*{\xdashrightarrow}[2][]{%
  \mathrel{%
    \mathpalette{\da@xarrow{#1}{#2}{}\da@rightarrow{\,}{}}{}%
  }%
}
\newcommand{\xdashleftarrow}[2][]{%
  \mathrel{%
    \mathpalette{\da@xarrow{#1}{#2}\da@leftarrow{}{}{\,}}{}%
  }%
}
\newcommand*{\da@xarrow}[7]{%
  % #1: below
  % #2: above
  % #3: arrow left
  % #4: arrow right
  % #5: space left 
  % #6: space right
  % #7: math style 
  \sbox0{$\ifx#7\scriptstyle\scriptscriptstyle\else\scriptstyle\fi#5#1#6\m@th$}%
  \sbox2{$\ifx#7\scriptstyle\scriptscriptstyle\else\scriptstyle\fi#5#2#6\m@th$}%
  \sbox4{$#7\dabar@\m@th$}%
  \dimen@=\wd0 %
  \ifdim\wd2 >\dimen@
    \dimen@=\wd2 %   
  \fi
  \count@=2 %
  \def\da@bars{\dabar@\dabar@}%
  \@whiledim\count@\wd4<\dimen@\do{%
    \advance\count@\@ne
    \expandafter\def\expandafter\da@bars\expandafter{%
      \da@bars
      \dabar@ 
    }%
  }%  
  \mathrel{#3}%
  \mathrel{%   
    \mathop{\da@bars}\limits
    \ifx\\#1\\%
    \else
      _{\copy0}%
    \fi
    \ifx\\#2\\%
    \else
      ^{\copy2}%
    \fi
  }%   
  \mathrel{#4}%
}
\makeatother
% \DeclareMathOperator{\rg}{rg}

\usepackage{mathtools}
\DeclarePairedDelimiter{\paren}{(}{)}
\DeclarePairedDelimiter{\abs}{\lvert}{\rvert}
\DeclarePairedDelimiter{\norm}{\lVert}{\rVert}
\DeclarePairedDelimiter{\innerproduct}{\langle}{\rangle}
\newcommand{\Of}[2]{{\operatorname{#1}} {\paren*{#2}}}
\newcommand{\of}[2]{{{{#1}} {\paren*{#2}}}}

\DeclareMathOperator{\Shim}{Shim}
\DeclareMathOperator{\sgn}{sgn}
\DeclareMathOperator{\fdeg}{fdeg}
\DeclareMathOperator{\SL}{SL}
\DeclareMathOperator{\slLie}{\mathfrak{s}\mathfrak{l}}
\DeclareMathOperator{\soLie}{\mathfrak{s}\mathfrak{o}}
\DeclareMathOperator{\spLie}{\mathfrak{s}\mathfrak{p}}
\DeclareMathOperator{\glLie}{\mathfrak{g}\mathfrak{l}}
\newcommand{\pn}[1]{{\color{ForestGreen} \sf PN: [#1]}}
\DeclareMathOperator{\Mp}{Mp}
\DeclareMathOperator{\Mat}{Mat}
\DeclareMathOperator{\GL}{GL}
\DeclareMathOperator{\Gr}{Gr}
\DeclareMathOperator{\GU}{GU}
\def\gl{\mathfrak{g}\mathfrak{l}}
\DeclareMathOperator{\odd}{odd}
\DeclareMathOperator{\even}{even}
\DeclareMathOperator{\GO}{GO}
\DeclareMathOperator{\good}{good}
\DeclareMathOperator{\bad}{bad}
\DeclareMathOperator{\PGO}{PGO}
\DeclareMathOperator{\htt}{ht}
\DeclareMathOperator{\height}{height}
\DeclareMathOperator{\Ass}{Ass}
\DeclareMathOperator{\coheight}{coheight}
\DeclareMathOperator{\GSO}{GSO}
\DeclareMathOperator{\SO}{SO}
\DeclareMathOperator{\so}{\mathfrak{s}\mathfrak{o}}
\DeclareMathOperator{\su}{\mathfrak{s}\mathfrak{u}}
\DeclareMathOperator{\ad}{ad}
% \DeclareMathOperator{\sc}{sc}
\DeclareMathOperator{\Ad}{Ad}
\DeclareMathOperator{\disc}{disc}
\DeclareMathOperator{\inv}{inv}
\DeclareMathOperator{\Pic}{Pic}
\DeclareMathOperator{\uc}{uc}
\DeclareMathOperator{\Cl}{Cl}
\DeclareMathOperator{\Clf}{Clf}
\DeclareMathOperator{\Hom}{Hom}
\DeclareMathOperator{\hol}{hol}
\DeclareMathOperator{\Heis}{Heis}
\DeclareMathOperator{\Haar}{Haar}
\DeclareMathOperator{\h}{h}
\def\sp{\mathfrak{s}\mathfrak{p}}
\DeclareMathOperator{\heis}{\mathfrak{h}\mathfrak{e}\mathfrak{i}\mathfrak{s}}
\DeclareMathOperator{\End}{End}
\DeclareMathOperator{\JL}{JL}
\DeclareMathOperator{\image}{image}
\DeclareMathOperator{\red}{red}
\def\div{\operatorname{div}}
\def\eps{\varepsilon}
\def\cHom{\mathcal{H}\operatorname{om}}
\DeclareMathOperator{\Ops}{Ops}
\DeclareMathOperator{\Symb}{Symb}
\def\boldGL{\mathbf{G}\mathbf{L}}
\def\boldSO{\mathbf{S}\mathbf{O}}
\def\boldU{\mathbf{U}}
\DeclareMathOperator{\hull}{hull}
\DeclareMathOperator{\LL}{LL}
\DeclareMathOperator{\PGL}{PGL}
\DeclareMathOperator{\class}{class}
\DeclareMathOperator{\lcm}{lcm}
\DeclareMathOperator{\spann}{span}
\DeclareMathOperator{\Exp}{Exp}
\DeclareMathOperator{\ext}{ext}
\DeclareMathOperator{\Ext}{Ext}
\DeclareMathOperator{\Tor}{Tor}
\DeclareMathOperator{\et}{et}
\DeclareMathOperator{\tor}{tor}
\DeclareMathOperator{\loc}{loc}
\DeclareMathOperator{\tors}{tors}
\DeclareMathOperator{\pf}{pf}
\DeclareMathOperator{\smooth}{smooth}
\DeclareMathOperator{\prin}{prin}
\DeclareMathOperator{\Kl}{Kl}
\newcommand{\kbar}{\mathchar'26\mkern-9mu k}
\DeclareMathOperator{\der}{der}
% \DeclareMathOperator{\abs}{abs}
\DeclareMathOperator{\Sub}{Sub}
\DeclareMathOperator{\Comp}{Comp}
\DeclareMathOperator{\Err}{Err}
\DeclareMathOperator{\dom}{dom}
\DeclareMathOperator{\radius}{radius}
\DeclareMathOperator{\Fitt}{Fitt}
\DeclareMathOperator{\Sel}{Sel}
\DeclareMathOperator{\rad}{rad}
\DeclareMathOperator{\id}{id}
\DeclareMathOperator{\Center}{Center}
\DeclareMathOperator{\Der}{Der}
\DeclareMathOperator{\U}{U}
% \DeclareMathOperator{\norm}{norm}
\DeclareMathOperator{\trace}{trace}
\DeclareMathOperator{\Equid}{Equid}
\DeclareMathOperator{\Feas}{Feas}
\DeclareMathOperator{\bulk}{bulk}
\DeclareMathOperator{\tail}{tail}
\DeclareMathOperator{\sys}{sys}
\DeclareMathOperator{\atan}{atan}
\DeclareMathOperator{\temp}{temp}
\DeclareMathOperator{\Asai}{Asai}
\DeclareMathOperator{\glob}{glob}
\DeclareMathOperator{\Kuz}{Kuz}
\DeclareMathOperator{\Irr}{Irr}
\newcommand{\rsL}{ \frac{ L^{(R)}(\Pi \times \Sigma, \std, \frac{1}{2})}{L^{(R)}(\Pi \times \Sigma, \Ad, 1)}  }
\DeclareMathOperator{\GSp}{GSp}
\DeclareMathOperator{\PGSp}{PGSp}
\DeclareMathOperator{\BC}{BC}
\DeclareMathOperator{\Ann}{Ann}
\DeclareMathOperator{\Gen}{Gen}
\DeclareMathOperator{\SU}{SU}
\DeclareMathOperator{\PGSU}{PGSU}
% \DeclareMathOperator{\gen}{gen}
\DeclareMathOperator{\PMp}{PMp}
\DeclareMathOperator{\PGMp}{PGMp}
\DeclareMathOperator{\PB}{PB}
\DeclareMathOperator{\ind}{ind}
\DeclareMathOperator{\Jac}{Jac}
\DeclareMathOperator{\jac}{jac}
\DeclareMathOperator{\im}{im}
\DeclareMathOperator{\Aut}{Aut}
\DeclareMathOperator{\Int}{Int}
\DeclareMathOperator{\PSL}{PSL}
\DeclareMathOperator{\co}{co}
\DeclareMathOperator{\irr}{irr}
\DeclareMathOperator{\prim}{prim}
\DeclareMathOperator{\bal}{bal}
\DeclareMathOperator{\baln}{bal}
\DeclareMathOperator{\dist}{dist}
\DeclareMathOperator{\RS}{RS}
\DeclareMathOperator{\Ram}{Ram}
\DeclareMathOperator{\Sob}{Sob}
\DeclareMathOperator{\Sol}{Sol}
\DeclareMathOperator{\soc}{soc}
\DeclareMathOperator{\nt}{nt}
\DeclareMathOperator{\mic}{mic}
\DeclareMathOperator{\Gal}{Gal}
\DeclareMathOperator{\st}{st}
\DeclareMathOperator{\std}{std}
\DeclareMathOperator{\diag}{diag}
\DeclareMathOperator{\Sym}{Sym}
\DeclareMathOperator{\gr}{gr}
\DeclareMathOperator{\aff}{aff}
\DeclareMathOperator{\Dil}{Dil}
\DeclareMathOperator{\Lie}{Lie}
\DeclareMathOperator{\Symp}{Symp}
\DeclareMathOperator{\Stab}{Stab}
\DeclareMathOperator{\St}{St}
\DeclareMathOperator{\stab}{stab}
\DeclareMathOperator{\codim}{codim}
\DeclareMathOperator{\linear}{linear}
\newcommand{\git}{/\!\!/}
\DeclareMathOperator{\geom}{geom}
\DeclareMathOperator{\spec}{spec}
\def\O{\operatorname{O}}
\DeclareMathOperator{\Au}{Aut}
\DeclareMathOperator{\Fix}{Fix}
\DeclareMathOperator{\Opp}{Op}
\DeclareMathOperator{\opp}{op}
\DeclareMathOperator{\Size}{Size}
\DeclareMathOperator{\Save}{Save}
% \DeclareMathOperator{\ker}{ker}
\DeclareMathOperator{\coker}{coker}
\DeclareMathOperator{\sym}{sym}
\DeclareMathOperator{\mean}{mean}
\DeclareMathOperator{\elliptic}{ell}
\DeclareMathOperator{\nilpotent}{nil}
\DeclareMathOperator{\hyperbolic}{hyp}
\DeclareMathOperator{\newvector}{new}
\DeclareMathOperator{\new}{new}
\DeclareMathOperator{\full}{full}
\newcommand{\qr}[2]{\left( \frac{#1}{#2} \right)}
\DeclareMathOperator{\unr}{u}
\DeclareMathOperator{\ram}{ram}
% \DeclareMathOperator{\len}{len}
\DeclareMathOperator{\fin}{fin}
\DeclareMathOperator{\cusp}{cusp}
\DeclareMathOperator{\curv}{curv}
\DeclareMathOperator{\rank}{rank}
\DeclareMathOperator{\rk}{rk}
\DeclareMathOperator{\pr}{pr}
\DeclareMathOperator{\Transform}{Transform}
\DeclareMathOperator{\mult}{mult}
\DeclareMathOperator{\Eis}{Eis}
\DeclareMathOperator{\reg}{reg}
\DeclareMathOperator{\sing}{sing}
\DeclareMathOperator{\alt}{alt}
\DeclareMathOperator{\irreg}{irreg}
\DeclareMathOperator{\sreg}{sreg}
\DeclareMathOperator{\Wd}{Wd}
\DeclareMathOperator{\Weil}{Weil}
\DeclareMathOperator{\Th}{Th}
\DeclareMathOperator{\Sp}{Sp}
\DeclareMathOperator{\Ind}{Ind}
\DeclareMathOperator{\Res}{Res}
\DeclareMathOperator{\ini}{in}
\DeclareMathOperator{\ord}{ord}
\DeclareMathOperator{\osc}{osc}
\DeclareMathOperator{\fluc}{fluc}
\DeclareMathOperator{\size}{size}
\DeclareMathOperator{\ann}{ann}
\DeclareMathOperator{\equ}{eq}
\DeclareMathOperator{\res}{res}
\DeclareMathOperator{\pt}{pt}
\DeclareMathOperator{\src}{source}
\DeclareMathOperator{\Zcl}{Zcl}
\DeclareMathOperator{\Func}{Func}
\DeclareMathOperator{\Map}{Map}
\DeclareMathOperator{\Frac}{Frac}
\DeclareMathOperator{\Frob}{Frob}
\DeclareMathOperator{\ev}{eval}
\DeclareMathOperator{\pv}{pv}
\DeclareMathOperator{\eval}{eval}
\DeclareMathOperator{\Spec}{Spec}
\DeclareMathOperator{\Speh}{Speh}
\DeclareMathOperator{\Spin}{Spin}
\DeclareMathOperator{\GSpin}{GSpin}
\DeclareMathOperator{\Specm}{Specm}
\DeclareMathOperator{\Sphere}{Sphere}
\DeclareMathOperator{\Sqq}{Sq}
\DeclareMathOperator{\Ball}{Ball}
\DeclareMathOperator\Cond{\operatorname{Cond}}
\DeclareMathOperator\proj{\operatorname{proj}}
\DeclareMathOperator\Swan{\operatorname{Swan}}
\DeclareMathOperator{\Proj}{Proj}
\DeclareMathOperator{\bPB}{{\mathbf P}{\mathbf B}}
\DeclareMathOperator{\Projm}{Projm}
\DeclareMathOperator{\Tr}{Tr}
\DeclareMathOperator{\Type}{Type}
\DeclareMathOperator{\Prop}{Prop}
\DeclareMathOperator{\vol}{vol}
\DeclareMathOperator{\covol}{covol}
\DeclareMathOperator{\Rep}{Rep}
\DeclareMathOperator{\Cent}{Cent}
\DeclareMathOperator{\val}{val}
\DeclareMathOperator{\area}{area}
\DeclareMathOperator{\nr}{nr}
\DeclareMathOperator{\CM}{CM}
\DeclareMathOperator{\CH}{CH}
\DeclareMathOperator{\tr}{tr}
\DeclareMathOperator{\characteristic}{char}
\DeclareMathOperator{\supp}{supp}


\theoremstyle{plain} \newtheorem{theorem} {Theorem} \newtheorem{conjecture} [theorem] {Conjecture} \newtheorem{corollary} [theorem] {Corollary} \newtheorem{proposition} [theorem] {Proposition} \newtheorem{fact} [theorem] {Fact}
\theoremstyle{definition} \newtheorem{definition} [theorem] {Definition} \newtheorem{hypothesis} [theorem] {Hypothesis} \newtheorem{assumptions} [theorem] {Assumptions}
\newtheorem{example} [theorem] {Example}
\newtheorem{assertion}[theorem] {Assertion}
\newtheorem{note}[theorem] {Note}
\newtheorem{conclusion}[theorem] {Conclusion}
\newtheorem{claim}            {Claim}
\newtheorem{homework} {Homework}
\newtheorem{exercise} {Exercise}  \newtheorem{question}[theorem] {Question}    \newtheorem{answer} {Answer}  \newtheorem{problem} {Problem}    \newtheorem{remark} [theorem] {Remark}
\newtheorem{notation} [theorem]           {Notation}
\newtheorem{terminology}[theorem]            {Terminology}
\newtheorem{convention}[theorem]            {Convention}
\newtheorem{motivation}[theorem]            {Motivation}


\newtheoremstyle{itplain} % name
{6pt}                    % Space above
{5pt\topsep}                    % Space below
{\itshape}                   % Body font
{}                           % Indent amount
{\itshape}                   % Theorem head font
{.}                          % Punctuation after theorem head
{5pt plus 1pt minus 1pt}                       % Space after theorem head
% {.5em}                       % Space after theorem head
{}  % Theorem head spec (can be left empty, meaning ‘normal’)

% \theoremstyle{mytheoremstyle}


\theoremstyle{itplain} %--default
% \theoremheaderfont{\itshape}
% \newtheorem{lemma}{Lemma}
\newtheorem{lemma}[theorem]{Lemma}
% \newtheorem{lemma}{Lemma}[subsubsection]

\newtheorem*{lemma*}{Lemma}
\newtheorem*{proposition*}{Proposition}
\newtheorem*{definition*}{Definition}
\newtheorem*{example*}{Example}

\newtheorem*{results*}{Results}
\newtheorem{results} [theorem] {Results}


\usepackage[displaymath,textmath,sections,graphics]{preview}
\PreviewEnvironment{align*}
\PreviewEnvironment{multline*}
\PreviewEnvironment{tabular}
\PreviewEnvironment{verbatim}
\PreviewEnvironment{lstlisting}
\PreviewEnvironment*{frame}
\PreviewEnvironment*{alert}
\PreviewEnvironment*{emph}
\PreviewEnvironment*{textbf}




\begin{document}

\author{ONPM, PN}
\title{Some notes on coherent states}


\begin{abstract}
  We summarize a paper of Sugita (``Proof of the generalized Lieb-Wehrl conjecture for integer indices larger than one'') and Delbourgo--Fox (``Maximum weight vectors possess minimal uncertainty'').
\end{abstract}

\maketitle

\section{The generalized Lieb-Wehrl conjecture for integer indices larger than one}\label{sec:cngub4lfbx}
We summarize Sugita \cite{MR1946863}.

Let $G$ be a compact connected Lie group.  Let $\pi$ be an irreducible unitary (complex) representation of $G$.

\begin{definition}\label{definition:cngsx0l9k6}
  We say that a unit vector $v \in \pi$ is a \emph{coherent state} if it is a highest weight vector with respect to some maximal torus and ordering.
\end{definition}

For convenience, let us now fix a maximal torus and ordering, so that we may speak of the highest weight $\lambda$ of $\pi$ and the line of highest weight vectors.  Coherent states are then the $G$-translates of unit vectors in that line.

We equip $G$ with its probability Haar $d g$.

\begin{theorem}\label{theorem:cngsx0yg7y}
  Let $q \in \mathbb{Z}_{\geq 2}$.  For a unit vector $v \in \pi$, set
  \begin{equation*}
    I(v) :=
    (\dim \pi)
    \int_{G}
    \left( \lvert \left\langle g v, v_\lambda \right\rangle \rvert^2 \right)^q
    \, d g.
  \end{equation*}
  (The normalizing factor is for later convenience.)  Then $I(v)$ achieves its maximum precisely when $v$ is a coherent state.
\end{theorem}
\begin{remark}\label{remark:cngsx0yf26}
  Using the convexity of the $q$th power function $x \mapsto x^q$, one deduces an ostensibly more general result concerning higher rank tensors in place of $v \otimes \overline{v}$.
\end{remark}

We turn to the proof of Theorem \ref{theorem:cngsx0yg7y}.

Let $\Pi$ be an irreducible unitary representation of $G$ with highest weight $q \lambda$.  Then $\Pi$ embeds as a subrepresentation of $\pi^{\otimes q}$, with multiplicity one.  We identify $\Pi$ with a subrepresentation of $\pi^{\otimes q}$:
\begin{equation*}
  \Pi \subseteq  \pi^{\otimes q}. 
\end{equation*}
\begin{lemma}\label{lemma:cngurf0apf}
  For unit vectors $v \in \pi$, the quantity $I(v)$ is maximized precisely when $v^{\otimes q} \in \Pi$.
\end{lemma}
\begin{proof}
  It is enough to show that
  \begin{equation}\label{eq:cngurmvhig}
    I(v) = \frac{\dim \pi}{\dim \Pi}
    \left\langle P v^{\otimes q}, v^{\otimes q} \right\rangle,
  \end{equation}
  where $P : \pi^{\otimes q} \rightarrow \Pi$ denotes the orthogonal projection, because the right hand side of this formula is clearly maximal precisely when $v^{\otimes q}$ lies in the image $\Pi$ of $P$.  To establish \eqref{eq:cngurmvhig}, we first apply the Schur orthogonality relations to see that
  \begin{equation*}
    P =(\dim \Pi) \int_{G} g v_\lambda^{\otimes q}
    \otimes \overline{g v_\lambda^{\otimes q}} \, d g. 
  \end{equation*}
  We conclude by inserting this formula for $P$ into the right hand side of \eqref{eq:cngurmvhig}.
\end{proof}

To complete the proof of Theorem \ref{theorem:cngsx0yg7y}, we reduce to establishing the following equivalence:
\begin{equation}\label{eq:cngsx4hl1k}
  v:\text{coherent} \iff v^{\otimes q} \in \Pi.
\end{equation}
The forward implication follows from the fact that $v_{\lambda}^{\otimes q} \in \Pi$.  It remains only to establish the converse.

To that end, we introduce the Casimir operator $\Omega$ for $G$, given by $\sum_{x \in \mathcal{B}(\mathfrak{g})} x^2$ for an orthonormal basis $\mathcal{B}(\mathfrak{g})$ of $\mathfrak{g} := \Lie(G)$ taken with respect to a $G$-invariant inner product.  This operator lies in the center of the universal enveloping algebra, hence acts on any given irreducible representation as multiplication by some scalar.

\begin{lemma}\label{lemma:cngsx4g21h}
  For unit vectors $v \in \pi$, the inner product
  \begin{equation*}
    \left\langle \Omega v^{\otimes q}, v^{\otimes q} \right\rangle
  \end{equation*}
  is maximized precisely when $v$ is coherent.
\end{lemma}
\begin{proof}
  We first compute that
  \begin{equation}\label{eq:cngsx5j54m}
    \left\langle \Omega v^{\otimes q}, v^{\otimes q} \right\rangle
    =
    q
    \Omega_\pi + q(q - 1) E(v),
  \end{equation}
  where $\Omega_\pi$ denotes the scalar by which $\Omega$ acts on $\pi$ and
  \begin{equation*}
    E(v) := \sum_{x \in \mathcal{B}(\mathfrak{g})}
    \left\lvert \langle x v, v \rangle \right\rvert^2.
  \end{equation*}
  Consider for instance the case $q = 2$.  For $x \in \mathfrak{g}$, we have
  \begin{align*}
    x^2 v^{\otimes 2} &= x \left( x v \otimes v +
                        v \otimes x v \right) \\
                      &=
                        x^2 v \otimes v + 2 x v \otimes x v + v \otimes x^2 v.
  \end{align*}
  Summing over $x \in \mathcal{B}(\mathfrak{g})$ gives
  \begin{align*}
    \Omega v^{\otimes 2} &= \Omega v \otimes v + 2 \sum_{x \in \mathcal{B}(\mathfrak{g})} x v \otimes x v+ v \otimes \Omega v  \\
                         &= 2 \Omega_\pi v \otimes v + 2 \sum_{x \in \mathcal{B}(\mathfrak{g})} x v \otimes x v.
  \end{align*}
  When $q = 3$, we obtain instead
  \begin{equation*}
    \Omega v^{\otimes 3} = 3 \Omega_\pi v^{\otimes 3} + 2 \sum_{x \in \mathcal{B}} \left(
      x v \otimes x v \otimes v
      + x v \otimes v \otimes x v
      + v \otimes x v \otimes x v
    \right).
  \end{equation*}
  We get a similar expression in general, which, after pairing with $v^{\otimes q}$, gives the claimed formula \eqref{eq:cngsx5j54m}.

  We conclude via appeal to the following lemma.
\end{proof}

\begin{lemma}\label{lemma:cngsx4g4hz}
  For unit vectors $v$, the quantity $E(v)$ is maximized precisely when $v$ is coherent.
\end{lemma}
\begin{proof}
  See Section \ref{sec:cngsx6buyy}.
\end{proof}


We may now complete the proof of the backwards implication in \eqref{eq:cngsx4hl1k}, hence of Theorem \ref{theorem:cngsx0yg7y}.  Suppose $v^{\otimes q} \in \Pi$.  Then, writing $\Omega_\Pi$ for the scalar by which $\Omega$ acts on the irreducible representation $\Pi$, we have $\Omega v^{\otimes q} = \Omega_\Pi v^{\otimes q}$ and $\Omega v_\lambda^{\otimes q} = \Omega_\Pi v_\lambda^{\otimes q}$, hence
\begin{equation*}
  \left\langle \Omega v^{\otimes q}, v^{\otimes q} \right\rangle
  =
  \left\langle \Omega v_\lambda^{\otimes q}, v_\lambda^{\otimes q} \right\rangle.
\end{equation*}
By two applications of Lemma \ref{lemma:cngsx4g21h}, we deduce that $v$ is coherent, as required.


\section{Coherent states minimize uncertainty}\label{sec:cngsx6buyy}
We summarize Delbourgo--Fox \cite{MR0480888}, and in particular, prove Lemma \ref{lemma:cngsx4g4hz}.

Let $G$ be a compact connected Lie group.  Let $\pi$ be an irreducible unitary representation of $G$.  Equip the Lie algebra $\mathfrak{g}$ with a $G$-invariant inner product, and let $\mathcal{B}(\mathfrak{g})$ be an orthonormal basis.

\begin{definition}\label{definition:cngsx6dtkh}
  For each unit vector $v \in \pi$, we define
  \begin{equation*}
    \Delta(v) := \sum_{
      x \in \mathcal{B}(\mathfrak{g})      
    }
    \lVert \bar{x} v \rVert^2
  \end{equation*}
  where $\bar{x} := x - \langle x v, v \rangle$.
\end{definition}
\begin{theorem}\label{theorem:cngsx6du1v}
  For unit vectors $v$, the quantity $\Delta(v)$ is minimized precisely when $v$ is a coherent state.
\end{theorem}

That Theorem \ref{theorem:cngsx6du1v} implies Lemma \ref{lemma:cngsx4g4hz} is immediate from the following:
\begin{lemma}\label{lemma:cngsx6fp2j}
  We have $\Delta(v) = - \Omega_\pi - E(v)$, where $\Omega_\pi$ denotes the Casimir eigenvalue for $\pi$.
\end{lemma}
\begin{proof}
  We note that $\bar{x} v = x v - \langle x v, v \rangle v$, thus $\langle \bar{x} v, v \rangle = 0$ and
  \begin{equation*}
    \lVert \bar{x} v \rVert^2
    =
    \langle \bar{x} v, x v \rangle.
  \end{equation*}
  By expanding the definition of $\bar{x}$, we obtain
  \begin{equation*}
    \langle x v, x v \rangle
    -
    \langle x v, v \rangle \langle v, x v \rangle.
  \end{equation*}
  Using here that $x$ acts via a skew-symmetric operator, we see that the first term equals $- \langle x^2 v, v \rangle$, while the second term  may be abbreviated to $\lvert \langle x v, v \rangle \rvert^2$.  Summing over $x$ leads to the required identity.
\end{proof}

We turn to the proof of Theorem \ref{theorem:cngsx6du1v}.

\begin{lemma}\label{lemma:cngsx64916}
  Let $\pi$ be a unitary representation of $G$, let $v \in \pi$ be a unit vector, let $x \in i \mathfrak{g}$.  Then the minimum over all (real) scalars $c$ of the quantity
  \begin{equation*}
    \lVert (x - c) v \rVert
  \end{equation*}
  is achieved by taking $c = \langle x v, v \rangle$.
\end{lemma}
\begin{proof}
  Since $x$ lies in $i \mathfrak{g}$, it acts via a self-adjoint operator, so we readily obtain
  \begin{equation}\label{eq:cngsx65skq}
    \lVert (x-c) v \rVert^2
    = \langle x v, x v \rangle
    - 2 c \langle x v, v \rangle
    + c^2.
  \end{equation}
  This is a quadratic polynomial whose minimum is attained at the critical point, which we solve for by taking a derivative.
\end{proof}

The definition of $\Delta(v)$ remains unchanged upon replacing $\mathfrak{g}$ with its imaginary multiple $i \mathfrak{g}$.  By Lemma~\ref{lemma:cngsx64916}, we see that the minimum of $\Delta(v)$ taken over unit vectors $v$ coincides with the minimum of
\begin{equation}\label{eq:cngub21394}
  \sum_{x \in \mathcal{B}(i \mathfrak{g})}
  \lVert (x - c_x) v \rVert^2
\end{equation}
taken over unit vectors $v$ and tuples of scalars $c_x$.  Such a minimum exists by continuity and compactness.  Let us consider one such minimum.  The above expression expands, as in \eqref{eq:cngsx65skq}, to
\begin{equation*}
  - \Omega_\pi + \sum_x c_x^2 - 2 \sum_{x} c_x \langle x v, v \rangle.
\end{equation*}
With the notation $y := \sum_x c_x x \in \mathfrak{g}$, the above may be written
\begin{equation*}
  - \Omega_\pi + \lVert y \rVert^2 - 2 \langle y v, v \rangle.
\end{equation*}
We choose a maximal torus whose Lie algebra contains $y$, and an ordering with respect to $y$ which is dominant.  Our assumptions imply that $\langle y v, v \rangle$ cannot be made larger by changing $v$, so $v$ must lie in the eigenspace for $y$ with largest eigenvalue.  We claim that this eigenspace is one-dimensional.  If the claim holds, then $y$ is a highest weight vector, so we are done.  Assume the claim fails.  Let us then modify $v$, if necessary, so that it is a highest weight vector for the given torus and ordering; we may do so without changing $\langle y v, v \rangle$, hence without changing our assumption that $v$ and $c_x$ realize the minimum.  We see then that by modifying $y$ without changing $\lVert y \rVert^2$, we may increase the size of $\langle y v, v \rangle$, contradicting the supposed minimality.  This completes the proof of Theorem \ref{theorem:cngsx6du1v}.



\section{Determining vectors by their matrix coefficients}\label{sec:cngub4lcsx}
It follows from Lemma \ref{lemma:cngsx64916} that for a given unit vector $v$, the quantity \eqref{eq:cngub21394} is minimized by taking $c_x = \langle x v, v \rangle$.  We record here that \emph{the latter quantities determine} $v$ \emph{up to multiplication by a unit scalar}.  Indeed, let $G$ be a compact connected Lie group, $\pi$ an irreducible representation, and $u, v \in \pi$ unit vectors with $\langle x u, u \rangle = \langle x v, v \rangle$ for all $x \in \mathfrak{g}$.  By expanding the exponential series, we see that $\langle \exp(x) u, u \rangle = \langle \exp(x) v, v \rangle$, so that the
$\langle g u, u \rangle = \langle g v, v \rangle$ holds near the identity on $G$.  Since matrix coefficients of finite-dimensional representations define analytic functions, we deduce the equality on all of $G$.  We conclude that $u$ and $v$ are proportional by appeal to the following:

\begin{proposition}\label{proposition:cngub293qk}
  Let $G$ be a compact group, let $\pi$ be an irreducible representation, and let $u$ and $v$ be nonzero vectors in $\pi$ with the property that
  \begin{equation}\label{eq:cngub3eab0}
    \langle g u, u \rangle = \langle g v, v \rangle
  \end{equation}
  for all $g \in G$.  Then $u$ and $v$ are proportional.
\end{proposition}
We record a proof after some lemmas.

\begin{lemma}[Schur's lemma]\label{lemma:cngubyxnij}
  Let $\pi$ be an irreducible representation of $G$.  Then $\End_G(\pi) = \mathbb{C}$, that is to say, any linear operator on $\pi$ that commutes with the action of $G$ is a scalar multiple of the identity.
\end{lemma}

\begin{lemma}\label{lemma:cngubytkmn}
  Let $\pi$ and $\sigma$ be unitary representations of $G$, with $\pi$ irreducible.  Let $u \in \pi$ and $v \in \sigma$ be nonzero vectors such that
  \begin{equation*}
    \langle g u, u \rangle = \langle g v, v \rangle.
  \end{equation*}
  Then there is a unique $G$-equivariant map $T : \pi \rightarrow \sigma$ that sends $u$ to $v$.  
\end{lemma}
\begin{proof}
  Any vector in $\pi$ may be written $\sum_{g \in G} c_g g u$ for some finitely-supported coefficients $c_g$.  We have no choice but to attempt to define
  \begin{equation*}
    T \left( \sum_{g \in G} c_g g u \right) := \sum_{g \in G} c_g g v.
  \end{equation*}
  We need to check that this is well-defined.  To see this, we use that the vanishing of the argument of $T$ may be detected via its inner product with itself, and then use that $u$ and $v$ have the same inner products to see that the right hand side must likewise vanish.
\end{proof}

\begin{proof}[Proof of Proposition~\ref{proposition:cngub293qk}]
  By Lemma \ref{lemma:cngubytkmn}, there is a unique $T \in \End_G(\pi)$ that maps $u$ to $v$.  By Schur's lemma, $T$ is a multiple of the identity.  [thinking-face]
\end{proof}


\bibliography{refs}{} \bibliographystyle{plain}
\end{document}
