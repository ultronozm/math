\documentclass[reqno]{amsart} \input{common.tex}

\title{Large values of Whittaker functions for $\GL_2$}

\begin{document}


\begin{abstract}
  Note written around 2016 where, following Templier \cite{MR3272013} and Saha \cite{MR3544630}, we exhibit large values of $p$-adic Whittaker functions for principal series representations of $\GL_2$.
\end{abstract}
\maketitle
\tableofcontents

\section{General notation}
\label{sec:org788739a}
Let $k$ be a local field.  Fix a nontrivial additive character $\psi$ of $k$.  When $k$ is non-archimedean, we assume $\psi$ is unramified and adopt the accompanying notation $\mathfrak{o}, \mathfrak{p}, q$ for its maximal order, maximal ideal and residue field cardinality.  Let $\chi_1, \chi_2$ be unitary characters of $k^\times$ with analytic conductors $C_1 := C(\chi_1)$, $C_2 := C(\chi_2)$.  Set $G := \GL_2(k)$.  We use the notation
\begin{equation*}
  n(x) := 
  \begin{pmatrix}
    1 & x \\
      & 1
  \end{pmatrix}
  ,
  \quad
  a(y) := 
  \begin{pmatrix}
    y &  \\
      & 1
  \end{pmatrix}
  .
\end{equation*}
Let $\pi := \mathcal{I}(\chi_1, \chi_2)$ be the induced principal series representation of conductor $C := C_1 C_2$.

We write the character group of $k^\times$ additively.  We write $\alpha$ for the character given by the normalized modulus $|.|$.  We denote by $\omega := \chi_1 + \chi_2$ the central character of $\pi$.  We realize $\pi$ in its induced model as a space of functions $f : G \rightarrow \mathbf{C}$ satisfying
\begin{equation*}
  f(n(x) a(y) z g) = y^{\alpha/2+\chi_1} z^{\omega} f(g).
\end{equation*}
Here $z \in k^\times$ identifies with an element of the center of $G$, and we use notation such as
\begin{equation*}
  y^{\alpha / 2 + \chi_1} := \lvert y \rvert^{1/2} \chi_1(y).
\end{equation*}

We denote in what follows by $\iota_k$ a constant such as $\zeta_k(1)$, $\zeta_k(2)$, or a product of (inverses of) similar quantities; we allow the precise value of $\iota_k$ to vary from one occurrence to another.

\section{Whittaker functions in principal series representations}
\label{sec:orgd52af82}
\subsection{Schwartz space parametrization}
\label{sec:orgd34bf57}
For \(\Phi \in \mathcal{S}(k^2)\), define the element \(f_\Phi \in \pi\) by the formula
\begin{equation*}
  f_\Phi(g) := \int_{z \in k^\times} \Phi(e_2 z g) z^{-\omega} \det(z g)^{\alpha/2+\chi_1}.
\end{equation*}
Thus if
\begin{equation*}
  g =
  \begin{pmatrix}
    \ast & \ast \\
    c & d
  \end{pmatrix}
  ,
\end{equation*}
then
\begin{equation*}
  f_\Phi(g) = \det(g)^{\alpha/2+\chi_1} \int_{z \in k^\times} \Phi(c z, d z) z^{\alpha+\chi_1-\chi_2}.
\end{equation*}
This association defines a surjective map \(\Phi \mapsto f_\Phi\) onto \(\pi\).

\subsection{The Whittaker intertwiner}
\label{sec:org908ce5e}
Embed \(f \in \pi\) into its \(\psi\)-Whittaker model via the map \(f \mapsto W_f\), where
\begin{equation*}
  W_f(g) := \int_{u \in k} f(w n(u) g) \psi(-u).
\end{equation*}
For \(\Phi \in \mathcal{S}(k^2)\), set \(W_{\Phi} := W_{f_{\Phi}}\).  For $g = \left(
  \begin{smallmatrix}
    a&b\\
    c&d
  \end{smallmatrix}
\right) \in G$, we compute that
\begin{equation*}
  W_\Phi(g) = \det(g)^{\alpha/2+\chi_1} \int_{z \in k^\times} \int _{u \in k} \Phi (z (a + c u), z (b + d u)) z^{\alpha+\chi_1-\chi_2} \psi(- u).
\end{equation*}
After the change of variables \(u \mapsto u/z\) followed by \(z \mapsto u z\), the above becomes
\begin{equation}\label{whit-general}
  W_\Phi(g)
  =
  \det(g)^{\alpha/2+\chi_1}
  \int_{z \in k^\times, u \in k}
  \Phi (u(a z + c), u(b z + d))
  (u z)^{\chi_1-\chi_2}
  \psi(- 1/z).
\end{equation}

\subsection{Specialization to diagonal matrices}
\label{sec:orgc9a42d8}
In the special case $g = a(y) := \left(
  \begin{smallmatrix}
    y&\\
     &1
  \end{smallmatrix}
\right)$, the formula (\ref{whit-general}) specializes to
\begin{align*}
  W_\Phi(a(y))
  &=
    y^{\alpha/2+\chi_1}
    \int_{z \in k^\times, u \in k}
    \Phi(u y z, u) (u z)^{\chi_1 - \chi_2} \psi(-1 / z),
\end{align*}
which after the substitutions $u \mapsto u / y z$, $z = 1/ y t$ gives
\begin{equation}\label{eq:whittaker-diag}
  W_\Phi(a(y))
  = y^{\alpha/2 + \chi_2}
  \int_{u,t \in k}
  \Phi(u, u t) u^{\chi_1 - \chi_2} \psi( - y t ).
\end{equation}

\subsection{Specialization to matrices in the open Bruhat cell}
The most relevant group element is
\begin{equation*}
  g = a(y) w n(x)
  =
  \begin{pmatrix}
    & -y \\
    1 & x
  \end{pmatrix}
  ,
\end{equation*}
for which (\ref{whit-general}) gives
\begin{equation*}
  W_\Phi(a(y) w n(x))
  =
  y^{\alpha/2+\chi_1}
  \int_{z \in k^\times}
  \int _{u \in k}
  \Phi (u, u(-y z + x))
  (u z)^{\chi_1-\chi_2} \psi(- 1/z).
\end{equation*} 
We substitute $z \mapsto z/y$ to rewrite this as
\begin{equation*}
  W_\Phi(a(y) w n(x))
  =
  y^{\alpha/2+\chi_2}
  \int_{z \in k^\times}
  \int _{u \in k}
  \Phi (u, u(x-z))
  (u z)^{\chi_1-\chi_2} \psi(- y/z)
\end{equation*}
and then substitute $z = x - t$ to obtain
\begin{equation*}
  W_\Phi(a(y) w n(x))
  =
  y^{\alpha/2+\chi_2}
  \int_{u,t \in k}
  \Phi (u, u t)
  u^{\chi_1-\chi_2}
  (x-t)^{\chi_1-\chi_2 - \alpha }
  \psi(-y (x-t)^{-1})
\end{equation*}
and finally $t \mapsto x t$ to conclude that
\begin{equation}\label{whit-open}
  W_\Phi(a(y) w n(x))
  = y^{\alpha/2 + \chi_2}
  \int_{u,t \in k}
  \Phi(u, u t x) (u x)^{\chi_1 - \chi_2} ( 1 - t)^{\chi_1 -
    \chi_2 - \alpha} 
  \psi \left( \frac{-y /x}{1 - t} \right).
\end{equation}


\section{The normalized newvector in a principal series representation}
\label{sec:org5339323}
We suppose in this section that \(k\) is non-archimedean.

\subsection{Choice of Schwartz function}
\label{sec:orgbd005ab}
Define $\Phi \in \mathcal{S}(k^2)$ by the formula
\begin{equation*}
  \Phi(c,d) :=
  C_2^{1/2}
  c^{-\chi_1} d^{\chi_2} 
  \cdot
  \begin{cases}
    1_{C_2 |c| = 1}& C_1 > 1
    \\
    1_{C_2 |c| \leq 1} & C_1 = 1
  \end{cases}
  \cdot
  \begin{cases}
    1_{|d| = 1}& C_2 > 1
    \\
    1_{|d| \leq 1} & C_2 = 1.
  \end{cases}
\end{equation*}
The restriction of $f_\Phi$ to $g = \left(
  \begin{smallmatrix}
    a&b\\
    c&d
  \end{smallmatrix}
\right) \in K := \GL_2(\mathfrak{o})$ is
\begin{equation*}
  f_\Phi(g)
  =
  \det(g)^{\alpha/2+\chi_1}
  \int_{z \in k^\times} \Phi(c z, d z) z^{\alpha + \chi_1 - \chi_2}.
\end{equation*}
This defines a newvector in $\pi$ with $L^2$-norm $\asymp 1$.  One can see this directly in the induced model (e.g., observe that it is supported on the same $B \times K_1(C)$-double coset modulo $\mathfrak{p}^{n_1 + n_2}$ as the one given in Schmidt's note \cite{Sch02}, or just verify directly that it satisfies the needed invariance properties and has norm $\asymp 1$).  We give another verification below.

We henceforth assume for simplicity that $C_1, C_2$ are both strictly greater than $1$, so that we may simplify the above formula to
\begin{equation*}
  \Phi(c,d) =
  C_2^{1/2}
  c^{-\chi_1} d^{\chi_2} 
  1_{C_2 |c| = 1}
  1_{|d| = 1}.
\end{equation*}
Put another way, we have
\begin{equation*}
  \Phi(u, u t) u^{\chi_1 - \chi_2}
  =
  C_2^{1/2}
  t^{\chi_2}
  1_{C_2 |u| = 1}
  1_{|t| = C_2}.
\end{equation*}

\subsection{Sanity check: the Whittaker function at diagonal matrices}
\label{sec:org12b4732}
The general formula (\ref{eq:whittaker-diag}) for the Schwartz function-parametrized Whittaker function at a diagonal element specialized to the choice of \(\Phi\) given above gives
\begin{equation*}
  W(a(y)) = C_2^{1/2} y^{\alpha/2+\chi_2} \int_{u,t \in k} t^{\chi_2} 1_{C_2 |u| = 1} 1_{|t| = C_2} \psi(-y t).
\end{equation*}
The integrand has support of volume \(\asymp 1\); the \(u\) integrand is constant, while the \(t\) integrand is a Gauss sum vanishing unless \(|y| = 1\), in which case it has magnitude \(\iota_k C_2^{-1/2}\).  The change of variables \(t \mapsto t/y\) and \(z := 1/t\) gives
\begin{align*}
  W(a(y))
  &=
    \iota_k
    C_2^{1/2}
    y^{\alpha/2}
    \int_{z \in k^\times}
    1_{C_2 |y z| = 1}
    z^{-\chi_2} \psi(-1/z),
\end{align*}
which by the stated properties of Gauss sums implies that
\begin{equation*}
  W(a(y)) = \iota_{k} 1_{\mathfrak{o}^\times}(y),
\end{equation*}
as expected.

\subsection{The Whittaker function at elements of the open Bruhat cell}
\label{sec:org665ed72}
As a variant of the formula for \(\Phi(u,u t)\) recorded above, we have
\begin{equation*}
  \Phi(u, u x t) (u x)^{\chi_1 - \chi_2} = C_2^{1/2} x^{\chi_1} t^{\chi_2} 1_{C_2 |u| = 1} 1_{|x t| = C_2}.
\end{equation*}
From the general formula for \(W(a(y) w n(x))\) recorded above in (\ref{whit-open}), it follows that
\begin{equation*}
  W(a(y) w n(x)) = C_2^{1/2} x^{\chi_1} y^{\alpha/2 + \chi_2} \int_{u,t \in k} t^{\chi_2} (1 - t)^{\chi_1 - \chi_2 - \alpha} 1_{C_2 |u| = 1} 1_{|x t| = C_2} \psi \left( \frac{-y/x}{1-t} \right).
\end{equation*}
The \$u\$-integral is constant with support of volume \(\iota_k C_2^{-1}\), while the support of the \(t\) integral has volume \(\iota_k C_2 / |x|\), so executing the \(u\) integral while renaming \(t \in k\) to \(z \in k^\times\) gives
\begin{equation*}
  W(a(y) w n(x)) = \iota_k \left( \frac{C_2 |y|}{|x|^2} \right)^{1/2} x^{\chi_1} y^{\chi_2} \int_{z \in k^\times} z^{\chi_2} 1_{|x z| = C_2} (1 - z)^{\chi_1 - \chi_2 - \alpha} \psi \left( \frac{-y/x}{1-z} \right)
\end{equation*}
Note that all calculations up until now basically work over an archimedean field, too; we just have to replace conditions like \(1_{|z| = 1}\) with a smooth cutoff.
\subsection{Final formula for the magnitude}
\label{sec:orgcd9e180}
In particular, taking absolute values in the final formula from section \ref{sec:org665ed72} gives
\begin{equation*}
  W(a(y) w n(x)) \asymp \left( \frac{C_2 |y|}{|x|^2} \right)^{1/2} \int_{z \in k^\times} 1_{|z| = C_2/|x|} z^{\chi_2} (1 - z)^{\chi_1 - \chi_2 - \alpha} \psi\left(- \frac{y/x}{1-z} \right).
\end{equation*}
It will also be convenient (for later purposes unrelated to the lower bound) to record the result of the substitution \(z \mapsto z^{-1}\):
\begin{equation*}
  W(a(y) w n(x)) \asymp \left( \frac{|y|}{C_2} \right)^{1/2} \int_{z \in k^\times} 1_{|z| = |x|/C_2} z^{-\chi_1} (1 - z)^{\chi_1 - \chi_2 - \alpha} \psi\left(\frac{z y/x}{1-z} \right).
\end{equation*}
\section{The lower bound}
\label{sec:org5058a58}
\subsection{Assumptions}
\label{sec:org638b8ee}
We consider in this section characters whose conductors satisfy
\begin{itemize}
\item \(C_1 > C_2 > 1\),
\item \(C_1\) is a square,
\item \(C_2 < C_1^{1/2}\),
\end{itemize}
and \(x,y\) satisfying
\begin{itemize}
\item \(C_1 (C_2 / |x|)^2 \leq 1\),
\item \(|y/x| (C_2/ |x|)^2 \leq 1\),
\item \(C_2 < |x| < C_1\).
\end{itemize}
\subsection{The model case}
\label{sec:orgfffc2ef}
It is clear that \(x,y\) satisfying the above assumptions exist; for instance, it suffices to require that
\begin{equation}\label{model-case}
  \lvert x \rvert = C_2 C_1^{1/2},  \lvert y \rvert = C_2 C_1^{3/2}.
\end{equation}

\subsection{Calculation}
\label{sec:orgf3b1c43}
Under the above assmptions, for any $z \in k^\times$ with $|z| = C_2/|x|$ one has $|1 - z| = 1$ and
\begin{equation*}
  (1 - z)^{\chi_1 - \chi_2} = \psi(\xi z)
\end{equation*}
for some $\xi \in k$ satisfying
\begin{equation*}
  |\xi| = C(\chi_1 - \chi_2) = C_1.
\end{equation*}
Moveover, by a quadratic Taylor expansion $(1 - z)^{-1} = 1 + z + z^2 ( 1 - z)^{-1}$ (using our assumptions to kill off the contribution of the remainder),
\begin{equation*}
  \psi
  \left(
    - \frac{y/x}{1 - z}
  \right)
  =
  \psi
  (-y/x)
  \psi(- y z / x)
\end{equation*}
and so by the first of the above formulas,
\begin{equation*}
  W(a(y) w n(x))
  \asymp
  \left( \frac{C_2 |y|}{|x|^2} \right)^{1/2}
  \int_{z \in k^\times}
  1_{|z| = C_2/|x|}
  z^{\chi_2} \psi((\xi - y/x) z).
\end{equation*}
The inner integral is a Gauss sum of size
\begin{equation*}
  \int_{z \in k^\times}
  1_{|z| = C_2/|x|}
  z^{\chi_2} \psi((\xi - y/x) z)
  \asymp
  1_{|\xi - y/x| \cdot C_2/|x| = C_2}
  C_2^{-1/2}
  =
  1_{|x \xi - y| = |x|^2} C_2^{-1/2}.
\end{equation*}
From our assumption $|\xi| = C_1 > |x|$ we see that $|x \xi - y| = |x|^2$ implies $|y| = C_1 |x|$ and thus
\begin{equation*}
  W(a(y) w n(x))
  \asymp
  \left( \frac{C_1}{|x|} \right)^{1/2}
  1_{|x \xi - y| = |x|^2}.
\end{equation*}
Choosing $x,y$ as in the model case above which satisfy $|x \xi - y| = |x|^2$ (which is clearly possible), we obtain
\begin{equation*}
  W(a(y) w n(x))
  \asymp
  \left( \frac{C_1^{1/2}}{C_2} \right)^{1/2}.
\end{equation*}

When is this big?  Well, for instance, if $C_2 = 1$ and $C_1$ is large.  But note that if $C_1$ and $C_2$ are the same size (e.g., if the central character is trivial), then the above is actually small, so we don't get values that are obviously large.

\bibliography{refs}{} \bibliographystyle{plain}
\end{document}
