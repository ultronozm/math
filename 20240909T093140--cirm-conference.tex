\documentclass[reqno]{amsart} \usepackage{graphicx, amsmath, amssymb, amsfonts, amsthm, stmaryrd, amscd}
\usepackage[usenames, dvipsnames]{xcolor}
\usepackage{tikz}
% \usepackage{tikzcd}
% \usepackage{comment}

% \let\counterwithout\relax
% \let\counterwithin\relax
% \usepackage{chngcntr}

\usepackage{enumerate}
% \usepackage{enumitem}
% \usepackage{times}
\usepackage[normalem]{ulem}
% \usepackage{minted}
% \usepackage{xypic}
% \usepackage{color}


% \usepackage{silence}
% \WarningFilter{latex}{Label `tocindent-1' multiply defined}
% \WarningFilter{latex}{Label `tocindent0' multiply defined}
% \WarningFilter{latex}{Label `tocindent1' multiply defined}
% \WarningFilter{latex}{Label `tocindent2' multiply defined}
% \WarningFilter{latex}{Label `tocindent3' multiply defined}
\usepackage{hyperref}
% \usepackage{navigator}


% \usepackage{pdfsync}
\usepackage{xparse}


\usepackage[all]{xy}
\usepackage{enumerate}
\usetikzlibrary{matrix,arrows,decorations.pathmorphing}



\makeatletter
\newcommand*{\transpose}{%
  {\mathpalette\@transpose{}}%
}
\newcommand*{\@transpose}[2]{%
  % #1: math style
  % #2: unused
  \raisebox{\depth}{$\m@th#1\intercal$}%
}
\makeatother


\makeatletter
\newcommand*{\da@rightarrow}{\mathchar"0\hexnumber@\symAMSa 4B }
\newcommand*{\da@leftarrow}{\mathchar"0\hexnumber@\symAMSa 4C }
\newcommand*{\xdashrightarrow}[2][]{%
  \mathrel{%
    \mathpalette{\da@xarrow{#1}{#2}{}\da@rightarrow{\,}{}}{}%
  }%
}
\newcommand{\xdashleftarrow}[2][]{%
  \mathrel{%
    \mathpalette{\da@xarrow{#1}{#2}\da@leftarrow{}{}{\,}}{}%
  }%
}
\newcommand*{\da@xarrow}[7]{%
  % #1: below
  % #2: above
  % #3: arrow left
  % #4: arrow right
  % #5: space left 
  % #6: space right
  % #7: math style 
  \sbox0{$\ifx#7\scriptstyle\scriptscriptstyle\else\scriptstyle\fi#5#1#6\m@th$}%
  \sbox2{$\ifx#7\scriptstyle\scriptscriptstyle\else\scriptstyle\fi#5#2#6\m@th$}%
  \sbox4{$#7\dabar@\m@th$}%
  \dimen@=\wd0 %
  \ifdim\wd2 >\dimen@
    \dimen@=\wd2 %   
  \fi
  \count@=2 %
  \def\da@bars{\dabar@\dabar@}%
  \@whiledim\count@\wd4<\dimen@\do{%
    \advance\count@\@ne
    \expandafter\def\expandafter\da@bars\expandafter{%
      \da@bars
      \dabar@ 
    }%
  }%  
  \mathrel{#3}%
  \mathrel{%   
    \mathop{\da@bars}\limits
    \ifx\\#1\\%
    \else
      _{\copy0}%
    \fi
    \ifx\\#2\\%
    \else
      ^{\copy2}%
    \fi
  }%   
  \mathrel{#4}%
}
\makeatother
% \DeclareMathOperator{\rg}{rg}

\usepackage{mathtools}
\DeclarePairedDelimiter{\paren}{(}{)}
\DeclarePairedDelimiter{\abs}{\lvert}{\rvert}
\DeclarePairedDelimiter{\norm}{\lVert}{\rVert}
\DeclarePairedDelimiter{\innerproduct}{\langle}{\rangle}
\newcommand{\Of}[2]{{\operatorname{#1}} {\paren*{#2}}}
\newcommand{\of}[2]{{{{#1}} {\paren*{#2}}}}

\DeclareMathOperator{\Shim}{Shim}
\DeclareMathOperator{\sgn}{sgn}
\DeclareMathOperator{\fdeg}{fdeg}
\DeclareMathOperator{\SL}{SL}
\DeclareMathOperator{\slLie}{\mathfrak{s}\mathfrak{l}}
\DeclareMathOperator{\soLie}{\mathfrak{s}\mathfrak{o}}
\DeclareMathOperator{\spLie}{\mathfrak{s}\mathfrak{p}}
\DeclareMathOperator{\glLie}{\mathfrak{g}\mathfrak{l}}
\newcommand{\pn}[1]{{\color{ForestGreen} \sf PN: [#1]}}
\DeclareMathOperator{\Mp}{Mp}
\DeclareMathOperator{\Mat}{Mat}
\DeclareMathOperator{\GL}{GL}
\DeclareMathOperator{\Gr}{Gr}
\DeclareMathOperator{\GU}{GU}
\def\gl{\mathfrak{g}\mathfrak{l}}
\DeclareMathOperator{\odd}{odd}
\DeclareMathOperator{\even}{even}
\DeclareMathOperator{\GO}{GO}
\DeclareMathOperator{\good}{good}
\DeclareMathOperator{\bad}{bad}
\DeclareMathOperator{\PGO}{PGO}
\DeclareMathOperator{\htt}{ht}
\DeclareMathOperator{\height}{height}
\DeclareMathOperator{\Ass}{Ass}
\DeclareMathOperator{\coheight}{coheight}
\DeclareMathOperator{\GSO}{GSO}
\DeclareMathOperator{\SO}{SO}
\DeclareMathOperator{\so}{\mathfrak{s}\mathfrak{o}}
\DeclareMathOperator{\su}{\mathfrak{s}\mathfrak{u}}
\DeclareMathOperator{\ad}{ad}
% \DeclareMathOperator{\sc}{sc}
\DeclareMathOperator{\Ad}{Ad}
\DeclareMathOperator{\disc}{disc}
\DeclareMathOperator{\inv}{inv}
\DeclareMathOperator{\Pic}{Pic}
\DeclareMathOperator{\uc}{uc}
\DeclareMathOperator{\Cl}{Cl}
\DeclareMathOperator{\Clf}{Clf}
\DeclareMathOperator{\Hom}{Hom}
\DeclareMathOperator{\hol}{hol}
\DeclareMathOperator{\Heis}{Heis}
\DeclareMathOperator{\Haar}{Haar}
\DeclareMathOperator{\h}{h}
\def\sp{\mathfrak{s}\mathfrak{p}}
\DeclareMathOperator{\heis}{\mathfrak{h}\mathfrak{e}\mathfrak{i}\mathfrak{s}}
\DeclareMathOperator{\End}{End}
\DeclareMathOperator{\JL}{JL}
\DeclareMathOperator{\image}{image}
\DeclareMathOperator{\red}{red}
\def\div{\operatorname{div}}
\def\eps{\varepsilon}
\def\cHom{\mathcal{H}\operatorname{om}}
\DeclareMathOperator{\Ops}{Ops}
\DeclareMathOperator{\Symb}{Symb}
\def\boldGL{\mathbf{G}\mathbf{L}}
\def\boldSO{\mathbf{S}\mathbf{O}}
\def\boldU{\mathbf{U}}
\DeclareMathOperator{\hull}{hull}
\DeclareMathOperator{\LL}{LL}
\DeclareMathOperator{\PGL}{PGL}
\DeclareMathOperator{\class}{class}
\DeclareMathOperator{\lcm}{lcm}
\DeclareMathOperator{\spann}{span}
\DeclareMathOperator{\Exp}{Exp}
\DeclareMathOperator{\ext}{ext}
\DeclareMathOperator{\Ext}{Ext}
\DeclareMathOperator{\Tor}{Tor}
\DeclareMathOperator{\et}{et}
\DeclareMathOperator{\tor}{tor}
\DeclareMathOperator{\loc}{loc}
\DeclareMathOperator{\tors}{tors}
\DeclareMathOperator{\pf}{pf}
\DeclareMathOperator{\smooth}{smooth}
\DeclareMathOperator{\prin}{prin}
\DeclareMathOperator{\Kl}{Kl}
\newcommand{\kbar}{\mathchar'26\mkern-9mu k}
\DeclareMathOperator{\der}{der}
% \DeclareMathOperator{\abs}{abs}
\DeclareMathOperator{\Sub}{Sub}
\DeclareMathOperator{\Comp}{Comp}
\DeclareMathOperator{\Err}{Err}
\DeclareMathOperator{\dom}{dom}
\DeclareMathOperator{\radius}{radius}
\DeclareMathOperator{\Fitt}{Fitt}
\DeclareMathOperator{\Sel}{Sel}
\DeclareMathOperator{\rad}{rad}
\DeclareMathOperator{\id}{id}
\DeclareMathOperator{\Center}{Center}
\DeclareMathOperator{\Der}{Der}
\DeclareMathOperator{\U}{U}
% \DeclareMathOperator{\norm}{norm}
\DeclareMathOperator{\trace}{trace}
\DeclareMathOperator{\Equid}{Equid}
\DeclareMathOperator{\Feas}{Feas}
\DeclareMathOperator{\bulk}{bulk}
\DeclareMathOperator{\tail}{tail}
\DeclareMathOperator{\sys}{sys}
\DeclareMathOperator{\atan}{atan}
\DeclareMathOperator{\temp}{temp}
\DeclareMathOperator{\Asai}{Asai}
\DeclareMathOperator{\glob}{glob}
\DeclareMathOperator{\Kuz}{Kuz}
\DeclareMathOperator{\Irr}{Irr}
\newcommand{\rsL}{ \frac{ L^{(R)}(\Pi \times \Sigma, \std, \frac{1}{2})}{L^{(R)}(\Pi \times \Sigma, \Ad, 1)}  }
\DeclareMathOperator{\GSp}{GSp}
\DeclareMathOperator{\PGSp}{PGSp}
\DeclareMathOperator{\BC}{BC}
\DeclareMathOperator{\Ann}{Ann}
\DeclareMathOperator{\Gen}{Gen}
\DeclareMathOperator{\SU}{SU}
\DeclareMathOperator{\PGSU}{PGSU}
% \DeclareMathOperator{\gen}{gen}
\DeclareMathOperator{\PMp}{PMp}
\DeclareMathOperator{\PGMp}{PGMp}
\DeclareMathOperator{\PB}{PB}
\DeclareMathOperator{\ind}{ind}
\DeclareMathOperator{\Jac}{Jac}
\DeclareMathOperator{\jac}{jac}
\DeclareMathOperator{\im}{im}
\DeclareMathOperator{\Aut}{Aut}
\DeclareMathOperator{\Int}{Int}
\DeclareMathOperator{\PSL}{PSL}
\DeclareMathOperator{\co}{co}
\DeclareMathOperator{\irr}{irr}
\DeclareMathOperator{\prim}{prim}
\DeclareMathOperator{\bal}{bal}
\DeclareMathOperator{\baln}{bal}
\DeclareMathOperator{\dist}{dist}
\DeclareMathOperator{\RS}{RS}
\DeclareMathOperator{\Ram}{Ram}
\DeclareMathOperator{\Sob}{Sob}
\DeclareMathOperator{\Sol}{Sol}
\DeclareMathOperator{\soc}{soc}
\DeclareMathOperator{\nt}{nt}
\DeclareMathOperator{\mic}{mic}
\DeclareMathOperator{\Gal}{Gal}
\DeclareMathOperator{\st}{st}
\DeclareMathOperator{\std}{std}
\DeclareMathOperator{\diag}{diag}
\DeclareMathOperator{\Sym}{Sym}
\DeclareMathOperator{\gr}{gr}
\DeclareMathOperator{\aff}{aff}
\DeclareMathOperator{\Dil}{Dil}
\DeclareMathOperator{\Lie}{Lie}
\DeclareMathOperator{\Symp}{Symp}
\DeclareMathOperator{\Stab}{Stab}
\DeclareMathOperator{\St}{St}
\DeclareMathOperator{\stab}{stab}
\DeclareMathOperator{\codim}{codim}
\DeclareMathOperator{\linear}{linear}
\newcommand{\git}{/\!\!/}
\DeclareMathOperator{\geom}{geom}
\DeclareMathOperator{\spec}{spec}
\def\O{\operatorname{O}}
\DeclareMathOperator{\Au}{Aut}
\DeclareMathOperator{\Fix}{Fix}
\DeclareMathOperator{\Opp}{Op}
\DeclareMathOperator{\opp}{op}
\DeclareMathOperator{\Size}{Size}
\DeclareMathOperator{\Save}{Save}
% \DeclareMathOperator{\ker}{ker}
\DeclareMathOperator{\coker}{coker}
\DeclareMathOperator{\sym}{sym}
\DeclareMathOperator{\mean}{mean}
\DeclareMathOperator{\elliptic}{ell}
\DeclareMathOperator{\nilpotent}{nil}
\DeclareMathOperator{\hyperbolic}{hyp}
\DeclareMathOperator{\newvector}{new}
\DeclareMathOperator{\new}{new}
\DeclareMathOperator{\full}{full}
\newcommand{\qr}[2]{\left( \frac{#1}{#2} \right)}
\DeclareMathOperator{\unr}{u}
\DeclareMathOperator{\ram}{ram}
% \DeclareMathOperator{\len}{len}
\DeclareMathOperator{\fin}{fin}
\DeclareMathOperator{\cusp}{cusp}
\DeclareMathOperator{\curv}{curv}
\DeclareMathOperator{\rank}{rank}
\DeclareMathOperator{\rk}{rk}
\DeclareMathOperator{\pr}{pr}
\DeclareMathOperator{\Transform}{Transform}
\DeclareMathOperator{\mult}{mult}
\DeclareMathOperator{\Eis}{Eis}
\DeclareMathOperator{\reg}{reg}
\DeclareMathOperator{\sing}{sing}
\DeclareMathOperator{\alt}{alt}
\DeclareMathOperator{\irreg}{irreg}
\DeclareMathOperator{\sreg}{sreg}
\DeclareMathOperator{\Wd}{Wd}
\DeclareMathOperator{\Weil}{Weil}
\DeclareMathOperator{\Th}{Th}
\DeclareMathOperator{\Sp}{Sp}
\DeclareMathOperator{\Ind}{Ind}
\DeclareMathOperator{\Res}{Res}
\DeclareMathOperator{\ini}{in}
\DeclareMathOperator{\ord}{ord}
\DeclareMathOperator{\osc}{osc}
\DeclareMathOperator{\fluc}{fluc}
\DeclareMathOperator{\size}{size}
\DeclareMathOperator{\ann}{ann}
\DeclareMathOperator{\equ}{eq}
\DeclareMathOperator{\res}{res}
\DeclareMathOperator{\pt}{pt}
\DeclareMathOperator{\src}{source}
\DeclareMathOperator{\Zcl}{Zcl}
\DeclareMathOperator{\Func}{Func}
\DeclareMathOperator{\Map}{Map}
\DeclareMathOperator{\Frac}{Frac}
\DeclareMathOperator{\Frob}{Frob}
\DeclareMathOperator{\ev}{eval}
\DeclareMathOperator{\pv}{pv}
\DeclareMathOperator{\eval}{eval}
\DeclareMathOperator{\Spec}{Spec}
\DeclareMathOperator{\Speh}{Speh}
\DeclareMathOperator{\Spin}{Spin}
\DeclareMathOperator{\GSpin}{GSpin}
\DeclareMathOperator{\Specm}{Specm}
\DeclareMathOperator{\Sphere}{Sphere}
\DeclareMathOperator{\Sqq}{Sq}
\DeclareMathOperator{\Ball}{Ball}
\DeclareMathOperator\Cond{\operatorname{Cond}}
\DeclareMathOperator\proj{\operatorname{proj}}
\DeclareMathOperator\Swan{\operatorname{Swan}}
\DeclareMathOperator{\Proj}{Proj}
\DeclareMathOperator{\bPB}{{\mathbf P}{\mathbf B}}
\DeclareMathOperator{\Projm}{Projm}
\DeclareMathOperator{\Tr}{Tr}
\DeclareMathOperator{\Type}{Type}
\DeclareMathOperator{\Prop}{Prop}
\DeclareMathOperator{\vol}{vol}
\DeclareMathOperator{\covol}{covol}
\DeclareMathOperator{\Rep}{Rep}
\DeclareMathOperator{\Cent}{Cent}
\DeclareMathOperator{\val}{val}
\DeclareMathOperator{\area}{area}
\DeclareMathOperator{\nr}{nr}
\DeclareMathOperator{\CM}{CM}
\DeclareMathOperator{\CH}{CH}
\DeclareMathOperator{\tr}{tr}
\DeclareMathOperator{\characteristic}{char}
\DeclareMathOperator{\supp}{supp}


\theoremstyle{plain} \newtheorem{theorem} {Theorem} \newtheorem{conjecture} [theorem] {Conjecture} \newtheorem{corollary} [theorem] {Corollary} \newtheorem{proposition} [theorem] {Proposition} \newtheorem{fact} [theorem] {Fact}
\theoremstyle{definition} \newtheorem{definition} [theorem] {Definition} \newtheorem{hypothesis} [theorem] {Hypothesis} \newtheorem{assumptions} [theorem] {Assumptions}
\newtheorem{example} [theorem] {Example}
\newtheorem{assertion}[theorem] {Assertion}
\newtheorem{note}[theorem] {Note}
\newtheorem{conclusion}[theorem] {Conclusion}
\newtheorem{claim}            {Claim}
\newtheorem{homework} {Homework}
\newtheorem{exercise} {Exercise}  \newtheorem{question}[theorem] {Question}    \newtheorem{answer} {Answer}  \newtheorem{problem} {Problem}    \newtheorem{remark} [theorem] {Remark}
\newtheorem{notation} [theorem]           {Notation}
\newtheorem{terminology}[theorem]            {Terminology}
\newtheorem{convention}[theorem]            {Convention}
\newtheorem{motivation}[theorem]            {Motivation}


\newtheoremstyle{itplain} % name
{6pt}                    % Space above
{5pt\topsep}                    % Space below
{\itshape}                   % Body font
{}                           % Indent amount
{\itshape}                   % Theorem head font
{.}                          % Punctuation after theorem head
{5pt plus 1pt minus 1pt}                       % Space after theorem head
% {.5em}                       % Space after theorem head
{}  % Theorem head spec (can be left empty, meaning ‘normal’)

% \theoremstyle{mytheoremstyle}


\theoremstyle{itplain} %--default
% \theoremheaderfont{\itshape}
% \newtheorem{lemma}{Lemma}
\newtheorem{lemma}[theorem]{Lemma}
% \newtheorem{lemma}{Lemma}[subsubsection]

\newtheorem*{lemma*}{Lemma}
\newtheorem*{proposition*}{Proposition}
\newtheorem*{definition*}{Definition}
\newtheorem*{example*}{Example}

\newtheorem*{results*}{Results}
\newtheorem{results} [theorem] {Results}


\usepackage[displaymath,textmath,sections,graphics]{preview}
\PreviewEnvironment{align*}
\PreviewEnvironment{multline*}
\PreviewEnvironment{tabular}
\PreviewEnvironment{verbatim}
\PreviewEnvironment{lstlisting}
\PreviewEnvironment*{frame}
\PreviewEnvironment*{alert}
\PreviewEnvironment*{emph}
\PreviewEnvironment*{textbf}



\begin{document}

\title{Notes from CIRM conference BB6, \emph{Automorphic Forms and Related Topics}} 

\begin{abstract}
  Random notes from the CIRM conference, Building Bridges 6th, EU/US Workshop on Automorphic Forms and Related Topics (BB6),
  9-13 September, 2024.  These notes are incomplete and have not been proofread.  Any errors should be assumed to be due to the note-taker.
\end{abstract}


\section{Anne-Maria Ernvall-hytönen, \textnormal{\emph{Lattices and modular forms in coset coding}}}

Wyner (1975).  Wiretap channel (Alice, Bob, Eve).  Gaussian noise with variance $\sigma^2$ (between Alice and Bob) and $\sigma_e^2$ (for Eve).  We assume $\sigma_e > \sigma$.  We would like to do this whole scheme in such a way that we can use the noise so that Bob can still receive the image, but Eve cannot.  We aim to do this with lattices $\Lambda \geq \Lambda_e$, where each coset corresponds to a code-letter.

Belfiore and Oggier (2010) (maybe \cite{MR4032962}?): secrecy gain.  Transmit codeword $x \in \mathbb{R}^n$.
\begin{equation*}
  \frac{1}{(\sigma_\Lambda \sqrt{2 \pi})^n}
  \int_{V_{\Lambda(x)}}
  e^{- \lVert y - x \rVert^2 / 2 \sigma^2} \, d y.
\end{equation*}

\begin{equation*}
  \frac{1}{(\sigma_\Lambda \sqrt{2 \pi})^n}
  \sum_{t \in \Lambda_e \cap R}
  \int_{V_{\Lambda(x + t)}}
  e^{- \lVert y - x \rVert^2 / 2 \sigma^2} \, d y.
\end{equation*}

End up trying to minimize
\begin{equation*}
  \theta_{\Lambda_e} \left( - \frac{1}{2 \pi i \sigma_e^2} \right).
\end{equation*}
Secrecy function
\begin{equation*}
  \Xi(y) :=
  \frac{\theta_{\lambda \mathbb{Z}^n}(y i)}{ \theta_\Lambda(y i)}.
\end{equation*}
Belfiore and Sole: For unimodular $\Lambda$, maximum at $y = 1$.  Even unimodular: polynomials in
\begin{equation*}
  E_4 = \frac{1}{2} \left( \vartheta_2^8 + \vartheta_3^8 + \vartheta_4^8  \right),
  \qquad
  \Lambda = \frac{1}{256} \vartheta_2^8 \vartheta_3^8 \vartheta_4^8.
\end{equation*}
Inverse of $\Xi(y)$ a polynomial in $\frac{\vartheta_4^4 \vartheta_2^4}{\vartheta_3^8}$.

(Maybe \cite{MR2966067} is a reference.)

$\ell$-modular: $1 / \sqrt{\ell}$, $\mathbb{Z} \oplus \sqrt{2} \mathbb{Z} \oplus 2 \mathbb{Z}$.

The connection to deeper mathematics is, what kinds of representations as polynomials do you have for various theta functions?

\section{Jolanta Marzec-ballesteros, \textnormal{\emph{Doubling method for self-dual linear codes}}}

Garrett, Piatetski-Shapiro--Rallis (1980's).  Integral of cusp form against restriction of Siegel-type Eisenstein series equals $L$-function attached to cusp form times cusp form or Eisenstein series attached to cusp form:
\begin{equation*}
  \left\langle E \left(
      \begin{pmatrix}
        g &  \\
          & g' \\
      \end{pmatrix}, s \right),
    f(g)\right\rangle
  = L(f, s) f(g ').
\end{equation*}

Done for $G$ symplectic, orthogonal, unitary over global field, also for congreunce subgroups.

Let's start with an overview.  Let $f$ be a cusp form on $G$, and $H$ a subgroup for which $G \times G \hookrightarrow H$.  Then form an Eisenstein series on $H$
\begin{equation*}
  E(h, s) = \sum_{\gamma \in P \backslash H} \phi(\gamma h, s).
\end{equation*}
Restrict to $h = \diag(g, g')$ and take inner product with $F(g)$.  This leads to an unfolding involving a sum over $\gamma \in P \backslash H /(G \times G)$.  In favorable cases, only one representative $\gamma_0$ contributes, leaving us with
\begin{align*}
  &\sum_{(k, k') \in G \times G}
    \left\langle \phi \left( \gamma_0
    \begin{pmatrix}
      k g        &  \\
                 & k ' g ' \\
    \end{pmatrix}, s \right), f(g) \right\rangle
  \\
  &=
    \sum_{\beta \in \gamma_0(G \times 1)}
    \psi(f) f |_\beta(g ') \\
  &= L(f, s) f(g ').
\end{align*}
We would like to do something similar, but now over finite fields.

A \emph{linear code of length} $2 n$ over a finite field $\mathbb{F}$ is a linear subspace $C \subseteq \mathbb{F}^{2 n}$.  We denote by $\langle , \rangle : C \times C \rightarrow \mathbb{F}$ the Euclidean inner product.  We say that $C$ is \emph{self-dual} if
\begin{equation*}
  C = C^\perp := \left\{ v \in \mathbb{F}^{2 n} : \langle v, C \rangle = 0 \right\}.
\end{equation*}
Then (the length $2 n$ is even and) $\dim C = n$.

The \emph{weight} of a codeword $c =(c_1, \dotsc, c_{2 n}) \in C$ is
\begin{equation*}
  \operatorname{wt}c = \# \left\{ i \in \{1, \dotsc, 2 n\} : c_i \neq 0 \right\}.
\end{equation*}

\emph{Weight enumerators} are certain homogeneous polynomials of degree $2 n$ in variables from the set $V = \left\{ x_\alpha : \alpha \in \mathbb{F}^g \right\}$, where $g \in \mathbb{N}$ is the genus.

The \emph{genus one weight enumerator} of a code $C \subseteq \mathbb{F}_2^{2 n}$ is a polynomial
\begin{equation*}
  W_1(C,(x_0, x_1))
  = \sum_{c \in C} x_0^{2 n - \operatorname{wt} c}
  x_1^{\operatorname{wt} c}.
\end{equation*}
The \emph{genus} $g$ \emph{weight enumerator} of a code $C \subseteq \mathbb{F}^{2 n}$ is a polynomial
\begin{equation*}
  W_g(C, x) = \sum_{(c^1, \dotsc, c^g) \in C^g }
  \prod_{\alpha \in \mathbb{F}^g}
  x_\alpha^{w_\alpha(c^1, \dotsc, c^g)}
\end{equation*}
of degree $2 n$, where $x =(x_\alpha)_{\alpha \in \mathbb{F}^g}$ and
\begin{equation*}
  w_\alpha(c^1, \dotsc, c^g)
  = \# \left\{ \text{rows $r$ in } (c_{i}^j)_{i = 1..2n}^{j = 1..g} : r = \alpha \right\}.
\end{equation*}

As an example, we give a basis for a Hamming code $H_8$ 5nof weight $8$, the span over $\mathbb{F}_2$ inside $\mathbb{F}_2^8$ of the vectors
\begin{equation*}
  \begin{pmatrix}
    1 \\ 0 \\ 0 \\ 1 \\ 0 \\ 1 \\ 1 \\ 0
  \end{pmatrix}
  \quad
  \begin{pmatrix}
    0 \\ 1 \\ 0 \\ 1 \\ 0 \\ 1 \\ 0 \\ 1
  \end{pmatrix}
  \quad
  \begin{pmatrix}
    0 \\ 0 \\ 1 \\ 1 \\ 0 \\ 0 \\ 1 \\ 1
  \end{pmatrix}
  \quad
  \begin{pmatrix}
    0 \\ 0 \\ 0 \\ 0 \\ 1 \\ 1 \\ 1 \\ 1
  \end{pmatrix}.
\end{equation*}
Then
\begin{align*}
  W_2(H_8,(x_{00}, x_{01}, x_{10}, x_{11}))
  &= \sum_{\alpha \in \mathbb{F}_2^2}
  x_\alpha^8 + 14
  \sum_{
    \substack{
      \alpha_1, \alpha_2 \in \mathbb{F}_2^2  \\
      \alpha_1 < \alpha_2      
    }
  }
  x_{\alpha_1}^4 x_{\alpha_2}^4
  + 168 x_{00}^2 x_{0 1}^2 x_{10}^2 x_{11}^2
  \\
  &=(8) +
    14(4, 4)
    + 168(2, 2, 2, 2).
\end{align*}
In general,
\begin{equation*}
  W_g(C) = \sum_A b_A \cdot(A),
\end{equation*}
where
\begin{equation*}
  A \in \left\{(a_0, \dotsc, a_{2^g - 1}) : \text{admissible tuples},
    \,
    \sum_{i = 0}^{2^g - 1} a_i = 2 n\right\}.
\end{equation*}

Some analogies with modular forms:
\begin{itemize}
\item $W_g(C)$ is like a modular form $f$ of genus $g$,
\item $\sum_A b_A \cdot(A)$ is like a Fourier expansion,
\item $(2 n)$ is like a constant term $a(0)$.
\end{itemize}
EXamples of cusp forms:
\begin{itemize}
\item $W_1(G_{24}) - W_1(H_8 \times H_8 \times H_8)
  = - 42(20, 4) + 168(16, 8) - 252(12, 12)$ is a cusp form of genus one.
\item $W_3(E_{16}) - W_3(H_8 \times H_8) = - 2688(9, 1, 1, 1, 1, 1, 1, 1) + \dotsb$ is a cusp form of genus $3$.
\end{itemize}

\begin{theorem}[Runge, 1996; Nebe, Rains, Sloane, 2006]
  We have
  \begin{equation*}
    \left\langle W_g(C) : \text{$C$ self-dual, over $\mathbb{F}$} \right\rangle
    =
    \left( \mathbb{C}[x_\alpha : \alpha \in \mathbb{F}^g] \right)^{\mathcal{C}_g},
  \end{equation*}
  where
  \begin{equation*}
    \mathcal{C}_g := \left\langle m_r, d_\phi, h_{\iota, u_{\iota}, v_{\iota}} : r \in \GL_g(\mathbb{F}), \phi, \iota \right\rangle
  \end{equation*}
  with
  \begin{equation*}
    m_r : x_\alpha \mapsto x_{r \alpha},
  \end{equation*}
  \begin{equation*}
    d_\phi : x_\alpha \mapsto e^{2 \pi i \phi(\alpha)} x_\alpha,
  \end{equation*}
  \begin{equation*}
    h_{\iota, u_\iota, v_\iota}
    : x_\alpha \mapsto \left( \# \iota \mathbb{F}^g \right)^{- 1/2}
    \sum_{w \in \iota \mathbb{F}^g}
    e^{\frac{2 \pi i}{p} \left\langle w, v_\iota \alpha \right\rangle} x_w + \dotsb.
  \end{equation*}
\end{theorem}
Consider the mean polynomial (``Siegel-type Eisenstein series'')
\begin{align*}
  M_{2 g}((2 n)) &= \sum_{\gamma \in P_{2 g} \backslash \mathcal{C}_{2 g}}
  (2 n)^\gamma
  \\
  &=
  \sum_{\gamma \in P_{2 g} \backslash \mathcal{C}_{2 g}} \sum_{\alpha \in \mathbb{F}^{2 g}}
  \left((x_\alpha)^\gamma \right)^{2 n}
  = \mathrm{const} \sum_{
    \substack{
      C \subset \mathbb{F}^{2 n}  \\
      \text{fixed type}      
    }
  }
  W_{2 g}(C, x),
\end{align*}
and an inner product defined on monomials.

What we prove with Bourganis in 2024 is the following:
\begin{theorem}
  Let $\mathcal{T}$ be a family of self-dual codes of length $2 n$ over a field $\mathbb{F}$.  Assume either that $\mathbb{F}$ has odd characteristic or is equal to $\mathbb{F}_2$.  Let $C \in \mathcal{T}$ be doubly-even, and fix $g \in \mathbb{N}$.  Then there exists an (explicit) constant $C$ such that for a cusp form $f \in \mathcal{T}$ of genus $r$, with $\deg f = 2 n$, we have
  \begin{equation*}
    \left\langle M_{2 g}((2 n))(x y), f(x) \right\rangle
    =
    \begin{cases}
      0      & \text{ if } r < g, \\
      C \cdot f(y)             & \text{ if } r = g
    \end{cases}.
  \end{equation*}
\end{theorem}

\section{Petru Constantinescu, \textnormal{\emph{Non-vanishing of geodesic periods of automorphic forms}}}
Preprint: \cite{2024arXiv2404.12982}, joint with Asbhj{\o}rn Nordentoft.

Class groups:
\begin{itemize}
\item $\Gamma = \PSL_2(\mathbb{Z})$, $K = \mathbb{Q}(\sqrt{D})$,
\item $\mathrm{Cl}_K$ class group, $h(D) = h_K = \lvert \mathrm{Cl}_K \rvert$ class number,
\item $\mathcal{Q}_D$: set of primitive integral binary quadratic forms of discriminant $D$,
  \begin{equation*}
    \mathcal{Q}_D = \left\{ Q(x, y)
      = a x^2 + b x y + c y^2 :(a, b, c) = 1,
      \,
      b^2 - 4 a c = D\right\}.
  \end{equation*}
\item Gauss: $\Gamma \circlearrowright \mathcal{Q}_D,$
  \begin{equation*}
    (Q . \gamma)
    \begin{pmatrix}
      x      \\
      y
    \end{pmatrix}
    = Q \left( \gamma
      \begin{pmatrix}
        x        \\
        y  \\
      \end{pmatrix} \right),
  \end{equation*}
  
\item isomorphism
  \begin{equation*}
    \mathrm{Cl}_K \xrightarrow{\cong} \Gamma \backslash \mathcal{Q}_D
  \end{equation*}
  \begin{equation*}
    A \mapsto[a, b, c].
  \end{equation*}
\end{itemize}

Heegner points and closed geodesics:
\begin{itemize}
\item $D < 0$: $A \in \mathrm{Cl_K} \rightsquigarrow$ Heegner point $z_A \in \Gamma \backslash \mathbb{H}$,
  \begin{equation*}
    [a, b, c] \rightsquigarrow \frac{- b - i \sqrt{\lvert D \rvert}}{2 a}.
  \end{equation*}
  \begin{equation*}
    h(D) = \lvert \mathrm{Cl}_K \rvert = \lvert D \rvert^{1/2 + o(1)}.
  \end{equation*}
\item $D > 0$: $A \in \mathrm{Cl}_K^+ \rightsquigarrow $ closed geodesic $C_A \subset \Gamma \backslash \mathbb{H}$,
  $[a, b, c] \rightsquigarrow$ semicircle with endpoints $\frac{- b \pm \sqrt{D}}{2 a}$.
  \begin{equation*}
    h(D) \log \eps_D = D^{1/2 + o(1)},
  \end{equation*}
  \begin{equation*}
    I(C_A) = 2 \log \eps_D.
  \end{equation*}
\end{itemize}

\begin{theorem}[Duke '88]
  Fix $\Omega \subset \PSL_2(\mathbb{Z}) \backslash \mathbb{H}$.

  \begin{itemize}
  \item $D < 0$: equidistribution of $z_A$, $A \in \Cl_{\mathbb{Q}(\sqrt{D})}$.
  \item $D > 0$: similar for closed geodesics.
  \end{itemize}
\end{theorem}


\textbf{Waldspurger formulas}.  Let $f$ be a nonzero Maass form.  Our goal is to study closed geodesic periods $\int_{C_A} f(z) \, \frac{\lvert d z \rvert}{y}$.

$\chi \in \widehat{\Cl_K} \rightsquigarrow $ $\theta_\chi$, the associated theta series (weight one on $\Gamma_0(D)$, nebentypus $\chi_D$).

\begin{theorem}[Waldspurger/Zhang/Popa]
  Let $D$ be a fundamental discriminant.  For $D < 0$,
  \begin{equation*}
    L(f \times \theta_\chi, \tfrac{1}{2})
    = \frac{C_f}{ D^{1/2}}
    \left| \sum_{A \in \Cl_K} \chi(A) f(z_A) \right|^2.
  \end{equation*}
  For $D > 0$,
  \begin{equation*}
    L(f \times \theta_\chi, \tfrac{1}{2}) = \frac{C_f}{ D^{1/2}}
    \left| \sum_{A \in \Cl_K^+} \chi(A) \int_{C_A} f(z)
      \, \frac{\lvert d z \rvert}{y}
    \right|^2.
  \end{equation*}
\end{theorem}

\begin{theorem}[Michel--Venkatesh '05]
  Let $\delta = 1/2700$.  For $D < 0$:
  \begin{equation*}
    \left| \left\{ \chi \in \widehat{\Cl_K} : L(f \times \theta_\chi, \tfrac{1}{2}) \neq 0 \right\} \right|
    \gg D^\delta.
  \end{equation*}
\end{theorem}
\begin{proof}[Sketch of proof]
  By orthogonality of characters,
  \begin{equation*}
    \frac{1}{h(D)}
    \sum_{\chi \in \widehat{\Cl_K}}
    L(f \times \theta_\chi, \tfrac{1}{2})
    = \frac{c_f}{ D^{1/2}}
    \sum_{A \in \Cl_K} \left| \dotsb \right|^2.
  \end{equation*}
  This converges by Duke's equidistribution theorem.  The conclusion then follows from subconvexity for Rankin--Selberg $L$-functions, due to Harcos--Michel.
\end{proof}

Same proof does not work for geodesics, cannot apply equidistribution and relate to subconvexity (square is outside integral).  

\begin{question}[Michel, Oberwolfach 2020]
  Let $K$ be a real quadratic field of discriminant $D > 0$, and assume that $h_K  \gg D^\delta$.  Does there exist $A \in \Cl_K^+$ such that
  \begin{equation*}
    \int_{C_A} f \, \frac{\lvert d z \rvert}{y} \neq 0?
  \end{equation*}
  Equivalently, does there exist $\chi \in \widehat{\Cl_K^+}$ such that $L(f \times \theta_\chi, \tfrac{1}{2}) \neq 0$?
\end{question}

\textbf{The prime geodesic theorem}.  Let $D > 0, A \in \Cl_K^+$.  This gives rise to a closed geodesic $C_A$, with $I(C_A) = 2 \log \eps_D$.  Sound--Young have the best estimate for their count.

\begin{theorem}[C--Nordentoft 2024]
  Let $f$ be a nonzero Maass form for $\SL_2(\mathbb{Z})$.  Then
  \begin{equation*}
    \# \left\{ C \in \mathcal{C}(X) :
      \int_{C}
      f(z)
      \, \frac{\lvert d z \rvert}{y} =  0\right\}
    \ll
    \frac{X}{(\log X)^{5/4}}.
  \end{equation*}
\end{theorem}

\begin{remark}
  We also obtain 100\% non-vanishing for periods of weight $k$ holomorphic cusp forms, for any Fuchsian group $\Gamma$.
\end{remark}

\begin{theorem}[C--Nordentoft 2024]
  For a positive proportion of positive discriminants $D > 0$ with $\eps_\Delta \leq X$, we get that there exists $\chi \in \widehat{\Cl_{\mathbb{Q}(\sqrt{D})}}$
  with $L(f \times \theta_\chi, \tfrac{1}{2} \neq 0)$.
\end{theorem}

We construct a bipartite graph on $X_N$ (double cosets in $\Gamma_\infty \backslash \Gamma / \Gamma_\infty$ with $c \leq N$) times $Y_N$ (conjugacy classes with trace bounded in magnitude by $N$).  This graph relates closed geodesic and vertical geodesics.


\section{An excised orthogonal model for families of cusp forms}
Talk by Zoe Batterman (Abstract), Akash Narayanan, Christopher Yao.  Joint with Owen Barrett, Aditya Jambhale, and Kishan Sharma.  Preprint: \cite{2024arXiv2407.14526}.

Conjecture (Montgomery--Dyson, 1970's): zeros of of zeta vs. GUE.

2005: S.J.\ Miller noticed a repulsion of the lowest-lying zeros near the central point of a family of even twists of a fixed elliptic curve $L$-function with finite conductor.

2011: Duenez, Huynh, Keating, Miller, Snaith: proposed an excised orthogonal model to capture the behavior of this repulsion.

\begin{question}
  How accurately do egienvalues of random matrices from classical compact groups model the lowest-lying zeros of families of $L$-functions associated to a cuspidal newform?
\end{question}

Let
\begin{equation*}
  S_k^{\new}(M, \chi_f) \ni f(z) = \sum_{n = 1}^\infty a_f(n) e^{2 \pi i n z},
\end{equation*}
$\lambda_f(n) = a_f(n) / n^{(k - 1)/2}$.
\begin{equation*}
  L(s, f) = \sum_{n \geq 1} \lambda_f(n) n^{- s},
\end{equation*}
Various specific families of twists $L(f \otimes \psi_d, s)$, match with classical compact groups:
\begin{itemize}
\item principal character, even twists vs.\ $\SO(\mathrm{even})$
\item principal character, odd twists vs.\ $\SO(\mathrm{odd})$
\item non-principal character, self-dual vs.\ $\Sp$
\item generic vs.\ $\U$
\end{itemize}

Pictures.

\section{On an extension of the Rohrlich-Jensen formula, \textnormal{\emph{Lejla Smajlović}}}
Joint work with James Cogdell, Jay Jorgensen.  Preprint: \cite{2021arXiv2101.09599}.

What is a Poisson--Jensen formula?  We will view it as a way to characterize meromorphic functions in terms of their divisors.  Some notation:
\begin{itemize}
\item $D_R = \left\{ z = x + i y \in \mathbb{C} : \lvert z \rvert < R \right\}$
\item $F$: a non-constant meromorphic function on $\overline{D_R}$,
  \begin{equation*}
    F(z) = c_F z^m + \O(z^{m + 1}), \quad z \rightarrow 0.
  \end{equation*}
\end{itemize}
Then
\begin{equation*}
  \int_0^{2 \pi}
  \log \lvert F(R e^{i \theta}) \rvert
  \, \frac{d \theta}{2 \pi}
  + \sum _{D_R} \dotsb = F(0).
\end{equation*}

Rohrlich, 1980's: a \emph{modular} generalization, characterizing modular forms via divisors.  Given $f$, a meromorphic function on $\mathbb{H}$ that is invariant by $\PSL(2,\mathbb{Z})$.  Assume $f$ is holomorphic at the cusp and that the Fourier expansion of $f$ at $\infty$ has constant term equal to one.  Then
\begin{equation*}
  \int_{\PSL(2,\mathbb{Z}) \backslash \mathbb{H}}
  \log \lvert f(z) \rvert
  \, \frac{d \mu(z)}{2 \pi}
  + \sum_{w \in \mathcal{F}}
  \frac{\ord_w(f)}{ \ord(w)} P(w) = 0.
\end{equation*}
Here
\begin{itemize}
\item $\ord_w(f)$ is the order of $f$ at $w$ as a meromorphic function,
\item $\ord(w)$ denotes the order of the \emph{point} $w$ with respect to the action of $\PSL(2,\mathbb{Z})$ on $\mathbb{H}$, and
\item $P(w) = \log \left( \lvert \eta(w) \rvert^4 \cdot \Im(w) \right)$ is the Kronecker limit function associated to the parabolic Eisenstein series on $\PSL(2,\mathbb{Z})$.  This is the function appearing as the next-order term in the expansion of the Eisenstein series as $s \rightarrow 1$.
\end{itemize}
Another way to interpret this formula is as follows.  We have
\begin{equation*}
  \left\langle 1, \log \lvert f(z) \rvert \right\rangle
  = \lim_{Y \rightarrow \infty} \int_{\mathcal{F}(Y)} 1 \cdot \log \lvert f(z) \rvert \, d \mu(z),
\end{equation*}
hence the formula reads
\begin{equation*}
  \left\langle 1, \log \lvert f(z) \rvert \right\rangle = - 2 \pi
  \sum_{w \in \mathcal{F}} \frac{\ord_w(f)}{ \ord(w)} P(w).
\end{equation*}
Here $\log \lvert f(z) \rvert$ can be replaced by
\begin{equation*}
  \log \lVert f(z) \rVert = \log \left( \Im z^k \lvert f(z) \rvert \right)
\end{equation*}
for weight $2 k$ meromorphic modular forms.  Generalizations:
\begin{itemize}
\item to other Fuchsian groups of the first kind, by Rohrlich.
\item to hyperbolic $3$-space, by Herrero, Imamoglu, von Pippich, Toth.
\end{itemize}

Further modular generalization by Bringman and Kane.  Keep the modular group setting, but evaluate more general inner products.
\begin{itemize}
\item $j(z) = q^{-1} + 744 + \O(q)$: Hauptmodul
\item $j_1(z) := j(z) - 744$
\item $j_n(z) := j_1 | T_n(z)$, for $n \geq 2$
\item $f$: weight $2 k$ meromorphic modular form with respect to $\PSL(2, \mathbb{Z})$, normalized so that $f(z) = 1 + \O(q)$
\end{itemize}
They evaluated the regularized scalar product in terms of the divisor of $f$, proving that
\begin{equation*}
  \left\langle j_n(z),
    \log \left((\Im(z))^k \lvert f(z) \rvert \right)\right\rangle
  = - 2 \pi \sum_{w \in \mathcal{F}}
  \frac{\ord_w(f)}{\ord(w)}
  \mathbf{j}_n(w)
  + \frac{k}{6} c_n,
\end{equation*}
where $\mathbf{j}_n$ is characterized in terms of differential operators by Bringman and Kane; application of our results yields a different expression for the same function, namely
\begin{equation*}
  \mathbf{j}_n(w) = 2 \pi \sqrt{n} \partial_s F_{- n}^{\PSL(2, \mathbb{Z})}(w, s) |_{s=1}
  - 24 \sigma(n) P(w).
\end{equation*}

What is our main goal?
\begin{itemize}
\item To extend the point of view that the Rohrlich--Jensen formula is the evaluation of a particular type of inner product.
\item To prove the extension of this formula in the setting of an arbitrary, not necessarily arithmetic, Fuchsian group of the first kind with one cusp.
\end{itemize}

We start with $j_n(z) = j_1 | T_n(z)$, which is the unique (up to constants) holomorphic function that is $\PSL(2,\mathbb{Z})$-invariant on $\mathbb{H}$ and whose expansion near $\infty$ is $q^{- n} + o(q^{-1})$.

These properties hold for the special value $s = 1$ of the Niebur--Poincar{\'e} series $F_{- n}^{\Gamma}(z, s)$, defined for any Fuchsian group $\Gamma$ of the first kind with oen cusp by
\begin{equation*}
  F_m^\Gamma(z, s)
  = \sum_{\gamma_\infty \backslash \Gamma}
  e \left( m \Re(\gamma z) \right) \left( \Im(\gamma z) \right)^{1/2} I_{s - 1/2} \left( 2 \pi \lvert m \rvert \Im(\gamma z) \right),
\end{equation*}
for $m \neq 0$.

It is an eigenfucntion of the hyperbolic Laplacian, and may be expressed in terms of $j_m$.

The term $\log \left( \lVert f(z) \rVert \right)$ is a bit complicated, involving findings by Jorgensen, von Pippich and the speaker, plus some additional work done in our paper.

\begin{proposition}
  Let $\Gamma$ be a cofinite Fuchsian group with one cusp at $\infty$ with identity as its scaling matrix.  Let $2 k \geq 0$ be an even integer, and let $f$ be aw eight $2 k$ meromorphic form which is $\Gamma$-invariant and with $q$-expansion at $\infty$ normalized so its constant term is equal to one.  Then, we can express $\log \left( \lVert f \rVert(z) \right)$ in terms of parabolic Eisenstein series and Green's functions, as
  \begin{equation*}
    - 2 k + 2 \pi \sum_{w \in \mathcal{F}_\Gamma}
    \frac{\ord_w(f)}{ \ord(w)} \lim_{s \rightarrow 1} \left( G_s^\Gamma(z, w) + \mathcal{E}_{\Gamma, \infty}^{\mathrm{par}(z, s)} \right)
    + \dotsb.
  \end{equation*}
\end{proposition}

Here
\begin{equation*}
  \mathcal{E}_\infty^{\mathrm{par}}(z, s) = \sum_{\Gamma_\infty \backslash \Gamma} \Im(\gamma z)^s.
\end{equation*}
The Kronecker limit formula says that
\begin{equation*}
  \mathcal{E}_\infty^{\mathrm{par}}(z, s) = \frac{1}{(s-1) \vol(\Gamma \backslash \mathbb{H})}
  + \beta - \frac{1}{\vol(\Gamma \backslash \mathbb{H})}
  \log \left( \lvert \eta_\infty^4(z) \rvert \Im(z) \right)
  + \O(s - 1).
\end{equation*}
Then we need to define $\beta$, and the Green's function $G_s(z, w)$, which is obtained by averaging the kernel $k_s(z, w)$; both functions are eigenfunctions of the Laplacian with eigenvalue $s(1 - s)$, and the Green's function has a specified singularity on the diagonal.  The Green's function also has a Laurent series expansion as $s \rightarrow 1$ that involves the parabolic Eisenstein series.  In particular,
\begin{equation*}
  \lim_{s \rightarrow 1} \left( G_s^\Gamma(z, w) + \mathcal{E}_{\Gamma, \infty}^{\mathrm{par}}(z, w) \right)
\end{equation*}
exists, with a logarithmic singularity on the diagonal.

The Rohrlich--Jensen formula can be understood through the study of the regularized inner product of this last limit with $F_{- n}(\cdot, 1)$.  Regularization is needed only in the cusp (because the logarithmic singularity is integrable).  We can thus write
\begin{align*}
  &\left\langle F_{- n}(\cdot, 1),
    \overline{ \lim_{s \rightarrow 1} \left( G_s^\Gamma(z, w) + \mathcal{E}_{\Gamma, \infty}^{\mathrm{par}}(z, w) \right)}
    \right\rangle \\
  &=
    \lim_{Y \rightarrow \infty}
    \int_{\mathcal{F}(Y)}
    F_{- n}(z, 1) \lim_{s \rightarrow 1} \left( G_s(z, w) + \mathcal{E}_\infty^{\mathrm{par}}(z, s) \right)
    \, d \mu(z).
\end{align*}
A key observation is that all terms are eigenfunctions of the Laplacian, hence one can seek to compute the inner product in a manner similar to that which yields the Maass--Selberg formula.  The key identity is that
\begin{align*}
  &\int_{\mathcal{F}(Y)}
  F_{- n}(z, 1) \lim_{s \rightarrow 1} \left( G_s(z, w) + \mathcal{E}_\infty^{\mathrm{par}}(z, s) \right)
    \, d \mu(z) \\
  &=
    \partial_s \left( - s(1 - s) \int_{\mathcal{F}(Y)}
    F_{- n}(z, 1) \left( G_s(z, w) + \mathcal{E}_\infty^{\mathrm{par}}(w , s) \, d \mu(z) \right)\right)|_{s=1}.
\end{align*}
We can now absorb $- s(1 - s)$ into the integrals after applying the hyperbolic Laplacian $\Delta$ to each of the two factors in the integrand.

Main results:

\begin{theorem}
  For any positive integer $n$ and any point $w \in \mathcal{F}$, we have
  \begin{equation*}
    \left\langle F_{- n}(\cdot, 1), \overline{\lim_{s \rightarrow 1} \left( G_s(\cdot, w) + \mathcal{E}_\infty^{\mathrm{par}}(\cdot, s) \right)} \right\rangle
    = - \partial_s F_{- n}(w, s) |_{s=1}.
  \end{equation*}
\end{theorem}
Combined with the previous proposition (describing $\log \lVert f(z) \rVert$), we get the Rohrlich--Jensen formula:
\begin{equation*}
  \left\langle F_{- n}(\cdot, 1), \log \lVert f \rVert \right\rangle
  = - 2 \pi \sum_{w \in \mathcal{F}}
  \frac{\ord_w(f)}{\ord(w)} \partial_s F_{- n}(w, s)|_{s=1}.
\end{equation*}

We have further results.  For instance, if $g$ is any $\Gamma$-invariant analytic function with a pole at $\infty$, then (Niebur)
\begin{equation*}
  g(z) = \sum_{n = 1}^K 2 \pi \sqrt{n} a_n F_{- n}(z, 1) + c(g)
\end{equation*}
for some constants $K$, $a_n$, and $c(g)$ depending only upon $g$.  Then, we have the identity
\begin{equation*}
  \left\langle g, \log \lvert f \rvert \right\rangle
  = - 2 \pi \sum_{w \in \mathcal{F}} \frac{\ord_w(f)}{\ord(w)}
  \left( 2 \pi \sum_{n = 1}^K \sqrt{n } a_n \partial_s F_{- n}(w, s) |_{s=1} - \dotsb \right).
\end{equation*}
Further found that the generating series for the Niebur Poincar{\'e} series at $s = 1$ is, in the $z$-variable (where we sum over $q_z$), the holomorphic part of the weight two biharmonic Maass form given by differentiating our linear combination of Eisenstein series and Green's function with respect to $z$.


\section{Shenghao Hua, \textnormal{\emph{Joint Value Distribution of Hecke--Maass Forms}}}
Preprint: \cite{2024arXiv2405.00996}

Motivation: semiclassical limit of solutions to Schroedinger equation.  Berry's random wave conjecture.  QUE, $L^p$-norms.

Joint Gaussian moments conjecture: statistical indepndence of values of large eigenfunctions (multiplied together and integrated against some test function).

\begin{theorem}
  Let $f$ and $g$ be normalized Hecke--MAass cusp forms.  Then if $2 t_f \leq t_g - t_g^\eps$, we have
  \begin{equation*}
    \int_{\Gamma \backslash \mathbb{H}}
    \psi f^2 g \ll t_g^{- A},
  \end{equation*}
  while if $2 t_f > t_g - t_g^\eps$, then, assuming GLH, we have
  \begin{equation*}
    \int_{\Gamma \backslash \mathbb{H}} \psi f^2 g
    = \int_{\Gamma \backslash \mathbb{H}} \psi g +
    \O(t_f^{- 1/4 + \eps}(1 + \lvert 2 t_f - t_g \rvert^{- 1/4})).
  \end{equation*}
\end{theorem}
Proof uses spectral expansion of $\psi$ into eigenfunctions $u$, then spectral expansion of $\int (u g)(f^2)$, reducing to the case $g=u$, then further spectral expansion of $\int(g^2)(f^2)$.

\begin{theorem}
  Assuming GRH and GRC, we have as $\max(t_f,t_g) \rightarrow \infty$
  \begin{equation*}
    \mathbb{E} f^2 g^2 = 1 + \O((\log(t_f + t_g))^{- 1/4 + \eps}).
  \end{equation*}
\end{theorem}
Proof eventually applies ultimately Soundararajan's method of moments (which requires GRH).

Work in progress: asymptotic formulas for certain averages of $L$-functions over toroidal families, going beyond \cite{2023arXiv2303.11664}.

\section{Algebraic proof of modular form inequalities for optimal sphere packings, \textnormal{\emph{Seewoo Lee}}}
Preprint: \cite{2024arXiv2406.14659}.

Motivating question: finding densest sphere packings.
\begin{itemize}
\item $d = 1$
\item $d = 2$: Thue 1910, $\Delta_2 = \frac{\pi}{2 \sqrt{3}}$
\item $d =3$: Kepler conjecture 1611, Hales 2005, $\Delta_3 = \frac{\pi}{3 \sqrt{2}}$, formally verified 2014 in Isabelle/HOL
\item (Korkine--Zolotareff, Blichfeldt, Cohn--Kumar): $D_4, D_5, D_6, E_7, E_8$ and Leech lattice
\item Viazovska, 2016 $\pi$-day on arXiv: $E_8$ lattice packing optimal, with $\Delta_8 = \frac{\pi^4}{388}$
\item Cohn--Kumar--Miller--Radchenko--Viazovska, March 21st 2016 on arXiv: $\Delta_{24} = \frac{\pi^{12}}{12!}$ Leech lattice (unique even unimodular with nonzero minimal length $2$)
\end{itemize}

Viazovska et al use:
\begin{theorem}[Cohn--Elkies, 2013]
  Let $r > 0$.  Assume there exists a nice function $f : \mathbb{R}^d \rightarrow \mathbb{R}$
  satisfying
  \begin{itemize}
  \item $f(0) = \hat{f}(0) > 0$,
  \item $f(x) \leq 0$ for all $\lVert x \rVert \geq r$,
  \item $\hat{f}(y) \geq 0$ for all $y \in \mathbb{R}^d$,
  \end{itemize}
  then
  \begin{equation*}
    \Delta_d \leq \vol(B_{r/2}^d) = \left( \frac{r}{2} \right)^d \frac{\pi^{d/2}}{(d / 2)!}.
  \end{equation*}
\end{theorem}
Proof is not hard (fits in one page).

This leads to the hunt for a ``magic function'' $f$ satisfying the indicated conditions.  BAsed on their numerical experiments, Cohn--Elkies conjectured that the optimal sphere packing can be achieved by a magic function when $d=2,8,24$.  The magic function $f$ and its Fourier transform $\hat{f}$ should vanish at the lattice points.

Viazovska et al constructed the magic functions for $d = 8,24$ using modular forms.  Decompose into Fourier eigenfunctions $f = f_+ + f_-$ (parity under Fourier transform).  Viazovska writes them as
\begin{equation*}
  f_{\pm}(x) = \sin^2 \left( \frac{\pi \lVert x \rVert^2}{2} \right)
  \int_0^\infty \varphi_{\pm}(t) e^{- \lVert x \rVert^2 t} \, d t,
\end{equation*}
where the $\sin^2$ factor enforces desired roots.

For $d = 8$, we have $\varphi_{\pm}(t) = t^2 \psi_{\pm}(i/t) $, where $\psi_{\pm}$ are defined in terms of standard modular forms.  We also need (for both $d = 8, 24$) a few non-obvious inequalities involving some of these modular forms.  One reason these inequalities are difficult is because they're inhomogeneous with respect to the weights of the forms involved.

Original proofs use bounds of Fourier coefficients of the form
\begin{equation*}
  \lvert c(n) \rvert \leq C_1 e^{C_2 \pi \sqrt{n}},
\end{equation*}
following from the Hardy--Ramanujan formula, reducing the question to finite calculations and interval arithmetic.

Ran Romik (2023) gave an alternative and much simpler proof for $d = 8$ that doe snot use any interval arithmetic, but still requires a ``calculator'' to check inequalities like
\begin{equation*}
  e^{3 \pi} \frac{9 \Gamma(1/4)^{16}}{8192 \pi^{12}} < 20480.
\end{equation*}

\begin{question}
  Can we prove such inequalities \emph{algebraically}?
\end{question}
The answer is yes, as we now explain.  To that end, we'll develop a theory of \emph{(completely) positive quasimodular forms}.

\begin{definition}
  Let $\Gamma \subseteq \SL_2(\mathbb{Z})$.  We say that a quasimodular form $F$
  is \emph{positive} if
  \begin{equation*}
    F(i t) \geq 0
  \end{equation*}
  for all $t > 0$.

  We call it \emph{completely positive} if it has nonnegative $q$-coefficients at $\infty$.
\end{definition}

Note that ``completely positive'' implies ``positive'', but the inclusion is strict in general.  For instance, the discriminant form has a product formula that tells you that it is positive, but you can check that it's not completely positive.

Easy facts:
\begin{theorem}
  \begin{itemize}
  \item Anti-derivative preserves positivity.
  \item Derivative preserves complete positivity.
  \item Serre derivative preserves complete positivity.
  \end{itemize}
\end{theorem}

``Nontrivial'' fact (that we won't use, and which also follows directly from Bernstein's theorem):
\begin{theorem}
  $F$ is completely positive if and only if all its derivatives are positive.
\end{theorem}

``Interesting'' fact that we will use:
\begin{theorem}
  Let $F = \sum_{n \geq n_0} a_n q^n$ with $a_{n _0 > 0}$.  If some Serre derivative of $F$ is positive, then so is $F$.
\end{theorem}

Examples?
\begin{definition}[Kaneko--Koike]
  For given weight $w$ and depth $s$, we can speak of \emph{extremal quasimodular forms of weight} $w$ \emph{and depth} $s$, which are those whose order of zeros at $\infty$ is as large as possible.
\end{definition}
It's been shown that these have almost all coefficients positive, and conjectured that they're all positive.

\begin{theorem}
  That conjecture is true for depth $1$ extremal forms.
\end{theorem}

We also have similar identities proving complete positively  of the depth $2$ extremal forms of weight $w \leq 14$, but not yet for general $w$.

This reduces some of the inequalities in the arguments of Viazovska et al to checking some limiting properties of ratios of modular functions.  For monotonicity, we check that derivatives are positive,  and for this it suffices to check the same for Serre derivatives of derivatives.

Possible future directions:
\begin{itemize}
\item Classify the completely positive forms?
\item Possible applications in other LP problems?
  \begin{itemize}
  \item Dual LP, uncertainty principle
  \item Any results that are ``uniform'' in dimensions
  \end{itemize}
\item Make a formalization of the proof (e.g., in Lean) easier?
\end{itemize}

\section{Ring of modular forms on certain unitary Shimura variety, \textnormal{\emph{Yuxin Lin }}}

Recall the definition of modular forms of weight $k$ and level $1$, as certain holomorphic functions $f : \mathbb{H} \rightarrow \mathbb{C}$.

More generally, let $\operatorname{Sh}(G)$ be a Shimura variety (of level $1$), where $G$ is a reductive group over $\mathbb{Q}$.  We can understand it as the moduli space of abelian varieties with additional structure.  Let $\pi : \mathcal{A} \rightarrow \operatorname{Sh}(G)$ be the universal abelian scheme of dimension $g$.

Let $\underline{\omega} := \pi_\ast \Omega_{\mathcal{A} / \mathrm{Sh}}$ be the Hodge bundle on $\operatorname{Sh}(G)$, and let $\lambda_{\mathrm{Sh}} := \Lambda^g \underline{\omega}$ be the Hodge line bundle on $\operatorname{Sh}(G)$.

\begin{definition}
  The ring of automorphic forms of scalar-valued weight is the space of global sections of the $k$-th tensor power of the Hodge line bundle, namely
  \begin{equation*}
    \oplus_k H^0(\mathrm{Sh}, \lambda^{\otimes k}).
  \end{equation*}
\end{definition}, 8, 12, 2, 16
An interesting question to ask is, what is the ring structure of such objects?  There are some relevant results along these lines.

First, when $G = \SL_2(\mathbb{Z})$, we have that $\operatorname{Sh}(G)$ is the modular curve, and the ring of modular forms is generated by $E_4$ and $E_6$ modulo a single relation:
\begin{equation*}
  \mathcal{M}(\SL_2(\mathbb{Z})) = \mathbb{C}[E_4, E_6] /(E_4^3 - E_6^2).
\end{equation*}



Going one dimension higher, for $G = \SL_2(\mathcal{O}_K)$ and $K = \mathbb{Q}(\sqrt{13})$, we have that $\operatorname{Sh}(G)$ is the Hilbert modular surface with real multiplication by $\mathcal{O}_K$.  Van Der Geer and Zagier show that the ring of symmetric Hilbert modular forms of even weight has $4$ generators at weights $4, 8, 12, 12$ respectively, with a relation at weight $24$, while the ring of Hilbert modular forms of even weight has $5$ generators of weights $4, 8, 12, 12, 16$, respectively, with two relations at weight $24$ and $32$.  To that end, they find the minimal model $Y$ of the compactification of $\operatorname{Sh}(G)$, then find the canonical divisor of $Y$ and compute the image $S$ of $Y$ under the canonical embedding, then realize sections of $H^0(\mathrm{Sh}, \lambda^{\otimes 2 k})$ as meromorphic sections of $\mathcal{O}_S(k)$ with certain multiplicity at the cusps.

This is the strategy we will imitate, but for certain unitary Shimura varieties rather than Hilbert modular surfaces.

To that end, we need to first introduce the moduli space of curves.

Let $G = \mathbb{Z} / d \mathbb{Z}$ be a finite cyclic group.
\begin{definition}
  \begin{itemize}
  \item $\mathcal{M}_G$: the moduli space of admissible stable $\mathbb{Z} / d \mathbb{Z}$ covers of $\mathbb{P}^1$.  Objects are pairs $(C / S, \iota : G \rightarrow \Aut_S(C))$, i.e., a family of curves over the base scheme together with an embedding of $G$ in their automorphism group.
  \item $\tilde{\mathcal{M}_G}$, the moduli space of admissible stable $\mathbb{Z} / d \mathbb{Z}$ covers of $\mathbb{P}^1$ with an ordering on the ramification points.  The objects are tuples $(C / S, \iota : G \rightarrow \Aut_S(C), \sigma_i)$.
  \item The forgetful morphism $\tilde{\mathcal{M}_G} \rightarrow \mathcal{M}_G$ is finite {\'e}tale.
  \end{itemize}
\end{definition}
The irreducible connected components of $\mathcal{M}_G$ are indexed by monodromy datum, which is a triple $(d, r, \underline{a})$, where
\begin{itemize}
\item $r$ is the number of branching points on $C / \iota(G)$,
\item $\underline{a} =(a(1), \dotsc, a(r)) \in G^r$ is the \emph{inertia type}, which records the character of $G$ at the tangent space of the $i$-th ramification point.
\end{itemize}

The \emph{Torelli morphism} $T$ associates the moduli of curves with the moduli of principally polarized abelian varieties:
\begin{equation*}
  \begin{CD}         
    \mathcal{M}(d, r, \underline{a})_{\mathbb{C}}  @>>> \operatorname{Sh}(\mathcal{D})_{\mathbb{C}}\\
    @VVV  @VVV \\
    \mathcal{M}_{g, \mathbb{C}} @>>T> \mathcal{A}_{g, \mathbb{C}}.\\
  \end{CD}
\end{equation*}
The image of the Torelli morphism is called the \emph{Torelli locus} inside $\mathcal{A}_{g, \mathbb{C}}$.  It turns out that the Jacobians coming out of $\mathcal{M}(d, r, \underline{a})$ have larger endomorphisms than generically, and in particular, at least contain the group algebra generated by $G$.  It's thus reasonable to speculate that the image of $T$ on $\mathcal{M}(d, r, a)$ will sit inside a \emph{special} subvariety of $\mathcal{A}_g$, and this special subvariety will be the focus of our study.

More precisely, $\operatorname{Sh}(\mathcal{D})$ is the smallest PEL type Shimura subvariety of $\mathcal{A}_{g, \mathbb{C}}$ containing $T(\mathcal{M}(d, r, \underline{a})_{\mathbb{C}})$.  It parametrizes abelian varieties with multiplication by $\mathbb{Z}[G]$ and signature given by $(d, r, \underline{a})$.

We focus on the family $\mathcal{M}[16] = \mathcal{M}(5, 5,(1, 1, 1, 1, 1))$.
\begin{itemize}
\item The associated Shimura variety $\operatorname{Sh}(G)$ is of \emph{unitary type}, whose reductive group $\mathcal{G}$ has $\mathbb{R}$ points given by $\GU(3, 0) \times \GU(2, 1)$.   Therefore, it is \emph{compact}.  The reason is that the signature of this Shimura variety can be computed, and the signature has a zero in one of the components, which means that the hermitian symmetric domain is compact, and hence this $\operatorname{Sh}(\mathcal{D})$ is compact.  In particular, this Shimura variety has no cusp.
\item A second reason why this family is so nice is that $\operatorname{Sh}(\mathcal{D})_{\mathbb{C}}$ has dimension $2$, which is equal to the dimension of $\mathcal{M}[16]$.  Therefore, the Torelli morphism $T$ gives an \emph{isomorphism}
  \begin{equation*}
    T : \mathcal{M}[16] \rightarrow \operatorname{Sh}(\mathcal{D}),
  \end{equation*}
  such that $T$ gives an isomorphism between the corresponding Hodge line bundles.  Thus, in order to compute global sections of the Hodge line bundle on $\operatorname{Sh}(\mathcal{D})$, it suffices to do the analogous thing on $\mathcal{M}[16]$, which is a bit simpler, but still tricky, due to nontrivial stabilizers.  To that end, we consider:
\item The finite {\'e}tale cover $\tilde{\mathcal{M}[16]}$, which has a natural forgetful morphism
  \begin{equation*}
    q : \tilde{\mathcal{M}[16]} \rightarrow \overline{M_{0, 5}},
  \end{equation*}
  that is bijective on field-valued points.  This is very nice, because we know the geometry of $\overline{M_{0, 5}}$: its Picard group is generated by ten divisors, each isomorphic to $\mathbb{P}^1$.
\end{itemize}

\begin{proposition}
  Let $\Delta$ be the boundary divisor of $\overline{M_{0, 5}}$, and let $\lambda_{\tilde{M}}$ be the Hodge divisor on $\tilde{\mathcal{M}[16]}$.  Under the forgetful morphism
  \begin{equation*}
    q : \tilde{\mathcal{M}[16]} \rightarrow \overline{M_{0, 5}},
  \end{equation*}
  we have that $q^\ast \Delta = 5 \lambda_{\tilde{M}}$.
\end{proposition}
\begin{proof}[Idea of proof]
  \begin{itemize}
  \item Use Grothendieck--Riemann--Roch to relate $\lambda_{\tilde{M}}$ with the \emph{boundary divisor} of $\tilde{\mathcal{M}}$.
  \item $q$ exhibits $\Delta$ as a multiple of the boundary of $\tilde{\mathcal{M}}[16]$.
  \end{itemize}
\end{proof}

Our next goal is to fill in the lifts
\begin{equation*}
  \begin{CD}         
    \tilde{\mathcal{M}[16]}  @>?>> \operatorname{Sh}_{\mathcal{K}}(\mathcal{D})\\
    @VVV  @VV?V \\
    \mathcal{M}[16] @>>T> \operatorname{Sh}(\mathcal{D}).\\
  \end{CD}
\end{equation*}
Giving such a cover of $T$ is the same as computing the level subgroup corresponding to this cover.  That level subgroup is basically a congruence subgroup:

\begin{proposition}
  The level subgroup $\mathcal{K}$ at the place $p$ satisfies:
  \begin{equation*}
    \mathcal{K}_p =
    \begin{cases}
      \mathcal{G}(\mathbb{Z}_p)      & \text{ if } p \neq 5, \\
      \mathcal{U}(\mathfrak{m})                                     & \text{ if } p = 5.
    \end{cases}
  \end{equation*}
\end{proposition}

Putting these parts together via the sequence of morphisms
\begin{equation*}
  \operatorname{Sh}_{\mathcal{K}}(\mathcal{D}) \xleftarrow{T} \tilde{\mathcal{M}[16]}
  \xrightarrow{q} \overline{M_{0, 5}}
  \xrightarrow{\lvert - K \rvert} S,
\end{equation*}
we obtain the following:
\begin{proposition}
  \begin{itemize}
  \item The weight $m$ automorphic forms on $\operatorname{Sh}_{\mathcal{K}}(\mathcal{D})$ is
    \begin{equation*}
      H^0(\operatorname{Sh}_{\mathcal{K}(\mathcal{D}), \lambda_{\mathrm{Sh}^{\otimes m}}})
      \cong H^0(S, \mathcal{O}_S(2 \lfloor \frac{m}{5} \rfloor))
      \cong \left( \mathbb{C}[x_0, \dotsc, x_5] \right) / (Q_{12} - Q_{i j}),
    \end{equation*}
    with explicit degrees.
  \item Similar for the ring of graded automorphic forms.
  \end{itemize}
\end{proposition}

From the moduli interpretation, we see that $\operatorname{Sh}(\mathcal{D})$ is the $\mathfrak{S}_5$-invariant of $\operatorname{Sh}_{\mathcal{K}}(\mathcal{D})$.   Therefore:
\begin{theorem}
  The ring of automorphic forms on $\operatorname{Sh}(\mathcal{D})$ is the $\mathfrak{S}_5$-invariant of $\operatorname{Sh}_{\mathcal{K}}(\mathcal{D})$.  In particular,
  \begin{equation*}
    \oplus_{m} H^0(\operatorname{Sh}(\mathcal{D}), \lambda^{\otimes m}) \cong \mathbb{C}[f_2, f_4, f_6, \eps] / \eps^{10}
    + \frac{1}{52}(f_2^2 - 4 f_4),
  \end{equation*}
  where the $f_{2n}$ are polynomials of degree $2 n$ in $x_i$, where everything has explicit degrees.
\end{theorem}

\section{Counting modular forms mod $p$ satisfying constraints at $p$, \textnormal{\emph{Samuele Anni}}}
Recall the definition of modular forms $f$ of weight $k$ on $\Gamma_0(n)$, with $q$-expansion $f(z) = \sum a_n q^n$.  We also know that there are Hecke operators $T_p$ acting on them.  We consider only cuspidal \emph{newforms} ($a_0 = 0$), \emph{normalized} ($a_1 = 1$), which are eigenforms for the Hecke operators and arise from level $n$ and not any smaller level.  We use the notation $S_k(n, \mathbb{C})$ and $S_k(n, \mathbb{C})^{\new}$.

\begin{definition}
  The \emph{Hecke algebra} $\mathbb{T}(n, k)$ is the $\mathbb{Z}$-subalgebra of $\End_{\mathbb{C}}(S_k(n, \mathbb{C}))$ generated by Hecke operators $T_p$ for each prime $p$.
\end{definition}

Newforms can be seen as ring homomorphisms $f : \mathbb{T}(n, k) \rightarrow \overline{\mathbb{Q}}$.

\begin{theorem}[Deligne, Serre, Shimura]
  Let $n$ and $k$ be positive integers.  Let $\mathbb{F}$ be a finite field of characteristic $\ell$, with $\ell \nmid n$, and
  let $f : \mathbb{T}(n, k) \twoheadrightarrow \mathbb{F}$ be a surjective ring homomorphism.  Then there is a unique continuous semisimple representation
  \begin{equation*}
    \bar{\rho} _f : \Gal(\bar{\mathbb{Q}} / \mathbb{Q}) \rightarrow \GL_2(\mathbb{F}),
  \end{equation*}
  unramified outside $n \ell$, such that for all $p$ not dividing $n \ell$, the trace of Frobenius at $p$ under $\bar{\rho}_f$ is $f(T_p)$, and the determinant is given in terms of the value of $f$ on the diamond operator (central character).
\end{theorem}

Computing $\bar{\rho}_f$ is ``difficult'', but theoretically it \emph{can be done in polynomial time} in $n, k, \# \mathbb{F}$:
\begin{itemize}
\item Edixhoven, Couveignes, de Jong, Merkl, Bruin, Bosman: $\# \mathbb{F} \leq 32$.
\item Mascot, Zeng, Tian: $\# \mathbb{F} \leq 53$.
\end{itemize}

Fix a prime $p \geq 5$, a level $N$ prime to $p$, and a weight $k \geq 2$.  Let $S_k := S_k(\Gamma_0(N p), \bar{\mathbb{Q}}_p)$ be the space of $p$-new (i.e., not coming from level $N$) cuspidal modular forms of level $N p$ and weight $k$ with coefficients in $\bar{\mathbb{Q}}_p$.

We now discuss the \textbf{Atkin--Lehner involution}.  There exist $x, y, z, t \in \mathbb{Z}$ for which the matrix
\begin{equation*}
  W_p =
  \begin{pmatrix}
    p x    & y \\
    N p z &  p t \\
  \end{pmatrix}
\end{equation*}
has determinant $p$.

The matrix $W_p$ normalizes the group $\Gamma_0(N p)$, and for any weight $k$, it induces a linear operator $w_p$ on the space of cusp forms $S_k$ that commutes with the Hecke operators $T_q$ for all $q \nmid N p$ and acts as its own inverse.

Any cusp form in $S_k$ that is an eigenform for all $T_q$ with $q \nmid N$ is also an eigenform for $w_p$, withe eigenvalue $\pm 1$.  This involution acts on the modular curve.

The Atkin--Lehner involution $w_p$ splits $S_k$ as a sum of plus and minus spaces:
\begin{equation*}
  S_k = S_k^+ \oplus S_k^-.
\end{equation*}
Since we have dimension formulas for $s_k := \dim S_k$, in order to understand the dimensions
\begin{equation*}
  s_k^{\pm} = \dim S_k^{\pm}
\end{equation*}
of the Atkin--Lehner eigenspaces, it suffices to understand the difference
\begin{equation*}
  d_k := s_k^+ - s_k^-.
\end{equation*}

We can get some idea from tables of examples (obtained empirically by looking at many more specific examples):
\begin{table}[h]
  \centering
  \begin{tabular}{|c|c|}
    \hline
    $p$ & $d_k$ \\
    \hline
    5   & $\pm 1$ \\
    11  & $\pm 2$ \\
    59  & $\pm 6$ \\
    101 & $\pm 7$ \\
    \hline
  \end{tabular}
  \caption{Values of $d_k$ for different $p$}
  \label{tab:dk_values}
\end{table}

Classical result that $d_k$ is constant in absolute value and alternates in sign.  Need to introduce modification $d_k^\ast$ taking into account Eisenstein series of weight two.  One can show:
\begin{theorem}[Fricke, Yamauchi, Momose, Ogg, Wakatsuki, Helfgott, Martin et al.]
  We have
  \begin{equation*}
    d_k^\ast =(- 1)^{k/2} \frac{\# \mathrm{FP}}{2},
  \end{equation*}
  where $\# \mathrm{FP}$ is the number of fixed points of the Atkin--Lehner involution $w_p$ on $X_0(N p)$.
\end{theorem}
The fixed points of $w_p$ on $X_0(N p)$ corresponds to elliptic curves with level structure and CM by $\sqrt{- p}$, in fact the $d_k$ are closely related to class numbers:
\begin{equation*}
  \# \mathrm{FP}
  = c_p \cdot h(\sqrt{-p}) \cdot \prod_{q \mid N, \text{prime}}
  \left( 1 + \qr{- 4 p}{q} \right).
\end{equation*}

What do we want to do?  Systems of mod-$p$ prime-to-$N p$ Hecke eigenvalues correspond to continuous semisimple Galois representations $\Gal(\bar{\mathbb{Q}} / \mathbb{Q})  \rightarrow \GL_2(\bar{\mathbb{F}}_p)$.  Can decompose according to these, and then further via Atkin--Lehner, giving spaces $S_{k, \bar{\rho}}^{\pm}$ and their dimensions $s_{k, \bar{\rho}}^{\pm}$ and differences $d_{k, \bar{\rho}}$.  As before, $k = 2$ and $\bar{\rho}$ forces us to make an adjustment, so let
\begin{equation*}
  d_{k, \bar{\rho}}^\ast :=
  \begin{cases}
    d_{k, \bar{\rho}} - 1    &  \text{ if } k = 2 \text{ and } \bar{\rho} = 1 \oplus \omega, \\
    d_{k, \bar{\rho}}                             & \text{ otherwise,}
  \end{cases}
\end{equation*}
where $\omega$ is the cyclotomic character.

\begin{theorem}[Anni, Ghitza, Medvedovsky]
  For $k \geq 2$ and any $\Gamma_0(N p)$-modular $\bar{\rho}$, we have
  \begin{equation*}
    d^\ast_{k + 2, \bar{\rho}[1]} = - d_{k, \bar{\rho}}^\ast.
  \end{equation*}
\end{theorem}

As an example, for $p = 5$, $N= 23$, we can check that for $k > 2$, we have $d_k = \pm 2$.

Bergdall and Pollack use the Ash--Stevens formula, a fundamentally characteristic $p$ technique for filtering cohomology of modular symbols, to derive their dimension formulas.  But Ash--Stevens has nothing to say about Atkin--Lehner, in part because Atkin--Lehner operator requires inverting $p$.  On the other hand, the classical complex methods -- trace formulas, Gauss--Bonnet, Riemann--Hurwitz -- do not know anything about $\bar{\rho}$.

What we do instead is to combine the \emph{trace formula} (Zagier--Cohen--Osterl{\'e}--Cohen--Str{\"o}mberg and Skoruppa--Zagier--Popa) with an \emph{algebra theorem}, an explicit refinement of Brauer--Nesbitt.

What is this explicit Brauer--Nesbitt?  See \cite{2022arXiv2207.07108}.
\begin{theorem}[AGM]
  Let $M$ and $N$ be two finite free $\mathbb{Z}_p$-modules of the sam erank $d$, each with an action of an operator $T$.  Then $\bar{M}^{\mathrm{ss}} \cong \bar{N}^{\mathrm{ss}}$ as $\mathbb{F}_p[T]$-modules if and only if for every $n$ with $1 \leq n \leq d$, we have
  \begin{equation*}
    \trace(T^n | M) = \trace(T^n | N),
  \end{equation*}
  (...).
\end{theorem}
\begin{corollary}
  If we have modules that injective modulo $p$ (but not necessarily in characteristic zero), then we can check that semisimplifications of certain residual quotients are the same by proving an inequality involving traces.
\end{corollary}

Now work with the trace formula.  We can deduce statements about dimensions.

\section{Integer partitions detect the primes, \textnormal{\emph{Jan-Willem Van Ittersum}}}
Preprint: \cite{2024arXiv2405.06451}.

We write $s_i$ for the sizes of the parts, and $m_i$ for the multiplicities of the part, with $s_1 > s_2 > \dotsb$.
\begin{definition}[MacMahon 1920]
  \begin{equation*}
    M_a(n) := \sum_{
      \substack{
        n = m_1 s_1 + \dotsb + m_a s_a  \\
        0 < s_1 < \dotsb < s_a        
      }
    } m_1 \dotsb m_a.
  \end{equation*}
\end{definition}
\begin{remark}
  This is the sum of multiplicity products of partitions of $n$ with $a$ part sizes.
\end{remark}
Consider
\begin{equation*}
  \psi(n) :=(n^2 - 3 n + 2) M_1(n) - 8 M_2(n).
\end{equation*}
\begin{example}
  $n = 3, a = 1$, in which case $\lambda =(3)$ or $\lambda =(1^3)$, in which case $M_1(3) = 1 + 3 = 4$,
  or $a = 2$, in which case $\lambda =(2, 1)$, so $M_2(3) = 1 \cdot 1 = 1$, and we compute that $\psi(3) = 2 M_1(3) - 8 M_2(3) = 0$.

  Continuing, take $n = 4, a = 1$, in which case $\lambda = (4)$, $\lambda = (2^2)$ $\lambda =(1^4)$, so $M_1(4) = 1 + 2 + 4 = 7$,
  or $a = 2$, in which case $\lambda =(3, 1)$ or $\lambda =(2, 1^2)$, so $M_2(4) = 1 \cdot 1 + 1 \cdot 2 = 3$, and so $\psi(4) = 6 M_1(4) - 8 M_2(4) = 18$.

  Continuing, one gets the table, for $n = 2..11$, of $\psi(n) = 0, 0, 18, 0, 120, 0, 270, 192, 504, 0$.
\end{example}

\begin{theorem}[Craig--vI--Ono] For positive integers $n$, we have
  \begin{enumerate}
  \item $(n^2 - 3 n + 2) M_1(n) - 8 M_2(n) \geq 0$,
  \item $(3 n^3 - 13 n^2 + 18 n - 8) M_1(n) +(12 n^2 - 120 n + 212) M_2(n) - 960 M_3(n) \geq 0$
  \end{enumerate}
  and for $n \geq 2$ these expressions vanish if and only if $n$ is prime.
\end{theorem}
\begin{remark}
  We call such an expression \emph{prime-detecting}.  The sum of two such expressions is likewise prime-detecting, and multiplying $f(n)$ for a polynomial $f$ yields a prime-detecting expression if $f(n) > 0$ for all $n$.
\end{remark}

We can get a table of prime-detecting expressions: the first two that appear above, then two more, $(126 n^5 - \dotsb) M_1(n) + \dotsb$ and $(300 n^8 - \dotsb) M_1(n) + \dotsb$.

\begin{conjecture}
  These are all such expressions (up to addition and multiplication as before).
\end{conjecture}

We then checked whether we could get more results of a similar shape, by considering a generalization of the MacMahon function:
\begin{definition}
  For $\ell \in \mathbb{Z}_{\geq 0}^a$, we define the \emph{generalized MacMahon partition function}
  \begin{equation*}
    M_{\underline{\ell}}(n) := \sum_{
      \substack{
        n = m_1 s_1 + \dotsb + m_a s_a  \\
        0 < s_1 < \dotsb < s_a        
      }
    }
    m_1^{\ell_1} \dotsb m_a^{\ell_a}.
  \end{equation*}
\end{definition}
You can think of this as a generalization of the divisor function, but for partitions.  For this new generalized partition function, much more is possible:
\begin{theorem}[Craig--vI--Ono] Let $d \geq 4$.
  \begin{enumerate}
  \item There exist $clu \underline{\ell} \in \mathbb{Z}$ such that $\sum_{\lvert \ell \rvert < d} c_{\underline{\ell}} M_{\underline{\ell}}(n) \geq 0$, where $\lvert \underline{\ell} \rvert = \ell_1 + \dotsb + \ell_a$.
  \item There are $\gg d^2$ linearly independent such expressions.
  \end{enumerate}
\end{theorem}
\begin{example}
  $63 M_{(2, 2)}(n) - 12 M_{(3,0)}(n) - \dotsb + 12 M_{(3,0,0)}(n) = \frac{11}{3} \psi(n)$.
\end{example}<++>

Goal of the talk now is to explain some of the ideas behind these results, which have a lot to do with modular forms.

\textbf{Prime-detecting quasimodular forms}.  Let
\begin{equation*}
  \mathcal{G}_k := - \frac{B_k}{2 k} + \sum_{n \geq  1} \sigma_{k - 1}(n) q^n,
  \qquad
  D := q \frac{\partial}{\partial q},
  \qquad q = e^{2 \pi i \tau}.  
\end{equation*}
Consider
\begin{align*}
  f_{k, \ell} &:=(D^{\ell} + 1) \mathcal{G}_{k + 1} -(D^k + 1) \mathcal{G}_{\ell + 1} \\
              &= \ast + \sum_{n \geq 1} \sum_{d \mid n} \left((n^{\ell} + 1) d^k -(n^k + 1) d^{\ell} \right) q^n.
\end{align*}
Note that for $d = 1$ one gets $n^{\ell} + 1 - n^k - 1 = n^{\ell} - n^k$.  Calculating similarly with $d = n$, we see that $f_{k, \ell}$ vanishes at prime coefficients.  It's also easy to see that its coefficients are positive.  But note that $f_{k, \ell}$ is not modular, and not of homogeneous weight.  Key features here are quasimodularity and mixed weight.

\begin{theorem}[Craig--vI--Ono] All prime-detecting forms in $\oplus_{k: \text{even}} \oplus_{n \geq 0} D^n \mathcal{G}_k$ are linear combinations of $D^n H_{k}$, where
  \begin{equation*}
    H_k =
    \begin{cases}
      \frac{1}{6}(D^2 - D + 1) \mathcal{G}_2 - \mathcal{G}_4      &  \text{ if } k = 6, \\
      \frac{1}{24} \left( - D^2 \mathcal{G}_{k - 6} +(D^2 + 1) \mathcal{G}_{k - 4} - \mathcal{G}_{k - 2} \right)                                                                  & \text{ if } k = 8.
    \end{cases}
  \end{equation*}
\end{theorem}

That's the first ingredient, but still need to say something about modularity of MacMahon functions.  Let
\begin{equation*}
  \mathcal{G}(\underline{\ell}) = \sum_{n \geq 0} M_{\underline{\ell}}(n) q^n \xrightarrow{q \rightarrow 1}
  \frac{(1 - q)^{\lvert \underline{\ell}  \rvert + a}}{\prod_i \ell_i !}
  \sum_{s_1 < \dotsb < s_a}
  \frac{1}{ s_1^{\ell_1 + 1} \dotsb s_a^{\ell_a + 1}}
  \quad
  (\ell_a \geq 1, \, \ell_i \geq 0).
\end{equation*}
The algebra $\mathcal{Z}_q := \left\langle \mathcal{G}(\underline{\ell}) \right\rangle_{\mathbb{Q}}$ was introduced by Bachmann--K{\"u}hn.  Facts:
\begin{itemize}
\item $\mathcal{Z}_q$ is a differential algebra (closed under differentiation and multiplication)
\item $\tilde{M} := \mathbb{Q}[\mathcal{G}_2 , \mathcal{G}_4, \mathcal{G}_6] \subseteq \mathcal{Z}_q$ (i.e., it contains the space of quasimodular forms).
\item (Hoffman--Ihara) We have
  \begin{equation*}
    \sum_{j \geq 0} \mathcal{G}(\underbrace{1, 1, \dotsc, 1}_{j}) x^j = \exp
    \left(
      \sum_{n \geq1}
      \frac{(- 1)^{n + 1}}{ n}
      x^n \underbrace
      {
        \sum_m \frac{q^m}{(1 - q^m)^{2n}}
      }_{
        \sum_{m, s \geq 1} \binom{s + n - 1}{ s - n} q^{m s} \in \tilde{M}
      }
    \right),
  \end{equation*}
  which is a linear combination of quasimodular Eisenstein series, hence $M_a(n)$ are coefficients of a quasimodular form.
\end{itemize}

\section{Supersingular abelian surfaces and where to find them, \textnormal{\emph{Gabriele Bogo}}}
Characteristic zero: $\End(E) \cong \mathbb{Z}$ or $\mathcal{O}_D$, order in an imaginary quadratic field.

Characteristic $p > 0$ (Hasse, 1936), can also be $\mathcal{O}_{\infty, p}$, order in a quaternion algebra over $\mathbb{Q}$.  In that case, we call it \emph{supersingular}.

Deuring 1941: for $p \geq 5$, the number of supersingular elliptic curves over $\mathbb{F}_{p^2}$ is
\begin{equation*}
  \lfloor \frac{p - 1}{12} \rfloor + \delta + \eps.
\end{equation*}
Another interpretation: writing $\mathcal{M}_{1, 1} \cong \mathbb{H} / \SL_2(\mathbb{Z})$ for the moduli space of elliptic curves over $\mathbb{C}$, there are three special points $\infty, i = \sqrt{- 1}, \rho =(- 1 + \sqrt{- 3}) / 2$, and we can think of Deuring's result as
\begin{equation*}
  \#(\mathrm{ss}_p)
  = \lfloor \frac{p - 1}{12} \rfloor + \delta + \eps
  \leq
  \lfloor
  \frac{- \chi(\mathbb{H} / \SL_2(\mathbb{Z})) \cdot(p - 1)}{2}\rfloor + 2.
\end{equation*}

Let's generalize to higher-dimensional varieties.  Let $\mathbb{Q}(\sqrt{D})$ be a real quadratic field, with ring of integers $\mathcal{O}_D$.  We consider a family $\mathcal{X} \rightarrow \mathcal{C}$ of (p.p.) abelian surfaces, over a curve $\mathcal{C}$, such that:
\begin{itemize}
\item the fibers $X_c$ are defined over a number field $k$, and
\item there exists an inclusion of rings $\mathcal{O}_D \hookrightarrow \End_k(X_c)$.
\end{itemize}
An abelian surface over a field of characteristic $p$ is \emph{supersingular} (resp.\ \emph{superspecial}) if it is isogenous (resp.\ isomorphic) to the product of two supersingular elliptic curves.

There are embeddings
\begin{equation*}
  \mathbb{H} / \Gamma = \mathcal{C} \hookrightarrow X_D(\mathbb{C}) = \mathbb{H}^2 / \SL_2(\mathcal{O}_D).
\end{equation*}
For example, if $\Gamma = \SL_2(\mathbb{Z})$, this is the inclusion of elliptic curves into the space of abelian surfaces with multiplication.  There are also modular curves $\Gamma_0(N)$.  But one can have more interesting examples, related to non-arithmetic curves.  To give an explicit example, consider the triangle group $\Gamma = \Delta(2, 5, \infty) \hookrightarrow X_5$, with explicit equation
\begin{equation*}
  y^2 =
  \begin{cases}
    x^5 - 5 x^3 + 5 x    
    - 2 t
    & \text{ if } t \neq \infty, \\
    x^5 - 1    & \text{ if } t = \infty.
  \end{cases}
\end{equation*}
We now take the reductions of the above inclusion modulo $p$, to a map
\begin{equation*}
  \overline{\mathcal{C}} \rightarrow X_D(\mathbb{F}). 
\end{equation*}
We want to study the fibers of this family that, after reduction modulo $p$, are supersingular.

To that end, consider a smooth algebraic curve in a Hilbert modular surface
\begin{equation*}
  \mathcal{C} \hookrightarrow X_D
\end{equation*}
with second Lyapunov exponent $\lambda_2 \in \mathbb{Q} \cap \left(0,1\right]$.  For example, for $\SL_2(\mathbb{Z})$, we have $\lambda_2 = 1$, and for modular curves, $\lambda_2 = 1$, while for
$\Delta(2, 5, \infty)$ we have $\lambda_2 = 1/3$, but for Teichmuller curves, you can have $1/3, 1/5, 1/7$.

\begin{theorem}[B--Li, 2024]
  The supersingular locus of $\mathcal{C}$ modulo $p$ has cardinality described by
  \begin{align*}
    \lfloor \frac{- \chi(\mathcal{C})(p - \lambda_2)}{2} \rfloor \leq
    \#\mathrm{ss}_p^{\mathcal{C}}
    \leq \lfloor
    \frac{- \chi(\mathcal{C})(p - 1)(\lambda_2 + 1)}{2}\rfloor + r, \quad \text{ if } \qr{D}{p} = -1,
  \end{align*}
  while
  \begin{equation*}
    \# \mathrm{ss}_p^{\mathcal{C}}
    \leq \lfloor \frac{- \chi(\mathcal{C})(p - 1) \lambda_2}{2} \rfloor + r,
    \quad \text{ if } \qr{D}{p} = 1.
  \end{equation*}
\end{theorem}

How to find supersingular abelian surfaces?  If $\mathcal{C} = \mathbb{H} / \Gamma$  is of genus zero with Hauptmodul $j$, then the supersingular locus can be described by a polynomial in $j$:
\begin{equation*}
  \mathrm{ss}_p^{\mathcal{C}}(j) :=
  \prod_{
    \substack{
      c \in \mathcal{C}  \\
      \overline{X_c} \text{ supersingular}      
    }
  }
  \left( j - j(c) \right).
\end{equation*}

\begin{theorem}[B.--Li, 2024]
  There are two families $\{A_{0, n}(j)\}_n$ and $\{A_{1, n}(j)\}_n$ of orthogonal polynomials (Atkin's polynomials) such that for every $p$ of good reduction,
  \begin{equation*}
    \mathrm{ss}_p^{\mathcal{C}}(j) \equiv
    \begin{cases}
      \lcm(A_{0, n_p}, A_{1, \tilde{n}_p})      & \text{ if } \qr{D}{p} = -1, \\
      \gcd(A_{0, n_p}, A_{1, \tilde{n}_p})                                                & \text{ if } \qr{D}{p} = 1
    \end{cases}
  \end{equation*}
  for explicit indices $n_p$ and $\tilde{n}_p$.
\end{theorem}

For example, consider $\Delta(2, 5, \infty)$.  Let $j$ be a Hauptmodul, with pole at $\infty$.  The scalar products are defined on $\mathbb{R}[j]$ by
\begin{equation*}
  \langle f, g \rangle_0 :=
  \int_{\pi / 5}^{\pi / 2}
  f(e^{i \theta})
  g(e^{i \theta}) \, d th,
\end{equation*}
\begin{equation*}
  \langle f, g \rangle_1 := \int_{2 \pi / 5}^{\pi / 2}
  f(e^{i \theta}) g(e^{i \theta}) \, d \theta.
\end{equation*}
The first polynomials $A_{0, n}(j)$ are: $A_{0, 0}(j) = 1$, $A_{0, 1}(j) = j - 9/20$, etc.  Taking $p = 13$, one finds that
\begin{equation*}
  \mathrm{ss}_{13}(j) \equiv A_{0, 3}(j) \equiv j(j + 4)(j - 1) \pmod{13}.
\end{equation*}

Next, real multiplication splits $H_{\mathrm{d R}}^{1}(X_c)$ into two eigenspaces.  The eigendifferentials induce two second order differnetial equations, called Picard--Fuchs differential equations.  TRuncation of the holomorphic solutions can be related to the supersingular locus of $\mathcal{C}$.
\begin{example}
  For $\mathcal{C} = \mathbb{H} / \Delta(2, 5, \infty)$,
  \begin{equation*}
    j^n \cdot {}_2 F_1 \left( \frac{7}{20}, \frac{3}{20}; 1 ; \frac{1}{j} \right)
    = U_{0, n}(j) + \O(j^{-1}),
  \end{equation*}
  etc.  Leads to a formula for $\mathrm{ss}_p^{\mathcal{C}}(j)$.
\end{example}

The partial Hasse invariants $h_1$ and $h_2$ are characteristic $p$ Hilbert modular forms of non-parallel weight.  They have the following properties:
\begin{itemize}
\item The divisor of $h_i$ is the component $D_i$ of the non-ordinary locus of $X_D(\mathbb{F})$.
\item Their $q$-expansion is constant, equal to $1$, at every cusp.
\end{itemize}
In particular, they do not lift to Hilbert modular forms in characteristic zeor.

\begin{theorem} [B--Li, 2024]
  Let $\mathcal{C} \cong \mathbb{H} / \Gamma \hookrightarrow X_D$ be a smooth algebraic curve with good reduction at $p$.  Then the partial Hasse invariants lift to (twisted) modular forms on $\Gamma$ in characteristic zero.
\end{theorem}

\section{Congruences for the number of 3- and 6-regular partitions and quadratic forms, \textnormal{\emph{Cristina Ballantine}}}

\begin{definition}
  A \emph{partition} of a positive number $n$ is a non-increasings equence of positive integers, named \emph{parts}, that add up to $n$.
\end{definition}
We denote by $p(n)$ the number of partitions of $n$.

\begin{example}
  $p(4) = 5$.
\end{example}

Ramanujan congruences:
\begin{equation*}
  p(5 n + 4) \equiv 0 \pmod{5},
\end{equation*}
\begin{equation*}
  p(7 n + 5) \equiv 0 \pmod{7},
\end{equation*}
\begin{equation*}
  p(11 n + 6) \equiv 0 \pmod{11}.
\end{equation*}
Ramanujan: it appears there are no equally simple properties for any moduli involving primes other than these.

Ahlgren and Ono (2001): there are such Ramanujan congruences modulo every prime $\ell \geq 5$ (actually mod $\ell^m$).

How about $\ell = 2, 3$?  Radu (2012, formerly Subbarao Conjecture): for $\ell \in \{2, 3\}$, there is no arithmetic progression $A n + B$ such that
$p(A n + B) \equiv 0 \pmod{\ell}$ for all $n$.

Conjecture (Parkin and Shank): approximately half the values of $p(n)$ are even.

Bellaiche and Nicolas 2016: the number of $n \leq x$ with $p(n)$ even is at least
\begin{equation*}
  0.069 \sqrt{x} \log \log x,
\end{equation*}
while the number that are odd is at least a constant multiple of $\sqrt{x}/ \log x$.

The reason we can do this using modular forms is that
\begin{equation*}
  \sum_{n = 0}^{\infty} p(n) q^n = \prod_{i = 1}^\infty \frac{1}{1 - q^i} = \frac{q^{1/24}}{\eta(q)}.
\end{equation*}

Let's write $Q(n)$ for the number of partitions with distinct parts.  Then
\begin{equation*}
  \sum_{n = 0} Q(n) q^n = \prod_{i = 1}^\infty(1 + q^i) \equiv \prod_{i = 1}^\infty(1 - q^i) \pmod{2}
  =
  \sum_{j \in \mathbb{Z}}(- 1)^j q^{j(3j - 1)/2},
\end{equation*}
so $Q(n)$ is even asymptotically 100\% of the time.

\begin{definition}
  An $r$\emph{-regular partition} is a partition in which no parts are divisible by $r$.
\end{definition}
\begin{notation}
  $b_r(n) := $ the number of $r$-regular partitions of $n$.
\end{notation}
$b_2(n)$ is even asymptotically 100\% of the time.

\begin{theorem}[Keith--Zanello 2022]
  If $p \equiv 13, 17,. 19, 23 \pmod{24}$ is prime, then
  \begin{equation*}
    b_3 \left( 2(p^2 n +  pk - 24^{-1}) \right) \equiv 0 \pmod{2}
  \end{equation*}
  for $1 \leq k \leq p - 1$, where the inverse is taken modulo $p$.
\end{theorem}
\begin{theorem}[Yao 2022]
  Let $p \geq 5$ for a prime.  Then we can construct more arithmetic progressions where this happens, different from the above.
  \begin{enumerate}
  \item If $b_3 \left( \frac{p^2 - 1}{12} \right) \equiv 1 \pmod{2}$, then we get a whole arithmetic progression.
  \item (...)
  \end{enumerate}
\end{theorem}

Experimentally, we found a bunch of arithmetic progressions on which $b_3$ was even.  For instance, for $\alpha \in \{1, 51, 76, 101\}$,
\begin{equation*}
  b_3 \left( 2(5^3 n + \alpha) \right) \equiv 0 \pmod{2}.
\end{equation*}
Conjectured that one can do this for every prime $p \geq 5$.

The bad news is that the congruences we found for $p = 5,7,11$ are special cases of Yao's theorem.  My coauthor did more experience and found other primes $p \in \{29, 59, \dotsc, 683\}$ where we have
\begin{equation*}
  b_3 \left( 2(p^2 n + p \alpha - 24^{-1}) \right) \equiv 0 \pmod{2},
\end{equation*}
where $0 \leq \alpha < p$, $\alpha \neq \lfloor p/24 \rfloor$, and $24^{-1}$ taken modulo $p$, that do not fit into Yao's theorem.

We proved the congruences algorithmically, using a result of Radu involving modular forms and Sturm's bound.

Can we describe these primes?  What is the general theorem?

Let $\mathcal{P}$ be the set of primes $p$ such that for some $j \in \{1, 4, 8\}$, the equation $x^2 + 24 \cdot 9 y^2 = j p$ has primitive solutions.


\section{Exact formulae for ranks of partitions, \textnormal{\emph{Qihang Sun}}}
Preprint: \cite{2024arXiv2406.06294}.

$p(n)$: partition function.  Ramanujan's congruence mod $5$ and $7$.

Dyson's rank: for each partition $\Lambda = \{\Lambda_1 \geq \Lambda_2 \geq \dotsb \geq \Lambda_k\}$, define $\rank(\Lambda) := \Lambda_1 - \kappa$ to be the largest part minus the number of parts.  Rank generating function $N(a, b, n)$.  Explains Ramanujan congruences, e.g.,
\begin{equation*}
  N(a, 5, 5 n + 4) = \frac{1}{5} p(5 n + 4),
\end{equation*}
\begin{equation*}
  N(a, 7, 7 n + 5) = \frac{1}{7} p(7 n + 5).
\end{equation*}
These conjectures were all proved by Swinnerton-Dyer in 1954.

We'll focus on the analytic side of those rank functions.  For $q = e^{2 \pi i z}$ and $w = \zeta_b^a$,
\begin{equation*}
  \mathcal{R}(w, q) = 1 + \sum_{n = 1}^\infty \sum_{m \in \mathbb{Z}} N(m, n)
  w^m q^n
  =
  1 + \sum_{n = 1}^\infty \frac{q^{n^2}}{(w q, q)_n(w^{-1} q, q)_n}
  =: 1 + \sum_{n = 1}^\infty A \left( \frac{a}{b}, n \right) q^n.
\end{equation*}

Examples:
\begin{itemize}
\item $R(1, q) = 1 + \sum p(n) q^n = q^{1/24} / \eta(z)$
\item $R(- 1, q) = f(q)$ (Ramanujan's $3$rd other mock theta function)
\item $R(\zeta_3, q) = R(\zeta_3^2, q) = \gamma(q)$ (another mock theta function)
\end{itemize}

Hardy--Ramanujan (1918):
\begin{equation*}
  p(n) \sim \frac{1}{4 n \sqrt{3}}
  \exp \left( \pi \sqrt{\frac{2 n}{3}} \right),
\end{equation*}
together with more precise result.  Rademacher (1937): exact formula.

$A(\tfrac{1}{2} , n) = N(0, 2, n) - N(1, 2, n)$.  Ramanujan claimed, Dragonette (1952) proved:
\begin{equation*}
  A(\tfrac{1}{2} , n) = \frac{(- 1)^{n - 1} e^{\pi \sqrt{\tilde{n}/6}}}{2 \sqrt{\tilde{n}}} + \dotsb.
\end{equation*}
Andrews (1966).

Conjectures by Andrews and Lewis, proved by Bringmann (2009).
\begin{equation*}
  A(\tfrac{1}{3} , n) = \frac{4 \sqrt{3}i}{(24 n - 1)^{1/2}}
  \sum_{3 \mid k}^{\lfloor \sqrt{n} \rfloor}
  \dotsb.
\end{equation*}
Should be an exact formula.

More generally, Bringmann gave asymptotics for $A(\tfrac{\ell}{u}, n)$.

Proofs use Maass Poincar{\'e} series.

\section{The rational torsion subgroups of the Drinfeld modular Jacobians for prime-power levels, \textnormal{\emph{Sheng-Yang Kevin Ho}}}
Preprint: \cite{2024arXiv2404.00738}.


$X_0(N)$: modular curve,
\begin{equation*}
  J_0(N)(\mathbb{Q}) \cong \mathbb{Z}^r \oplus \mathcal{T}(N).
\end{equation*}
Can we compute the rational torsion subgroup?

Let $\mathcal{C}(N)$ denote the rational cuspidal divisor class group for $X_0(N)$.  By a theorem of Manin and Drinfeld, we have
\begin{equation*}
  \mathcal{C}(N) \subseteq \mathcal{T}(N).
\end{equation*}

Conjecture (generalized Ogg's conjecture): for any positive integer $N$, we have $\mathcal{C}(N) = \mathcal{T}(N)$.

Strategy:
\begin{enumerate}
\item Compute $\mathcal{C}(N)$ explicitly to give a lower bound for $\mathcal{T}(N)$.
\item Study the Eisenstein ideal of the Hecke algebra of level $N$ to give an upper bound for $\mathcal{T}(N)$.
\end{enumerate}

Pass to the function field setting.

\section{A variation of a theme after Dirichlet, \textnormal{\emph{Chung Hang Kwan}}}

\begin{equation*}
  \int_0^1 \int_0^\infty \Phi \left[
    \begin{pmatrix}
      y_0      &  &  \\
               & y_0 &  \\
               &  & 1 \\
    \end{pmatrix}
    \begin{pmatrix}
      1      & u & 0 \\
      0 & 1 & 0 \\
      0 & 0 & 1 \\
    \end{pmatrix}\right]
  \,d^\times y_0
  e(- u) \, d u,
\end{equation*}
\begin{equation*}
  \int_0^\infty \int_0^1 \Phi \left[
    \begin{pmatrix}
      1      & u & \ast \\
      0             & 1 & \ast \\
      0             & 0 & 1 \\
    \end{pmatrix}
    \begin{pmatrix}
      y_0      & 0 & 0 \\
      0 & y_0 & 0 \\
      0 & 0 & 1 \\
    \end{pmatrix}\right]
  e(- u)
  \, d u
  \,d^\times y_0.
\end{equation*}


\section{On Hecke eigenvalues of Ikeda lifts, \textnormal{\emph{Nagarjuna Chary Addanki}}}

Preprint: \cite{2024arXiv2401.08855}.

$S_k(\Gamma_2)$ decomposes into a Maass space and its complement.  The Maass space is generated by Saito--Kurokawa lifts.  These lifts are characterized by the signs of their eigenvalues: $F$ is a Saito--Kurokawa lift if and only if all the eigenvalues are positive.  The question is whether there is a similar space in higher dimensions.  Ikeda lifts are generalizations of these Saito--Kurokawa lifts, so we can: are the eigenvalues of such lifts positive?

We're going to talk about Siegel modular forms, which are defined on the Siegel upper half-space
\begin{equation*}
  \mathbb{H}_n := \left\{ Z \in M_n(\mathbb{C}), \, {}^t Z = Z, \, \Im(Z) > 0  \right\}.
\end{equation*}
The symplectic group $\GSp_{2 n}(R) \leq \GL_{2 n}(R)$ is given by the condition ${}^t g J g = \mu(g) J$, where
\begin{equation*}
  J =
  \begin{pmatrix}
    0  & 1 \\
    -1 & 0 \\
  \end{pmatrix}.
\end{equation*}
We set $\Gamma_n := \Sp_{2n}(\mathbb{Z})$.  A Siegel modular form $F$ of weight $k$ over $\Gamma_n$ is defined in the usual way.  The space of such forms, $M_k(\Gamma_n)$, is finite-dimensional.  Elements $F$ admit a Fourier expansion
\begin{equation*}
  F(Z) = \sum_{
    \substack{
      T = T^t,  \\
      T \geq 0, \\
      \text{half-integral}
    }
  }
  A(T) e^{2 \pi i \trace(T Z)}.
\end{equation*}
We can define a notion of cusp form ($A(T) = 0$ unless $T >0$) and Petersson inner product.

For each $g \in G(\mathbb{Q})^+ \cap M_{2 n}(\mathbb{Z})$, there is an associated Hecke operator $T(g)$ on $M_k(\Gamma_n)$.  The Hecke algebra, $\mathcal{H}_n$ is generated by $T(g)$.
\begin{itemize}
\item $\mathcal{H}_n$ is commutative.
\item To each $m \in \mathbb{N}$ we may associate $T(m) \in \mathcal{H}_n$ such that $T(m n) = T(m) T(n)$ when $(m, n) = 1$.
\item Each $T(g)$ is self-adjoint with respect to the Petersson inner product.
\end{itemize}
Andrianov 1973: basis of simultaneous eigenfunctions.

Eigenvalues may be given in terms of the Satake parameters.  Let $F$ be a Hecke eigenform and $\lambda_F(g)$ the eigenvalue corresponding to the operator $T(g)$.  For any $g$ with $\mu(g) = p^r$, depending on $F$, there are $n + 1$ complex numbers
\begin{equation*}
  \left( a_{0, p}^{(F)}, a_{1, p}^{(F)}, \dotsc, a_{n, p}^{(F)} \right)
\end{equation*}
satisfying, with
\begin{equation*}
  \Gamma_n g \Gamma_n = \sqcup_i \Gamma_n g_i ,
  \quad
  g_i =
  \begin{pmatrix}
    A_i    &  B_i \\
    0           & D_i \\
  \end{pmatrix},
  \quad
  D_i =
  \begin{pmatrix}
    p^{d_{i 1}}    & \dotsb & \ast \\
    0                   & \dotsb & \ast \\
    0                   & \dotsb & p^{d_{i n}} \\
  \end{pmatrix},
\end{equation*}
we can explicitly describe the Hecke eigenvalue by
\begin{equation*}
  \lambda(g) = a_{0, p}^r \sum_i \prod_{j = 1}^n
  (a_{j, p}^{- j})^{d_{i j}}.
\end{equation*}.

Andrianov, 1974: if $F \in S_k(\Gamma_n)$ is a Hecke eigenform, then
\begin{equation*}
  \sum_{r = 0}^\infty \lambda_F(p^r) p^{- r s}
  = \frac{P_{F, p}(p^{- s})}{Q_{F, p}(p^{- s})},
\end{equation*}

\begin{equation*}
  Q_{F, p}(p^{- s}) = \dotsb.
\end{equation*}

Let $f \in S_k(\Gamma_1)$ be a Hecke eigenform.
\begin{theorem}[Ikeda, 2001]
  Assume that $n \equiv k \pmod{2}$.  Then
  \begin{equation*}
    F_f(Z) = \sum_{B \in \mathcal{S}_{2 n}(\mathbb{Z})^+}
    A(B) e^{2 \pi i \trace(B Z)},
    \quad
    Z \in \mathbb{H}_{2 n}
  \end{equation*}
  is a Hecke eigenform in $S_{k + n}(\Gamma_{2 n})$ and the standard $L$-function of $F$ is equal to
  \begin{equation*}
    \zeta(s) \prod_{i = 1}^{2 n}
    L(s + k + n - i, f, \mathrm{spin}).
  \end{equation*}
  Here $S_{2 n}(\mathbb{Z})^+$ is the set of all positive definite half-integral matrices of size $2 n$.
\end{theorem}

To compute the spin $L$-function of the Ikeda lift, move to the language of automorphic representations.  Let $F_f \in S_{k + n}(\Gamma_{2 n})$ be the Ikeda lift of $f \in S_{2 k}(\Gamma_1)$.  Let $\Pi$ and $\pi$ be the associated automorphic representations.
\begin{theorem}[Schmidt, 2003]
  We have
  \begin{equation*}
    L_p(s, \Pi, \mathrm{spin})
    = \prod_{j = 0}^n
    \prod_{
      \substack{
        r = j(j - 2 n)  \\
        \text{Step }2
      }
    }^{j(2n - j)}
    L_p(s + \tfrac{r}{2}, \pi, \Sym^{n - j})^{\beta(r, j, n)}.
  \end{equation*}
  Here $\beta(r, j, n)$ is given explicitly.
\end{theorem}

\begin{theorem}[A, 2024]
  For $n \equiv k \pmod{2}$, let $F_f \in S_{k + n}(\Gamma_{2 n})$ be the Ikeda lift of $f \in S_{2 k}(\Gamma_1)$.  For all large enough primes $p$, we have $\lambda_{F_f}(p) \geq 0$.
\end{theorem}

\begin{lemma}
  If $F_f \in S_{k + 2}(\Gamma_4)$ is the Ikeda lift of $f \in S_{2 k}(\Gamma_1)$, then $\lambda_{F_f}(p)$ is positive for all primes $p$.
\end{lemma}

For $\lambda_F(p^r)$ with $r > 1$, we use
\begin{equation*}
  \sum_{r = 0}^\infty \lambda_F(p^r) p^{- r s}
  = \frac{P_{F, p}(p^{- s})}{Q_{F, p}(p^{- s})},
\end{equation*}
and we need knowledge of both numerator and denominator.  The denominator can be computed using Schmidt's formluaf or the $L$-function, but the numerator is known only for genus at most $4$.

Let $T(p^4)$ be Hecke operators in $\mathcal{H}_4$.  Vankov (2011) found polynomials $P_p$ and $Q_p$ over the Hecke algebra such that
\begin{equation*}
  \sum_{r \geq 0} T(p^r) x^r = \frac{P_p(x)}{ Q_p(x)}.
\end{equation*}

$P_{p, F}(x)$ can be retrieved from $P_p(x)$.

$\lambda_F(p^r)$ for genus $4$ Ikeda lifts?  We do the following.
\begin{itemize}
\item Express $\frac{1}{Q_{p, F}(p^{- s})}$ as partial fractions.
\item Write each partial fraction as a power series in $p^{- s}$.
\item Multiply by $P_{p, F}(p^{- s})$.
\item Compare the coefficients of $p^{- r s}$ on both sides.
\end{itemize}
This yields a formula for $\lambda_{F_f}(p^r)$ in terms of some $c_i \in \mathbb{Z}[a, a^{-1}]$ that are bounded, uniformly in $r$.  This lead sto:
\begin{theorem}[A, 2024]
  Let $F_f$ be a Hecke eigenform in $S_{k + 2}(\Gamma_4)$, the Ikeda lift of $f \in S_{2 k}(\Gamma_1)$.  FOr a fixed $r$, the number $\lambda_{F_f}(p^r)$ is positive for all large enough primes $p$.
\end{theorem}


\section{Abelian covers of P1 of p-ordinary Ekedahl--Oort type, \textnormal{\emph{Deepesh Singhal }}}

Preprint: \cite{2023arXiv2303.13350}.

Fix a prime $p$.  Let $A$ be a $g$-dimensional principally polarized abelian variety defined over $\bar{\mathbb{F}}_p$.  There are three invariants attached to $A$:
\begin{itemize}
\item $p$-rank: $f(A) = \dim_{\bar{\mathbb{F}}_p}(\mu_p, A)$.  Note that
  \begin{itemize}
  \item $\# A[p](\bar{\mathbb{F}}_p) = p^f$, 
  \item $0 \leq f \leq g$.
  \end{itemize}
\item Newton polygon: the isogeny class of the $p$-divisible group $A[p^\infty]$, which can be presented as a sequence of $2 g$ rational numbers $N P(A) :=(\nu_1, \dotsc, \nu_{2 g})$.
\item Ekedahl--Oort type: the isomorphism class of the $p$-kernel $A[p]$, which can be presented as an element $w(A) \in \mathfrak{S}_{2 g}$.
\end{itemize}

When $g =1$, $A$ is an elliptic curve:
\begin{itemize}
\item $f(A) = 0$ iff $N P(A) =(\tfrac{1}{2}, \tfrac{1}{2})$ iff $w(A) = \id$ iff $A$ is supersingular.
\item $f(A) = 1$ iff $N P(A) =(0, 1)$ iff $w(A) =(21)$ iff $A$ is ordinary.
\end{itemize}

When $g \geq 2$, these are distinct invariants with some relation among each other:
\begin{itemize}
\item $p$-rank can be computed from either the Newton polygon or the $\mathrm{EO}$-type.
\item $f(A) = \# \{i : v_i = 0\}$.
\end{itemize}

One has the \emph{Torelli morphism}  from the moduli of curves to the moduli of abelian varieties:
\begin{equation*}
  \text{Hurwitz space} \xrightarrow{T} \text{Shimura variety}
\end{equation*}
\begin{equation*}
  C \xrightarrow{T} J(C).
\end{equation*}
\begin{itemize}
\item Fact: $\dim(\im(T)) < \dim(\text{ambient Shimura variety})$.
\item For $S = \mathcal{A}_{g, 1}$, it is known which $p$-rank, Newton polygons and $\mathrm{EO}$-types give \emph{nonempty} stratum.
\end{itemize}
\begin{question}
  Which stratum in $\mathcal{S}$ has nonempty intersection with the Torelli locus?
\end{question}

We turn to the Hurwitz space of cyclic covers of $\mathbb{P}^1$.  Fix $m$ such that $p \nmid m$.  $\mathcal{M}_{\mu_m}$ is the moduli space of $\mu_m$ covers of $\mathbb{P}^1$.
\begin{itemize}
\item $\mathcal{M}_{\mu_m} = \cup \mathcal{M}(m, r, \underline{a})$ is the decomposition of irreducible components.
\item $r$ is the number of branching points.
\item $\underline{a} =(a(1), a(2), \dotsc, a(r))$ is the monodromy data of each branching point.
\item Looking at one of the irreducible components, we have have the restricted Torelli morphism
  \begin{equation*}
    \mathcal{M}(m, r, \underline{a}) \xrightarrow{T} \operatorname{Sh}(H, \mu).
  \end{equation*}
\end{itemize}
Here
\begin{itemize}
\item $\operatorname{Sh}(H, \mu)$ is the smallest PEL type Shimura variety containing the image of $T$, with $\mu$ determined by the data $\underline{a}$.
\item The set of the Newton polygons occurring is known and denoted by $B(H_{\mathbb{Q}_p}, \mu)$.  However, very little is known about which of these Newton polygons have a stratum that intersects the Torelli locus.
\item The set $B(H_{\mathbb{Q}_p}, \mu)$ has a partial ordering, and $\nu(H, \mu)$ is the unique maximal element in this set.  It is called the $\mu$\emph{-ordinary} Newton polygon.
\end{itemize}

\begin{example}
  Consider the moduli space of $\mu_7$-covers of $\mathbb{P}^1$, with $r =4$ branching points, and monodromy datum $(3, 2, 1, 1)$.  If $p \equiv 6 \pmod{7}$, then there are three Newton polygons:
  \begin{itemize}
  \item the $\mu$-ordinary $\mu =(0,1)^4 \oplus(\tfrac{1}{2}, \tfrac{1}{2})^2$, with $p$-rank $4$.
  \item $\nu =(0, 1)^2 \oplus(\tfrac{1}{2}, \tfrac{1}{2} )^4$, with $p$-rank $2$.
  \item the basic $\beta =(\tfrac{1}{2}, \tfrac{1}{2} )^6$, with $p$-rank $0$.
  \end{itemize}
\end{example}

We refine Bouw's result and show the sharpness of the upper bound on Newton polygon given by $B(\mathbb{H}_{\mathbb{Q}_p}, \mu)$:
\begin{theorem}[Y.\ Lin, E.\ Mantovan, S.,\ 2023]
  Fix a prime $p$ and let $m \in \mathbb{N}$ such that $p \nmid m$ and $p > m(r - 2)$.  Then for any $\mathcal{M}(m, r, \underline{a})$ such that $r \leq 5$, we have
  \begin{equation*}
    T(\mathcal{M}(G, r, \underline{a}))[\nu(H, \mu)] \neq \emptyset.
  \end{equation*}
  In addition, we also show (...).
\end{theorem}

Comments:
\begin{itemize}
\item Same conclusion holds for abelian covers.
\item The statement of the theorem likely remains true without the assumptions that $p > m(r - 2)$ and $r \leq 5$.  However, these conditions are needed for our technique to work.
\item (...)
\end{itemize}

Proof strategy:
\begin{theorem}[Moonen, 2004]
  For a PEL-type Shimura variety $\operatorname{Sh}(H, \mu)$, we have
  \begin{equation*}
    \nu(H, \mu) \text{ stratum }
    =
    \text{ maximal EO stratum }
    \subseteq \text{ maximal $p$-rank stratum.}
  \end{equation*}
\end{theorem}
To prove the non-emptiness of the intersection with maximal EO stratum, we use the equivalence of categories between
\begin{itemize}
\item $p$-kernel of polarized abelian variety,
\item polarized mod $p$ Dieudonn{\'e} module, 
\item Hasse--Witt triples.
\end{itemize}
We translate the condition of having maximal EO type to linear algebra conditions on the Hasse--Witt triple.  By explicitly computing them, we are able to verify the rank conditions.

\section{Hyperbolic Counting Problems, \textnormal{\emph{Marius Voskou}}}
Preprint: \cite{2024arXiv2407.03134}.

We start with an interlude on hyperbolic geometry.  Work with $\PSL_2(\mathbb{R})$ (orientation-preserving isometries), lattice points $\gamma \cdot z$, where $\gamma \in \Gamma$, a discrete cofinite subgroup of $\PSL_2(\mathbb{R})$, and $z \in \mathbb{H}$.

Consider a hyperbolic circle of center $w$, radius $R$.

This looks like a Euclidean circle; the only difference is that the center is slightly lower than what you'd expect, which has to do with the fact that distances blow up as you get closer to the $x$-axis.

We're now ready to formulate our first hyperbolic counting problem: take
\begin{equation*}
  N(z, w, R) := \# \left\{ \gamma z : \gamma \in \Gamma, \dist(w, \gamma z) \leq R \right\}.
\end{equation*}

The same method as in the Euclidean counting problem does, in principle work.  We get a main term comparable to the area, an error term comparable to the circumference.  The problem is that in hyperbolic geometry, these are comparable -- both are comparable to $e^R$ -- so the error term is not smaller than the main term.  Instead, we use the spectral theory of automorphic forms.

To that end, the first step is to write our count in terms of the automorphic kernel:
\begin{equation*}
  N(z, w, R) = K(z, w)
  = \sum_{\gamma} \mathbf{1}(\dist(w, \gamma z) < R).
\end{equation*}
We know that automorphic kernels have nice spectral expansions in terms of eigenfunctions of the hyperbolic Laplacian.  Since it's a 20 minute talk, we'll just take spectral theory as a black box and not say much about it.  The thing we want to note is that the hyperbolic Laplacian has two kinds of eigenvalues:
\begin{itemize}
\item large eigenvalues: $\lambda \geq 1/4$, $\lambda = s(1 - s), s = 1/2 + i t$.
\item small eigenvalues: $\lambda < 1/4, \lambda = s(1 - s), s \in \left(1/2,1\right]$, of which there are finitely many.
\end{itemize}
Using this spectral expansion, Selberg in the 70's proved an asymptotic formula for $N(z, w, R)$:
\begin{theorem}[Selberg 1970, Gunther 1980, Good 1983]
  We have
  \begin{equation*}
    N(z, w, \log X)
    = \frac{2 \pi}{\area(\mathbb{H} / \Gamma)}
    X + \sum_{s_j}
    c_j(z, w) X^{s_j} + \O(X^{2/3}),
  \end{equation*}
  where the sum is over finitely many real numbers $s_j \in(1/2, 1)$, corresponding to the small eigenvalues.
\end{theorem}

Now, let's move to a slightly more complicated problem: let's ask what happens if instead of points, we have geodesics.  For fixed (closed) geodesics $\ell_1, \ell_2$, how many geodesics $\gamma \cdot \ell_1$ have distance $\leq R$ from $\ell_2$?  For simplicity, take $\ell_1 = \ell_2 = \ell$.

[Picture of a pair of closed geodeics intersecting.]

The distance between closed geodesics is the length of the common perpendicular.  We want to count how many closed geodesics are at given distance from one closed geodesic.

Our strategy is again to use the spectral theory of automorphic forms.  We can write the count as the integral of an automorphic kernel:
\begin{equation*}
  N(\ell_1, \ell_2, R) = \int_{\ell_1} \int_{\ell_2}
  K(z, w)
  \, d s(z)
  \, d s(w).
\end{equation*}
This leads to an asymptotic formula:
\begin{theorem}[Good 1983, Lekkas--Petridis 2024]
  We have
  \begin{equation*}
    N(\ell, \log X)
    = \frac{2(\len \ell)^2}{\pi \area(\mathbb{H} / \Gamma)} X
    + \sum_{s_j}(\ell) X^{s_j} + E(X),
  \end{equation*}
  where $E(X) = \O(X^{2/3})$.
\end{theorem}

Let's say a few words about why it's necessary to have a second proof of this.  The first proof by Good was hard to read, and some people still question whether it was valid.  In particular, if you want to do refinements of this or apply his methods to obtain related results, it's difficult.  Petridis--Lekkas found simpler methods that also apply to other problems.

\begin{conjecture}
  We have $E(X) = \O(X^{1/2 + \eps})$.  (And similarly for the previous problem, and many similar problems.)
\end{conjecture}

This has been around for many decades, and we have zero improvements on $X^{2/3}$.

What we're going to try to do now is to motivate this conjecture.  To that end, we'll start by saying why we can't do any better:
\begin{theorem}[V.\ 2024]
  We  have
  \begin{equation*}
    E(X) = \Omega \left( X^{1/2}(\log \log X)^{1/4 - \delta} \right).
  \end{equation*}  
\end{theorem}

\begin{theorem}[Lekkas--Petridis 2024]
  We have
  \begin{equation*}
    \left( \frac{1}{X} \int_X^{2 X} \lvert E(u) \rvert^2 \, d u \right)^{1/2} \ll X^{1/2} \log X.
  \end{equation*}  
\end{theorem}

Now, let's restrict to the case of a primitive hyperbolic element
\begin{equation*}
  h =
  \begin{pmatrix}
    m    & 0 \\
    0 & m^{-1} \\
  \end{pmatrix}.
\end{equation*}
For $\ell = \mathcal{I} / \langle h \rangle$, we have
\begin{equation*}
  \cosh \dist(\ell, \gamma \ell) = \max \left( \lvert a d + b c \rvert, 1 \right),
\end{equation*}
so our counting problem can be rephrased as follows: how many double cosets $\gamma \in \langle h \rangle \backslash \Gamma / \langle h \rangle$ with $\lvert a d + b c \rvert < X$?  We can also ask, are there more $\gamma$ with $a d + b c > 0$ than $a d + b c < 0$, or less?  Equivalently, are there more $a, d$ having the same sign than different, or less?  Geometric interpretation: the sign corresponds to the direction of the image $\gamma \cdot \ell$, and whether it lies on the left or the right of $\ell$.  This behaves differently as we tilde $\ell$ to some $\ell_\theta$.  Idea: consider
\begin{equation*}
  \frac{\partial^2}{\partial \theta \partial \phi}
  N(\ell_\theta, \ell_\phi, R).  
\end{equation*}

Let's finish with a nice arithmetic application.  Fix a prime $p$.  For $\Gamma$ an appropriate quaternion order, and $m = \left( 1 + \sqrt{2} \right)^2$:
\begin{theorem}[V.\ 2023, $p=5$; Hejhal 1982]
  We have
  \begin{equation*}
    \sum_{n \leq X} \mathcal{N}(n) \mathcal{N}(p n \pm 1)
    =
    \frac{p}{p + \qr{2}{p}}
    \cdot \left( \frac{\log m}{ \pi} \right)^2 X +
    \sum_{1/2 < s_j < 1}
    a_j^{\pm} X^{s_j} + \O(X^{2/3}),
  \end{equation*}
  where $a_j^{\pm}$ is real and $\mathcal{N}(n)$ is the number of $\mathfrak{a} \leq \mathbb{Z}[\sqrt{2}]$ with $\norm(\mathfrak{a}) = n$.
\end{theorem}

That's all.
\bibliography{refs}{} \bibliographystyle{plain}
\end{document}
