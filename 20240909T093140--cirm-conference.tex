\documentclass[reqno]{amsart} \input{common.tex}

\begin{document}

\title{Notes from CIRM conference BB6, \emph{Automorphic Forms and Related Topics}} 

\begin{abstract}
  Random notes from the CIRM conference, Building Bridges 6th, EU/US Workshop on Automorphic Forms and Related Topics (BB6),
  9-13 September, 2024.  These notes are incomplete and have not been proofread.  Any errors should be assumed to be due to the note-taker.
\end{abstract}


\section{Anne-Maria Ernvall-hytönen, \textnormal{\emph{Lattices and modular forms in coset coding}}}

Wyner (1975).  Wiretap channel (Alice, Bob, Eve).  Gaussian noise with variance $\sigma^2$ (between Alice and Bob) and $\sigma_e^2$ (for Eve).  We assume $\sigma_e > \sigma$.  We would like to do this whole scheme in such a way that we can use the noise so that Bob can still receive the image, but Eve cannot.  We aim to do this with lattices $\Lambda \geq \Lambda_e$, where each coset corresponds to a code-letter.

Belfiore and Oggier (2010) (maybe \cite{MR4032962}?): secrecy gain.  Transmit codeword $x \in \mathbb{R}^n$.
\begin{equation*}
  \frac{1}{(\sigma_\Lambda \sqrt{2 \pi})^n}
  \int_{V_{\Lambda(x)}}
  e^{- \lVert y - x \rVert^2 / 2 \sigma^2} \, d y.
\end{equation*}

\begin{equation*}
  \frac{1}{(\sigma_\Lambda \sqrt{2 \pi})^n}
  \sum_{t \in \Lambda_e \cap R}
  \int_{V_{\Lambda(x + t)}}
  e^{- \lVert y - x \rVert^2 / 2 \sigma^2} \, d y.
\end{equation*}

End up trying to minimize
\begin{equation*}
  \theta_{\Lambda_e} \left( - \frac{1}{2 \pi i \sigma_e^2} \right).
\end{equation*}
Secrecy function
\begin{equation*}
  \Xi(y) :=
  \frac{\theta_{\lambda \mathbb{Z}^n}(y i)}{ \theta_\Lambda(y i)}.
\end{equation*}
Belfiore and Sole: For unimodular $\Lambda$, maximum at $y = 1$.  Even unimodular: polynomials in
\begin{equation*}
  E_4 = \frac{1}{2} \left( \vartheta_2^8 + \vartheta_3^8 + \vartheta_4^8  \right),
  \qquad
  \Lambda = \frac{1}{256} \vartheta_2^8 \vartheta_3^8 \vartheta_4^8.
\end{equation*}
Inverse of $\Xi(y)$ a polynomial in $\frac{\vartheta_4^4 \vartheta_2^4}{\vartheta_3^8}$.

(Maybe \cite{MR2966067} is a reference.)

$\ell$-modular: $1 / \sqrt{\ell}$, $\mathbb{Z} \oplus \sqrt{2} \mathbb{Z} \oplus 2 \mathbb{Z}$.

The connection to deeper mathematics is, what kinds of representations as polynomials do you have for various theta functions?

\section{Jolanta Marzec-ballesteros, \textnormal{\emph{Doubling method for self-dual linear codes}}}

Garrett, Piatetski-Shapiro--Rallis (1980's).  Integral of cusp form against restriction of Siegel-type Eisenstein series equals $L$-function attached to cusp form times cusp form or Eisenstein series attached to cusp form:
\begin{equation*}
  \left\langle E \left(
      \begin{pmatrix}
        g &  \\
          & g' \\
      \end{pmatrix}, s \right),
    f(g)\right\rangle
  = L(f, s) f(g ').
\end{equation*}

Done for $G$ symplectic, orthogonal, unitary over global field, also for congreunce subgroups.

Let's start with an overview.  Let $f$ be a cusp form on $G$, and $H$ a subgroup for which $G \times G \hookrightarrow H$.  Then form an Eisenstein series on $H$
\begin{equation*}
  E(h, s) = \sum_{\gamma \in P \backslash H} \phi(\gamma h, s).
\end{equation*}
Restrict to $h = \diag(g, g')$ and take inner product with $F(g)$.  This leads to an unfolding involving a sum over $\gamma \in P \backslash H /(G \times G)$.  In favorable cases, only one representative $\gamma_0$ contributes, leaving us with
\begin{align*}
  &\sum_{(k, k') \in G \times G}
    \left\langle \phi \left( \gamma_0
    \begin{pmatrix}
      k g        &  \\
                 & k ' g ' \\
    \end{pmatrix}, s \right), f(g) \right\rangle
  \\
  &=
    \sum_{\beta \in \gamma_0(G \times 1)}
    \psi(f) f |_\beta(g ') \\
  &= L(f, s) f(g ').
\end{align*}
We would like to do something similar, but now over finite fields.

A \emph{linear code of length} $2 n$ over a finite field $\mathbb{F}$ is a linear subspace $C \subseteq \mathbb{F}^{2 n}$.  We denote by $\langle , \rangle : C \times C \rightarrow \mathbb{F}$ the Euclidean inner product.  We say that $C$ is \emph{self-dual} if
\begin{equation*}
  C = C^\perp := \left\{ v \in \mathbb{F}^{2 n} : \langle v, C \rangle = 0 \right\}.
\end{equation*}
Then (the length $2 n$ is even and) $\dim C = n$.

The \emph{weight} of a codeword $c =(c_1, \dotsc, c_{2 n}) \in C$ is
\begin{equation*}
  \operatorname{wt}c = \# \left\{ i \in \{1, \dotsc, 2 n\} : c_i \neq 0 \right\}.
\end{equation*}

\emph{Weight enumerators} are certain homogeneous polynomials of degree $2 n$ in variables from the set $V = \left\{ x_\alpha : \alpha \in \mathbb{F}^g \right\}$, where $g \in \mathbb{N}$ is the genus.

The \emph{genus one weight enumerator} of a code $C \subseteq \mathbb{F}_2^{2 n}$ is a polynomial
\begin{equation*}
  W_1(C,(x_0, x_1))
  = \sum_{c \in C} x_0^{2 n - \operatorname{wt} c}
  x_1^{\operatorname{wt} c}.
\end{equation*}
The \emph{genus} $g$ \emph{weight enumerator} of a code $C \subseteq \mathbb{F}^{2 n}$ is a polynomial
\begin{equation*}
  W_g(C, x) = \sum_{(c^1, \dotsc, c^g) \in C^g }
  \prod_{\alpha \in \mathbb{F}^g}
  x_\alpha^{w_\alpha(c^1, \dotsc, c^g)}
\end{equation*}
of degree $2 n$, where $x =(x_\alpha)_{\alpha \in \mathbb{F}^g}$ and
\begin{equation*}
  w_\alpha(c^1, \dotsc, c^g)
  = \# \left\{ \text{rows $r$ in } (c_{i}^j)_{i = 1..2n}^{j = 1..g} : r = \alpha \right\}.
\end{equation*}

As an example, we give a basis for a Hamming code $H_8$ 5nof weight $8$, the span over $\mathbb{F}_2$ inside $\mathbb{F}_2^8$ of the vectors
\begin{equation*}
  \begin{pmatrix}
    1 \\ 0 \\ 0 \\ 1 \\ 0 \\ 1 \\ 1 \\ 0
  \end{pmatrix}
  \quad
  \begin{pmatrix}
    0 \\ 1 \\ 0 \\ 1 \\ 0 \\ 1 \\ 0 \\ 1
  \end{pmatrix}
  \quad
  \begin{pmatrix}
    0 \\ 0 \\ 1 \\ 1 \\ 0 \\ 0 \\ 1 \\ 1
  \end{pmatrix}
  \quad
  \begin{pmatrix}
    0 \\ 0 \\ 0 \\ 0 \\ 1 \\ 1 \\ 1 \\ 1
  \end{pmatrix}.
\end{equation*}
Then
\begin{align*}
  W_2(H_8,(x_{00}, x_{01}, x_{10}, x_{11}))
  &= \sum_{\alpha \in \mathbb{F}_2^2}
  x_\alpha^8 + 14
  \sum_{
    \substack{
      \alpha_1, \alpha_2 \in \mathbb{F}_2^2  \\
      \alpha_1 < \alpha_2      
    }
  }
  x_{\alpha_1}^4 x_{\alpha_2}^4
  + 168 x_{00}^2 x_{0 1}^2 x_{10}^2 x_{11}^2
  \\
  &=(8) +
    14(4, 4)
    + 168(2, 2, 2, 2).
\end{align*}
In general,
\begin{equation*}
  W_g(C) = \sum_A b_A \cdot(A),
\end{equation*}
where
\begin{equation*}
  A \in \left\{(a_0, \dotsc, a_{2^g - 1}) : \text{admissible tuples},
    \,
    \sum_{i = 0}^{2^g - 1} a_i = 2 n\right\}.
\end{equation*}

Some analogies with modular forms:
\begin{itemize}
\item $W_g(C)$ is like a modular form $f$ of genus $g$,
\item $\sum_A b_A \cdot(A)$ is like a Fourier expansion,
\item $(2 n)$ is like a constant term $a(0)$.
\end{itemize}
EXamples of cusp forms:
\begin{itemize}
\item $W_1(G_{24}) - W_1(H_8 \times H_8 \times H_8)
  = - 42(20, 4) + 168(16, 8) - 252(12, 12)$ is a cusp form of genus one.
\item $W_3(E_{16}) - W_3(H_8 \times H_8) = - 2688(9, 1, 1, 1, 1, 1, 1, 1) + \dotsb$ is a cusp form of genus $3$.
\end{itemize}

\begin{theorem}[Runge, 1996; Nebe, Rains, Sloane, 2006]
  We have
  \begin{equation*}
    \left\langle W_g(C) : \text{$C$ self-dual, over $\mathbb{F}$} \right\rangle
    =
    \left( \mathbb{C}[x_\alpha : \alpha \in \mathbb{F}^g] \right)^{\mathcal{C}_g},
  \end{equation*}
  where
  \begin{equation*}
    \mathcal{C}_g := \left\langle m_r, d_\phi, h_{\iota, u_{\iota}, v_{\iota}} : r \in \GL_g(\mathbb{F}), \phi, \iota \right\rangle
  \end{equation*}
  with
  \begin{equation*}
    m_r : x_\alpha \mapsto x_{r \alpha},
  \end{equation*}
  \begin{equation*}
    d_\phi : x_\alpha \mapsto e^{2 \pi i \phi(\alpha)} x_\alpha,
  \end{equation*}
  \begin{equation*}
    h_{\iota, u_\iota, v_\iota}
    : x_\alpha \mapsto \left( \# \iota \mathbb{F}^g \right)^{- 1/2}
    \sum_{w \in \iota \mathbb{F}^g}
    e^{\frac{2 \pi i}{p} \left\langle w, v_\iota \alpha \right\rangle} x_w + \dotsb.
  \end{equation*}
\end{theorem}
Consider the mean polynomial (``Siegel-type Eisenstein series'')
\begin{align*}
  M_{2 g}((2 n)) &= \sum_{\gamma \in P_{2 g} \backslash \mathcal{C}_{2 g}}
  (2 n)^\gamma
  \\
  &=
  \sum_{\gamma \in P_{2 g} \backslash \mathcal{C}_{2 g}} \sum_{\alpha \in \mathbb{F}^{2 g}}
  \left((x_\alpha)^\gamma \right)^{2 n}
  = \mathrm{const} \sum_{
    \substack{
      C \subset \mathbb{F}^{2 n}  \\
      \text{fixed type}      
    }
  }
  W_{2 g}(C, x),
\end{align*}
and an inner product defined on monomials.

What we prove with Bourganis in 2024 is the following:
\begin{theorem}
  Let $\mathcal{T}$ be a family of self-dual codes of length $2 n$ over a field $\mathbb{F}$.  Assume either that $\mathbb{F}$ has odd characteristic or is equal to $\mathbb{F}_2$.  Let $C \in \mathcal{T}$ be doubly-even, and fix $g \in \mathbb{N}$.  Then there exists an (explicit) constant $C$ such that for a cusp form $f \in \mathcal{T}$ of genus $r$, with $\deg f = 2 n$, we have
  \begin{equation*}
    \left\langle M_{2 g}((2 n))(x y), f(x) \right\rangle
    =
    \begin{cases}
      0      & \text{ if } r < g, \\
      C \cdot f(y)             & \text{ if } r = g
    \end{cases}.
  \end{equation*}
\end{theorem}

\section{Petru Constantinescu, \textnormal{\emph{Non-vanishing of geodesic periods of automorphic forms}}}
Preprint: \cite{2024arXiv2404.12982}, joint with Asbhj{\o}rn Nordentoft.

Class groups:
\begin{itemize}
\item $\Gamma = \PSL_2(\mathbb{Z})$, $K = \mathbb{Q}(\sqrt{D})$,
\item $\mathrm{Cl}_K$ class group, $h(D) = h_K = \lvert \mathrm{Cl}_K \rvert$ class number,
\item $\mathcal{Q}_D$: set of primitive integral binary quadratic forms of discriminant $D$,
  \begin{equation*}
    \mathcal{Q}_D = \left\{ Q(x, y)
      = a x^2 + b x y + c y^2 :(a, b, c) = 1,
      \,
      b^2 - 4 a c = D\right\}.
  \end{equation*}
\item Gauss: $\Gamma \circlearrowright \mathcal{Q}_D,$
  \begin{equation*}
    (Q . \gamma)
    \begin{pmatrix}
      x      \\
      y
    \end{pmatrix}
    = Q \left( \gamma
      \begin{pmatrix}
        x        \\
        y  \\
      \end{pmatrix} \right),
  \end{equation*}
  
\item isomorphism
  \begin{equation*}
    \mathrm{Cl}_K \xrightarrow{\cong} \Gamma \backslash \mathcal{Q}_D
  \end{equation*}
  \begin{equation*}
    A \mapsto[a, b, c].
  \end{equation*}
\end{itemize}

Heegner points and closed geodesics:
\begin{itemize}
\item $D < 0$: $A \in \mathrm{Cl_K} \rightsquigarrow$ Heegner point $z_A \in \Gamma \backslash \mathbb{H}$,
  \begin{equation*}
    [a, b, c] \rightsquigarrow \frac{- b - i \sqrt{\lvert D \rvert}}{2 a}.
  \end{equation*}
  \begin{equation*}
    h(D) = \lvert \mathrm{Cl}_K \rvert = \lvert D \rvert^{1/2 + o(1)}.
  \end{equation*}
\item $D > 0$: $A \in \mathrm{Cl}_K^+ \rightsquigarrow $ closed geodesic $C_A \subset \Gamma \backslash \mathbb{H}$,
  $[a, b, c] \rightsquigarrow$ semicircle with endpoints $\frac{- b \pm \sqrt{D}}{2 a}$.
  \begin{equation*}
    h(D) \log \eps_D = D^{1/2 + o(1)},
  \end{equation*}
  \begin{equation*}
    I(C_A) = 2 \log \eps_D.
  \end{equation*}
\end{itemize}

\begin{theorem}[Duke '88]
  Fix $\Omega \subset \PSL_2(\mathbb{Z}) \backslash \mathbb{H}$.

  \begin{itemize}
  \item $D < 0$: equidistribution of $z_A$, $A \in \Cl_{\mathbb{Q}(\sqrt{D})}$.
  \item $D > 0$: similar for closed geodesics.
  \end{itemize}
\end{theorem}


\textbf{Waldspurger formulas}.  Let $f$ be a nonzero Maass form.  Our goal is to study closed geodesic periods $\int_{C_A} f(z) \, \frac{\lvert d z \rvert}{y}$.

$\chi \in \widehat{\Cl_K} \rightsquigarrow $ $\theta_\chi$, the associated theta series (weight one on $\Gamma_0(D)$, nebentypus $\chi_D$).

\begin{theorem}[Waldspurger/Zhang/Popa]
  Let $D$ be a fundamental discriminant.  For $D < 0$,
  \begin{equation*}
    L(f \times \theta_\chi, \tfrac{1}{2})
    = \frac{C_f}{ D^{1/2}}
    \left| \sum_{A \in \Cl_K} \chi(A) f(z_A) \right|^2.
  \end{equation*}
  For $D > 0$,
  \begin{equation*}
    L(f \times \theta_\chi, \tfrac{1}{2}) = \frac{C_f}{ D^{1/2}}
    \left| \sum_{A \in \Cl_K^+} \chi(A) \int_{C_A} f(z)
      \, \frac{\lvert d z \rvert}{y}
    \right|^2.
  \end{equation*}
\end{theorem}

\begin{theorem}[Michel--Venkatesh '05]
  Let $\delta = 1/2700$.  For $D < 0$:
  \begin{equation*}
    \left| \left\{ \chi \in \widehat{\Cl_K} : L(f \times \theta_\chi, \tfrac{1}{2}) \neq 0 \right\} \right|
    \gg D^\delta.
  \end{equation*}
\end{theorem}
\begin{proof}[Sketch of proof]
  By orthogonality of characters,
  \begin{equation*}
    \frac{1}{h(D)}
    \sum_{\chi \in \widehat{\Cl_K}}
    L(f \times \theta_\chi, \tfrac{1}{2})
    = \frac{c_f}{ D^{1/2}}
    \sum_{A \in \Cl_K} \left| \dotsb \right|^2.
  \end{equation*}
  This converges by Duke's equidistribution theorem.  The conclusion then follows from subconvexity for Rankin--Selberg $L$-functions, due to Harcos--Michel.
\end{proof}

Same proof does not work for geodesics, cannot apply equidistribution and relate to subconvexity (square is outside integral).  

\begin{question}[Michel, Oberwolfach 2020]
  Let $K$ be a real quadratic field of discriminant $D > 0$, and assume that $h_K  \gg D^\delta$.  Does there exist $A \in \Cl_K^+$ such that
  \begin{equation*}
    \int_{C_A} f \, \frac{\lvert d z \rvert}{y} \neq 0?
  \end{equation*}
  Equivalently, does there exist $\chi \in \widehat{\Cl_K^+}$ such that $L(f \times \theta_\chi, \tfrac{1}{2}) \neq 0$?
\end{question}

\textbf{The prime geodesic theorem}.  Let $D > 0, A \in \Cl_K^+$.  This gives rise to a closed geodesic $C_A$, with $I(C_A) = 2 \log \eps_D$.  Sound--Young have the best estimate for their count.

\begin{theorem}[C--Nordentoft 2024]
  Let $f$ be a nonzero Maass form for $\SL_2(\mathbb{Z})$.  Then
  \begin{equation*}
    \# \left\{ C \in \mathcal{C}(X) :
      \int_{C}
      f(z)
      \, \frac{\lvert d z \rvert}{y} =  0\right\}
    \ll
    \frac{X}{(\log X)^{5/4}}.
  \end{equation*}
\end{theorem}

\begin{remark}
  We also obtain 100\% non-vanishing for periods of weight $k$ holomorphic cusp forms, for any Fuchsian group $\Gamma$.
\end{remark}

\begin{theorem}[C--Nordentoft 2024]
  For a positive proportion of positive discriminants $D > 0$ with $\eps_\Delta \leq X$, we get that there exists $\chi \in \widehat{\Cl_{\mathbb{Q}(\sqrt{D})}}$
  with $L(f \times \theta_\chi, \tfrac{1}{2} \neq 0)$.
\end{theorem}

We construct a bipartite graph on $X_N$ (double cosets in $\Gamma_\infty \backslash \Gamma / \Gamma_\infty$ with $c \leq N$) times $Y_N$ (conjugacy classes with trace bounded in magnitude by $N$).  This graph relates closed geodesic and vertical geodesics.


\section{An excised orthogonal model for families of cusp forms}
Talk by Zoe Batterman (Abstract), Akash Narayanan, Christopher Yao.  Joint with Owen Barrett, Aditya Jambhale, and Kishan Sharma.  Preprint: \cite{2024arXiv2407.14526}.

Conjecture (Montgomery--Dyson, 1970's): zeros of of zeta vs. GUE.

2005: S.J.\ Miller noticed a repulsion of the lowest-lying zeros near the central point of a family of even twists of a fixed elliptic curve $L$-function with finite conductor.

2011: Duenez, Huynh, Keating, Miller, Snaith: proposed an excised orthogonal model to capture the behavior of this repulsion.

\begin{question}
  How accurately do egienvalues of random matrices from classical compact groups model the lowest-lying zeros of families of $L$-functions associated to a cuspidal newform?
\end{question}

Let
\begin{equation*}
  S_k^{\new}(M, \chi_f) \ni f(z) = \sum_{n = 1}^\infty a_f(n) e^{2 \pi i n z},
\end{equation*}
$\lambda_f(n) = a_f(n) / n^{(k - 1)/2}$.
\begin{equation*}
  L(s, f) = \sum_{n \geq 1} \lambda_f(n) n^{- s},
\end{equation*}
Various specific families of twists $L(f \otimes \psi_d, s)$, match with classical compact groups:
\begin{itemize}
\item principal character, even twists vs.\ $\SO(\mathrm{even})$
\item principal character, odd twists vs.\ $\SO(\mathrm{odd})$
\item non-principal character, self-dual vs.\ $\Sp$
\item generic vs.\ $\U$
\end{itemize}

Pictures.

\section{On an extension of the Rohrlich-Jensen formula, \textnormal{\emph{Lejla Smajlović}}}
Joint work with James Cogdell, Jay Jorgensen.  Preprint: \cite{2021arXiv2101.09599}.

What is a Poisson--Jensen formula?  We will view it as a way to characterize meromorphic functions in terms of their divisors.  Some notation:
\begin{itemize}
\item $D_R = \left\{ z = x + i y \in \mathbb{C} : \lvert z \rvert < R \right\}$
\item $F$: a non-constant meromorphic function on $\overline{D_R}$,
  \begin{equation*}
    F(z) = c_F z^m + \O(z^{m + 1}), \quad z \rightarrow 0.
  \end{equation*}
\end{itemize}
Then
\begin{equation*}
  \int_0^{2 \pi}
  \log \lvert F(R e^{i \theta}) \rvert
  \, \frac{d \theta}{2 \pi}
  + \sum _{D_R} \dotsb = F(0).
\end{equation*}

Rohrlich, 1980's: a \emph{modular} generalization, characterizing modular forms via divisors.  Given $f$, a meromorphic function on $\mathbb{H}$ that is invariant by $\PSL(2,\mathbb{Z})$.  Assume $f$ is holomorphic at the cusp and that the Fourier expansion of $f$ at $\infty$ has constant term equal to one.  Then
\begin{equation*}
  \int_{\PSL(2,\mathbb{Z}) \backslash \mathbb{H}}
  \log \lvert f(z) \rvert
  \, \frac{d \mu(z)}{2 \pi}
  + \sum_{w \in \mathcal{F}}
  \frac{\ord_w(f)}{ \ord(w)} P(w) = 0.
\end{equation*}
Here
\begin{itemize}
\item $\ord_w(f)$ is the order of $f$ at $w$ as a meromorphic function,
\item $\ord(w)$ denotes the order of the \emph{point} $w$ with respect to the action of $\PSL(2,\mathbb{Z})$ on $\mathbb{H}$, and
\item $P(w) = \log \left( \lvert \eta(w) \rvert^4 \cdot \Im(w) \right)$ is the Kronecker limit function associated to the parabolic Eisenstein series on $\PSL(2,\mathbb{Z})$.  This is the function appearing as the next-order term in the expansion of the Eisenstein series as $s \rightarrow 1$.
\end{itemize}
Another way to interpret this formula is as follows.  We have
\begin{equation*}
  \left\langle 1, \log \lvert f(z) \rvert \right\rangle
  = \lim_{Y \rightarrow \infty} \int_{\mathcal{F}(Y)} 1 \cdot \log \lvert f(z) \rvert \, d \mu(z),
\end{equation*}
hence the formula reads
\begin{equation*}
  \left\langle 1, \log \lvert f(z) \rvert \right\rangle = - 2 \pi
  \sum_{w \in \mathcal{F}} \frac{\ord_w(f)}{ \ord(w)} P(w).
\end{equation*}
Here $\log \lvert f(z) \rvert$ can be replaced by
\begin{equation*}
  \log \lVert f(z) \rVert = \log \left( \Im z^k \lvert f(z) \rvert \right)
\end{equation*}
for weight $2 k$ meromorphic modular forms.  Generalizations:
\begin{itemize}
\item to other Fuchsian groups of the first kind, by Rohrlich.
\item to hyperbolic $3$-space, by Herrero, Imamoglu, von Pippich, Toth.
\end{itemize}

Further modular generalization by Bringman and Kane.  Keep the modular group setting, but evaluate more general inner products.
\begin{itemize}
\item $j(z) = q^{-1} + 744 + \O(q)$: Hauptmodul
\item $j_1(z) := j(z) - 744$
\item $j_n(z) := j_1 | T_n(z)$, for $n \geq 2$
\item $f$: weight $2 k$ meromorphic modular form with respect to $\PSL(2, \mathbb{Z})$, normalized so that $f(z) = 1 + \O(q)$
\end{itemize}
They evaluated the regularized scalar product in terms of the divisor of $f$, proving that
\begin{equation*}
  \left\langle j_n(z),
    \log \left((\Im(z))^k \lvert f(z) \rvert \right)\right\rangle
  = - 2 \pi \sum_{w \in \mathcal{F}}
  \frac{\ord_w(f)}{\ord(w)}
  \mathbf{j}_n(w)
  + \frac{k}{6} c_n,
\end{equation*}
where $\mathbf{j}_n$ is characterized in terms of differential operators by Bringman and Kane; application of our results yields a different expression for the same function, namely
\begin{equation*}
  \mathbf{j}_n(w) = 2 \pi \sqrt{n} \partial_s F_{- n}^{\PSL(2, \mathbb{Z})}(w, s) |_{s=1}
  - 24 \sigma(n) P(w).
\end{equation*}

What is our main goal?
\begin{itemize}
\item To extend the point of view that the Rohrlich--Jensen formula is the evaluation of a particular type of inner product.
\item To prove the extension of this formula in the setting of an arbitrary, not necessarily arithmetic, Fuchsian group of the first kind with one cusp.
\end{itemize}

We start with $j_n(z) = j_1 | T_n(z)$, which is the unique (up to constants) holomorphic function that is $\PSL(2,\mathbb{Z})$-invariant on $\mathbb{H}$ and whose expansion near $\infty$ is $q^{- n} + o(q^{-1})$.

These properties hold for the special value $s = 1$ of the Niebur--Poincar{\'e} series $F_{- n}^{\Gamma}(z, s)$, defined for any Fuchsian group $\Gamma$ of the first kind with oen cusp by
\begin{equation*}
  F_m^\Gamma(z, s)
  = \sum_{\gamma_\infty \backslash \Gamma}
  e \left( m \Re(\gamma z) \right) \left( \Im(\gamma z) \right)^{1/2} I_{s - 1/2} \left( 2 \pi \lvert m \rvert \Im(\gamma z) \right),
\end{equation*}
for $m \neq 0$.

It is an eigenfucntion of the hyperbolic Laplacian, and may be expressed in terms of $j_m$.

The term $\log \left( \lVert f(z) \rVert \right)$ is a bit complicated, involving findings by Jorgensen, von Pippich and the speaker, plus some additional work done in our paper.

\begin{proposition}
  Let $\Gamma$ be a cofinite Fuchsian group with one cusp at $\infty$ with identity as its scaling matrix.  Let $2 k \geq 0$ be an even integer, and let $f$ be aw eight $2 k$ meromorphic form which is $\Gamma$-invariant and with $q$-expansion at $\infty$ normalized so its constant term is equal to one.  Then, we can express $\log \left( \lVert f \rVert(z) \right)$ in terms of parabolic Eisenstein series and Green's functions, as
  \begin{equation*}
    - 2 k + 2 \pi \sum_{w \in \mathcal{F}_\Gamma}
    \frac{\ord_w(f)}{ \ord(w)} \lim_{s \rightarrow 1} \left( G_s^\Gamma(z, w) + \mathcal{E}_{\Gamma, \infty}^{\mathrm{par}(z, s)} \right)
    + \dotsb.
  \end{equation*}
\end{proposition}

Here
\begin{equation*}
  \mathcal{E}_\infty^{\mathrm{par}}(z, s) = \sum_{\Gamma_\infty \backslash \Gamma} \Im(\gamma z)^s.
\end{equation*}
The Kronecker limit formula says that
\begin{equation*}
  \mathcal{E}_\infty^{\mathrm{par}}(z, s) = \frac{1}{(s-1) \vol(\Gamma \backslash \mathbb{H})}
  + \beta - \frac{1}{\vol(\Gamma \backslash \mathbb{H})}
  \log \left( \lvert \eta_\infty^4(z) \rvert \Im(z) \right)
  + \O(s - 1).
\end{equation*}
Then we need to define $\beta$, and the Green's function $G_s(z, w)$, which is obtained by averaging the kernel $k_s(z, w)$; both functions are eigenfunctions of the Laplacian with eigenvalue $s(1 - s)$, and the Green's function has a specified singularity on the diagonal.  The Green's function also has a Laurent series expansion as $s \rightarrow 1$ that involves the parabolic Eisenstein series.  In particular,
\begin{equation*}
  \lim_{s \rightarrow 1} \left( G_s^\Gamma(z, w) + \mathcal{E}_{\Gamma, \infty}^{\mathrm{par}}(z, w) \right)
\end{equation*}
exists, with a logarithmic singularity on the diagonal.

The Rohrlich--Jensen formula can be understood through the study of the regularized inner product of this last limit with $F_{- n}(\cdot, 1)$.  Regularization is needed only in the cusp (because the logarithmic singularity is integrable).  We can thus write
\begin{align*}
  &\left\langle F_{- n}(\cdot, 1),
    \overline{ \lim_{s \rightarrow 1} \left( G_s^\Gamma(z, w) + \mathcal{E}_{\Gamma, \infty}^{\mathrm{par}}(z, w) \right)}
    \right\rangle \\
  &=
    \lim_{Y \rightarrow \infty}
    \int_{\mathcal{F}(Y)}
    F_{- n}(z, 1) \lim_{s \rightarrow 1} \left( G_s(z, w) + \mathcal{E}_\infty^{\mathrm{par}}(z, s) \right)
    \, d \mu(z).
\end{align*}
A key observation is that all terms are eigenfunctions of the Laplacian, hence one can seek to compute the inner product in a manner similar to that which yields the Maass--Selberg formula.  The key identity is that
\begin{align*}
  &\int_{\mathcal{F}(Y)}
  F_{- n}(z, 1) \lim_{s \rightarrow 1} \left( G_s(z, w) + \mathcal{E}_\infty^{\mathrm{par}}(z, s) \right)
    \, d \mu(z) \\
  &=
    \partial_s \left( - s(1 - s) \int_{\mathcal{F}(Y)}
    F_{- n}(z, 1) \left( G_s(z, w) + \mathcal{E}_\infty^{\mathrm{par}}(w , s) \, d \mu(z) \right)\right)|_{s=1}.
\end{align*}
We can now absorb $- s(1 - s)$ into the integrals after applying the hyperbolic Laplacian $\Delta$ to each of the two factors in the integrand.

Main results:

\begin{theorem}
  For any positive integer $n$ and any point $w \in \mathcal{F}$, we have
  \begin{equation*}
    \left\langle F_{- n}(\cdot, 1), \overline{\lim_{s \rightarrow 1} \left( G_s(\cdot, w) + \mathcal{E}_\infty^{\mathrm{par}}(\cdot, s) \right)} \right\rangle
    = - \partial_s F_{- n}(w, s) |_{s=1}.
  \end{equation*}
\end{theorem}
Combined with the previous proposition (describing $\log \lVert f(z) \rVert$), we get the Rohrlich--Jensen formula:
\begin{equation*}
  \left\langle F_{- n}(\cdot, 1), \log \lVert f \rVert \right\rangle
  = - 2 \pi \sum_{w \in \mathcal{F}}
  \frac{\ord_w(f)}{\ord(w)} \partial_s F_{- n}(w, s)|_{s=1}.
\end{equation*}

We have further results.  For instance, if $g$ is any $\Gamma$-invariant analytic function with a pole at $\infty$, then (Niebur)
\begin{equation*}
  g(z) = \sum_{n = 1}^K 2 \pi \sqrt{n} a_n F_{- n}(z, 1) + c(g)
\end{equation*}
for some constants $K$, $a_n$, and $c(g)$ depending only upon $g$.  Then, we have the identity
\begin{equation*}
  \left\langle g, \log \lvert f \rvert \right\rangle
  = - 2 \pi \sum_{w \in \mathcal{F}} \frac{\ord_w(f)}{\ord(w)}
  \left( 2 \pi \sum_{n = 1}^K \sqrt{n } a_n \partial_s F_{- n}(w, s) |_{s=1} - \dotsb \right).
\end{equation*}
Further found that the generating series for the Niebur Poincar{\'e} series at $s = 1$ is, in the $z$-variable (where we sum over $q_z$), the holomorphic part of the weight two biharmonic Maass form given by differentiating our linear combination of Eisenstein series and Green's function with respect to $z$.


\section{Shenghao Hua, \textnormal{\emph{Joint Value Distribution of Hecke--Maass Forms}}}
Preprint: \cite{2024arXiv2405.00996}

Motivation: semiclassical limit of solutions to Schroedinger equation.  Berry's random wave conjecture.  QUE, $L^p$-norms.

Joint Gaussian moments conjecture: statistical indepndence of values of large eigenfunctions (multiplied together and integrated against some test function).

\begin{theorem}
  Let $f$ and $g$ be normalized Hecke--MAass cusp forms.  Then if $2 t_f \leq t_g - t_g^\eps$, we have
  \begin{equation*}
    \int_{\Gamma \backslash \mathbb{H}}
    \psi f^2 g \ll t_g^{- A},
  \end{equation*}
  while if $2 t_f > t_g - t_g^\eps$, then, assuming GLH, we have
  \begin{equation*}
    \int_{\Gamma \backslash \mathbb{H}} \psi f^2 g
    = \int_{\Gamma \backslash \mathbb{H}} \psi g +
    \O(t_f^{- 1/4 + \eps}(1 + \lvert 2 t_f - t_g \rvert^{- 1/4})).
  \end{equation*}
\end{theorem}
Proof uses spectral expansion of $\psi$ into eigenfunctions $u$, then spectral expansion of $\int (u g)(f^2)$, reducing to the case $g=u$, then further spectral expansion of $\int(g^2)(f^2)$.

\begin{theorem}
  Assuming GRH and GRC, we have as $\max(t_f,t_g) \rightarrow \infty$
  \begin{equation*}
    \mathbb{E} f^2 g^2 = 1 + \O((\log(t_f + t_g))^{- 1/4 + \eps}).
  \end{equation*}
\end{theorem}
Proof eventually applies ultimately Soundararajan's method of moments (which requires GRH).

Work in progress: asymptotic formulas for certain averages of $L$-functions over toroidal families, going beyond \cite{2023arXiv2303.11664}.

\section{Algebraic proof of modular form inequalities for optimal sphere packings, \textnormal{\emph{Seewoo Lee}}}
Preprint: \cite{2024arXiv2406.14659}.

Motivating question: finding densest sphere packings.
\begin{itemize}
\item $d = 1$
\item $d = 2$: Thue 1910, $\Delta_2 = \frac{\pi}{2 \sqrt{3}}$
\item $d =3$: Kepler conjecture 1611, Hales 2005, $\Delta_3 = \frac{\pi}{3 \sqrt{2}}$, formally verified 2014 in Isabelle/HOL
\item (Korkine--Zolotareff, Blichfeldt, Cohn--Kumar): $D_4, D_5, D_6, E_7, E_8$ and Leech lattice
\item Viazovska, 2016 $\pi$-day on arXiv: $E_8$ lattice packing optimal, with $\Delta_8 = \frac{\pi^4}{388}$
\item Cohn--Kumar--Miller--Radchenko--Viazovska, March 21st 2016 on arXiv: $\Delta_{24} = \frac{\pi^{12}}{12!}$ Leech lattice (unique even unimodular with nonzero minimal length $2$)
\end{itemize}

Viazovska et al use:
\begin{theorem}[Cohn--Elkies, 2013]
  Let $r > 0$.  Assume there exists a nice function $f : \mathbb{R}^d \rightarrow \mathbb{R}$
  satisfying
  \begin{itemize}
  \item $f(0) = \hat{f}(0) > 0$,
  \item $f(x) \leq 0$ for all $\lVert x \rVert \geq r$,
  \item $\hat{f}(y) \geq 0$ for all $y \in \mathbb{R}^d$,
  \end{itemize}
  then
  \begin{equation*}
    \Delta_d \leq \vol(B_{r/2}^d) = \left( \frac{r}{2} \right)^d \frac{\pi^{d/2}}{(d / 2)!}.
  \end{equation*}
\end{theorem}
Proof is not hard (fits in one page).

This leads to the hunt for a ``magic function'' $f$ satisfying the indicated conditions.  BAsed on their numerical experiments, Cohn--Elkies conjectured that the optimal sphere packing can be achieved by a magic function when $d=2,8,24$.  The magic function $f$ and its Fourier transform $\hat{f}$ should vanish at the lattice points.

Viazovska et al constructed the magic functions for $d = 8,24$ using modular forms.  Decompose into Fourier eigenfunctions $f = f_+ + f_-$ (parity under Fourier transform).  Viazovska writes them as
\begin{equation*}
  f_{\pm}(x) = \sin^2 \left( \frac{\pi \lVert x \rVert^2}{2} \right)
  \int_0^\infty \varphi_{\pm}(t) e^{- \lVert x \rVert^2 t} \, d t,
\end{equation*}
where the $\sin^2$ factor enforces desired roots.

For $d = 8$, we have $\varphi_{\pm}(t) = t^2 \psi_{\pm}(i/t) $, where $\psi_{\pm}$ are defined in terms of standard modular forms.  We also need (for both $d = 8, 24$) a few non-obvious inequalities involving some of these modular forms.  One reason these inequalities are difficult is because they're inhomogeneous with respect to the weights of the forms involved.

Original proofs use bounds of Fourier coefficients of the form
\begin{equation*}
  \lvert c(n) \rvert \leq C_1 e^{C_2 \pi \sqrt{n}},
\end{equation*}
following from the Hardy--Ramanujan formula, reducing the question to finite calculations and interval arithmetic.

Ran Romik (2023) gave an alternative and much simpler proof for $d = 8$ that doe snot use any interval arithmetic, but still requires a ``calculator'' to check inequalities like
\begin{equation*}
  e^{3 \pi} \frac{9 \Gamma(1/4)^{16}}{8192 \pi^{12}} < 20480.
\end{equation*}

\begin{question}
  Can we prove such inequalities \emph{algebraically}?
\end{question}
The answer is yes, as we now explain.  To that end, we'll develop a theory of \emph{(completely) positive quasimodular forms}.

\begin{definition}
  Let $\Gamma \subseteq \SL_2(\mathbb{Z})$.  We say that a quasimodular form $F$
  is \emph{positive} if
  \begin{equation*}
    F(i t) \geq 0
  \end{equation*}
  for all $t > 0$.

  We call it \emph{completely positive} if it has nonnegative $q$-coefficients at $\infty$.
\end{definition}

Note that ``completely positive'' implies ``positive'', but the inclusion is strict in general.  For instance, the discriminant form has a product formula that tells you that it is positive, but you can check that it's not completely positive.

Easy facts:
\begin{theorem}
  \begin{itemize}
  \item Anti-derivative preserves positivity.
  \item Derivative preserves complete positivity.
  \item Serre derivative preserves complete positivity.
  \end{itemize}
\end{theorem}

``Nontrivial'' fact (that we won't use, and which also follows directly from Bernstein's theorem):
\begin{theorem}
  $F$ is completely positive if and only if all its derivatives are positive.
\end{theorem}

``Interesting'' fact that we will use:
\begin{theorem}
  Let $F = \sum_{n \geq n_0} a_n q^n$ with $a_{n _0 > 0}$.  If some Serre derivative of $F$ is positive, then so is $F$.
\end{theorem}

Examples?
\begin{definition}[Kaneko--Koike]
  For given weight $w$ and depth $s$, we can speak of \emph{extremal quasimodular forms of weight} $w$ \emph{and depth} $s$, which are those whose order of zeros at $\infty$ is as large as possible.
\end{definition}
It's been shown that these have almost all coefficients positive, and conjectured that they're all positive.

\begin{theorem}
  That conjecture is true for depth $1$ extremal forms.
\end{theorem}

We also have similar identities proving complete positively  of the depth $2$ extremal forms of weight $w \leq 14$, but not yet for general $w$.

This reduces some of the inequalities in the arguments of Viazovska et al to checking some limiting properties of ratios of modular functions.  For monotonicity, we check that derivatives are positive,  and for this it suffices to check the same for Serre derivatives of derivatives.

Possible future directions:
\begin{itemize}
\item Classify the completely positive forms?
\item Possible applications in other LP problems?
  \begin{itemize}
  \item Dual LP, uncertainty principle
  \item Any results that are ``uniform'' in dimensions
  \end{itemize}
\item Make a formalization of the proof (e.g., in Lean) easier?
\end{itemize}

\section{Ring of modular forms on certain unitary Shimura variety, \textnormal{\emph{Yuxin Lin }}}

Recall the definition of modular forms of weight $k$ and level $1$, as certain holomorphic functions $f : \mathbb{H} \rightarrow \mathbb{C}$.

More generally, let $\operatorname{Sh}(G)$ be a Shimura variety (of level $1$), where $G$ is a reductive group over $\mathbb{Q}$.  We can understand it as the moduli space of abelian varieties with additional structure.  Let $\pi : \mathcal{A} \rightarrow \operatorname{Sh}(G)$ be the universal abelian scheme of dimension $g$.

Let $\underline{\omega} := \pi_\ast \Omega_{\mathcal{A} / \mathrm{Sh}}$ be the Hodge bundle on $\operatorname{Sh}(G)$, and let $\lambda_{\mathrm{Sh}} := \Lambda^g \underline{\omega}$ be the Hodge line bundle on $\operatorname{Sh}(G)$.

\begin{definition}
  The ring of automorphic forms of scalar-valued weight is the space of global sections of the $k$-th tensor power of the Hodge line bundle, namely
  \begin{equation*}
    \oplus_k H^0(\mathrm{Sh}, \lambda^{\otimes k}).
  \end{equation*}
\end{definition}, 8, 12, 2, 16
An interesting question to ask is, what is the ring structure of such objects?  There are some relevant results along these lines.

First, when $G = \SL_2(\mathbb{Z})$, we have that $\operatorname{Sh}(G)$ is the modular curve, and the ring of modular forms is generated by $E_4$ and $E_6$ modulo a single relation:
\begin{equation*}
  \mathcal{M}(\SL_2(\mathbb{Z})) = \mathbb{C}[E_4, E_6] /(E_4^3 - E_6^2).
\end{equation*}



Going one dimension higher, for $G = \SL_2(\mathcal{O}_K)$ and $K = \mathbb{Q}(\sqrt{13})$, we have that $\operatorname{Sh}(G)$ is the Hilbert modular surface with real multiplication by $\mathcal{O}_K$.  Van Der Geer and Zagier show that the ring of symmetric Hilbert modular forms of even weight has $4$ generators at weights $4, 8, 12, 12$ respectively, with a relation at weight $24$, while the ring of Hilbert modular forms of even weight has $5$ generators of weights $4, 8, 12, 12, 16$, respectively, with two relations at weight $24$ and $32$.  To that end, they find the minimal model $Y$ of the compactification of $\operatorname{Sh}(G)$, then find the canonical divisor of $Y$ and compute the image $S$ of $Y$ under the canonical embedding, then realize sections of $H^0(\mathrm{Sh}, \lambda^{\otimes 2 k})$ as meromorphic sections of $\mathcal{O}_S(k)$ with certain multiplicity at the cusps.

This is the strategy we will imitate, but for certain unitary Shimura varieties rather than Hilbert modular surfaces.

To that end, we need to first introduce the moduli space of curves.

Let $G = \mathbb{Z} / d \mathbb{Z}$ be a finite cyclic group.
\begin{definition}
  \begin{itemize}
  \item $\mathcal{M}_G$: the moduli space of admissible stable $\mathbb{Z} / d \mathbb{Z}$ covers of $\mathbb{P}^1$.  Objects are pairs $(C / S, \iota : G \rightarrow \Aut_S(C))$, i.e., a family of curves over the base scheme together with an embedding of $G$ in their automorphism group.
  \item $\tilde{\mathcal{M}_G}$, the moduli space of admissible stable $\mathbb{Z} / d \mathbb{Z}$ covers of $\mathbb{P}^1$ with an ordering on the ramification points.  The objects are tuples $(C / S, \iota : G \rightarrow \Aut_S(C), \sigma_i)$.
  \item The forgetful morphism $\tilde{\mathcal{M}_G} \rightarrow \mathcal{M}_G$ is finite {\'e}tale.
  \end{itemize}
\end{definition}
The irreducible connected components of $\mathcal{M}_G$ are indexed by monodromy datum, which is a triple $(d, r, \underline{a})$, where
\begin{itemize}
\item $r$ is the number of branching points on $C / \iota(G)$,
\item $\underline{a} =(a(1), \dotsc, a(r)) \in G^r$ is the \emph{inertia type}, which records the character of $G$ at the tangent space of the $i$-th ramification point.
\end{itemize}

The \emph{Torelli morphism} $T$ associates the moduli of curves with the moduli of principally polarized abelian varieties:
\begin{equation*}
  \begin{CD}         
    \mathcal{M}(d, r, \underline{a})_{\mathbb{C}}  @>>> \operatorname{Sh}(\mathcal{D})_{\mathbb{C}}\\
    @VVV  @VVV \\
    \mathcal{M}_{g, \mathbb{C}} @>>T> \mathcal{A}_{g, \mathbb{C}}.\\
  \end{CD}
\end{equation*}
The image of the Torelli morphism is called the \emph{Torelli locus} inside $\mathcal{A}_{g, \mathbb{C}}$.  It turns out that the Jacobians coming out of $\mathcal{M}(d, r, \underline{a})$ have larger endomorphisms than generically, and in particular, at least contain the group algebra generated by $G$.  It's thus reasonable to speculate that the image of $T$ on $\mathcal{M}(d, r, a)$ will sit inside a \emph{special} subvariety of $\mathcal{A}_g$, and this special subvariety will be the focus of our study.

More precisely, $\operatorname{Sh}(\mathcal{D})$ is the smallest PEL type Shimura subvariety of $\mathcal{A}_{g, \mathbb{C}}$ containing $T(\mathcal{M}(d, r, \underline{a})_{\mathbb{C}})$.  It parametrizes abelian varieties with multiplication by $\mathbb{Z}[G]$ and signature given by $(d, r, \underline{a})$.

We focus on the family $\mathcal{M}[16] = \mathcal{M}(5, 5,(1, 1, 1, 1, 1))$.
\begin{itemize}
\item The associated Shimura variety $\operatorname{Sh}(G)$ is of \emph{unitary type}, whose reductive group $\mathcal{G}$ has $\mathbb{R}$ points given by $\GU(3, 0) \times \GU(2, 1)$.   Therefore, it is \emph{compact}.  The reason is that the signature of this Shimura variety can be computed, and the signature has a zero in one of the components, which means that the hermitian symmetric domain is compact, and hence this $\operatorname{Sh}(\mathcal{D})$ is compact.  In particular, this Shimura variety has no cusp.
\item A second reason why this family is so nice is that $\operatorname{Sh}(\mathcal{D})_{\mathbb{C}}$ has dimension $2$, which is equal to the dimension of $\mathcal{M}[16]$.  Therefore, the Torelli morphism $T$ gives an \emph{isomorphism}
  \begin{equation*}
    T : \mathcal{M}[16] \rightarrow \operatorname{Sh}(\mathcal{D}),
  \end{equation*}
  such that $T$ gives an isomorphism between the corresponding Hodge line bundles.  Thus, in order to compute global sections of the Hodge line bundle on $\operatorname{Sh}(\mathcal{D})$, it suffices to do the analogous thing on $\mathcal{M}[16]$, which is a bit simpler, but still tricky, due to nontrivial stabilizers.  To that end, we consider:
\item The finite {\'e}tale cover $\tilde{\mathcal{M}[16]}$, which has a natural forgetful morphism
  \begin{equation*}
    q : \tilde{\mathcal{M}[16]} \rightarrow \overline{M_{0, 5}},
  \end{equation*}
  that is bijective on field-valued points.  This is very nice, because we know the geometry of $\overline{M_{0, 5}}$: its Picard group is generated by ten divisors, each isomorphic to $\mathbb{P}^1$.
\end{itemize}

\begin{proposition}
  Let $\Delta$ be the boundary divisor of $\overline{M_{0, 5}}$, and let $\lambda_{\tilde{M}}$ be the Hodge divisor on $\tilde{\mathcal{M}[16]}$.  Under the forgetful morphism
  \begin{equation*}
    q : \tilde{\mathcal{M}[16]} \rightarrow \overline{M_{0, 5}},
  \end{equation*}
  we have that $q^\ast \Delta = 5 \lambda_{\tilde{M}}$.
\end{proposition}
\begin{proof}[Idea of proof]
  \begin{itemize}
  \item Use Grothendieck--Riemann--Roch to relate $\lambda_{\tilde{M}}$ with the \emph{boundary divisor} of $\tilde{\mathcal{M}}$.
  \item $q$ exhibits $\Delta$ as a multiple of the boundary of $\tilde{\mathcal{M}}[16]$.
  \end{itemize}
\end{proof}

Our next goal is to fill in the lifts
\begin{equation*}
  \begin{CD}         
    \tilde{\mathcal{M}[16]}  @>?>> \operatorname{Sh}_{\mathcal{K}}(\mathcal{D})\\
    @VVV  @VV?V \\
    \mathcal{M}[16] @>>T> \operatorname{Sh}(\mathcal{D}).\\
  \end{CD}
\end{equation*}
Giving such a cover of $T$ is the same as computing the level subgroup corresponding to this cover.  That level subgroup is basically a congruence subgroup:

\begin{proposition}
  The level subgroup $\mathcal{K}$ at the place $p$ satisfies:
  \begin{equation*}
    \mathcal{K}_p =
    \begin{cases}
      \mathcal{G}(\mathbb{Z}_p)      & \text{ if } p \neq 5, \\
      \mathcal{U}(\mathfrak{m})                                     & \text{ if } p = 5.
    \end{cases}
  \end{equation*}
\end{proposition}

Putting these parts together via the sequence of morphisms
\begin{equation*}
  \operatorname{Sh}_{\mathcal{K}}(\mathcal{D}) \xleftarrow{T} \tilde{\mathcal{M}[16]}
  \xrightarrow{q} \overline{M_{0, 5}}
  \xrightarrow{\lvert - K \rvert} S,
\end{equation*}
we obtain the following:
\begin{proposition}
  \begin{itemize}
  \item The weight $m$ automorphic forms on $\operatorname{Sh}_{\mathcal{K}}(\mathcal{D})$ is
    \begin{equation*}
      H^0(\operatorname{Sh}_{\mathcal{K}(\mathcal{D}), \lambda_{\mathrm{Sh}^{\otimes m}}})
      \cong H^0(S, \mathcal{O}_S(2 \lfloor \frac{m}{5} \rfloor))
      \cong \left( \mathbb{C}[x_0, \dotsc, x_5] \right) / (Q_{12} - Q_{i j}),
    \end{equation*}
    with explicit degrees.
  \item Similar for the ring of graded automorphic forms.
  \end{itemize}
\end{proposition}

From the moduli interpretation, we see that $\operatorname{Sh}(\mathcal{D})$ is the $\mathfrak{S}_5$-invariant of $\operatorname{Sh}_{\mathcal{K}}(\mathcal{D})$.   Therefore:
\begin{theorem}
  The ring of automorphic forms on $\operatorname{Sh}(\mathcal{D})$ is the $\mathfrak{S}_5$-invariant of $\operatorname{Sh}_{\mathcal{K}}(\mathcal{D})$.  In particular,
  \begin{equation*}
    \oplus_{m} H^0(\operatorname{Sh}(\mathcal{D}), \lambda^{\otimes m}) \cong \mathbb{C}[f_2, f_4, f_6, \eps] / \eps^{10}
    + \frac{1}{52}(f_2^2 - 4 f_4),
  \end{equation*}
  where the $f_{2n}$ are polynomials of degree $2 n$ in $x_i$, where everything has explicit degrees.
\end{theorem}

\section{Counting modular forms mod $p$ satisfying constraints at $p$, \textnormal{\emph{Samuele Anni}}}
Recall the definition of modular forms $f$ of weight $k$ on $\Gamma_0(n)$, with $q$-expansion $f(z) = \sum a_n q^n$.  We also know that there are Hecke operators $T_p$ acting on them.  We consider only cuspidal \emph{newforms} ($a_0 = 0$), \emph{normalized} ($a_1 = 1$), which are eigenforms for the Hecke operators and arise from level $n$ and not any smaller level.  We use the notation $S_k(n, \mathbb{C})$ and $S_k(n, \mathbb{C})^{\new}$.

\begin{definition}
  The \emph{Hecke algebra} $\mathbb{T}(n, k)$ is the $\mathbb{Z}$-subalgebra of $\End_{\mathbb{C}}(S_k(n, \mathbb{C}))$ generated by Hecke operators $T_p$ for each prime $p$.
\end{definition}

Newforms can be seen as ring homomorphisms $f : \mathbb{T}(n, k) \rightarrow \overline{\mathbb{Q}}$.

\begin{theorem}[Deligne, Serre, Shimura]
  Let $n$ and $k$ be positive integers.  Let $\mathbb{F}$ be a finite field of characteristic $\ell$, with $\ell \nmid n$, and
  let $f : \mathbb{T}(n, k) \twoheadrightarrow \mathbb{F}$ be a surjective ring homomorphism.  Then there is a unique continuous semisimple representation
  \begin{equation*}
    \bar{\rho} _f : \Gal(\bar{\mathbb{Q}} / \mathbb{Q}) \rightarrow \GL_2(\mathbb{F}),
  \end{equation*}
  unramified outside $n \ell$, such that for all $p$ not dividing $n \ell$, the trace of Frobenius at $p$ under $\bar{\rho}_f$ is $f(T_p)$, and the determinant is given in terms of the value of $f$ on the diamond operator (central character).
\end{theorem}

Computing $\bar{\rho}_f$ is ``difficult'', but theoretically it \emph{can be done in polynomial time} in $n, k, \# \mathbb{F}$:
\begin{itemize}
\item Edixhoven, Couveignes, de Jong, Merkl, Bruin, Bosman: $\# \mathbb{F} \leq 32$.
\item Mascot, Zeng, Tian: $\# \mathbb{F} \leq 53$.
\end{itemize}

Fix a prime $p \geq 5$, a level $N$ prime to $p$, and a weight $k \geq 2$.  Let $S_k := S_k(\Gamma_0(N p), \bar{\mathbb{Q}}_p)$ be the space of $p$-new (i.e., not coming from level $N$) cuspidal modular forms of level $N p$ and weight $k$ with coefficients in $\bar{\mathbb{Q}}_p$.

We now discuss the \textbf{Atkin--Lehner involution}.  There exist $x, y, z, t \in \mathbb{Z}$ for which the matrix
\begin{equation*}
  W_p =
  \begin{pmatrix}
    p x    & y \\
    N p z &  p t \\
  \end{pmatrix}
\end{equation*}
has determinant $p$.

The matrix $W_p$ normalizes the group $\Gamma_0(N p)$, and for any weight $k$, it induces a linear operator $w_p$ on the space of cusp forms $S_k$ that commutes with the Hecke operators $T_q$ for all $q \nmid N p$ and acts as its own inverse.

Any cusp form in $S_k$ that is an eigenform for all $T_q$ with $q \nmid N$ is also an eigenform for $w_p$, withe eigenvalue $\pm 1$.  This involution acts on the modular curve.

The Atkin--Lehner involution $w_p$ splits $S_k$ as a sum of plus and minus spaces:
\begin{equation*}
  S_k = S_k^+ \oplus S_k^-.
\end{equation*}
Since we have dimension formulas for $s_k := \dim S_k$, in order to understand the dimensions
\begin{equation*}
  s_k^{\pm} = \dim S_k^{\pm}
\end{equation*}
of the Atkin--Lehner eigenspaces, it suffices to understand the difference
\begin{equation*}
  d_k := s_k^+ - s_k^-.
\end{equation*}

We can get some idea from tables of examples (obtained empirically by looking at many more specific examples):
\begin{table}[h]
  \centering
  \begin{tabular}{|c|c|}
    \hline
    $p$ & $d_k$ \\
    \hline
    5   & $\pm 1$ \\
    11  & $\pm 2$ \\
    59  & $\pm 6$ \\
    101 & $\pm 7$ \\
    \hline
  \end{tabular}
  \caption{Values of $d_k$ for different $p$}
  \label{tab:dk_values}
\end{table}

Classical result that $d_k$ is constant in absolute value and alternates in sign.  Need to introduce modification $d_k^\ast$ taking into account Eisenstein series of weight two.  One can show:
\begin{theorem}[Fricke, Yamauchi, Momose, Ogg, Wakatsuki, Helfgott, Martin et al.]
  We have
  \begin{equation*}
    d_k^\ast =(- 1)^{k/2} \frac{\# \mathrm{FP}}{2},
  \end{equation*}
  where $\# \mathrm{FP}$ is the number of fixed points of the Atkin--Lehner involution $w_p$ on $X_0(N p)$.
\end{theorem}
The fixed points of $w_p$ on $X_0(N p)$ corresponds to elliptic curves with level structure and CM by $\sqrt{- p}$, in fact the $d_k$ are closely related to class numbers:
\begin{equation*}
  \# \mathrm{FP}
  = c_p \cdot h(\sqrt{-p}) \cdot \prod_{q \mid N, \text{prime}}
  \left( 1 + \qr{- 4 p}{q} \right).
\end{equation*}

What do we want to do?  Systems of mod-$p$ prime-to-$N p$ Hecke eigenvalues correspond to continuous semisimple Galois representations $\Gal(\bar{\mathbb{Q}} / \mathbb{Q})  \rightarrow \GL_2(\bar{\mathbb{F}}_p)$.  Can decompose according to these, and then further via Atkin--Lehner, giving spaces $S_{k, \bar{\rho}}^{\pm}$ and their dimensions $s_{k, \bar{\rho}}^{\pm}$ and differences $d_{k, \bar{\rho}}$.  As before, $k = 2$ and $\bar{\rho}$ forces us to make an adjustment, so let
\begin{equation*}
  d_{k, \bar{\rho}}^\ast :=
  \begin{cases}
    d_{k, \bar{\rho}} - 1    &  \text{ if } k = 2 \text{ and } \bar{\rho} = 1 \oplus \omega, \\
    d_{k, \bar{\rho}}                             & \text{ otherwise,}
  \end{cases}
\end{equation*}
where $\omega$ is the cyclotomic character.

\begin{theorem}[Anni, Ghitza, Medvedovsky]
  For $k \geq 2$ and any $\Gamma_0(N p)$-modular $\bar{\rho}$, we have
  \begin{equation*}
    d^\ast_{k + 2, \bar{\rho}[1]} = - d_{k, \bar{\rho}}^\ast.
  \end{equation*}
\end{theorem}

As an example, for $p = 5$, $N= 23$, we can check that for $k > 2$, we have $d_k = \pm 2$.

Bergdall and Pollack use the Ash--Stevens formula, a fundamentally characteristic $p$ technique for filtering cohomology of modular symbols, to derive their dimension formulas.  But Ash--Stevens has nothing to say about Atkin--Lehner, in part because Atkin--Lehner operator requires inverting $p$.  On the other hand, the classical complex methods -- trace formulas, Gauss--Bonnet, Riemann--Hurwitz -- do not know anything about $\bar{\rho}$.

What we do instead is to combine the \emph{trace formula} (Zagier--Cohen--Osterl{\'e}--Cohen--Str{\"o}mberg and Skoruppa--Zagier--Popa) with an \emph{algebra theorem}, an explicit refinement of Brauer--Nesbitt.

What is this explicit Brauer--Nesbitt?  See \cite{2022arXiv2207.07108}.
\begin{theorem}[AGM]
  Let $M$ and $N$ be two finite free $\mathbb{Z}_p$-modules of the sam erank $d$, each with an action of an operator $T$.  Then $\bar{M}^{\mathrm{ss}} \cong \bar{N}^{\mathrm{ss}}$ as $\mathbb{F}_p[T]$-modules if and only if for every $n$ with $1 \leq n \leq d$, we have
  \begin{equation*}
    \trace(T^n | M) = \trace(T^n | N),
  \end{equation*}
  (...).
\end{theorem}
\begin{corollary}
  If we have modules that injective modulo $p$ (but not necessarily in characteristic zero), then we can check that semisimplifications of certain residual quotients are the same by proving an inequality involving traces.
\end{corollary}

Now work with the trace formula.  We can deduce statements about dimensions.

\section{Integer partitions detect the primes, \textnormal{\emph{Jan-Willem Van Ittersum}}}
Preprint: \cite{2024arXiv2405.06451}.

We write $s_i$ for the sizes of the parts, and $m_i$ for the multiplicities of the part, with $s_1 > s_2 > \dotsb$.
\begin{definition}[MacMahon 1920]
  \begin{equation*}
    M_a(n) := \sum_{
      \substack{
        n = m_1 s_1 + \dotsb + m_a s_a  \\
        0 < s_1 < \dotsb < s_a        
      }
    } m_1 \dotsb m_a.
  \end{equation*}
\end{definition}
\begin{remark}
  This is the sum of multiplicity products of partitions of $n$ with $a$ part sizes.
\end{remark}
Consider
\begin{equation*}
  \psi(n) :=(n^2 - 3 n + 2) M_1(n) - 8 M_2(n).
\end{equation*}
\begin{example}
  $n = 3, a = 1$, in which case $\lambda =(3)$ or $\lambda =(1^3)$, in which case $M_1(3) = 1 + 3 = 4$,
  or $a = 2$, in which case $\lambda =(2, 1)$, so $M_2(3) = 1 \cdot 1 = 1$, and we compute that $\psi(3) = 2 M_1(3) - 8 M_2(3) = 0$.

  Continuing, take $n = 4, a = 1$, in which case $\lambda = (4)$, $\lambda = (2^2)$ $\lambda =(1^4)$, so $M_1(4) = 1 + 2 + 4 = 7$,
  or $a = 2$, in which case $\lambda =(3, 1)$ or $\lambda =(2, 1^2)$, so $M_2(4) = 1 \cdot 1 + 1 \cdot 2 = 3$, and so $\psi(4) = 6 M_1(4) - 8 M_2(4) = 18$.

  Continuing, one gets the table, for $n = 2..11$, of $\psi(n) = 0, 0, 18, 0, 120, 0, 270, 192, 504, 0$.
\end{example}

\begin{theorem}[Craig--vI--Ono] For positive integers $n$, we have
  \begin{enumerate}
  \item $(n^2 - 3 n + 2) M_1(n) - 8 M_2(n) \geq 0$,
  \item $(3 n^3 - 13 n^2 + 18 n - 8) M_1(n) +(12 n^2 - 120 n + 212) M_2(n) - 960 M_3(n) \geq 0$
  \end{enumerate}
  and for $n \geq 2$ these expressions vanish if and only if $n$ is prime.
\end{theorem}
\begin{remark}
  We call such an expression \emph{prime-detecting}.  The sum of two such expressions is likewise prime-detecting, and multiplying $f(n)$ for a polynomial $f$ yields a prime-detecting expression if $f(n) > 0$ for all $n$.
\end{remark}

We can get a table of prime-detecting expressions: the first two that appear above, then two more, $(126 n^5 - \dotsb) M_1(n) + \dotsb$ and $(300 n^8 - \dotsb) M_1(n) + \dotsb$.

\begin{conjecture}
  These are all such expressions (up to addition and multiplication as before).
\end{conjecture}

We then checked whether we could get more results of a similar shape, by considering a generalization of the MacMahon function:
\begin{definition}
  For $\ell \in \mathbb{Z}_{\geq 0}^a$, we define the \emph{generalized MacMahon partition function}
  \begin{equation*}
    M_{\underline{\ell}}(n) := \sum_{
      \substack{
        n = m_1 s_1 + \dotsb + m_a s_a  \\
        0 < s_1 < \dotsb < s_a        
      }
    }
    m_1^{\ell_1} \dotsb m_a^{\ell_a}.
  \end{equation*}
\end{definition}
You can think of this as a generalization of the divisor function, but for partitions.  For this new generalized partition function, much more is possible:
\begin{theorem}[Craig--vI--Ono] Let $d \geq 4$.
  \begin{enumerate}
  \item There exist $clu \underline{\ell} \in \mathbb{Z}$ such that $\sum_{\lvert \ell \rvert < d} c_{\underline{\ell}} M_{\underline{\ell}}(n) \geq 0$, where $\lvert \underline{\ell} \rvert = \ell_1 + \dotsb + \ell_a$.
  \item There are $\gg d^2$ linearly independent such expressions.
  \end{enumerate}
\end{theorem}
\begin{example}
  $63 M_{(2, 2)}(n) - 12 M_{(3,0)}(n) - \dotsb + 12 M_{(3,0,0)}(n) = \frac{11}{3} \psi(n)$.
\end{example}<++>

Goal of the talk now is to explain some of the ideas behind these results, which have a lot to do with modular forms.

\textbf{Prime-detecting quasimodular forms}.  Let
\begin{equation*}
  \mathcal{G}_k := - \frac{B_k}{2 k} + \sum_{n \geq  1} \sigma_{k - 1}(n) q^n,
  \qquad
  D := q \frac{\partial}{\partial q},
  \qquad q = e^{2 \pi i \tau}.  
\end{equation*}
Consider
\begin{align*}
  f_{k, \ell} &:=(D^{\ell} + 1) \mathcal{G}_{k + 1} -(D^k + 1) \mathcal{G}_{\ell + 1} \\
              &= \ast + \sum_{n \geq 1} \sum_{d \mid n} \left((n^{\ell} + 1) d^k -(n^k + 1) d^{\ell} \right) q^n.
\end{align*}
Note that for $d = 1$ one gets $n^{\ell} + 1 - n^k - 1 = n^{\ell} - n^k$.  Calculating similarly with $d = n$, we see that $f_{k, \ell}$ vanishes at prime coefficients.  It's also easy to see that its coefficients are positive.  But note that $f_{k, \ell}$ is not modular, and not of homogeneous weight.  Key features here are quasimodularity and mixed weight.

\begin{theorem}[Craig--vI--Ono] All prime-detecting forms in $\oplus_{k: \text{even}} \oplus_{n \geq 0} D^n \mathcal{G}_k$ are linear combinations of $D^n H_{k}$, where
  \begin{equation*}
    H_k =
    \begin{cases}
      \frac{1}{6}(D^2 - D + 1) \mathcal{G}_2 - \mathcal{G}_4      &  \text{ if } k = 6, \\
      \frac{1}{24} \left( - D^2 \mathcal{G}_{k - 6} +(D^2 + 1) \mathcal{G}_{k - 4} - \mathcal{G}_{k - 2} \right)                                                                  & \text{ if } k = 8.
    \end{cases}
  \end{equation*}
\end{theorem}

That's the first ingredient, but still need to say something about modularity of MacMahon functions.  Let
\begin{equation*}
  \mathcal{G}(\underline{\ell}) = \sum_{n \geq 0} M_{\underline{\ell}}(n) q^n \xrightarrow{q \rightarrow 1}
  \frac{(1 - q)^{\lvert \underline{\ell}  \rvert + a}}{\prod_i \ell_i !}
  \sum_{s_1 < \dotsb < s_a}
  \frac{1}{ s_1^{\ell_1 + 1} \dotsb s_a^{\ell_a + 1}}
  \quad
  (\ell_a \geq 1, \, \ell_i \geq 0).
\end{equation*}
The algebra $\mathcal{Z}_q := \left\langle \mathcal{G}(\underline{\ell}) \right\rangle_{\mathbb{Q}}$ was introduced by Bachmann--K{\"u}hn.  Facts:
\begin{itemize}
\item $\mathcal{Z}_q$ is a differential algebra (closed under differentiation and multiplication)
\item $\tilde{M} := \mathbb{Q}[\mathcal{G}_2 , \mathcal{G}_4, \mathcal{G}_6] \subseteq \mathcal{Z}_q$ (i.e., it contains the space of quasimodular forms).
\item (Hoffman--Ihara) We have
  \begin{equation*}
    \sum_{j \geq 0} \mathcal{G}(\underbrace{1, 1, \dotsc, 1}_{j}) x^j = \exp
    \left(
      \sum_{n \geq1}
      \frac{(- 1)^{n + 1}}{ n}
      x^n \underbrace
      {
        \sum_m \frac{q^m}{(1 - q^m)^{2n}}
      }_{
        \sum_{m, s \geq 1} \binom{s + n - 1}{ s - n} q^{m s} \in \tilde{M}
      }
    \right),
  \end{equation*}
  which is a linear combination of quasimodular Eisenstein series, hence $M_a(n)$ are coefficients of a quasimodular form.
\end{itemize}

\section{Supersingular abelian surfaces and where to find them, \textnormal{\emph{Gabriele Bogo}}}
Characteristic zero: $\End(E) \cong \mathbb{Z}$ or $\mathcal{O}_D$, order in an imaginary quadratic field.

Characteristic $p > 0$ (Hasse, 1936), can also be $\mathcal{O}_{\infty, p}$, order in a quaternion algebra over $\mathbb{Q}$.  In that case, we call it \emph{supersingular}.

Deuring 1941: for $p \geq 5$, the number of supersingular elliptic curves over $\mathbb{F}_{p^2}$ is
\begin{equation*}
  \lfloor \frac{p - 1}{12} \rfloor + \delta + \eps.
\end{equation*}
Another interpretation: writing $\mathcal{M}_{1, 1} \cong \mathbb{H} / \SL_2(\mathbb{Z})$ for the moduli space of elliptic curves over $\mathbb{C}$, there are three special points $\infty, i = \sqrt{- 1}, \rho =(- 1 + \sqrt{- 3}) / 2$, and we can think of Deuring's result as
\begin{equation*}
  \#(\mathrm{ss}_p)
  = \lfloor \frac{p - 1}{12} \rfloor + \delta + \eps
  \leq
  \lfloor
  \frac{- \chi(\mathbb{H} / \SL_2(\mathbb{Z})) \cdot(p - 1)}{2}\rfloor + 2.
\end{equation*}

Let's generalize to higher-dimensional varieties.  Let $\mathbb{Q}(\sqrt{D})$ be a real quadratic field, with ring of integers $\mathcal{O}_D$.  We consider a family $\mathcal{X} \rightarrow \mathcal{C}$ of (p.p.) abelian surfaces, over a curve $\mathcal{C}$, such that:
\begin{itemize}
\item the fibers $X_c$ are defined over a number field $k$, and
\item there exists an inclusion of rings $\mathcal{O}_D \hookrightarrow \End_k(X_c)$.
\end{itemize}
An abelian surface over a field of characteristic $p$ is \emph{supersingular} (resp.\ \emph{superspecial}) if it is isogenous (resp.\ isomorphic) to the product of two supersingular elliptic curves.

There are embeddings
\begin{equation*}
  \mathbb{H} / \Gamma = \mathcal{C} \hookrightarrow X_D(\mathbb{C}) = \mathbb{H}^2 / \SL_2(\mathcal{O}_D).
\end{equation*}
For example, if $\Gamma = \SL_2(\mathbb{Z})$, this is the inclusion of elliptic curves into the space of abelian surfaces with multiplication.  There are also modular curves $\Gamma_0(N)$.  But one can have more interesting examples, related to non-arithmetic curves.  To give an explicit example, consider the triangle group $\Gamma = \Delta(2, 5, \infty) \hookrightarrow X_5$, with explicit equation
\begin{equation*}
  y^2 =
  \begin{cases}
    x^5 - 5 x^3 + 5 x    
    - 2 t
    & \text{ if } t \neq \infty, \\
    x^5 - 1    & \text{ if } t = \infty.
  \end{cases}
\end{equation*}
We now take the reductions of the above inclusion modulo $p$, to a map
\begin{equation*}
  \overline{\mathcal{C}} \rightarrow X_D(\mathbb{F}). 
\end{equation*}
We want to study the fibers of this family that, after reduction modulo $p$, are supersingular.

To that end, consider a smooth algebraic curve in a Hilbert modular surface
\begin{equation*}
  \mathcal{C} \hookrightarrow X_D
\end{equation*}
with second Lyapunov exponent $\lambda_2 \in \mathbb{Q} \cap \left(0,1\right]$.  For example, for $\SL_2(\mathbb{Z})$, we have $\lambda_2 = 1$, and for modular curves, $\lambda_2 = 1$, while for
$\Delta(2, 5, \infty)$ we have $\lambda_2 = 1/3$, but for Teichmuller curves, you can have $1/3, 1/5, 1/7$.

\begin{theorem}[B--Li, 2024]
  The supersingular locus of $\mathcal{C}$ modulo $p$ has cardinality described by
  \begin{align*}
    \lfloor \frac{- \chi(\mathcal{C})(p - \lambda_2)}{2} \rfloor \leq
    \#\mathrm{ss}_p^{\mathcal{C}}
    \leq \lfloor
    \frac{- \chi(\mathcal{C})(p - 1)(\lambda_2 + 1)}{2}\rfloor + r, \quad \text{ if } \qr{D}{p} = -1,
  \end{align*}
  while
  \begin{equation*}
    \# \mathrm{ss}_p^{\mathcal{C}}
    \leq \lfloor \frac{- \chi(\mathcal{C})(p - 1) \lambda_2}{2} \rfloor + r,
    \quad \text{ if } \qr{D}{p} = 1.
  \end{equation*}
\end{theorem}

How to find supersingular abelian surfaces?  If $\mathcal{C} = \mathbb{H} / \Gamma$  is of genus zero with Hauptmodul $j$, then the supersingular locus can be described by a polynomial in $j$:
\begin{equation*}
  \mathrm{ss}_p^{\mathcal{C}}(j) :=
  \prod_{
    \substack{
      c \in \mathcal{C}  \\
      \overline{X_c} \text{ supersingular}      
    }
  }
  \left( j - j(c) \right).
\end{equation*}

\begin{theorem}[B.--Li, 2024]
  There are two families $\{A_{0, n}(j)\}_n$ and $\{A_{1, n}(j)\}_n$ of orthogonal polynomials (Atkin's polynomials) such that for every $p$ of good reduction,
  \begin{equation*}
    \mathrm{ss}_p^{\mathcal{C}}(j) \equiv
    \begin{cases}
      \lcm(A_{0, n_p}, A_{1, \tilde{n}_p})      & \text{ if } \qr{D}{p} = -1, \\
      \gcd(A_{0, n_p}, A_{1, \tilde{n}_p})                                                & \text{ if } \qr{D}{p} = 1
    \end{cases}
  \end{equation*}
  for explicit indices $n_p$ and $\tilde{n}_p$.
\end{theorem}

For example, consider $\Delta(2, 5, \infty)$.  Let $j$ be a Hauptmodul, with pole at $\infty$.  The scalar products are defined on $\mathbb{R}[j]$ by
\begin{equation*}
  \langle f, g \rangle_0 :=
  \int_{\pi / 5}^{\pi / 2}
  f(e^{i \theta})
  g(e^{i \theta}) \, d th,
\end{equation*}
\begin{equation*}
  \langle f, g \rangle_1 := \int_{2 \pi / 5}^{\pi / 2}
  f(e^{i \theta}) g(e^{i \theta}) \, d \theta.
\end{equation*}
The first polynomials $A_{0, n}(j)$ are: $A_{0, 0}(j) = 1$, $A_{0, 1}(j) = j - 9/20$, etc.  Taking $p = 13$, one finds that
\begin{equation*}
  \mathrm{ss}_{13}(j) \equiv A_{0, 3}(j) \equiv j(j + 4)(j - 1) \pmod{13}.
\end{equation*}

Next, real multiplication splits $H_{\mathrm{d R}}^{1}(X_c)$ into two eigenspaces.  The eigendifferentials induce two second order differnetial equations, called Picard--Fuchs differential equations.  TRuncation of the holomorphic solutions can be related to the supersingular locus of $\mathcal{C}$.
\begin{example}
  For $\mathcal{C} = \mathbb{H} / \Delta(2, 5, \infty)$,
  \begin{equation*}
    j^n \cdot {}_2 F_1 \left( \frac{7}{20}, \frac{3}{20}; 1 ; \frac{1}{j} \right)
    = U_{0, n}(j) + \O(j^{-1}),
  \end{equation*}
  etc.  Leads to a formula for $\mathrm{ss}_p^{\mathcal{C}}(j)$.
\end{example}

The partial Hasse invariants $h_1$ and $h_2$ are characteristic $p$ Hilbert modular forms of non-parallel weight.  They have the following properties:
\begin{itemize}
\item The divisor of $h_i$ is the component $D_i$ of the non-ordinary locus of $X_D(\mathbb{F})$.
\item Their $q$-expansion is constant, equal to $1$, at every cusp.
\end{itemize}
In particular, they do not lift to Hilbert modular forms in characteristic zeor.

\begin{theorem} [B--Li, 2024]
  Let $\mathcal{C} \cong \mathbb{H} / \Gamma \hookrightarrow X_D$ be a smooth algebraic curve with good reduction at $p$.  Then the partial Hasse invariants lift to (twisted) modular forms on $\Gamma$ in characteristic zero.
\end{theorem}

\section{Congruences for the number of 3- and 6-regular partitions and quadratic forms, \textnormal{\emph{Cristina Ballantine}}}

\begin{definition}
  A \emph{partition} of a positive number $n$ is a non-increasings equence of positive integers, named \emph{parts}, that add up to $n$.
\end{definition}
We denote by $p(n)$ the number of partitions of $n$.

\begin{example}
  $p(4) = 5$.
\end{example}

Ramanujan congruences:
\begin{equation*}
  p(5 n + 4) \equiv 0 \pmod{5},
\end{equation*}
\begin{equation*}
  p(7 n + 5) \equiv 0 \pmod{7},
\end{equation*}
\begin{equation*}
  p(11 n + 6) \equiv 0 \pmod{11}.
\end{equation*}
Ramanujan: it appears there are no equally simple properties for any moduli involving primes other than these.

Ahlgren and Ono (2001): there are such Ramanujan congruences modulo every prime $\ell \geq 5$ (actually mod $\ell^m$).

How about $\ell = 2, 3$?  Radu (2012, formerly Subbarao Conjecture): for $\ell \in \{2, 3\}$, there is no arithmetic progression $A n + B$ such that
$p(A n + B) \equiv 0 \pmod{\ell}$ for all $n$.

Conjecture (Parkin and Shank): approximately half the values of $p(n)$ are even.

Bellaiche and Nicolas 2016: the number of $n \leq x$ with $p(n)$ even is at least
\begin{equation*}
  0.069 \sqrt{x} \log \log x,
\end{equation*}
while the number that are odd is at least a constant multiple of $\sqrt{x}/ \log x$.

The reason we can do this using modular forms is that
\begin{equation*}
  \sum_{n = 0}^{\infty} p(n) q^n = \prod_{i = 1}^\infty \frac{1}{1 - q^i} = \frac{q^{1/24}}{\eta(q)}.
\end{equation*}

Let's write $Q(n)$ for the number of partitions with distinct parts.  Then
\begin{equation*}
  \sum_{n = 0} Q(n) q^n = \prod_{i = 1}^\infty(1 + q^i) \equiv \prod_{i = 1}^\infty(1 - q^i) \pmod{2}
  =
  \sum_{j \in \mathbb{Z}}(- 1)^j q^{j(3j - 1)/2},
\end{equation*}
so $Q(n)$ is even asymptotically 100\% of the time.

\begin{definition}
  An $r$\emph{-regular partition} is a partition in which no parts are divisible by $r$.
\end{definition}
\begin{notation}
  $b_r(n) := $ the number of $r$-regular partitions of $n$.
\end{notation}
$b_2(n)$ is even asymptotically 100\% of the time.

\begin{theorem}[Keith--Zanello 2022]
  If $p \equiv 13, 17,. 19, 23 \pmod{24}$ is prime, then
  \begin{equation*}
    b_3 \left( 2(p^2 n +  pk - 24^{-1}) \right) \equiv 0 \pmod{2}
  \end{equation*}
  for $1 \leq k \leq p - 1$, where the inverse is taken modulo $p$.
\end{theorem}
\begin{theorem}[Yao 2022]
  Let $p \geq 5$ for a prime.  Then we can construct more arithmetic progressions where this happens, different from the above.
  \begin{enumerate}
  \item If $b_3 \left( \frac{p^2 - 1}{12} \right) \equiv 1 \pmod{2}$, then we get a whole arithmetic progression.
  \item (...)
  \end{enumerate}
\end{theorem}

Experimentally, we found a bunch of arithmetic progressions on which $b_3$ was even.  For instance, for $\alpha \in \{1, 51, 76, 101\}$,
\begin{equation*}
  b_3 \left( 2(5^3 n + \alpha) \right) \equiv 0 \pmod{2}.
\end{equation*}
Conjectured that one can do this for every prime $p \geq 5$.

The bad news is that the congruences we found for $p = 5,7,11$ are special cases of Yao's theorem.  My coauthor did more experience and found other primes $p \in \{29, 59, \dotsc, 683\}$ where we have
\begin{equation*}
  b_3 \left( 2(p^2 n + p \alpha - 24^{-1}) \right) \equiv 0 \pmod{2},
\end{equation*}
where $0 \leq \alpha < p$, $\alpha \neq \lfloor p/24 \rfloor$, and $24^{-1}$ taken modulo $p$, that do not fit into Yao's theorem.

We proved the congruences algorithmically, using a result of Radu involving modular forms and Sturm's bound.

Can we describe these primes?  What is the general theorem?

Let $\mathcal{P}$ be the set of primes $p$ such that for some $j \in \{1, 4, 8\}$, the equation $x^2 + 24 \cdot 9 y^2 = j p$ has primitive solutions.


\section{Exact formulae for ranks of partitions, \textnormal{\emph{Qihang Sun}}}
Preprint: \cite{2024arXiv2406.06294}.

$p(n)$: partition function.  Ramanujan's congruence mod $5$ and $7$.

Dyson's rank: for each partition $\Lambda = \{\Lambda_1 \geq \Lambda_2 \geq \dotsb \geq \Lambda_k\}$, define $\rank(\Lambda) := \Lambda_1 - \kappa$ to be the largest part minus the number of parts.  Rank generating function $N(a, b, n)$.  Explains Ramanujan congruences, e.g.,
\begin{equation*}
  N(a, 5, 5 n + 4) = \frac{1}{5} p(5 n + 4),
\end{equation*}
\begin{equation*}
  N(a, 7, 7 n + 5) = \frac{1}{7} p(7 n + 5).
\end{equation*}
These conjectures were all proved by Swinnerton-Dyer in 1954.

We'll focus on the analytic side of those rank functions.  For $q = e^{2 \pi i z}$ and $w = \zeta_b^a$,
\begin{equation*}
  \mathcal{R}(w, q) = 1 + \sum_{n = 1}^\infty \sum_{m \in \mathbb{Z}} N(m, n)
  w^m q^n
  =
  1 + \sum_{n = 1}^\infty \frac{q^{n^2}}{(w q, q)_n(w^{-1} q, q)_n}
  =: 1 + \sum_{n = 1}^\infty A \left( \frac{a}{b}, n \right) q^n.
\end{equation*}

Examples:
\begin{itemize}
\item $R(1, q) = 1 + \sum p(n) q^n = q^{1/24} / \eta(z)$
\item $R(- 1, q) = f(q)$ (Ramanujan's $3$rd other mock theta function)
\item $R(\zeta_3, q) = R(\zeta_3^2, q) = \gamma(q)$ (another mock theta function)
\end{itemize}

Hardy--Ramanujan (1918):
\begin{equation*}
  p(n) \sim \frac{1}{4 n \sqrt{3}}
  \exp \left( \pi \sqrt{\frac{2 n}{3}} \right),
\end{equation*}
together with more precise result.  Rademacher (1937): exact formula.

$A(\tfrac{1}{2} , n) = N(0, 2, n) - N(1, 2, n)$.  Ramanujan claimed, Dragonette (1952) proved:
\begin{equation*}
  A(\tfrac{1}{2} , n) = \frac{(- 1)^{n - 1} e^{\pi \sqrt{\tilde{n}/6}}}{2 \sqrt{\tilde{n}}} + \dotsb.
\end{equation*}
Andrews (1966).

Conjectures by Andrews and Lewis, proved by Bringmann (2009).
\begin{equation*}
  A(\tfrac{1}{3} , n) = \frac{4 \sqrt{3}i}{(24 n - 1)^{1/2}}
  \sum_{3 \mid k}^{\lfloor \sqrt{n} \rfloor}
  \dotsb.
\end{equation*}
Should be an exact formula.

More generally, Bringmann gave asymptotics for $A(\tfrac{\ell}{u}, n)$.

Proofs use Maass Poincar{\'e} series.

\section{The rational torsion subgroups of the Drinfeld modular Jacobians for prime-power levels, \textnormal{\emph{Sheng-Yang Kevin Ho}}}
Preprint: \cite{2024arXiv2404.00738}.


$X_0(N)$: modular curve,
\begin{equation*}
  J_0(N)(\mathbb{Q}) \cong \mathbb{Z}^r \oplus \mathcal{T}(N).
\end{equation*}
Can we compute the rational torsion subgroup?

Let $\mathcal{C}(N)$ denote the rational cuspidal divisor class group for $X_0(N)$.  By a theorem of Manin and Drinfeld, we have
\begin{equation*}
  \mathcal{C}(N) \subseteq \mathcal{T}(N).
\end{equation*}

Conjecture (generalized Ogg's conjecture): for any positive integer $N$, we have $\mathcal{C}(N) = \mathcal{T}(N)$.

Strategy:
\begin{enumerate}
\item Compute $\mathcal{C}(N)$ explicitly to give a lower bound for $\mathcal{T}(N)$.
\item Study the Eisenstein ideal of the Hecke algebra of level $N$ to give an upper bound for $\mathcal{T}(N)$.
\end{enumerate}

Pass to the function field setting.

\section{A variation of a theme after Dirichlet, \textnormal{\emph{Chung Hang Kwan}}}

\begin{equation*}
  \int_0^1 \int_0^\infty \Phi \left[
    \begin{pmatrix}
      y_0      &  &  \\
               & y_0 &  \\
               &  & 1 \\
    \end{pmatrix}
    \begin{pmatrix}
      1      & u & 0 \\
      0 & 1 & 0 \\
      0 & 0 & 1 \\
    \end{pmatrix}\right]
  \,d^\times y_0
  e(- u) \, d u,
\end{equation*}
\begin{equation*}
  \int_0^\infty \int_0^1 \Phi \left[
    \begin{pmatrix}
      1      & u & \ast \\
      0             & 1 & \ast \\
      0             & 0 & 1 \\
    \end{pmatrix}
    \begin{pmatrix}
      y_0      & 0 & 0 \\
      0 & y_0 & 0 \\
      0 & 0 & 1 \\
    \end{pmatrix}\right]
  e(- u)
  \, d u
  \,d^\times y_0.
\end{equation*}


\section{On Hecke eigenvalues of Ikeda lifts, \textnormal{\emph{Nagarjuna Chary Addanki}}}

Preprint: \cite{2024arXiv2401.08855}.

$S_k(\Gamma_2)$ decomposes into a Maass space and its complement.  The Maass space is generated by Saito--Kurokawa lifts.  These lifts are characterized by the signs of their eigenvalues: $F$ is a Saito--Kurokawa lift if and only if all the eigenvalues are positive.  The question is whether there is a similar space in higher dimensions.  Ikeda lifts are generalizations of these Saito--Kurokawa lifts, so we can: are the eigenvalues of such lifts positive?

We're going to talk about Siegel modular forms, which are defined on the Siegel upper half-space
\begin{equation*}
  \mathbb{H}_n := \left\{ Z \in M_n(\mathbb{C}), \, {}^t Z = Z, \, \Im(Z) > 0  \right\}.
\end{equation*}
The symplectic group $\GSp_{2 n}(R) \leq \GL_{2 n}(R)$ is given by the condition ${}^t g J g = \mu(g) J$, where
\begin{equation*}
  J =
  \begin{pmatrix}
    0  & 1 \\
    -1 & 0 \\
  \end{pmatrix}.
\end{equation*}
We set $\Gamma_n := \Sp_{2n}(\mathbb{Z})$.  A Siegel modular form $F$ of weight $k$ over $\Gamma_n$ is defined in the usual way.  The space of such forms, $M_k(\Gamma_n)$, is finite-dimensional.  Elements $F$ admit a Fourier expansion
\begin{equation*}
  F(Z) = \sum_{
    \substack{
      T = T^t,  \\
      T \geq 0, \\
      \text{half-integral}
    }
  }
  A(T) e^{2 \pi i \trace(T Z)}.
\end{equation*}
We can define a notion of cusp form ($A(T) = 0$ unless $T >0$) and Petersson inner product.

For each $g \in G(\mathbb{Q})^+ \cap M_{2 n}(\mathbb{Z})$, there is an associated Hecke operator $T(g)$ on $M_k(\Gamma_n)$.  The Hecke algebra, $\mathcal{H}_n$ is generated by $T(g)$.
\begin{itemize}
\item $\mathcal{H}_n$ is commutative.
\item To each $m \in \mathbb{N}$ we may associate $T(m) \in \mathcal{H}_n$ such that $T(m n) = T(m) T(n)$ when $(m, n) = 1$.
\item Each $T(g)$ is self-adjoint with respect to the Petersson inner product.
\end{itemize}
Andrianov 1973: basis of simultaneous eigenfunctions.

Eigenvalues may be given in terms of the Satake parameters.  Let $F$ be a Hecke eigenform and $\lambda_F(g)$ the eigenvalue corresponding to the operator $T(g)$.  For any $g$ with $\mu(g) = p^r$, depending on $F$, there are $n + 1$ complex numbers
\begin{equation*}
  \left( a_{0, p}^{(F)}, a_{1, p}^{(F)}, \dotsc, a_{n, p}^{(F)} \right)
\end{equation*}
satisfying, with
\begin{equation*}
  \Gamma_n g \Gamma_n = \sqcup_i \Gamma_n g_i ,
  \quad
  g_i =
  \begin{pmatrix}
    A_i    &  B_i \\
    0           & D_i \\
  \end{pmatrix},
  \quad
  D_i =
  \begin{pmatrix}
    p^{d_{i 1}}    & \dotsb & \ast \\
    0                   & \dotsb & \ast \\
    0                   & \dotsb & p^{d_{i n}} \\
  \end{pmatrix},
\end{equation*}
we can explicitly describe the Hecke eigenvalue by
\begin{equation*}
  \lambda(g) = a_{0, p}^r \sum_i \prod_{j = 1}^n
  (a_{j, p}^{- j})^{d_{i j}}.
\end{equation*}.

Andrianov, 1974: if $F \in S_k(\Gamma_n)$ is a Hecke eigenform, then
\begin{equation*}
  \sum_{r = 0}^\infty \lambda_F(p^r) p^{- r s}
  = \frac{P_{F, p}(p^{- s})}{Q_{F, p}(p^{- s})},
\end{equation*}

\begin{equation*}
  Q_{F, p}(p^{- s}) = \dotsb.
\end{equation*}

Let $f \in S_k(\Gamma_1)$ be a Hecke eigenform.
\begin{theorem}[Ikeda, 2001]
  Assume that $n \equiv k \pmod{2}$.  Then
  \begin{equation*}
    F_f(Z) = \sum_{B \in \mathcal{S}_{2 n}(\mathbb{Z})^+}
    A(B) e^{2 \pi i \trace(B Z)},
    \quad
    Z \in \mathbb{H}_{2 n}
  \end{equation*}
  is a Hecke eigenform in $S_{k + n}(\Gamma_{2 n})$ and the standard $L$-function of $F$ is equal to
  \begin{equation*}
    \zeta(s) \prod_{i = 1}^{2 n}
    L(s + k + n - i, f, \mathrm{spin}).
  \end{equation*}
  Here $S_{2 n}(\mathbb{Z})^+$ is the set of all positive definite half-integral matrices of size $2 n$.
\end{theorem}

To compute the spin $L$-function of the Ikeda lift, move to the language of automorphic representations.  Let $F_f \in S_{k + n}(\Gamma_{2 n})$ be the Ikeda lift of $f \in S_{2 k}(\Gamma_1)$.  Let $\Pi$ and $\pi$ be the associated automorphic representations.
\begin{theorem}[Schmidt, 2003]
  We have
  \begin{equation*}
    L_p(s, \Pi, \mathrm{spin})
    = \prod_{j = 0}^n
    \prod_{
      \substack{
        r = j(j - 2 n)  \\
        \text{Step }2
      }
    }^{j(2n - j)}
    L_p(s + \tfrac{r}{2}, \pi, \Sym^{n - j})^{\beta(r, j, n)}.
  \end{equation*}
  Here $\beta(r, j, n)$ is given explicitly.
\end{theorem}

\begin{theorem}[A, 2024]
  For $n \equiv k \pmod{2}$, let $F_f \in S_{k + n}(\Gamma_{2 n})$ be the Ikeda lift of $f \in S_{2 k}(\Gamma_1)$.  For all large enough primes $p$, we have $\lambda_{F_f}(p) \geq 0$.
\end{theorem}

\begin{lemma}
  If $F_f \in S_{k + 2}(\Gamma_4)$ is the Ikeda lift of $f \in S_{2 k}(\Gamma_1)$, then $\lambda_{F_f}(p)$ is positive for all primes $p$.
\end{lemma}

For $\lambda_F(p^r)$ with $r > 1$, we use
\begin{equation*}
  \sum_{r = 0}^\infty \lambda_F(p^r) p^{- r s}
  = \frac{P_{F, p}(p^{- s})}{Q_{F, p}(p^{- s})},
\end{equation*}
and we need knowledge of both numerator and denominator.  The denominator can be computed using Schmidt's formluaf or the $L$-function, but the numerator is known only for genus at most $4$.

Let $T(p^4)$ be Hecke operators in $\mathcal{H}_4$.  Vankov (2011) found polynomials $P_p$ and $Q_p$ over the Hecke algebra such that
\begin{equation*}
  \sum_{r \geq 0} T(p^r) x^r = \frac{P_p(x)}{ Q_p(x)}.
\end{equation*}

$P_{p, F}(x)$ can be retrieved from $P_p(x)$.

$\lambda_F(p^r)$ for genus $4$ Ikeda lifts?  We do the following.
\begin{itemize}
\item Express $\frac{1}{Q_{p, F}(p^{- s})}$ as partial fractions.
\item Write each partial fraction as a power series in $p^{- s}$.
\item Multiply by $P_{p, F}(p^{- s})$.
\item Compare the coefficients of $p^{- r s}$ on both sides.
\end{itemize}
This yields a formula for $\lambda_{F_f}(p^r)$ in terms of some $c_i \in \mathbb{Z}[a, a^{-1}]$ that are bounded, uniformly in $r$.  This lead sto:
\begin{theorem}[A, 2024]
  Let $F_f$ be a Hecke eigenform in $S_{k + 2}(\Gamma_4)$, the Ikeda lift of $f \in S_{2 k}(\Gamma_1)$.  FOr a fixed $r$, the number $\lambda_{F_f}(p^r)$ is positive for all large enough primes $p$.
\end{theorem}


\section{Abelian covers of P1 of p-ordinary Ekedahl--Oort type, \textnormal{\emph{Deepesh Singhal }}}

Preprint: \cite{2023arXiv2303.13350}.

Fix a prime $p$.  Let $A$ be a $g$-dimensional principally polarized abelian variety defined over $\bar{\mathbb{F}}_p$.  There are three invariants attached to $A$:
\begin{itemize}
\item $p$-rank: $f(A) = \dim_{\bar{\mathbb{F}}_p}(\mu_p, A)$.  Note that
  \begin{itemize}
  \item $\# A[p](\bar{\mathbb{F}}_p) = p^f$, 
  \item $0 \leq f \leq g$.
  \end{itemize}
\item Newton polygon: the isogeny class of the $p$-divisible group $A[p^\infty]$, which can be presented as a sequence of $2 g$ rational numbers $N P(A) :=(\nu_1, \dotsc, \nu_{2 g})$.
\item Ekedahl--Oort type: the isomorphism class of the $p$-kernel $A[p]$, which can be presented as an element $w(A) \in \mathfrak{S}_{2 g}$.
\end{itemize}

When $g =1$, $A$ is an elliptic curve:
\begin{itemize}
\item $f(A) = 0$ iff $N P(A) =(\tfrac{1}{2}, \tfrac{1}{2})$ iff $w(A) = \id$ iff $A$ is supersingular.
\item $f(A) = 1$ iff $N P(A) =(0, 1)$ iff $w(A) =(21)$ iff $A$ is ordinary.
\end{itemize}

When $g \geq 2$, these are distinct invariants with some relation among each other:
\begin{itemize}
\item $p$-rank can be computed from either the Newton polygon or the $\mathrm{EO}$-type.
\item $f(A) = \# \{i : v_i = 0\}$.
\end{itemize}

One has the \emph{Torelli morphism}  from the moduli of curves to the moduli of abelian varieties:
\begin{equation*}
  \text{Hurwitz space} \xrightarrow{T} \text{Shimura variety}
\end{equation*}
\begin{equation*}
  C \xrightarrow{T} J(C).
\end{equation*}
\begin{itemize}
\item Fact: $\dim(\im(T)) < \dim(\text{ambient Shimura variety})$.
\item For $S = \mathcal{A}_{g, 1}$, it is known which $p$-rank, Newton polygons and $\mathrm{EO}$-types give \emph{nonempty} stratum.
\end{itemize}
\begin{question}
  Which stratum in $\mathcal{S}$ has nonempty intersection with the Torelli locus?
\end{question}

We turn to the Hurwitz space of cyclic covers of $\mathbb{P}^1$.  Fix $m$ such that $p \nmid m$.  $\mathcal{M}_{\mu_m}$ is the moduli space of $\mu_m$ covers of $\mathbb{P}^1$.
\begin{itemize}
\item $\mathcal{M}_{\mu_m} = \cup \mathcal{M}(m, r, \underline{a})$ is the decomposition of irreducible components.
\item $r$ is the number of branching points.
\item $\underline{a} =(a(1), a(2), \dotsc, a(r))$ is the monodromy data of each branching point.
\item Looking at one of the irreducible components, we have have the restricted Torelli morphism
  \begin{equation*}
    \mathcal{M}(m, r, \underline{a}) \xrightarrow{T} \operatorname{Sh}(H, \mu).
  \end{equation*}
\end{itemize}
Here
\begin{itemize}
\item $\operatorname{Sh}(H, \mu)$ is the smallest PEL type Shimura variety containing the image of $T$, with $\mu$ determined by the data $\underline{a}$.
\item The set of the Newton polygons occurring is known and denoted by $B(H_{\mathbb{Q}_p}, \mu)$.  However, very little is known about which of these Newton polygons have a stratum that intersects the Torelli locus.
\item The set $B(H_{\mathbb{Q}_p}, \mu)$ has a partial ordering, and $\nu(H, \mu)$ is the unique maximal element in this set.  It is called the $\mu$\emph{-ordinary} Newton polygon.
\end{itemize}

\begin{example}
  Consider the moduli space of $\mu_7$-covers of $\mathbb{P}^1$, with $r =4$ branching points, and monodromy datum $(3, 2, 1, 1)$.  If $p \equiv 6 \pmod{7}$, then there are three Newton polygons:
  \begin{itemize}
  \item the $\mu$-ordinary $\mu =(0,1)^4 \oplus(\tfrac{1}{2}, \tfrac{1}{2})^2$, with $p$-rank $4$.
  \item $\nu =(0, 1)^2 \oplus(\tfrac{1}{2}, \tfrac{1}{2} )^4$, with $p$-rank $2$.
  \item the basic $\beta =(\tfrac{1}{2}, \tfrac{1}{2} )^6$, with $p$-rank $0$.
  \end{itemize}
\end{example}

We refine Bouw's result and show the sharpness of the upper bound on Newton polygon given by $B(\mathbb{H}_{\mathbb{Q}_p}, \mu)$:
\begin{theorem}[Y.\ Lin, E.\ Mantovan, S.,\ 2023]
  Fix a prime $p$ and let $m \in \mathbb{N}$ such that $p \nmid m$ and $p > m(r - 2)$.  Then for any $\mathcal{M}(m, r, \underline{a})$ such that $r \leq 5$, we have
  \begin{equation*}
    T(\mathcal{M}(G, r, \underline{a}))[\nu(H, \mu)] \neq \emptyset.
  \end{equation*}
  In addition, we also show (...).
\end{theorem}

Comments:
\begin{itemize}
\item Same conclusion holds for abelian covers.
\item The statement of the theorem likely remains true without the assumptions that $p > m(r - 2)$ and $r \leq 5$.  However, these conditions are needed for our technique to work.
\item (...)
\end{itemize}

Proof strategy:
\begin{theorem}[Moonen, 2004]
  For a PEL-type Shimura variety $\operatorname{Sh}(H, \mu)$, we have
  \begin{equation*}
    \nu(H, \mu) \text{ stratum }
    =
    \text{ maximal EO stratum }
    \subseteq \text{ maximal $p$-rank stratum.}
  \end{equation*}
\end{theorem}
To prove the non-emptiness of the intersection with maximal EO stratum, we use the equivalence of categories between
\begin{itemize}
\item $p$-kernel of polarized abelian variety,
\item polarized mod $p$ Dieudonn{\'e} module, 
\item Hasse--Witt triples.
\end{itemize}
We translate the condition of having maximal EO type to linear algebra conditions on the Hasse--Witt triple.  By explicitly computing them, we are able to verify the rank conditions.

\section{Hyperbolic Counting Problems, \textnormal{\emph{Marius Voskou}}}
Preprint: \cite{2024arXiv2407.03134}.

We start with an interlude on hyperbolic geometry.  Work with $\PSL_2(\mathbb{R})$ (orientation-preserving isometries), lattice points $\gamma \cdot z$, where $\gamma \in \Gamma$, a discrete cofinite subgroup of $\PSL_2(\mathbb{R})$, and $z \in \mathbb{H}$.

Consider a hyperbolic circle of center $w$, radius $R$.

This looks like a Euclidean circle; the only difference is that the center is slightly lower than what you'd expect, which has to do with the fact that distances blow up as you get closer to the $x$-axis.

We're now ready to formulate our first hyperbolic counting problem: take
\begin{equation*}
  N(z, w, R) := \# \left\{ \gamma z : \gamma \in \Gamma, \dist(w, \gamma z) \leq R \right\}.
\end{equation*}

The same method as in the Euclidean counting problem does, in principle work.  We get a main term comparable to the area, an error term comparable to the circumference.  The problem is that in hyperbolic geometry, these are comparable -- both are comparable to $e^R$ -- so the error term is not smaller than the main term.  Instead, we use the spectral theory of automorphic forms.

To that end, the first step is to write our count in terms of the automorphic kernel:
\begin{equation*}
  N(z, w, R) = K(z, w)
  = \sum_{\gamma} \mathbf{1}(\dist(w, \gamma z) < R).
\end{equation*}
We know that automorphic kernels have nice spectral expansions in terms of eigenfunctions of the hyperbolic Laplacian.  Since it's a 20 minute talk, we'll just take spectral theory as a black box and not say much about it.  The thing we want to note is that the hyperbolic Laplacian has two kinds of eigenvalues:
\begin{itemize}
\item large eigenvalues: $\lambda \geq 1/4$, $\lambda = s(1 - s), s = 1/2 + i t$.
\item small eigenvalues: $\lambda < 1/4, \lambda = s(1 - s), s \in \left(1/2,1\right]$, of which there are finitely many.
\end{itemize}
Using this spectral expansion, Selberg in the 70's proved an asymptotic formula for $N(z, w, R)$:
\begin{theorem}[Selberg 1970, Gunther 1980, Good 1983]
  We have
  \begin{equation*}
    N(z, w, \log X)
    = \frac{2 \pi}{\operatorname{area}(\mathbb{H} / \Gamma)}
    X + \sum_{s_j}
    c_j(z, w) X^{s_j} + \O(X^{2/3}),
  \end{equation*}
  where the sum is over finitely many real numbers $s_j \in(1/2, 1)$, corresponding to the small eigenvalues.
\end{theorem}

Now, let's move to a slightly more complicated problem: let's ask what happens if instead of points, we have geodesics.  For fixed (closed) geodesics $\ell_1, \ell_2$, how many geodesics $\gamma \cdot \ell_1$ have distance $\leq R$ from $\ell_2$?  For simplicity, take $\ell_1 = \ell_2 = \ell$.

[Picture of a pair of closed geodeics intersecting.]

The distance between closed geodesics is the length of the common perpendicular.  We want to count how many closed geodesics are at given distance from one closed geodesic.

Our strategy is again to use the spectral theory of automorphic forms.  We can write the count as the integral of an automorphic kernel:
\begin{equation*}
  N(\ell_1, \ell_2, R) = \int_{\ell_1} \int_{\ell_2}
  K(z, w)
  \, d s(z)
  \, d s(w).
\end{equation*}
This leads to an asymptotic formula:
\begin{theorem}[Good 1983, Lekkas--Petridis 2024]
  We have
  \begin{equation*}
    N(\ell, \log X)
    = \frac{2(\operatorname{len} \ell)^2}{\pi \operatorname{area}(\mathbb{H} / \Gamma)} X
    + \sum_{s_j}(\ell) X^{s_j} + E(X),
  \end{equation*}
  where $E(X) = \O(X^{2/3})$.
\end{theorem}

Let's say a few words about why it's necessary to have a second proof of this.  The first proof by Good was hard to read, and some people still question whether it was valid.  In particular, if you want to do refinements of this or apply his methods to obtain related results, it's difficult.  Petridis--Lekkas found simpler methods that also apply to other problems.

\begin{conjecture}
  We have $E(X) = \O(X^{1/2 + \eps})$.  (And similarly for the previous problem, and many similar problems.)
\end{conjecture}

This has been around for many decades, and we have zero improvements on $X^{2/3}$.

What we're going to try to do now is to motivate this conjecture.  To that end, we'll start by saying why we can't do any better:
\begin{theorem}[V.\ 2024]
  We  have
  \begin{equation*}
    E(X) = \Omega \left( X^{1/2}(\log \log X)^{1/4 - \delta} \right).
  \end{equation*}  
\end{theorem}

\begin{theorem}[Lekkas--Petridis 2024]
  We have
  \begin{equation*}
    \left( \frac{1}{X} \int_X^{2 X} \lvert E(u) \rvert^2 \, d u \right)^{1/2} \ll X^{1/2} \log X.
  \end{equation*}  
\end{theorem}

Now, let's restrict to the case of a primitive hyperbolic element
\begin{equation*}
  h =
  \begin{pmatrix}
    m    & 0 \\
    0 & m^{-1} \\
  \end{pmatrix}.
\end{equation*}
For $\ell = \mathcal{I} / \langle h \rangle$, we have
\begin{equation*}
  \cosh \dist(\ell, \gamma \ell) = \max \left( \lvert a d + b c \rvert, 1 \right),
\end{equation*}
so our counting problem can be rephrased as follows: how many double cosets $\gamma \in \langle h \rangle \backslash \Gamma / \langle h \rangle$ with $\lvert a d + b c \rvert < X$?  We can also ask, are there more $\gamma$ with $a d + b c > 0$ than $a d + b c < 0$, or less?  Equivalently, are there more $a, d$ having the same sign than different, or less?  Geometric interpretation: the sign corresponds to the direction of the image $\gamma \cdot \ell$, and whether it lies on the left or the right of $\ell$.  This behaves differently as we tilde $\ell$ to some $\ell_\theta$.  Idea: consider
\begin{equation*}
  \frac{\partial^2}{\partial \theta \partial \phi}
  N(\ell_\theta, \ell_\phi, R).  
\end{equation*}

Let's finish with a nice arithmetic application.  Fix a prime $p$.  For $\Gamma$ an appropriate quaternion order, and $m = \left( 1 + \sqrt{2} \right)^2$:
\begin{theorem}[V.\ 2023, $p=5$; Hejhal 1982]
  We have
  \begin{equation*}
    \sum_{n \leq X} \mathcal{N}(n) \mathcal{N}(p n \pm 1)
    =
    \frac{p}{p + \qr{2}{p}}
    \cdot \left( \frac{\log m}{ \pi} \right)^2 X +
    \sum_{1/2 < s_j < 1}
    a_j^{\pm} X^{s_j} + \O(X^{2/3}),
  \end{equation*}
  where $a_j^{\pm}$ is real and $\mathcal{N}(n)$ is the number of $\mathfrak{a} \leq \mathbb{Z}[\sqrt{2}]$ with $\norm(\mathfrak{a}) = n$.
\end{theorem}

That's all.

\section{False indefinite theta functions and partitions separated by parity, \textnormal{\emph{Kathrin Bringmann}}}

Today we'll first say a bit about modular forms and partitions, but we'll probably skip most of that, since we've already seen it.  In particular, we need the Dedekind eta function $\eta(q) = q^{1/24} \prod(1 - q^n)$ and the theta function $\Theta(q) = \sum q^{n^2}$, together with the partition generating function $P(q) = \sum p(n) q^n$.  By work of Hardy and Ramanujan who developed the circle method, we have asymptotics for $p(n)$, which Rademacher improved to an exact formula: letting
\begin{equation*}
  A_k(n) := \sum_{h(k)^\ast} \omega_{h, k} e^{- \frac{2 \pi i n h}{k}},
\end{equation*}
and the Bessel function $I_\alpha$, Rademacher obtained
\begin{equation*}
  p(n) = \frac{2 \pi}{(24 n - 1)^{3/4}}
  \sum_{k \geq 1} \frac{A_k(n)}{n}
  I_{3/2} \left( \frac{4 \pi \sqrt{n}}{k} \right).
\end{equation*}

\textbf{Goal}: determine asymptotic behavior of $a(n)$ as $n \rightarrow \infty$.

Let $A(q) := \sum_{n \geq 0} a(n) q^n$ denote the generating function.  Then we can recover $a(n)$ via Cauchy.  The Hardy--Ramanujan asymptotics are obtained using
\begin{equation*}
  I_\kappa(x) \sim \frac{e^x}{ \sqrt{2 \pi x}}.
\end{equation*}
Rademacher--Zuckerman obtained exact formulas for Fourier coefficients of modular forms of negative weights, and Bringmann--Ono addressed the case of weight zero.

Ramanujan's lost notebook introduced the $q$-hypergeometric series $\sigma(q)$, with coefficients $S(n)$.
\begin{theorem}[Andrews--Dyson--Hickerson]
  We have $S(n) = T(24 n + 1)$, where $T(m)$ is the number of solutions to a certain Pell's equation.
\end{theorem}
\begin{definition}
  We say that a $q$-series is \emph{lacunary} if its coefficients are almost always zero.
\end{definition}
\begin{corollary}
  $\sigma(q)$ is lacunary.  $S(n)$ obtains every integer infinitely often.
\end{corollary}
Key identity: an expression for $\sigma(q)$ as a false indefinite theta function (not quite a modular form).

Now, what are the partition statistics that we're interested in?  We're interested in partitions separated by parity: even parts larger than odd parts (or reversed).  You can also require that parts be or not be distinct,.  In particular, we'r einterested in $p_{o d}^{e u}(n)$, the number of partitions with odd parts smaller than even parts and odd parts distinct.  For instance, for $n = 5$, we get $5$, $4 + 1$, $2 + 2 + 1$, so $p_{o d}^{e u}(5) = 3$.

The associated generating function $F_{o d}^{e u}(q)$ is related to $\sigma$ via
\begin{equation*}
  \frac{1}{(q^2, q^2)_\infty} \left( 1 - \dotsb \right),
\end{equation*}
where $\sigma$ is a simple expression involving $\sigma$.

\begin{theorem}[B--Craig--Nazaroglu]
  As $n \rightarrow \infty$, we have
  \begin{equation*}
    \sim \frac{e^{\pi \sqrt{n/3}}}{2 \sqrt{3} n}.
  \end{equation*}
\end{theorem}
\begin{proof}
  Use that coefficients are monotonic, Tauberian theorem, Euler--Maclaurin.
\end{proof}

Let's now tweak the combinatorics a little bit: take $p_{e u}^{o u}(n)$ denote the number of partitions with even parts smaller than odd parts, then the corresponding generating function is
\begin{equation*}
  F_{e u}^{o u}(q) =   \frac{1}{(1 - q)(q^2 ; q^2)_\infty}.
\end{equation*}

We can also define $p_{e u}^{o d}$, whose generating function is
\begin{equation*}
  \frac{1}{(q^2 ; q^2)_\infty} \sum_{n \geq 0} q^{n^2}.
\end{equation*}

Next, we look at $p_{e d}^{o u}$ (even parts smaller than odd parts, parts distinct), where the generating function is
\begin{equation*}
  \frac{1}{2(q; q^2)_\infty}
  \left((q; - q)_\infty + 1 - \sum_{n \geq 0}(1 -(- 1)^n q^n )(- 1)^{\dotsb} q^{\dotsb} \right).
\end{equation*}
This isn't modular, but you can improve it to obtain some sort of modular by introducing, for $z \in \mathbb{C}$ and $\tau \in \mathbb{H}$,
\begin{equation*}
  \psi(z, \tau) := i \sum_{n \in \mathbb{Z}} \sgn \left( n + \tfrac{1}{2} \right)(- 1)^n
  \zeta^{n + 1/2} q^{\tfrac{1}{2}(n + \tfrac{1}{2})^2}.  
\end{equation*}
We can recover this as the limit as $t \rightarrow \infty$ of a $\hat{\psi}(z; \tau, \tau + i t)$, where:
\begin{theorem}[B.--Nazaroglu]
  $\hat{\psi}$ has modular transformation properties.
\end{theorem}

Next, we want to look at $p_{e d}^{o d}$: partitions where even parts are smaller than odd parts, odd parts distinct.  The generating function is
\begin{equation*}
  \frac{(- q; q^2)_\infty}{ 1 - q}
  - \frac{q(- q^2 ; q^2)_\infty}{1 - q},
\end{equation*}
which is just some modular object, so you can use the circle method.

Similarly for a related case.

Next, let's look at $p_{o u}^{e d}$: odd parts smaller than even parts.  The generating function involves Ramanujan's mock theta function $f(q)$, which Zwegers showed how to understand (by completing by adding a second piece, given by an integral of a theta function of weight $3 /2$).

Harmonic Maass forms are $\hat{f}$ that transform like a modular form of weight $1/2$, annihilated by hyperbolic Laplacian of weight $1/2$.  B.--Mahlburg developed a theory of coefficients of mock modular forms multiplied by a modular form.  B.--Craig--Nazaroglu obtained leading asymptotics for all coefficients; can one obtain refinements for all of them?

$\sigma$ is not enough; one needs also its companion
\begin{equation*}
  \sigma^\ast(q) = \sum_{n \geq 1}
  \frac{(- 1)^n q^{n^2}}{(q; q^2)_n}.  
\end{equation*}
Let $a(n)$ its coefficients.
\begin{theorem}[Cohen]
  These are the coefficients of a Maass form.
\end{theorem}

\begin{definition}[Zagier]
  $f : \mathbb{Q} \rightarrow \mathbb{C}$ is a quantum modular form if $h_\gamma(x) := f(x) - f|_k \gamma(x)$ is nice on some nice subsets of $\mathbb{R}$.
\end{definition}

For $q$ a root of unity, we have $\sigma(q) = - \sigma^\ast(q^{-1})$.  We define $f$ in terms of $\sigma$ and $\sigma^\ast$, and also a function $R_\tau(z)$, and a measure defined by a bracket.  Then we can concisely express the transformation law of $f$ in terms of these definitions.  In particular, we deduce that $f$ is a quantum modular form.  This leads to improved asymptotics for the coefficients.  First, we split off components that we understand.  Let $r_o(n)$ denote the number of partitions of $n$ into distinct odd parts.  Its generating function is $\prod_{n \geq 0}(1 + q^{2 n + 1})$.  We can decompose into even and odd parts.

Our main theorem:
\begin{theorem}[B.--Craig--Nazaroglu]
  Very good asymptotics for $\alpha_j(n)$.
\end{theorem}

Work in progress: recover determine asymptotic expansion.

Idea of proof: use Maass forms.  Build a vector-valued form out of $\sigma$.  Circle method.


\section{On the local functional equation for archimedean exterior square L-function of $\mathrm{GL}_n$, \textnormal{\emph{Ravi Raghunathan}}}

The purpose of the talk is to complete the archimedean theory of the exterior square $L$-function of $\GL_r$ via the theory of integral representations due to Jacquet and Shalika.
\begin{itemize}
\item $F$: local, characteristic zero
\item $\psi_F : F \rightarrow \mathbb{C}$, nontrivial
\item $\pi$: irreducible admissible representation of $G_r := \GL_r(F)$.
\end{itemize}
Focus on the case $F$ archimedean, since the results are new only in that case.

The proofs are usually simpler when $r$ is odd, so we'll focus on the case $r = 2 n$.

The basic idea is as in Tate's thesis.  To each irreducible admissible representation $\pi$ of $G_r$, the local Langlands correspondence attaches an $n$-dimensional representation $\rho(\pi)$ of the Weil group (resp.\ Weil--Deligne group) when $F$ is archimedean (resp.\ $p$-adic).  For a character $\chi : F^\times \rightarrow \mathbb{C}$, the twisted exterior square $L$-function and $\eps$-factor of $\pi$ are defined via the local Langlands correspondence:
\begin{equation*}
  L(s, \pi, \wedge^2, \chi) := L(s, \wedge^2(\rho(\pi)) \otimes \chi),
\end{equation*}
\begin{equation*}
  \eps(s, \pi, \wedge^2, \chi) := \eps(s, \wedge^2(\rho(\pi)) \otimes \chi, \psi_F).
\end{equation*}
The right hand sides are the Artin $L$-function and $\eps$-factor.  When the character $\chi$ is trivial, we drop it from the notation.

\begin{example}
  If $\pi_p$ is an unramified representation of $G_n(\mathbb{Q}_p)$, then $\rho(\pi) = \chi_1 \oplus \dotsb \oplus \chi_n$, $\chi_i = \lvert . \rvert_p^{\lambda_i}$, then with $\alpha_i(p) = p^{- \lambda_i}$, we have
  \begin{equation*}
    L_G(s, \pi_p, \wedge^2) = \prod_{i < j} \frac{1}{ \left( 1 - \alpha_i(p) \alpha_j(p) p^{- s} \right)}.
  \end{equation*}
  In fact, the above formula is valid (even for finite extensions of $\mathbb{Q}_p$) as long as $\chi_i \chi_j$ is ramified, whenever $\chi_i$ and $\chi_j$ are both ramified.
\end{example}

If $F$ is $p$-adic, we denote by $\mathfrak{o}_F$ the ring of integers, and $q$ the cardinality of the residue field.  We let $N_r$ denote the standard maximal unipotent subgroup, and $Z_r$ the center of $G_r$.  Let $M_r$ denote the space of all $r \times r$ matrices, and $V_r$ the subspace of upper-triangular $r \times r$ matrices.

We set
\begin{equation*}
  w_n =
  \begin{pmatrix}
    &  & 1 \\
    & \dotsb &  \\
    1 &  &  \\
  \end{pmatrix},
  \quad
  w_{n, n} :=
  \begin{pmatrix}
    0    & 1_n \\
    1_n         & 0 \\
  \end{pmatrix}.
\end{equation*}
We view these as elements of $G_n$ or $G_r$, respectively.

Okay, one more thing.  We will assume our representations are generic, i.e., have a Whittaker model, which means you can embed the representation into a space of functions on $G$.  We extend $\psi_F$ to a character of $N_r$ by setting $\psi(u) := \prod_{j = 1}^{r - 1} \psi_F(u_{j, j + 1})$ for $u \in N_r$.  We will assume that $\psi$ is an irreducible generic representation of $G_n$ with respect to $\psi$ as above.  We denote the Harish--Chandra Schwartz space of Whittaker functions on $\pi$ by $\mathcal{W}(\pi, \psi)$, and refer to it in this talk simply as the Whittaker model of $\pi$.

When $F$ is archimedean, the space $\mathcal{W}(\pi, \psi)$ consists of smooth functions satisfying a moderate growth condition, transformation property with respect to $N_r$ and $\psi$, and is a Frech{\'e}t space.

We turn to the integral representation of Jacquet and Shalika.  For each $W \in \mathcal{W}(\pi, \psi)$, we define the function $\tilde{W}$ on $G_r$ by
\begin{equation*}
  \tilde{W}(g) := W(w_n {}^{\iota} g), \quad
  {}^\iota g := {}^t g^{-1}.
\end{equation*}
The Whittaker model of $\tilde{\pi}_v$, $\mathcal{W}(\tilde{\pi}, \bar{\psi})$, consists precisely of the set of functions $\{\tilde{W} : W \in \mathcal{W}(\pi, \psi)\}$.  Let $\mathcal{S}(F^n)$ be the space of Schwartz functions.  Use self-dual Haar measures.  Then the integral is
\begin{equation*}
  J(s, W, \phi) = \int_{N_n \backslash G_n}
  \int_{V_n \backslash M_n}
  W \left( \sigma
    \begin{pmatrix}
      1_n      & X \\
      0 & 1_n \\
    \end{pmatrix}
    \begin{pmatrix}
      g & 0 \\
      0 & g \\
    \end{pmatrix}\right)
  \psi(- \trace X)
  \, d X
  \phi(e_n g)
  \lvert \det g \rvert^s \, d g
\end{equation*}
for $W \in \mathcal{W}(\pi, \psi)$ and $\phi \in \mathcal{S}(F^n)$ and $s \in \mathbb{C}$ and $\sigma$ the permutation matrix given by
\begin{equation*}
  \begin{pmatrix}
    1 & 2 & \cdots & n & \mid & n+1 & n+2 & \cdots & 2n \\
    1 & 3 & \cdots & 2n-1 & \mid & 2 & 4 & \cdots & 2n
  \end{pmatrix}
\end{equation*}

\begin{proposition}[Jacquet--Shalika 1990]
  \begin{enumerate}
  \item   For a generic irreducible representation $\pi$ of $G_n$, the integrals $J(s, W, \phi)$ converge absolutely for $\Re(s) > 1 - \eta$ for some $\eta > 0$.
  \item The integrals admit meromorphic continuation.
  \item The integrals define elements of $\mathbb{C}(q^{- s})$.
  \end{enumerate}
\end{proposition}
We can thus take the fractional ideal of $\mathbb{C}[q^s, q^{- s}]$ generated by these integrals.  The generator can be taken (by results of Belt) to be of the form $1 / P(q^{-s})$.  This is then defined to be the Jacquet--Shalika exterior square $L$-value.  Are these the same as the Artin $L$-functions?  Yes:

\begin{theorem}[Kewat--R, Y.\ Jo]\label{theorem:cnojbgtred}
  Let $F$ be a $p$-adic field.  Then with $\pi$ as above,
  \begin{equation*}
    L_{JS}(s, \pi, \wedge^2)
    =
    L(s, \pi, \wedge^2).
  \end{equation*}
\end{theorem}

We now want to do the same thing in the archimedean setting, where less is known.  Implicit in Jacquet--Shalika is that these integrals admit meromorphic continuation.

First main result:

\begin{theorem}\label{theorem:cnojbgttz1}
  For $r = 2 n$, the quotients
  \begin{equation*}
    \Xi(s, W, \phi) := \frac{J(s, W, \phi)}{ L(s, \pi, \wedge^2)}
  \end{equation*}
  are entire and of finite order in vertical strips.  Moreover, if $\pi$ is ``nearly tempered'', then for every $s_0 \in \mathbb{C}$, we can choose $W \in \mathcal{W}(\pi, \psi)$ and $\phi \in \mathcal{S}(F^n)$ such that $\Xi(s_0, W, \phi) \neq 0$.
\end{theorem}

Here ``nearly tempered'' means that every pole of the $L$-function is realized by one of the integrals, so that the integrals really do capture the $L$-function more-or-less completely.

Second main result is the local functional equation:
\begin{theorem}
  With conditions as above, we have
  \begin{equation*}
    \frac{J(1 - s, \rho(w_{n, n}) \tilde{W}, \hat{\phi})}{L(1 - s, \tilde{\pi}, \wedge^2)}
    = \eps_{J S}(s, \pi, \wedge^2, \psi)
    \frac{J(s, W, \phi)}{L(s, \pi, \wedge^2)},
  \end{equation*}
  where $\eps_{J S}$ is of the form $A_0 B_0^s$.
\end{theorem}
For $F$ $p$-adic this was proved by Matringe, and in the odd case by Cogdell--???.

The main new tool we will use in the proofs of our results is the following very nice observation of Beuzart--Plessis (2021), which is deduced from Theorem 2 of Finis--Lapid--Muller (2015).

\begin{theorem}[Beuzart--Plessis]
  Let $k$ be a number field and let $w_0$ and $w_1$ be places of $k$ with $w_1$ non-archimedean.  Let $V$ be any open subset of $\operatorname{Temp}(\GL_n(k_{w_0}))$.  There exists a cuspidal automorphic representation $\Pi = \otimes_w ' \Pi_v$ of $\GL_n(\mathbb{A}_k)$ such that $\Pi_{w_0} \in V$ and $\Pi_w$ is unramified for every finite place $w \neq w_1$.
\end{theorem}

The nice new part of this theorem is that you can embed archimedean representations as well.  Moreover, there is this density that happens -- the moment you have an open set, you can find something there that can be pushed inside a global representation.  Once you have this, you're in great shape, because we were already (in Kewat--R) able to do this whenever we could push our representations into global ones.  Indeed, Beuzart--Plessis used the above theorem to prove analogues of our main results (Theorems \ref{theorem:cnojbgttz1} and \ref{theorem:cnojbgtred}) for the Asai $L$-function.

Let's say a bit about the proof.  Directly imitating Beuzart--Plessis allows us to prove a functional equation, but we don't' know that the $\eps$-factor that we get matches the Artin $\eps$-factor.  However, using the embedding theorem for different pairs of places (three times) allows us to show that the two $\eps$-factors are in fact the same.  What remains to show:
\begin{itemize}
\item Nonvanishing of $\Xi(s_0, W_v, \phi_v)$ in general.
\item Showing that there exist finitely many Whittaker and Schwartz functions that realize the $L$-factor.
\end{itemize}
Humphries and Jo \cite{2021arXiv2112.06860} prove the second assertion up to a polynomial factor, using a single test vector.

\section{Traces of Poincare series at square discriminants and Fourier coefficients of mock modular forms, \textnormal{\emph{Vaibhav Kalia}}}
(Joint work with PhD supervisor Balesh Kumar.)

[Oops, started late]

$\eps_\delta = 1$ if $\delta \equiv 1(4)$, and $\eps_d = i$ otherwise.

Fix $d, D \equiv 0, 1 (4)$; call them \emph{discriminants} if they are nonzero.

Fix $\tau \in \mathbb{H}$.

Let $k \in \tfrac{1}{2} \mathbb{Z}$. Assume $\Gamma = \SL_2(\mathbb{Z})$ if $k \in \mathbb{Z}$, else $\Gamma = \Gamma_0(4)$.

We define the slash operator and Kohnen's space $M_k^{!, +}$.

Let $j$ be Klein's $j$-function with $q$-expansion
\begin{equation*}
  j(\tau) = q^{-1} + 744 + 196884q + \dotsb.
\end{equation*}
Zagier 2002: we can give a basis for $M_{3/2}^{!, +}$ as $\mathcal{G}_d$, indexed by $0 < d \equiv 0, 1(4)$, where
\begin{equation*}
  \mathcal{G}_d(\tau) = q^{- d} + \sum_{0 > D \equiv 0, 1 (4)} a(d, D) q^{- D},
\end{equation*}
where
\begin{equation*}
  a(d, D) = \operatorname{Tr}_{D, d}(j - 744)
  := \frac{1}{\sqrt{d}}
  \sum_{Q \in \PSL_2(\mathbb{Z}) \backslash \mathcal{Q}_{d D}}
  \frac{\chi_d(Q)}{ \lvert \Gamma_Q \rvert}(j - 744)(\tau_Q).
\end{equation*}
Further, if for $0 > d \equiv 0, 1(4)$ we define
\begin{equation*}
  \mathcal{F}_d(\tau) := q^d + \sum_{0 < D \equiv 0 , 1(4)} a(D, d) q^D,
\end{equation*}
then Borcherds (1995) basis for $M_{1/2}^{!, +}$ is given by $\mathcal{F}_d$ for $0 > d \equiv 0,1(4)$.

\begin{question}
  Are there analogues of the functions $\operatorname{Tr}_{D, d}(\bullet)$ when $d D > 0$ for weight zero functions?  Are they related to any famiy of modular forms?
\end{question}
\begin{answer}
  Yes:  harmonic weak Maass forms and mock modular forms.
\end{answer}

We recall the definition of a weight $k$ harmonic weak Maass form on $\Gamma$.  It's any smooth function $f : \mathbb{H} \rightarrow \mathbb{C}$ that satisfies $f |_k A(\tau) = f(\tau)$, annihilated by $\Delta_k$ (hyperbolic Laplacian), and has an exponential growth condition at all cusps.  We denote by $\mathcal{H}_k ^!$ the space of such weight $k$ harmonic Maass forms on $\Gamma$.

By a \emph{sesqui} harmonic Maasss form of weight $k$, we mean the same thing, but without the second condition concerning $\Delta_k$, and instead $\xi_k \circ \Delta_k(f) = 0$, where $\xi_k$ is as in the definition of the shadow.

The Fourier expansion of $f \in \mathcal{H}_k ^!$ at $\infty$ with $k \neq 1$ is given by
\begin{equation*}
  f(\tau) = \sum_{
    \substack{
      n \in \mathbb{Z}  \\
      n \gg - \infty      
    }
  } c_f^+(n) q^n + c_f^-(0) y^{1 - k}
  + \sum_{
    \substack{
      n \in \mathbb{Z} - \{0\}  \\
      n \ll \infty      
    }
  } c_f^-(n)
  W_k(2 \pi n y) q^n.
\end{equation*}

Following Zagier (2007), a \emph{mock modular form} of weight $k$ for $\Gamma$ is the holomorphic part $f^+$ of $f \in \mathcal{H}_k ^!$.  It turns out that $\xi_k(f)$ is a weakly holmorphic of weight $2-k$, called the \emph{shadow} of $f$.

Duke--Imamoglu--Toth (2011) constructed a basis $\mathcal{F}_d$, indexed by $d < 0$, for mock modular forms, where
\begin{equation*}
  \mathcal{F}_d = \sum_{D > 0} a(D, d) q^D
\end{equation*}
with $a(d, D)$ defined in terms of the cycle integrals
\begin{equation*}
  (2 \pi)^{-1} \sum_{Q \in \Gamma \backslash \mathcal{Q}_{d D}}
  \chi_d(Q) \int_{C_Q}(j - 744)(\tau)
  \, \frac{d \tau}{ Q(\tau, 1)}
\end{equation*}
for $d , d > 0$ with $d D \neq \square$.

To interpret $a(D, d)$ when $d D = \square$, appeal to BFI2015 and Andersen2015.

JKK2013--2014: basis $\mathcal{G}_d$, indexed by $d > 0$, with coefficients $b(d, D)$ given in terms of twisted traces of cycle integrals of $\hat{\mathbb{J}}_1(\tau)$, where $\Delta_0(\hat{\mathbb{J}}_1) = - j + 720$ (done in the case where $d, D < 0$ and $d D \neq \square$).

What should $b(d, D)$ be when $d, D < 0$ and $d D = \square$?  We'll see in a moment.

We'll need the notion of Niebur Poincar{\'e} series (1973), which we've already seen, defined for $\Re(s) > 1$ by
\begin{equation*}
  G_{- 1}(\tau, s) = \sum_{A \in \Gamma_\infty \backslash \Gamma}
  e \left( - \Re(A \tau) \right)
  \phi_{- 1, s} \left( \Im(A \tau) \right).
\end{equation*}
Here
\begin{equation*}
  \phi_{- 1, s}(y) := 2 \pi y^{1/2} I_{s - 1/2}(2 \pi y).
\end{equation*}
It defines a smooth weight zero function for $\Gamma$.

The traces of cycle integrals of these Niebur Poincar{\'e} series we'll discuss now.
\begin{itemize}
\item If $0 < d, D$ with $D$ fundamental and $d D \neq \square$, we have (DIT 2011)
  \begin{equation*}
    \operatorname{Tr}_{d, D}(G_{- 1}(\tau, s)) := \frac{1}{2 \pi}
    \sum_{Q \in \Gamma \backslash \mathcal{Q}_{d D}}
    \chi_D(Q) \int_{C_Q} G_{- 1}(\tau, s)
    \, \frac{d \tau}{ Q(\tau, 1)}.
  \end{equation*}
\item With $d, D < 0$ and $D$ fundamental and $d D \neq \square$, due to $\chi_D(- Q) = \sgn(D) \chi_D(Q)$, we can define
  \begin{equation*}
    \operatorname{Tr}_{d, D}(G_{- 1}(\tau, s)) =
    \frac{1}{ \pi}
    \sum_{Q \in \Gamma_\infty \backslash \mathcal{Q}_{d D}^+}
    \chi_D(Q) \int_{S_Q} e(- \Re(\tau))
    \phi_{- 1, s}(\Im (\tau))
    \, \frac{d \tau}{ Q(\tau, 1)}.
  \end{equation*}
\item Superscripted $+$ means $a > 0$.
\end{itemize}

If $0 < d$, $D  = \square$ and $D$ fundamental, then the above trace is not defined. Andersen 2015:
\begin{equation*}
  \operatorname{Tr}_{d, D}(G_{- 1}(\tau, s))
  :=
  \frac{1}{2 \pi}
  \sum_{Q \in \Gamma \backslash \mathcal{Q}_{d D}}
  \chi_D(Q) \int_{C_Q} G_{- 1, Q}(\dotsb).
\end{equation*}
ETC.


Modified traces:
\begin{equation*}
  \operatorname{Tr}_{d, D}(G_{- 1}(\tau, s)) = \dotsb,
\end{equation*}
\begin{equation*}
  \tilde{\operatorname{Tr}}_{d, D}(G_{- 1}(\tau, s)) := \frac{1}{ \pi}
  \sum_{Q \in \Gamma_\infty \backslash \mathcal{Q}_{d D}^+}
  \chi_D(Q) \int_{C_Q} e(- \Re(\tau))
  \phi_{- 1, s} (\Im \tau)
  \, \frac{d \tau}{ Q(\tau, 1)}.
\end{equation*}

JKK2014: $\hat{\mathbb{J}}_1(\tau, s) := \partial_s G_{- 1}(\tau, s)$ and proved that $\hat{\mathbb{J}}_1(\tau) := \hat{\mathbb{J}}_1(\tau, 1)$ is a sesqui-harmonic Maass form of weight zero for $\Gamma$.  We define
\begin{equation*}
  \tilde{\operatorname{Tr}}_{d, D}(\hat{\mathbb{J}}_1(\tau)) :=
  \partial \P \tilde{\operatorname{Tr}}_{d, D}(G_{- 1}(\tau, s))|_{s=1}
\end{equation*}
when  $d , D < 0$ with $d D = \square$.

\begin{theorem}[-, Kumar 2024]
  For $d, D$ negative with $D$ fundamental and $d D = \square$, we have
  \begin{equation*}
    b(d, D) = - 8 \sqrt{d D} \tilde{\operatorname{Tr}}_{d,D}(\dotsb),
  \end{equation*}
  and also ETC.
\end{theorem}

Idea of the proof:
\begin{itemize}
\item $\tilde{\operatorname{Tr}}_{d, D}(G_{- 1}(\tau, s)) = \alpha(s, d D) \sum_{c = 1}^\infty \frac{S(d, D, 4 c)}{c^{1/2}}J_{s - 1/2} \left( \frac{\pi \sqrt{d D}}{c} \right)$, where
  \begin{equation*}
    S(d, D, 4 a)
    := \sum_{
      \substack{
        b (4 a)  \\
        (b^2 - d D) / 4 a \in \mathbb{Z}        
      }
    }
    \chi_D \left( \left[ a, b, \frac{b^2 - d D}{4 a} \right] \right)
    e \left( \frac{b}{2 a} \right).
  \end{equation*}
\item $\tilde{\operatorname{Tr}}_{d, D}(\hat{\mathbb{J}}_1 \dotsb)$, etc.
\end{itemize}


We can further relate these traces to the regularized inner product.  Let $f, g \in M_{1/2}^{!, +}$ (resp.\ $M_0 ^!$).  There exists $G \in \mathcal{H}_{3/2}^{!, +}$ (resp.\ $\mathcal{H}_2^!$) such that $\xi_-(G) = g$ (BDS2017).  Then
\begin{equation*}
  \langle f, g \rangle = \rho_k \sum_{n \in \mathbb{Z}} c_f(n) \dotsb.
\end{equation*}
ETC.  Relate to Borcherds.

Idea of the proof:
\begin{itemize}
\item Since mock modular forms $\mathcal{G}_d$ with shadow $\mathcal{F}_d$ has
  \begin{equation*}
    b(d, D) = 4 \sqrt{d D} c_{\lvert d \rvert}(\lvert D \rvert) + 192 \pi H(\lvert d \rvert)
    H(\lvert D \rvert),
  \end{equation*}
  we get by a computation
  \begin{equation*}
    \langle \mathcal{F}_d, \mathcal{F}_D \rangle = 2 \sqrt{\lvert D d \rvert}
    c_{\lvert d \rvert}(\lvert D \rvert) + 96 \pi H(\dotsb),
  \end{equation*}
  we get ETC.
\end{itemize}

Next:
\begin{corollary}
  Let $D$ be a negative fundamental discriminant.  Then we have
  \begin{equation*}
    \tilde{\operatorname{Tr}}_{D, D}(\hat{\mathbb{J}}_1(\tau))
    =
    - \frac{1}{4 \pi \sqrt{\lvert D \rvert}}
    \sum_{n > 0}
    \qr{D}{n}
    \frac{\langle j_n + 24 \sigma(n), j_1 + 24 \rangle}{n}
    e^{- 2 \pi n / \lvert D \rvert}
    - \dotsb.
  \end{equation*}
\end{corollary}
Idea of the proof: ANS2021 computed inner products of $\mathcal{F}_D$ in terms of exponential integrals, and then explicitly computed $\tilde{J}$, defined such that $\xi_2(\tilde{J}) = j_1 = j - 744$, to be given by $4 \pi h_1 - 8 \pi E_2^\ast$, where $h_1 \in H_2 ^!$ (DIT2016) satisfies $\xi_2(4 \pi h_1) = j_1 + 24$, and
\begin{equation*}
  E_2^\ast(\tau) := 1 - 24 \sum_{n \geq 1} \sigma(n) q^n.
\end{equation*}
We can thus write this inner product in terms of the Fourier coefficients of the harmonic Maass forms $h_1$.  We then just need to use that $\xi_2(4 \pi h_1) = j_1 + 24$ to write $c_{4 \pi h_1}^-(n) = 4 \pi n c_{j_1 + 24}(- n)$, then similarly for $+$.

As final remarks:
\begin{itemize}
\item For $n  < 0$, the Fourier coefficients $c_{j_1 + 24}(- n)$ of $j_1 + 24$ is the Rademacher--Petersson formula
  \begin{equation*}
    \frac{2 \pi}{ \sqrt{- n}}
    \sum_{c> 0}
    \frac{K(- 1, - n, c)}{c}
    I_1 \left( \frac{4 \pi \sqrt{- n}}{c} \right).
  \end{equation*}
\item   The regularized inner product $\langle j_n + 24 \sigma(n), j_1 + 24 \rangle$ can also be interpreted in terms of the Rademacher--ETC.
\end{itemize}


\section{Limiting behaviour and modular completions of MacMahon-like q-series, \textnormal{\emph{Badri Vishal Pandey}}}
Preprint: \cite{2024arXiv2402.08340}.

\begin{definition}[MacMahon's $q$-series]
  \begin{equation*}
    \mathcal{A}_a(q) := \sum_{1 \leq n_1 < \dotsb < n_a}
    \frac{q^{n_1 + \dotsb + n_a}}{(1 - q^{n_1})^{a_1} \dotsb}.
  \end{equation*}
\end{definition}
\begin{example}
  \begin{equation*}
    \mathcal{A}_1(q) = \sum_{n \geq 1} \frac{q^n}{(1 - q^n)^{2}}
    = \frac{1}{24}
    - \frac{1}{24} E_2(\tau).
  \end{equation*}
  This is a quasimodular form.
\end{example}

Motivation of our work is the following theorem:
\begin{theorem}[Amdeberhan--Ono--Singh]
  Let $a \in \mathbb{N}$.  We have
  \begin{equation*}
    q^{- \frac{a(a + 1)}{2}} \mathcal{A}_a(q) = \prod_{n \geq 1} \frac{1}{(1 - q^n)^3} + \O(\dotsb).
  \end{equation*}
\end{theorem}
\begin{example}
  We have
  \begin{equation*}
    \prod_{n \geq 1} \frac{1}{(1 - q^n)^3}
    = 1 + 3 q + 9 q^2 + 22 q^4 + \dotsb,
  \end{equation*}
  and these agree to increasing order with $q^{-1} \mathcal{A}_1$, $q^{- 3} \mathcal{A}_2$, $q^{- 6} \mathcal{A}_3$, $q^{-10} \mathcal{A}_4$, $q^{-15} \mathcal{A}_5$, $q^{- 21} \mathcal{A}_6$, etc.
\end{example}

Let $p_k(n)$ count the partitions of $n$ with parts having $k$-colors.  Its generating function is
\begin{equation*}
  \sum_{n \geq 0} p_k(n) q^n = \prod_{n \geq 1}
  \frac{1}{(1 - q^n)^k}.
\end{equation*}
\begin{question}[Ono]
  Find a family of functions that approximate $k$-colored partitions.
\end{question}

\begin{definition}
  For $a, r, s \in \mathbb{N}$ and $k \in \mathbb{Z}$, define the $q$-series
  \begin{equation*}
    \mathcal{A}_{a, k, r}(q) := \sum_{1 \leq n_1 < \dotsb < n_a}
    \frac{q^{r(n_1 + \dotsb + n_a)}}{(1 - q^{n_1})^k \dotsb} =: \sum_{n \geq 0} c_{a, k, r}(n) q^n,
  \end{equation*}
  \begin{equation*}
    \mathcal{B}_{a, k, r, s}(q) = \sum_{\dotsb}
    \frac{q^{r(n_1^2 + \dotsb + n_a^2) + s(n_1 + \dotsb + n_a)}}{\dotsb}
    =: \sum_{n \geq 0} d_{a,k,r,s}(n) q^n.
  \end{equation*}
\end{definition}

\begin{theorem}[Bringmann, Craig, van-Ittersum, P.]
  Let $a, r, s \in \mathbb{N}$ and $k \in \mathbb{Z}$.  We have
  \begin{equation*}
    q^{- \frac{r a(a + 1)}{2}} \mathcal{A}_{a, k, r}(q) = \prod_{n \geq 1}
    \frac{1}{(1 - q^{r n})(1 - q^n)^k}
    + \O(q^{a + 1}),
  \end{equation*}
  and similarly for $\mathcal{B}$.
\end{theorem}
\begin{example}
  We have
  \begin{equation*}
    \frac{1}{(1 - q^n)^4} = 1 + 4 q + 14 q^2 + 40 q^3 + \dotsb,
  \end{equation*}
  approximately to increasing accuracy by $q^{-1} \mathcal{A}_{1, 3, 1}$, $q^{- 3} \mathcal{A}_{2, 3, 1}$, $q^{- 6} \mathcal{A}_{3, 3, 1}$, and so on.
\end{example}

We want to study these functions $\mathcal{A}$ and $\mathcal{B}$.

\begin{theorem}[Bringmann, Craig, van-Ittersum, P.]
  If $a, r \in \mathbb{N}$, then $\mathcal{A}_{a, 2 r, r}$ is a quasimodular form of (mixed) weight $2 a r$.
\end{theorem}
\begin{example}
  Formulas for $\mathcal{A}_{2, 4, 2}$ and $\mathcal{A}_{2, 4, 3}$ in terms of $G_4, G_6, G_2, G_3, G_7, G_5$.
\end{example}
\begin{remark}
  The proof uses the quasi-shuffle algebra.
\end{remark}
\begin{remark}
  If $k \neq 2 r$, then odd Eisenstein series are involved.
\end{remark}

\begin{corollary}[Bringmann, Craig, van-Ittersum, P.]
  Let $a, r, m \in \mathbb{N}$.  There are infinitely many non-nested AP $A n + B$ with
  \begin{equation*}
    c_{a, 2 r, r}(A n + B) \equiv 0 \pmod{m}.
  \end{equation*}
\end{corollary}
\begin{conjecture}[Bringmann, Craig, van-Ittersum, P.]
  Let $a,k,r,s \in \mathbb{N}$.  For any prime $p$, we have
  \begin{equation*}
    c_{a, k , r}(p^{\alpha + 1} n + p^\alpha \beta) \equiv 0 \pmod{p^{\nu_p(k) - \alpha}}.
  \end{equation*}
  ETC.
\end{conjecture}

We move now to results on $\mathcal{B}$, which turns out to have modular properties.
\begin{definition}
  We call a function $\hat{f}(\tau, \bar{\tau})$ a \emph{completion} of $f(\tau)$ if
  \begin{equation*}
    \lim_{\bar{\tau} \rightarrow - i \infty}
    \hat{f}(\tau, \bar{\tau})
    = f(\tau).
  \end{equation*}
\end{definition}
\begin{theorem}
  Let $k, r, s \in \mathbb{N}$.
  \begin{enumerate}
  \item The function $\mathcal{B}_{a, k, r, s}$ is a polynomial in $\mathcal{B}_{k, r, s} := \mathcal{B}_{1, k, r, s}$ of degree $a$.
  \item $\mathcal{B}_{k, r, s} + \mathcal{B}_{k, r, k - s}$ are a mixed weight linear combination of:
    \begin{enumerate}[(a)]
    \item functions admitting a modular completion
    \item\label{enumerate:cnojbnhici} powers of $q$ times theta functions
    \end{enumerate}
  \item If $k = s + 1$, then \eqref{enumerate:cnojbnhici} in the above linear combination vanishes.
  \end{enumerate}
\end{theorem}

Define Zwegers' multivariable Appell function for $\ell \in \mathbb{N}$ by
\begin{equation*}
  A_{\ell}(z_1, z_2 \tau)
  = \zeta_1^{\ell / 2}
  \sum_{n \in \mathbb{Z}}
  \frac{(- 1)^{\ell n} q^{\frac{\ell n(n + 1)2}{}} \dotsb}{\dotsb}.
\end{equation*}

Define some $R$.  Then $\hat{A}_{\ell}$ is given in terms of $A_{\ell}$ and $\vartheta$ and $R$.

Theorem of Zwegers describes transformation of $\hat{A}_{\ell}$.

Now we briefly describe the idea of the proof of our result.  Taylor expansion where Taylor coefficients turn out to be similar functions:
\begin{equation*}
  \exp \left( \sum_{n \geq 1}
    \frac{(- 1)^{n + 1}}{ n}
    \mathcal{B}_{n k, n r, n s}(q) X^n\right)
  = \sum_{j \geq 0}
  \mathcal{B}_{j, k, r, s}(q) X^j.
\end{equation*}
From this we get that the coefficients on the right hand side are indeed polynomials in terms of the coefficients on the left hand side.  This gives the first part.  To get the other parts, we use that
\begin{align*}
  \mathcal{B}_{j, r, s - k + j}(q) + \mathcal{B}_{j, r, k - s}(q)
  &= \sum_{n \in \mathbb{Z} - \{0\}}
    \frac{q^{r n^2 +(s - k + j) n}}{(1 - q^n)^j}
  \\
  &= - \frac{1}{j !}
    \left[ \frac{\partial^i}{\partial \zeta^j}(1 - \zeta) \sum_{n \in \mathbb{Z}}
    \frac{q^{r n^2 +(s + 1 - k) n}}{1 - \zeta q^n}\right]|_{z=0}
  \\
  &=
    - \frac{1}{j!}
    \left[ \frac{\partial^j}{\partial \zeta^j}(1 - \zeta) \zeta^{- r}
    A_{2 r}(z,(- k - r + s + 1) \tau  \tau)\right]|_{z=0}.
\end{align*}
We get
\begin{equation*}
  \mathcal{B}_{k, r, k - 1}(q) + \mathcal{B}_{k, r, 1}(q)
  = - \frac{1}{k!}
  \sum_{\ell = 0}^k
  \frac{\alpha_{1/2}(k, \ell)}{(2 \pi i)^{\ell}}
  f_{r, \ell }(\tau),
\end{equation*}
where
\begin{equation*}
  f_{r, \ell}(\tau) :=
  \left[ \frac{\partial^{\ell}}{ \partial z^{\ell}}
    \left( \left( \zeta^{-1/2} - \zeta^{1/2} \right) \zeta^{- r} A_{2 r}(z, - r \tau, \tau) \right)\right]_{z = 0}
\end{equation*}
and $\alpha_{1/2}(k, \ell)$ is a generalized Stirling number defined by a recurrence.  Enough to show that $f_{r, \ell}(\tau)$ can be completed.

Fact: Taylor coefficients of a Jacobi form are quasi-modular.

\begin{example}[$k=2=s+1, r=1$]
  \begin{enumerate}
  \item We write $\mathcal{B}_{2, 2, 1, 1}(q)$ as the coefficient of $X^2$ in
    \begin{equation*}
      \exp \left( \mathcal{B}_{2, 1, 1}(q) X - \tfrac{1}{2} \mathcal{B}_{4, 2, 2}(q) X^2 + \O(X^3) \right),
    \end{equation*}
    which we compute to be $\tfrac{1}{2} \mathcal{B}_{2, 1, 1}^2(q) - \tfrac{1}{2} \mathcal{B}_{4, 2, 2}(q)$.
  \item ETC.
  \end{enumerate}
\end{example}


\section{Explicit lower bounds on the conductor of abelian varieties with specified bad reduction, \textnormal{\emph{Pierre Tchamitchian}}}

The first part is about $L$-functions.  Consider a ``good'' $L$-function: there exists
\begin{itemize}
\item $N > 0$, the conductor, 
\item $(d_1, d_2)$ with $d = d_1 + 2 d_2$, the signature,
\item $\mu_j$ and $\nu_k$, the spectral parameters ($j \leq d_1, k \leq d_2$)
\end{itemize}
such that the completed $L$-function
\begin{equation*}
  \Lambda(s) = N^{s/2}
  \prod_{j \leq d_1}
  \Gamma_{\mathbb{R}}(s + \mu_j) \prod_{k \leq d_2} \Gamma_{\mathbb{C}}(s + \nu_k) L(s)
\end{equation*}
satisfies
\begin{equation*}
  \Lambda(s) = \eps \tilde{\Lambda}(1 - s)
\end{equation*}
for some $\eps$, and $L(s)$ admits an Euler product
\begin{equation*}
  L(s) = \prod_p F_p(p^{- s})^{-1}
  = \prod_p
  \left( 1 - \alpha_{1, p} p^{- s} \right)^{-1}
  \dotsb
  \left( 1 - \alpha_{d_p, p} p^{- s} \right)^{-1},
\end{equation*}
absolutely convergent for $\sigma > 1$.

\begin{definition}
  We say that $F : \mathbb{R} \rightarrow \mathbb{R}$ is a \emph{Mestre test function} if there exists $\eps > 0$ such that
  \begin{enumerate}[(i)]
  \item $F(x) \exp \left((\tfrac{1}{2} + \eps) \lvert x \rvert \right)$ is integrable
  \item $F(x) \exp \left((\tfrac{1}{2} + \eps) x \right)$ has bounded variation, is integrable, and the value at each point is the average of its right and left limits
  \item $(F(x) - F(0)) / x$ has bounded variation.
  \end{enumerate}
\end{definition}
Then  we get the Weil--Mestre explicit formula: for $A = \pi^{- d/2} 2^{- d_2} N^{1/2}$
and $\Phi(s) = \int F(x) e^{(s - 1/2)x}\, d x$, we have
\begin{equation*}
  \sum_\rho \Phi(\rho) - \sum_\mu \Phi(\mu) + \sum_{j \leq d_1}
  I(\tfrac{1}{2}, \mu_j /2) + I(1/2, \overline{\mu_j}/ 2)
  + \sum_{k \leq d_2} J(1, \nu_k) + J(1, \overline{\nu_k})
\end{equation*}
\begin{equation*}
  = 2 F(0) \log(A) - \sum_{p, i, m \geq 1}
  \left( \alpha_{i, p}^m F(m \log p)
    + \overline{\alpha_{i, p}}
    ^m F(- m \log p)
  \right)
  \frac{\log p}{ p^{m/2}}.
\end{equation*}
Morally, the proof is a Fourier transform, as in the prime number theorem.  You integrate the function $\Phi(s) \Lambda '(s) / \Lambda(s)$ along the rectangular contour with real parts $- \eps$ and $1 + \eps$.  Relate left and right hadn sides using functional equation.  Compute explicitly the right hand side using the Euler product, and use a Fourier transform.  Because of this, we use GRH.

Trick: choose $F(x)$ such that $\Re(\Phi(s)) > 0$ for $s$ on the critical line.  Equivalently, choose $F(x)$ with a positive Fourier transform.  Then we have the \emph{Weil--Mestre explicit inequality}
\begin{multline*}
  2 F(0) \log(A) \geq \sum_{j \leq d_1}
  I(\tfrac{1}{2}, \mu_j /2)
  + I(\tfrac{1}{2}, \overline{\mu_j}/2)
  + \sum_{k \leq d_2} J(1, \nu_k) + J(1, \overline{\nu_k}) \\
  + \sum_{p, i, m \geq 1} \left( \alpha_{i, p}^m F(m \log p) +
    \overline{\alpha_{i, p}}^m F(- m \log p)\right)
  \frac{\log p}{p^{m/2}}.
\end{multline*}
\begin{theorem}[Poitou, Odlyzko, 1976]
  Let $K$ be a numebr field of degree $n$ and discriminant $d_K$, with $r_1$ real embeddings and $r_2$ complex embeddings.  Then, for $n$ big enough,
  \begin{equation*}
    \lvert d_K \rvert^{1/n} >(215.3)^{r_1 / n}(44.7)^{r_2 /n}.
  \end{equation*}
\end{theorem}

Let's talk about bad reduction of elliptic curves.  Take $E$ over $\mathbb{Q}$, say
\begin{equation*}
  y^2 = x^3 + a x + b
\end{equation*}
with $4 a^3 + 27 b^2 \neq 0$.  By reducing the coefficients (of a minimal model at $p$) modulo $p$, one gets a curve $E_p$ over $\mathbb{F}_p$.  Then we end up in one of the following cases:
\begin{itemize}
\item $E_p$ is an elliptic curve over $\mathbb{F}_p$
\item $E_p$ is a singular curve, with $E_{p, n s} \cong \mathbb{G}_m$
\item $E_p$ is a singular curve, and $E_{p, ns} \cong \mathbb{G}_a$
\end{itemize}
Set respectively $n_p = 0, 1, 2$ depending upon the reduction (unless $p \in \{2,3\}$).

The \emph{conductor} of $E$ of the elliptic curve is defined by $N_E := \prod p^{n_p}$.

We define the $L$-series in terms of the $a_p := p + 1 - \# E_p(\mathbb{F}_p)$, by $L_p(s) = 1 - a_p p^{- s} + p^{1 - 2 s}$ if $E$ has good redcution, omitting the final term if bad reduction.  We multiply these together to get $L_E(s)$.  Then we have the modularity theorem, saying that these admit an analytic continuation.  We also have the Mordell--Weil theorem and BSD conjecture.

If $F$ is even, has positive Fourier transform, and support in $[-1, 1]$, and write $F_\lambda(x) = F(x / \lambda)$ for $\lambda > 0$, then we have the Mestre inequality.  It involves
\begin{equation*}
  M_{\lambda, F} = 2 \log(2 \pi) + 2 I_{F_\lambda, 2} = 2 \left( \dotsb \right).
\end{equation*}

Instead of looking at all elliptic curves, we can only look at the ones with some prescribed reduction.

We can now do the same things for abelian varieties.  Looking at the bad reduction, there's a smooth part, torus part and unipotent part.  We associate $L$-functions and Satake parameters.  We have a Mestre inequality (1986): the conductor satisfies $N_A \geq 10.323^g$.

LMFDB.

Everything that we described here over $\mathbb{Q}$ also works over number fields.

\section{Analytic ranks of modular forms with nontrivial character, \textnormal{\emph{Maarten Derickx}}}
Link to actual slides: \url{https://bit.ly/analytic-ranks}.

We write $S_2(\Gamma)^{\mathrm{zero}}$ to consist of those newforms $f$ for which $L(f, 1) = 0$, where $\Gamma = \Gamma_0(N)$ or $\Gamma_1(N)$.

\begin{conjecture}[Brumer]
  Roughly half the modular forms should have positive analytic rank, and roughly half should have analytic rank zero:
  \begin{equation*}
    \# S_2(\Gamma_0(p))^{\mathrm{zero}} \sim \frac{1}{2} \# S_2(\Gamma_0(p)).
  \end{equation*}
\end{conjecture}

The sign functional equation forces
\begin{equation*}
  \# S_2(\Gamma_0(p))^{\mathrm{zero}}
  \geq \left( \tfrac{1}{2} - \eps \right)
  \# S_2(\Gamma_0(p))^{\new},
  \qquad \text{ for }
\end{equation*}<++>


[Okay, he's showing some slides of the LMFDB.]

\begin{theorem}[Michel--Kowalski]
  We have
  \begin{equation*}
    \# S_2(\Gamma_0(p))^{\mathrm{zero}}
    \leq \left( \frac{5}{6} + \eps \right) \# S_2(\Gamma_0(p))^{\new}.
  \end{equation*}
\end{theorem}

[More LMFDB.  The $\chi$ with $r_\chi(p) := \# S_2(p, \chi)^{\mathrm{zero}} > 0$ for the primes $p < 100000$.  Pretty rare that there's more than one relevant modular form per character.]

\begin{conjecture}[D.,\ Stoll]\label{conjecture:cnojrtinnd}
  If $\# S_2(p, \chi)^{\mathrm{zero}} > 0$, then $\ord \chi \leq 30$, or at least $\ord \chi$ grows super slowly with respect to $p$.
\end{conjecture}
This is like the equivalent of Brumer's conjecture with nontrivial character.

[More LMFDB discussion.  Examples with higher order should come from elliptic curves with higher rank.]

Let's give an application of Conjecture \ref{conjecture:cnojrtinnd}.  Let $p$ be a prime, $K$ a number field of degree $d$, $E$ an elliptic curve over $K$.
\begin{theorem}[Oesterle]
  Suppose $p \mid E(K)_{\tors}$, then $p <(3^{d/2} + 1)^2$.
\end{theorem}
Set
\begin{equation*}
  r_1(p) := \# S_2(\Gamma_1(p))^{\mathrm{zero}}
  -
  \# S_2(\Gamma_0(p))^{\mathrm{zero}}
  =
  \sum_{\chi \neq \chi_{\text{triv}}} \# S_2(p, \chi)^{\mathrm{zero}}.
\end{equation*}
\begin{theorem}[D.\ Stoll]
  Suppose $r_1(p) \leq \log_6(p) - 4$ for $p$ large enough.  Then for $p$ large enough, we have $p \mid E(K)_{\tors}$ implies $p \leq 6 d + 1$.
\end{theorem}

(...)

\begin{example}
  Let $p =99991$.  Suppose there exists a number field $K$ of degree $d$, and an elliptic curve over $K$, such that if $p \mid E(K)_{\tors}$, then $p \geq 33330$.  This is sharp, since complex multiplication gives examples with $p \mid E(K)_{\tors}$ and $d = 33330$.
\end{example}

\begin{proposition}[D.,\ Stoll]
  For every prime $p = m^2 \pm 108$, there is a genus $2$ curve $C_p$ such that
  \begin{itemize}
  \item $\mathbb{Z}[\zeta_3]$ is contained in $\End \Jac(C_p)$ (i.e., $\Jac(C_p)$ is of $\GL_2$ type)
  \item $\Jac(C_p)$ is modular of level $p$.
  \item If $\Jac(C_p)$ is simple, then the associated modular form has character of order $3$ or $6$.
  \end{itemize}
\end{proposition}

\begin{proposition}
  Every character $\chi$ of order $6$ in the table satisfies $p = m^2 \pm 108$, and Mordell--Weil rank $\Jac(C_p)(\mathbb{Q}) > 0$.
\end{proposition}

\begin{lemma}[D.,\ Stoll]
  Assume Bunyakovsky's conjecture.  Then there are infinitely many characters $\chi :(\mathbb{Z} / p \mathbb{Z})^\ast \rightarrow \mathbb{C}^\times$ of order $6$ such that $\# S_2(p, \chi)^{\mathrm{zero}} \geq 1$.
\end{lemma}

We did something similar for characters of order $3$, twisting elliptic curves.
\begin{lemma}[D.,\ Stoll]
  Let $\chi$ mod $p$ be of order $3$.  Then there exists $f \in S_2(p, \chi)^{\mathrm{zero}}$ with coefficients in $\mathbb{Q}(\zeta_3)$ iff there is an elliptic curve $E$ over $\mathbb{Q}$ such that
  \begin{enumerate}
  \item $E$ has conductor $p^2$,
  \item (...)
  \end{enumerate}
\end{lemma}

Let $\chi$ be of order $n$.  Let $\mathbb{S}_2(p, \chi)$ denote the space of cuspidal modular symbols with coefficients in $\mathbb{Q}(\zeta_n)$, and $\mathbb{T}_\chi \subset \End S_2(p, \chi)$ the Hecke algebra.
\begin{lemma}[The winding element $e$]
  There is an explicit $e \in \mathbb{S}_2(p, \chi)$ such that
  \begin{equation*}
    \dim_{\mathbb{Q}(\zeta_n)}(\mathbb{T}_\chi \otimes \mathbb{Q})e +
    \# S_2(p, \chi)^{\mathrm{zero}}
    = \# S_2(p, \chi)^{\new}
    = \dim S_2(p, \chi).
  \end{equation*}
\end{lemma}
\begin{itemize}
\item $\dim_{\mathbb{Q}(\zeta_n)}(\mathbb{T}_\chi \otimes \mathbb{Q}) e$ is computable with magma in reasonable time when $n$ is small enough and $p < 100000$.
\item If $n \asymp p$, then $\dim_{\mathbb{Q}} \mathbb{Q}(\zeta_n) \asymp p$ which makes the computation unfeasible for $p \asymp 100000$.
\item For $n$ large, work over $\mathbb{F}_p$ for a $p$ that splits completely in $\mathbb{Q}(\zeta_n)$.
\end{itemize}

\section{Sharp conditional bounds for moments of products of automorphic L-functions, \textnormal{\emph{Markus Valås Hagen}}}

Want to study
\begin{equation*}
  I_k(T) := \int_T^{2 T}
  \left| \zeta(\tfrac{1}{2} + i t) \right|^{2 k} \, d t.
\end{equation*}
Gives us a lot of information:
\begin{itemize}
\item Taking $k$ large, we amplify the peaks, giving a way to study extreme values.
\item Showing that $I_k(T) \ll T^{1 + \eps}$ for each $\eps > 0$ and $k \in \mathbb{N}$ is equivalent to knowing the Lindel\"{o}f hypothesis, i.e., that $\zeta(\tfrac{1}{2} + i t) \ll t^\eps$ for each $\eps > 0$.
\end{itemize}
\begin{conjecture}
  $I_k(T) \sim C_k T(\log T)^{k^2}$, for any real $k > 0$, where $C_k$ is predicted precisely via random matrix theory.
\end{conjecture}

Moments are closely related to the distribution of $\zeta$.  Selberg central limit theorem: for fixed $V \in \mathbb{R}$ and large $T$, we have
\begin{equation*}
  \Phi_T(V) := \frac{1}{T}
  \vol \left(
    t \in[T, 2 T] :
    \frac{\log \lvert \zeta(\tfrac{1}{2} + i t) \rvert}{ \sqrt{\tfrac{1}{2} \log \log T}} \geq V\right)
  \sim \frac{1}{\sqrt{2 \pi}}
  \int_V^\infty e^{- x^2 /2} \, d x.
\end{equation*}
In other words, $\log \lvert \zeta(\tfrac{1}{2} + i t) \rvert$, sampled for $t \in[T, 2 T]$, is approximately a normal variable with mean zero and variance $\tfrac{1}{2} \log \log T$

The order of magnitude of $I_k(T)$ is closely related to $\Phi_T(V)$ for $V \asymp k \sqrt{\log \log T}$.

What do we know about $I_k(T)$?
\begin{itemize}
\item Asymptotics for $k = 1, 2$
\item $I_k(T) \asymp T (\log T)^{k^2}$ for $0 < k \leq 2$
\item Assuming RH, $I_k(T) \ll T(\log T)^{k^2}$ for all $k > 0$.
\end{itemize}

$L$-functions in general are  attached to some object $X$, and are used to study that object.  $X$ can be an elliptic curve, a modular form, a number field, etc.  Also has pleasant analytic properties.  It is believed that all $L$-functions should arise from automorphic $L$-functions, or products of them.  This gives, at least for the speaker, a reason to be interested in automorphic $L$-functions.  GRH (Grand Riemann Hypothesis) asserts the Riemann Hypothesis for all automorphic $L$-functions, i.e., all nontrivial zeros $\rho$ satisfy $\Re(\rho) = 1/2$.

Automorphic $L$-functions also have Dirichlet series and Euler product for $\Re(s) > 1$, say
\begin{equation*}
  L(s) = \sum_{n \geq 1}
  \frac{a_L(n)}{ n^s}.
\end{equation*}
In our proof, we need control over these coefficients.  If $\Lambda_L(n)$ denotes the coefficients of the Dirichlet series of $L'/L(s)$, then the Generalized Ramanujan Conjecture gives the pointwise bound
\begin{equation*}
  \lvert \Lambda_L(n) \rvert \leq m_L \Lambda(n),
\end{equation*}
where $m_L$ is a fixed constant depending only on $L$.  There are also milder alternatives to GRC, such as the so-called ``Hypothesis H'' of Rudnick--Sarnak, which may be understood as an average version of GRC.  In particular, GRC implies Hypothesis H.

From now on, ``$L$-function'' means ``automorphic $L$-function.''

It's expected that moments of $L$-functions behave like those of the zeta function, i.e.,
\begin{equation*}
  \int_T^{2 T} \left| L(\tfrac{1}{2} + i t) \right|^{2 k} \sim C_{k, L} T(\log T)^{k^2}.
\end{equation*}
We want to study moments, but with several $L$-functions at once, e.g.,
\begin{equation*}
  \int_T^{2 T}
  \left| L_1(\tfrac{1}{2} + i t) \right|^{2 k_1}
  \left| L_2(\tfrac{1}{2} + i t) \right|^{2 k_2} \, d t.
\end{equation*}
This sheds light on the independence between them.  Given two distinct irreducible $L$-functions, we expect \emph{independence} between them.  In particular,
\begin{multline*}
  \frac{1}{T} \int_T^{2 T}
  \left| L_1(\tfrac{1}{2} + i t) \right|^{2 k_1}
  \left| L_2(\tfrac{1}{2} + i t) \right|^{2 k_2} \, d t
  \\
  \approx
  \frac{1}{T}
  \int_T^{2 T}
  \left| L_1(\tfrac{1}{2} + i t) \right|^{2 k_1} \, d t
  \times \frac{1}{T}
  \int_T^{2 T}
  \left| L_2(\tfrac{1}{2} + i t) \right|^{2 k_2} \, d t
  \asymp(\log T)^{k_1^2 + k_2^2}.
\end{multline*}

Milinovich--Turnage-Butterbaugh (2014): Let $k _1, \dotsc, k_r > 0$ be real numbers.  Let $L_1, \dotsc, L_r$ be distinct irreducible cuspidal automorphic $L$-functions with unitary central characters.  Assume the Generalized Riemann Hypothesis and Hypothesis H.  Then, for each $\eps > 0$, we have the almost sharp bound
\begin{equation*}
  \int_T^{2 T} \prod_{i = 1}^r \left| L_i(\tfrac{1}{2} + i t) \right|^{2 k_i} \, d t
  \ll T(\log T)^{k_1^2 + \dotsb + k_r^2 + \eps}.
\end{equation*}

\begin{theorem}[H.\ 2024+]
  Keep the assumptions above, but assume GRC instead of Hypothesis H.  Then
  \begin{equation*}
    T(\log T)^{k_1^2 + \dotsb + k_r^2} \ll \int_T^{2 T}
    \prod_{i = 1}^r
    \left| L_i(\tfrac{1}{2} + it) \right|^{2 k_i}
    \ll T(\log T)^{k_1^2 + \dotsb + k_r^2}.
  \end{equation*}
\end{theorem}

Let's discuss some applications.  For $\Re(s) > 1$ and $0 < \alpha \leq 1$, the Hurwitz zeta function in defined by
\begin{equation*}
  \zeta(s, \alpha) = \sum_{n \geq 0} \frac{1}{(n + \alpha)^s}.
\end{equation*}
Admits an analytic continuation to $\mathbb{C} - \{1\}$.  These are interesting functions because the analogue of the RH fails when $\alpha \neq 1/2, 1$, but $\zeta(s, \alpha)$ has many similarities with $\zeta$ either way.  From now on, let $\alpha = a/q$ ($\gcd(a, q) = 1, 1 \leq a \leq q$).  Then it's conjectured that
\begin{equation*}
  \int_T^{2 T} \left| \zeta(\tfrac{1}{2} + i t, \alpha) \right|^{2 k}
  \, d t
  \sim c_{k, \alpha} T(\log T)^{k^2}.
\end{equation*}
\begin{itemize}
\item Proven for $k = 1, 2$ unconditionally (Andersson, Rane, Sahay).  
\item Assuming RH for every Dirichlet $L$-function modulo $q$, Sahay (2023) proved for every positive integer $k$ and $\eps > 0$, one has
  \begin{equation*}
    T(\log T)^{k^2} \ll \int_T^{2 T}
    \left| \zeta(\tfrac{1}{2} + i t, \alpha) \right|^{2 k} \, d t
    \ll
    T(\log T)^{k^2 + \eps}.
  \end{equation*}
\end{itemize}
\begin{theorem}[H.\ 2024+]
  Let $\alpha = a/q$, where $\gcd(a, q) = 1$, $1 \leq a \leq q$.  Assume RH for all Dirichlet $L$-functions modulo $q$.  Then for all $k > 1/2$, we have
  \begin{equation*}
    T(\log T)^{k^2} \ll \int_T^{2 T}
    \left| \zeta(\tfrac{1}{2} + i t, \alpha) \right|^{2 k} \, d t
    \ll
    T(\log T)^{k^2}.
  \end{equation*}
\end{theorem}
\begin{proof}[Proof idea for upper bound, $k$ an integer]
  By orthogonality of Dirichlet characters, we have
  \begin{equation*}
    \zeta(\tfrac{1}{2} + i t, \tfrac{a}{q})
    =
    \frac{q    ^{1/2 + i t}}{\varphi(q)}
    \sum_{\chi(q)} \overline{\chi(a)}
    L(\tfrac{1}{2} + i t, \chi).
  \end{equation*}
  Taking $k$-th power and using the multnomial theorem, we get a sum over different products of Dirichlet $L$-functions with different powers.  Taking the second moment, we have to compute (up to order) many terms of the form (which we know from our main theorem)
  \begin{equation*}
    \prod_{i = 1}^r
    \left| L_i(\tfrac{1}{2} + it) \right|^{2 k_i}.
  \end{equation*}
\end{proof}

Another application concerns large deviations.  Selberg CLT for several $L$-functions at once:
\begin{equation*}
  \frac{1}{T} \vol \left( t \in[T, 2 T] :
    \frac{\log \left| L_j(\tfrac{1}{2} + i t) \right|}{\sqrt{\tfrac{1}{2} \log \log T}} \geq V_j, j = 1, \dotsc, r\right)
  \sim \prod_{j = 1}^{r}
  \int_{V_j}^\infty e^{- x^2 / 2} \, d x
\end{equation*}
for fixed $V_j$.  Controls typical fluctuations.  Conditional result (Selberg + Bombieri--Hejhal), requiring a strong zero density estimate.
\begin{theorem}[H.\ 2024+]
  Assume GRH and GRC.  Let $V_j \asymp \sqrt{\log \log T}$.  Then
  \begin{equation*}
    \frac{1}{T}
    \log \left( t \in[T, 2 T] : \frac{\log \lvert L_j(\tfrac{1}{2} + i t) \rvert}{ \sqrt{\log \log T}} \geq V_j, 1 \leq j \leq r \right)
    \asymp \exp \left( -(1 + o (1))
      \frac{V_1^2 + \dotsb + V_r^2}{2}\right).
  \end{equation*}
\end{theorem}
\begin{proof}[Sketch of proof]
  The large deviations $V_j \asymp \sqrt{\log \log T}$ are controlled by the moments.  Let $V_j = C_j \sqrt{\log \log T}$.  Then
  \begin{align*}
    \frac{1}{T}
    \vol \left( t \in[T, 2 T] :
      \frac{\log \lvert L_j(\tfrac{1}{2} + i t) \rvert}{ \sqrt{\tfrac{1}{2} \log \log T}} \geq V_j, 1 \leq j \leq r\right)
    &=
    \frac{1}{T}
    \vol \left( t \in[T, 2 T] :
      \lvert L_j(\tfrac{1}{2} + i t) \rvert \geq(\log T)^{\sqrt{C_j^2 /2}}
      , 1 \leq j \leq r\right).
  \end{align*}
  Now apply Chebyshev, then our knowledge of moments, then choose $k_j = C_j / \sqrt{2}$ to get the desired bound.
\end{proof}

[Slide that says ``Do I have more time?''.]

Let's now discuss the proof strategy for moments.  The lower bound proof strategy is due to Heath--Brown.  For $\zeta$, the argument starts with
\begin{equation*}
  \int_T^{2 T}
  \left| \zeta(\tfrac{1}{2} + i t) \right|^{2 k} \, d t
  \geq
  \int_T^{2 T}
  \left| \sum_{n \leq T}
    \frac{d_k(n)}{n^{1/2 + i t}}\right|^2 \, d t
  - \int_T^{2 T}
  \left| \zeta(\tfrac{1}{2} + i t)
    - \sum_{n \leq T}
    \frac{d_k(n)}{n^{1/2 + i t}}\right|^2 \,d t.
\end{equation*}
First term on the RHS can be computed by Montgomery--Vaughan mean value theorem.  Second term is dealt with by means of a version of Hadamard three lines theorem, moving the line of integration to $\Re(s) = 5/4$, where the corresponding integral is small.  For the upper bound, lots of ideas are combined:
\begin{itemize}
\item Soundararajan and Chandee: on GRH, bounding $L$-functions on the critical line by just an Euler product (with no information from zeros)
\item Radziwill: integrating over certain subsets where our $L$-functions behave ``typically'', instead of the whole of $[T, 2 T]$.
\item Harper/Radziwill--Soundararajan: splitting up Dirichlet polynomial into many parts with progressively smaller variance.
\end{itemize}

\section{Sub-Weyl bound for GL(2) L-functions, \textnormal{\emph{Prahlad Sharma}}}

\subsection{Background}

Hardy--Littlewood (1921) (written down by Landau):
\begin{equation*}
  \zeta(\tfrac{1}{2} + i t) \ll_\eps t^{1/6 + \eps}, \quad t \geq 1.
\end{equation*}
Uses Weyl differencing technique to estimate $\sum_{n \sim N} n^{i t}$.  Since then, several improvements have been done to this bound, leading to many important techniques for estimating exponential sums.  We'll only mention some of them.  One is Bombieri--Iwaniec's bound, refined by Huxley (2005), giving
\begin{equation*}
  \zeta(\tfrac{1}{2} + i t) \ll_\eps t^{1/6 - 13/1230 + \eps}, \quad t \geq 1.
\end{equation*}
The state of the art is due to Bourgain (2017), who obtained
\begin{equation*}
  \zeta(\tfrac{1}{2} + i t) \ll_\eps t^{1/6 - 1/84 + \eps}, \qquad t \geq 1,
\end{equation*}
using decoupling method from harmonic analysis.

For general automorphic $L$-functions, one has the convexity bound
(Phragmen--Lindel\"{o}f convexity principle plus functional equation)
\begin{equation*}
  L(\tfrac{1}{2} + i t, F) \ll_\eps \mathfrak{q}(t, F)^{1/4 + \eps},
\end{equation*}
while the analogue of the Weyl bound is
\begin{equation*}
  L(\tfrac{1}{2} + i t, F) \ll_\eps \mathfrak{q}(t, F)^{1/6 + \eps}.
\end{equation*}
A ``sub--Weyl'' bound is an improvement on the $1/6$ exponent.

There are important applications of subconvexity bounds.
\begin{itemize}
\item A.\ Good (1982): Weyl bound for $f$ holomorphic of level $1$,

  \begin{equation*}
    L(\tfrac{1}{2} + i t, f) \ll_\eps(t^2)^{1/6 + \eps}.
  \end{equation*}
\item K.\ Aggarwal and S.\ Singh (2017): Weyl bound for $\GL(2)$ $t$-aspect, using the ``$\GL(2)$ delta method''.
\item K.\ Aggarwal (2018): Weyl bound for $\GL(2)$ $t$-aspect using the ``trivial delta method''.
\item V.\ Blomer, S.\ Jana, P.\ Nelson (2021): Weyl bound for $\GL(2) \times \GL(2) \times \GL(2)$ using the period integral approach.
\end{itemize}
In particular, we haven't yet seen a sub-Weyl bound for $\GL(2)$.

\subsection{Main result}

Using a further refinement of the trivial delta method, we obtain:
\begin{theorem}[R.\ Holowinsky, R.\ Munshi, S., J.\ Streipel]
  Let $f$ be a $\SL_2(\mathbb{Z})$ form (cuspidal or Eisenstein).  We have
  \begin{equation*}
    L(\tfrac{1}{2} + i t, f) \ll_{\eps, f}
    (t^2)^{1/6 - 1/96 + \eps}.
  \end{equation*}
\end{theorem}
Some remarks:
\begin{itemize}
\item Taking $f$ to be Eisenstein, we get a sub-Weyl bound for zeta, between Bourgain's and Bombieri--Iwaniec's:
  \begin{equation*}
    \frac{1}{6} - \frac{1}{84}
    < \frac{1}{6} - \frac{1}{96}
    < \frac{1}{6} - \frac{1}{168}.
  \end{equation*}
\item The limiting bound of the method is $\zeta(\tfrac{1}{2} + i t) \ll_\eps t^{1/8 + \eps}$.  Speaker thinks that with more effort, they could improve upon Bourgain's bound.
\item The method is soft, using no exponent pair estimates.
\end{itemize}

\subsection{Proof sketch}

\subsubsection{Setup}

By the approximate functional equation,
\begin{equation*}
  L(\tfrac{1}{2} + i t, f) \approx t^{-1/2} \sum_{n \asymp t}
  \lambda(n) n^{- it}.
\end{equation*}
Trivial estimation gives the convexity bound
\begin{equation*}
  L(\tfrac{1}{2} + i t, f) \ll t^{1/2}.
\end{equation*}
Need to save $t^{1/6}$ plus something more in
\begin{equation*}
  S := \sum_{n \asymp t} \lambda(n) n^{- i t}
\end{equation*}
to get sub--Weyl.

\subsubsection{Trivial delta}

Let's now discuss the $\delta$-symbol.  Let $\delta : \mathbb{Z} \rightarrow \{0, 1\}$ denote the characteristic function of $0$:
\begin{equation*}
  \delta(k) =
  \begin{cases}
    1 & \text{ if } k = 0, \\
    0 & \text{ otherwise.}
  \end{cases}
\end{equation*}
Then
\begin{equation*}
  S = \sum_{m \asymp t} \sum_{n \asymp t}
  \lambda(m) \delta(m - n) n^{- it}.
\end{equation*}
The \emph{trivial delta symbol} comes from observing that for $p > t$, we have
\begin{equation*}
  \delta(m - n) = \frac{1}{p}
  \sum_{b(p)} e \left( \frac{b(m - n)}{p} \right).
\end{equation*}

To optimize the delta symbol, we employ ``conductor lowering''.  We first use
\begin{equation*}
  m - n \ll t / K  \leftrightsquigarrow
  \frac{1}{K}
  \int_{v \sim K}
  (m / n)^{i v} \, d v
\end{equation*}
for some $K < t$, and then detect the smaller equation
\begin{equation*}
  m - n = 0.
\end{equation*}
This reduces the number of variables we are introducing (saving from the $v$-integral) and reduces the conductor size arising from the delta symbol expansion.

We are led to proceed with
\begin{equation*}
  S = \frac{1}{K}
  \sum_{m \asymp t} \sum_{n \asymp t}
  \lambda(m)
  \delta(m - n)
  n^{- i t}
  \int_{v \asymp K}
  (m / n)^{i v} \, d v.
\end{equation*}
We choose a set of primes $\mathcal{P}$ of size $P$ such that
\begin{equation*}
  P > t /K.
\end{equation*}
Then
\begin{equation*}
  \delta(m - n) = \frac{1}{\lvert \mathcal{P} \rvert}
  \sum_{p \in \mathcal{P}}
  \frac{1}{p}
  \sum_{b(p)}
  e \left( \frac{b(m - n)}{p} \right).
\end{equation*}
Substituting, we arrive at
\begin{equation*}
  S = \frac{1}{K}
  \int_{v \asymp K}
  \frac{1}{ P}
  \sum_{p \in \mathcal{P}}
  \frac{1}{p} \sum_{b(p)}
  \sum_{m \asymp t}
  \lambda(m)
  e \left( \frac{b m}{p} \right)
  m^{i v}
  \sum_{n \asymp t}
  e \left( \frac{- b n}{p} \right)
  n^{- i(t + v)}
  \, d v.
\end{equation*}
Dualizing (Voronoi and Poisson), get
\begin{equation*}
  S = \frac{t^{3/2}}{K^{3/2} P^3}
  \sum_{p \in \mathcal{P}}
  \sum_{m \asymp p^2 K^2 / t}
  \sum_{n \asymp p}
  \lambda(m)
  e \left( \frac{\bar{n}m}{p} \right)
  (n/p)^{i t}
  e^{i \phi(m, n, p)}.
\end{equation*}
Trivially,
\begin{equation*}
  S \ll t^{1/2} K^{1/2} P = t \cdot \frac{K^{1/2} P}{t^{1/2}}.
\end{equation*}
We are $\frac{K^{1/2} P}{t^{1/2}}$ away from subconvexity.

Everything so far has been as in previous works.
\begin{equation*}
  \left( \sum_{m \asymp P^2 K^2 / t} \left| \lambda(m) \right|^2 \right)^{1/2}
  \left( \sum_{m \asymp P^2 K^2 / t}
    \left| \sum_{p \in \mathcal{P}} \sum_{n \asymp P}
      e \left( \frac{\bar{n} m}{ p} \right)
      (n /p)^{i t} e^{i \phi(m, n, p)}\right|^2\right)^{1/2}.
\end{equation*}
\begin{itemize}
\item Diagonal contributes $P^2 \cdot P^2  K^2 / t$.
\item Off-diagonal contributes $P^4 \cdot \sqrt{K}$.
\item Therefore
  \begin{equation*}
    S \ll t^{1/2} K^{1/2} + \frac{t}{K^{1/4}}.
  \end{equation*}
  With the optimal choice $K = t^{2/3}$, we get
  \begin{equation*}
    S \ll t^{1 - 1/6}.
  \end{equation*}
\end{itemize}

\subsubsection{Going beyond Weyl}

We go back to the step just before, and use Diophantine approximation.  Write
\begin{equation*}
  S = \frac{t^{3/2}}{K^{3/2} P^3}
  \sum_{p \in \mathcal{P}}
  \sum_{n \asymp p}
  \left( \frac{n}{p} \right)^{i t}
  \sum_{m \asymp p^2 K^2 / t}
  \lambda(m)
  e \left( \frac{\bar{n} m}{ p} \right)
  e^{i \phi(m, n, p)}.
\end{equation*}
Fix $Q \geq 1$.  For each $(n, p)$, choose $(a, q)$, with $q \leq Q$, such that
\begin{equation*}
  \frac{\bar{n}}{p} = \frac{a}{q} + \beta, \quad
  \beta \leq \frac{1}{q Q}.
\end{equation*}
Then
\begin{equation*}
  e \left( \frac{\bar{n}m}{p} \right)
  = e \left( \frac{a m}{q} \right)
  e(m \beta).
\end{equation*}
Choose $Q$ such that
\begin{equation*}
  m \beta \asymp K \implies Q \asymp P \sqrt{\frac{K}{t}},
\end{equation*}
which is smaller than $P$, although the analytic conductor remains the same.  The $\GL(2)$ sum now becomes
\begin{equation*}
  \sum_{m} \lambda(m) e \left( \frac{\bar{n} m}{p} \right)
  e^{i \phi(m, n, p)}
  =
  \sum_{m } \lambda(m)
  e \left( \frac{a m}{q} \right)
  e(m \beta)
  e^{i \phi(m,n , p)},
\end{equation*}
where the conductors are $P^2 \cdot K^2$ for the LHS and $Q^2 \cdot K^2$ for the RHS.  After Voronoi once more, we obtain
\begin{equation*}
  S = \frac{t}{K P^2} \sum_{p \in \mathcal{P}} \sum_{n \asymp p}
  (n /p)^{i t}
  \sum_{m \asymp K}
  \lambda(m)
  e \left( \frac{\bar{a} m}{q} \right) \mathcal{J}(m, n, p).
\end{equation*}
Trivially at this stage, we get
\begin{equation*}
  S \ll t,
\end{equation*}
which gives convexity.

We observe that
\begin{equation*}
  e \left( \frac{\bar{a} m}{q} \right)
\end{equation*}
is still additive with respect to $m$.

We now apply Cauchty--Schwarz followed by Poisson summation, giving
\begin{equation*}
  \left( \sum_{m \asymp K}
    \left| \lambda(m) \right|^2\right)^{1/2}
  \left( \sum_{m \asymp K}
    \left| \sum_{p \in \mathcal{P}} \sum_{n \asymp P}
      e \left( \frac{\bar{a} m}{q} \right)
      (n/p)^{i t}
      \mathcal{J}(m, n, p)\right|^2\right)^{1/2}.
\end{equation*}
The diagonal contributes $P^2 \cdot K$.  The off-diagonal gives $P^4 \cdot \sqrt{K}$.  Therefore
\begin{equation*}
  S \ll \frac{t}{P} + \frac{t}{ K^{1/4}}.
\end{equation*}
Here the diagonal is negligibly small, while the off-diagonal is the same as earlier.

With the best possible choice $K = t^{1 - \eps}$, we get
\begin{equation*}
  S \ll t^{1 - 1/4},
\end{equation*}
which gives
\begin{equation*}
  L(\tfrac{1}{2} + i t, f) \ll t^{1/4 + \eps},
\end{equation*}
\begin{equation*}
  \zeta(\tfrac{1}{2} + i t) \ll t^{1/8 + \eps}.
\end{equation*}
Some degeneracies arise in the non-generic cases $q \leq Q t^{- \delta}$, $\beta \ll \frac{t^{- \delta}}{Q^2}$.  These are relatively simpler when $K \ll t^{3/4}$.  Thus, for now, we obtain
\begin{equation*}
  S \ll \frac{t}{(t^{3/4})^{1/4}} = t^{1 - 3/16},
\end{equation*}
that is to say,
\begin{equation*}
  L(\tfrac{1}{2} + i t, f)
  \ll t^{1/3 - 1/48 + \eps},
\end{equation*}
\begin{equation*}
  \zeta(\tfrac{1}{2} + i t, f)
  \ll t^{1/6 - 1/96 + \eps}.
\end{equation*}

\section{Hybrid subconvex bounds for selfdual $GL_3$ L-functions via $\GL_3 \times \mathrm{GL}_2$ vs. $\GL_4 \times \GL_1$ spectral reciprocity, \textnormal{\emph{Soumendra Ganguly}}}

Let's start by discussing Motohashi's formula (1997).  It concerns
\begin{equation*}
  \sum_{f \in \mathcal{F}_1}
  \frac{L(\tfrac{1}{2} , f)^3}{ L(1, \ad f)}
  h(t_f) + \dotsb
  = \text{main term }
  +
  \int_{\mathbb{R}}
  \left| \zeta(\tfrac{1}{2} + i t) \right|^4 \mathcal{H}(t) \, d t.
\end{equation*}
Here $f$ runs over an orthonormal basis for the space of Hecke--Maass forms for $\SL(2,\mathbb{Z})$, $h$ is some nice enough function, and $\mathcal{H}$ is some integral transform of $h$.

[Ed: Motohashi's formula goes the other way]

Kwan (2021), Humphries--Khan (2022), Wu (2023): various formulas of the shape
\begin{equation*}
  \sum_f \frac{L(\tfrac{1}{2}, F \otimes f)}{L(1, \ad f)} h(t_f)
  + \dotsb
  = \text{main term } +
  \int_{\mathbb{R}} L(\tfrac{1}{2} + i t, F)
  \zeta(\tfrac{1}{2} - i t) \mathcal{H}(t) \, d t.
\end{equation*}
These recover Motohashi's formula by specializing to $F$ an Eisenstein series.

We focus on the approach of Humphries--Khan, which works only $F$ self-dual, but works for a broad range of test functions.

We consider generalizations to higher levels.  This is joint work with Humphries and Lin.  Preprint: \cite{2024arXiv2408.00596}.  We consider levels $q = q_1 q_2$, where $(q_1, q_2) = 1$, and take for $\chi_1$ a primitive Dirichlet character modulo $q_1$.

We consider the family
\begin{equation*}
  \mathcal{F}_2 = \left\{ g \otimes \chi_1 : g \in \mathcal{B}(q, \bar{\chi}_1^2) \right\}.
\end{equation*}
We show that for $F$ self-dual,
\begin{multline}\label{eq:cnojr4ohm4}
  \sum_{f \in \mathcal{F}_2} \frac{L(\tfrac{1}{2}, F \otimes f)}{L(1, \ad f)} h(t_f)
  + \dotsb
  \\
  = \text{main } +
  \frac{q_2^{1/2}}{ q_1}
  \sum_{\psi_1(q_1)}
  \int_{\mathbb{R}}
  L(\tfrac{1}{2} + i t, F \otimes \psi_1)
  L(\tfrac{1}{2} - i t, \overline{\psi_1})
  \mathcal{Z}_{\chi_1}(\psi_1, t)
  \mathcal{H}(t) \, d t
  + \dotsb.
\end{multline}
Here $\dotsb$ includes contributions from lower level, holomorphic forms, and the continuous spectrum.  Also,
\begin{equation*}
  \mathcal{Z}_{\chi_1}(\psi_1, t) \approx g(\chi_1, \psi_1),
\end{equation*}
which appears in the work of Conrey--Iwaniec and Petrow--Young.

As an application of this, we show the following subconvexity bounds:
\begin{equation*}
  L(\tfrac{1}{2} + i t, F \otimes \chi)
  \ll_{F, \eps}
  \left( q(\lvert t \rvert + 1)\right)^{3/5 + \eps}
  \left(
    1 +
    \frac{q^{2/5} }{q_1^{1/2}(\lvert t \rvert + 1)^{1/10}}
    +
    \frac{q_1^{1/8}}{(q(\lvert t \rvert + 1))^{1/10}}
  \right),
\end{equation*}
where $F$ is fixed and self-dual, while $\chi$ is a primitive Dirichlet character of conductor $q$ (any positive integer).

Also, for $f \in \mathcal{B}^\ast(q^2, \chi_0)$ such that $f \otimes \bar{\chi}$ has level dividing $q$, we have
\begin{equation*}
  L(\tfrac{1}{2}, F \otimes f)
  \ll_{F, \eps}
  \left( q(\lvert t_f \rvert + 1) \right)^{6/5 + \eps}
  \left( 1 + \frac{q^4}{q_1(\lvert t_f \rvert + 1)^{1/5}}
    + \frac{q_1^{1/4}}{(q(\lvert t \rvert + 1))^{1/5}}\right).
\end{equation*}

These bounds are hybrid in the aspects of $q$, $t_f$ and $t$.  Taking $q_1 = q$, we get a bound of
\begin{equation*}
  q^{5/8 + \eps}(\lvert t \rvert + 1)^{3/5 + \eps}.
\end{equation*}
If we further assume
\begin{equation*}
  q^{4/5}(\lvert t \rvert + 1)^{- 1/5} \ll q_1 \ll q^{4/5}(\lvert t \rvert + 1)^{4/5},
\end{equation*}
then the bound becomes
\begin{equation*}
  (q(\lvert t \rvert + 1))^{3/5 + \eps}.
\end{equation*}

For the proof of the spectral reciprocity formula \ref{eq:cnojr4ohm4}, we replace $1/2$ with $s$ of large enough real part, then open the Dirichlet series and apply Kuznetsov.  The diagonal term gives the main term.  For the off-diagonal term, we open the Kloosterman sums and apply Voronoi.  After some manipulations, we find two Dirichlet series, which eventually yields the $L$-factors on the right hand side, together with the integral transform.

\section{Subconvex bounds for $\SO(n+1) \times \SO(n)$, \textnormal{\emph{Blanca Gil Rosell}}}
Aim is to find an analogue of subconvex bounds for unitary groups $\U_{n + 1} \times \U_n$, but for orthogonal groups $\SO_{n + 1} \times \SO_n$.

Branching coefficients: for automorphic forms $\varphi \in \pi$ on $G$ and $\varphi_H \in \pi_H$ on $H$, then
\begin{equation*}
  \left| \int_{H(\mathbb{Q}) \backslash H(\mathbb{A})}
    \varphi \cdot \varphi_H\right|^2
  = \mathcal{L}(\pi \times \pi_H, \tfrac{1}{2}) \times (\text{local integrals}).
\end{equation*}
Conjecture of Ichino--Ikeda is that $\mathcal{L}(\pi \times \pi_H, \tfrac{1}{2})$ is essentially $L(\pi \times \pi_H, \tfrac{1}{2})$.  Essentially known for unitary groups, widely open for orthogonal groups.

For $f \in G(\mathbb{A})$,
\begin{equation*}
  \sum_\pi \sum_{\varphi \in \mathcal{B}(\pi)}
  \rho(f) \varphi(x) \overline{\varphi(y)}
  = \sum_{\gamma \in G(\mathbb{Q})}
  f(x^{-1} \gamma y).
\end{equation*}
Choose $f$ and $\varphi_H$ ``microlocalized'' (Nelson--Venkatesh) and use the ``amplification method'' (Duke--Friedlander--Iwaniec, Iwaniec--Sarnak),
\begin{equation*}
  \sum_\pi L(\pi \times \pi_H, \tfrac{1}{2}) w_{f, \varphi_H}(\pi)
  = \int_{x, y \in H(\mathbb{Q}) \backslash H(\mathbb{A})}
  \varphi_H(x) \overline{\varphi_H(y)}
  \sum_{\gamma \in G(\mathbb{Q})} f(x^{-1} \gamma y) \, d x \, d y.
\end{equation*}

Reduces to a transversality problem that we now describe.  Let $F$ be af ield, $V =  V_H \oplus F_e$ a quadratic space.
\begin{itemize}
\item $(G, H) =(\SO_{n + 1}, \SO_n)$,
\item $\mathfrak{g}, \mathfrak{h}$: Lie algebras,
\item $\tau \in \mathfrak{g} \rightsquigarrow \tau_H \in \mathfrak{h}$: upper left-block, i.e.,
  \begin{equation*}
    \tau =
    \begin{pmatrix}
      \tau_H      & \ast \\
      \ast                  & \ast \\
    \end{pmatrix}.
  \end{equation*}
  Nelson--Venkatesh: choose $\tau$ (resp.\ $\tau_H$) to be microlocal parameters attached to $\pi$ (resp.\ $\pi_H$).  Roughly speaking, their eigenvalues are the parameters of the representations.
\end{itemize}

Call $\tau$ \emph{stable} if $\gcd(P_\tau, P_{\tau_H}) = 1$.

Bounds for geometric side of relative trace formula boil reduce, via Cauchy--Schwarz, to solving the \emph{transversality problem}, which is to show that for $a \in G - H Z$, there exists $z \in H_{\tau_H}$ such that $a z \notin H G_\tau$.
\begin{itemize}
\item Marshall: depth aspect, $\tau$ and $\tau_H$ semisimple
\item Nelson: depth aspect
\item Nelson--Hu: horizontal aspects
\end{itemize}

The Lie algebra problem is to show that for each $x \in \mathfrak{g}_\tau - \mathfrak{z}$, there exists $z \in \mathfrak{h}_{\tau_H}$ such that $[x,[z,\tau]] \notin [\mathfrak{h},\tau]$.

\begin{conjecture}[Nelson 2024, private communication]
  The Lie algebra problem can probably be solved for $(\SO_{n+1}, \SO_n)$, and would be a good first PhD project.
\end{conjecture}

\begin{example}
  The conjecture is false for $(\SO_6, \SO_5)$, by explicit numerical computation.
\end{example}

So for the orthogonal case, we need to address the transversality problem more directly.  Experiments with this thus far, some experiments with Gr\"{o}bner bases.

\section{Subconvexity for Rankin--Selberg $L$-functions, \textnormal{\emph{Sumit Kumar}}}

Let's start with the setup.
\begin{itemize}
\item $f$: Hecke cusp form for $\Gamma_0(p)$ and nebentypus $\chi$.
\item $g$: Hecke cusp form for $\SL_2(\mathbb{Z})$ and trivial nebentypus.
\item Consider the Dirichlet series
  \begin{equation*}
    L(s, f \otimes g) = L(2 s, \chi)
    \sum_{n = 1}^\infty \frac{\lambda_f(n) \lambda_g(n)}{ n^s},
    \quad
    \Re(s) > 1,
  \end{equation*}
  where the $\lambda(n)$ are the normalized Fourier coefficients.  It satisfies a functional equation and has meromorphic continuation to $\mathbb{C}$.
\end{itemize}

We want to estimate $L(\tfrac{1}{2}, f \otimes g)$ as $p \rightarrow \infty$ with $g$ fixed.

Functional equation yields the following approximateion by a finite Dirichlet polynomial, of length roughly the square-root of the conductor:
\begin{equation*}
  L(\tfrac{1}{2}, f \otimes g) \approx \sum_{n \ll p^{1 + \eps}}
  \frac{\lambda_f(n) \lambda_g(n)}{ n^{1/2}} W(n).
\end{equation*}
Here $W$ is a smooth function satisfying some nice decay conditions.  Using this approximation, one can show that
\begin{equation*}
  L(\tfrac{1}{2}, f \otimes g) \ll_{g, \eps} p^{1/2 + \eps} =
  (p^2)^{1/4 + \eps},
\end{equation*}
the convexity bound.  Our aim is to show that
\begin{equation}\label{eq:cnojsamf9d}
  L(\tfrac{1}{2}, f \otimes g) \ll_g(p^2)^{1/4 - \delta} \text{ for some } \delta > 0.
\end{equation}
\begin{itemize}
\item The first estimate of the form \eqref{eq:cnojsamf9d} was shown by Kowalski--Michel--Vanderkam \cite{KMV02} when $f$ is holomorphic and $\chi$ has conductor $\ll p^{1/2}$.  
\item Michel \cite{Mi04} removed the the latter condition and obtained \eqref{eq:cnojsamf9d} with $\delta = 1/2114$ for $g$ holomorphic and $\chi$ non-trivial.  
\item Harcos--Michel \cite{MR2207235} proved \eqref{eq:cnojsamf9d} with $\delta = 1/2826$ for $g$ holomorphic/Maass cusp forms and non-trivial $\chi$.
\item Michel--Venkatesh \cite{michel-2009} generalized all the above results (uniform in all aspects of $f$).
\end{itemize}
Michel and Michel--Harcos use spectral theory of automorphic forms.  Michel--Venkatesh use period integral approach (geometric arguments).  In this talk, we will use Munshi's delta symbol approach to prove our theorem.

\begin{theorem}[K--Aggarwal--Kwan--Leung--Li--Young]
  Let $f$ be a Hecke cusp form of prime level $p$ and nebentypus $\chi$, and let $g$ be a fixed $\GL(2)$ form.  Then we have
  \begin{equation*}
    L(\tfrac{1}{2}, f \otimes g) \ll_{g, \eps}(p^2)^{1/4 - 1/1100 + \eps}.
  \end{equation*}
  For $\chi$ trivial, we have
  \begin{equation*}
    L(\tfrac{1}{2}, f \otimes g) \ll_{g, \eps} \dotsb.
  \end{equation*}
\end{theorem}
We use the $\delta$-method along with amplification.  Key input is DFI estimates of bilinear sum
\begin{equation*}
  \sum_n a(n) \sum_{m} b(m)
  e \left( \frac{c \bar{m}}{n} \right).
\end{equation*}

Let's now say something about the $\delta$-method.  For $n \in \mathbb{Z}$, $\delta(n) = 1$ if $n = 0$ and $\delta(n) = 0$ otherwise.  We think $\delta(n) = \int_0^1 e(n x) \, d x$.  Delta method is a variant of classical ricle method with no ``minor arc'' contributions.  We use the Fourier expansion (DFI) for $\lvert n \rvert \leq N$,
\begin{equation*}
  \delta(n)
  = \frac{1}{Q}
  \sum_{q \leq Q}
  \frac{1}{q}
  \sum_{a(q)^*}
  e \left( \frac{a n}{q} \right)
  \int_{\mathbb{R}}
  g(q, x)
  e \left( \frac{n x}{q Q} \right)
  \, d x,
\end{equation*}
where $g(q, x)$ is a smooth function and $Q \leq \sqrt{N}$.

Let's now give a sketch of the proof.  On applying the approximate functional equation, we see that
\begin{equation*}
  L(\tfrac{1}{2} , f \otimes g) \ll \sup_{N \ll p} \frac{S(N)}{\sqrt{N}},
\end{equation*}
where
\begin{equation*}
  S(N) = \sum_n \lambda_f(n) \lambda_g(n) W(n / N).
\end{equation*}
Here $W$ is a fixed bump function, taking the value one on some interval.  The goal is to show that
\begin{equation*}
  S(N) \ll N^{1 - \delta} \qquad \text{ for } N \approx p, \quad \text{ some } \delta > 0.
\end{equation*}
To that end, we write
\begin{align*}
  S(N) &=
         \sum_n \lambda_f(n) \lambda_g(n) W(n/N)
  \\
       &=
         \sum_n \sum_m \lambda_f(n) \lambda_g(m)
         W(n/N) W_1(m/N)
         \delta(m - n)
  \\
       &=
         \frac{1}{Q}
         \sum_q \int_{\mathbb{R}}
         \frac{g(q, x)}{q}
         \sum_{a(q)^\ast}
         S_1(a, q, x) S_2(a, q, x) \, d x.
\end{align*}
Here $Q = \sqrt{N}$.  (To be precise, we have suppressed the amplifier from our sketch of the argument, but that is used to treat the diagonal.)  Also
\begin{equation*}
  S_1(a, q, x) = \sum_{m} \lambda_f(m) e \left( \frac{a m}{q} \right)
  W_1(m, q, x),
\end{equation*}
\begin{equation*}
  S_2(a, q, x) = \sum_n \lambda_g(n) e \left( \frac{- a n}{q} \right)
  W_2(n, q, x).
\end{equation*}
So what we have done is to separate out the coefficients $\lambda_f$ and $\lambda_g$.  The next step is to analyze $S_1$ and $S_2$ using Voronoi summation.  In the generic case $q \sim Q = \sqrt{p}$, we see that
\begin{equation*}
  S_1(\dotsb) \rightsquigarrow \frac{N \chi(- q)}{Q \sqrt{p}}
  \sum_{m \asymp p}
  \lambda_f(m)
  e \left( \frac{- \overline{a p} m}{q} \right).
\end{equation*}
(This step doesn't lose or gain anything.)  Similarly,
\begin{equation*}
  S_2(\dotsb) \rightsquigarrow \frac{N}{Q} \sum_{m \asymp 1} \lambda_f(n) e \left( \frac{\bar{a} m}{q} \right).
\end{equation*}
(Here we have gained a lot.)  Combining everything together so far, we arrive at
\begin{equation*}
  S(N) \rightsquigarrow \frac{1}{\sqrt{p}}
  \sum_{m \asymp p} \lambda_f(m)
  \sum_q \chi(- q) \sum_{n \asymp 1} \lambda_f(n) c_q(p n - m).
\end{equation*}
Here
\begin{equation*}
  c_q(n) = \sum_{a (q)}^\ast e \left( \frac{a n}{q} \right)
\end{equation*}
is the Ramanujan sum.  Thus far, we have ``shown'' that
\begin{equation*}
  S(N) \ll \sqrt{p} Q \sqrt{Q} = N p^{1/4},
\end{equation*}
assuming square-root savings in $c_q(\dotsb)$.  (Of course, we get more saving than that for $c_q$, but we're doing a bunch of averaging.)  Next, we apply Cuachy to the $m$-sum to obtain
\begin{equation*}
  \lvert S(N) \rvert^2 \ll \sum_m
  \left|
    \sum_q
    \chi(- q)
    \sum_{n \asymp 1}
    \lambda_f(n) c_q(p n - m)\right|^2.
\end{equation*}
We now open the square and apply Poisson summation to the $m$-sum:
\begin{multline*}
  \sum_{m \asymp p} c_{q_1}(p n_1 - m)
  \overline{  c_{q_2}(p n_2 - m)}
  \\
  \rightsquigarrow p
  \sqrt{q_1 q_2} \delta(q_1 = q_2, n_1 = n_2)
  + \sum_{m \asymp 1}
  e \left( \frac{\overline{q_2} m p n_1}{q_1} +
    \frac{\overline{q_1} m p n_2}{ q_2}\right).
\end{multline*}
The remaining sum over $q_1$ and $q_2$ is given by
\begin{equation*}
  B = \sum_{q_1 \asymp Q}
  \sum_{q_2 \asymp Q} \chi(q_1) \overline{\chi}(q_2) e \left( \frac{a \overline{q_1}}{ q_2} \right),
  \quad a = p m(n_2 - n_1).
\end{equation*}
Here we apply DFI estimate $B \ll Q^{2 - 1/48}$ \cite{MR1437494}.  The proof follows.


\section{Archimedean Distinguished Representations and Exceptional Poles, \textnormal{\emph{
      Akash Yadav}}}

Preprint: \cite{2024arXiv2401.09063}

Let $F$ be a non-archimedean local field, $E$ either $F \times F$ (split case) or a quadratic extension (inert case).  $\pi$: irreducible generic representation of $\GL_n(E)$.

\begin{definition}
  A representation $\pi$ is said to be $\GL_n(F)$-distinguished if $\Hom_{\GL_n(F)}(\pi,1) \neq 0$.
\end{definition}
This is related to Rankin--Selberg and Asai $L$-functions according as we are in the split or inert case.

\begin{itemize}
\item $\psi : E \rightarrow S^1$ fixed additive character
\item $\mathcal{W}(\pi, \psi)$ Whittaker model of $\pi$ 
\item $\mathcal{S} = \mathcal{S}(F^n)$: Schwartz space
\item $e_n =(0, 0, \dotsc, 1) \in F^n$
\item $N_n \leq \GL_n$: subgroup of unipotent upper-triangular matrices
\end{itemize}
We consider the family of Rankin--Selberg integrals in the split case (resp.\ Flicker integrals in the inert case)
\begin{equation*}
  \mathcal{J}(\pi) = \left\{ I(s, W, \Phi) : W \in \mathcal{W}(\pi, \psi), \Phi \in \mathcal{S} \right\},
\end{equation*}
\begin{equation*}
  I(s, W, \Phi) := \int_{N_n(F) \backslash \GL_n(F)} W(g)
  \Phi(e_n g)
  \lvert \det g \rvert^s_F \, d g.
\end{equation*}

Jacquet, Piatetski-Shapiro and Shalika (resp.\ Beuzart--Plessis)showed that these integrals converge for $\Re(s)$ large enough and are holomorphic multiples of the Rankin--Selberg (resp.\ Asai) $L$-functions.  We have a filtration $\mathcal{S} = \mathcal{S}^0 \supset \dotsb \supset \mathcal{S} ^m$, where $\mathcal{S}^m$ is the set of vanishing functions to order at least $m$ at $0$.  At a pole $s_0$ of maximal order for $\mathcal{J}(\pi)$, each integral as above has a Laurent expansion
\begin{equation}\label{eq:cnojsuo68y}
  \frac{B_{s_0}(W, \Phi)}{(s - s_0)^d} + \dotsb,
\end{equation}
where $B_{s_0}$ is bilinear.  ``Exceptional pole of level $d$ at $s_0$'' means $B_{s_0} \neq 0$ [I think; slides were pretty fast].

\begin{theorem}[Y., 2024]\label{theorem:cnojsu3jzl}
  Let $\pi$ be an irreducible generic representation of $\GL_n(E)$, assumed to be \emph{nearly tempered} in the inert case.  Then $\pi$ is distinguished if and only if the family of integrals $\mathcal{J}(\pi)$ has an exceptional pole of level zero at $0$.
\end{theorem}

Here we say that $\pi$ is \emph{nearly tempered} if
\begin{equation*}
  \pi \cong   \left( \pi_1 \otimes \lvert \det \rvert_E^{r_1} \right)
  \times  \dotsb \times
  \left( \pi_k \otimes \lvert \det \rvert_E^{r_k} \right),
\end{equation*}
where each $\gamma_i \in \mathbb{R}$ or $\lvert \gamma_i \rvert < 1/4$.

In 2010, Matringe established for $E/F$ quadratic non-archimedean that $\pi$ is distinguished iff its Asai $L$-function has san exceptional pole at $0$.  In the non-archimedean setting, the Asai $L$-function serves as the greatest common divisor of the family of Flicker integrals, ensuring that the quotient of these integrals and the Asai $L$-function is nonvanishing.  But in the archimedean case, the nonvanishing of this quotient is known only for nearly tempered representations (in the inert case).

We'll first prove the ``if'' direction of this theorem.  Suppose $s = s_0$ is a pole of maximal order $d$ for $\mathcal{J}(\pi)$, then we have an expansion as in \eqref{eq:cnojsuo68y}, where
\begin{equation*}
  B_{s_0}(g \cdot W, g \cdot \Phi)
  =
  \lvert \det \rvert^{s_0}  B_{s_0}(W, \Phi).
\end{equation*}
Note that $\mathcal{S}^1$ is a codimension $1$ subspace of $\mathcal{S}$, being the kernel of the evaluation at zero map $\mathcal{S} \rightarrow \mathbb{C}$.  Therefore, if $s = s_0$ is an exceptional pole of level $0$ for the family $\mathcal{J}(\pi)$, then $B_{s_0}$ is of the form
\begin{equation*}
  B_{s_0}(W, \Phi) = \lambda_{s_0}(W) \Phi(0),
\end{equation*}
where $\lambda_{s_0}$ is nonzero and satisfies
\begin{equation*}
  \lambda_{s_0}(g \cdot W) = \lvert \det g \rvert^{- s_0} \lambda_{s_0}(W).
\end{equation*}
Taking $s_0$, we see that $\lambda_0 \neq 0$, and so $\pi$ is $\GL_n(F)$-distinguished.

In the remainder of the talk, we focus on proving the ``only if'' direction.  We focus on the split case; the inert case will follow from a similar argument.

Let $W_F$ be the Weil group of $F$, either $\mathbb{C}^\times$ if $F = \mathbb{C}$ or $\mathbb{C}^\times \sqcup j \mathbb{C}^\times$ for $F = \mathbb{R}$, where $j^2 =  - 1$ and $j w j^{-1} = \bar{w}$.  To any continuous finite-dimensional $\phi : W_F \rightarrow \GL(V)$, Tate associates $L(s, \phi)$ and $\eps(s, \phi, \psi ')$.  Let $\pi = \pi_1 \otimes \pi_2$: irreducible representation of $\GL_n(E) = \GL_n(F) \times \GL_n(F)$.  Let $\phi_1, \phi_2$ denote the associated Langlands parameter, so that $\phi_1 \otimes \phi_2$ is the parameter giving rise to the Rankin--Selberg $L$-function.  In 1983 and 2010, Jacquet, Piatetski-Shapiro and Shalika resolved the analytic properties of Rankin--Selberg integrals.
\begin{theorem}[JPSS 1983, Jacquet 2010]
  For $\pi = \pi_1 \otimes \pi_@$, $W$ and $\Phi$ as above:
  \begin{enumerate}
  \item The local integrals $I(s, W, \Phi)$ converges for $\Re(s)$ large enough and extend meromorphically.
  \item We have the local functional equation.
  \item The normalized local integrals are holomorphic, and for any $s_0 \in \mathbb{C}$, do not vanish identically.
  \end{enumerate}
\end{theorem}

We turn to \emph{continuity of integrals}.  Let
\begin{equation*}
  P _n =
  \begin{pmatrix}
    \ast    & \ast & \dotsb & \ast \\
    \dotsb            & \dotsb & \dotsb & \ast \\
    \ast            & \ast & \ast & \ast \\
    0            & 0 & 0 & 1 \\
  \end{pmatrix}
\end{equation*}
denote the mirabolic subgroup of $\GL_n$.  For $W \in \mathcal{W}(\pi, \psi_n)$, consider the local Rankin--Selberg integrals
\begin{equation*}
  I(s, W, \Phi).
\end{equation*}

\begin{proposition}[Cogdell 2005, Chai 2015]
  The normalized local Rankin--Selberg integrals are continuous.
  \begin{equation*}
    \frac{I(s, W, \Phi)}{L(s, \pi, \mathrm{RS})}.
  \end{equation*}
\end{proposition}
\begin{proposition}[Beuzart-Plessis, 2001]
  The normalized Rankin--Selberg integrals don't vanish identically.  
\end{proposition}
\begin{proposition}[Ehud Moshe Baruch, 2003]
  For a distinguished representation $\pi$, any $P_n(F)$-invariant linear form on $\pi$ is actually $\GL_n(F)$-invariant.
\end{proposition}

We now give the proof of Theorem \ref{theorem:cnojsu3jzl}  Let $\pi = \pi_1 \otimes \pi_2$, with irreducible generic representations of $\GL_n(F)$.  We need to prove that if $\pi$ is distingiushed, then $s = 0$ is an exceptional pole of level $0$ for the family $\mathcal{J}(\pi)$.  For $\Re(s)$ sufficiently negative, consider the integral
\begin{equation*}
  I(1 - s, \tilde{W},\tilde{\Phi}) = \int_{N_n(F) \backslash \GL_n(F)}
  \tilde{W}(g) \hat{\Phi}(e_n g) \lvert \det g \rvert^{1 - s} \, d g.
\end{equation*}
We assume that $W$ is a pure tensor of $K_n$-finite components.  Then we can write
\begin{equation*}
  \tilde{W}_1(g k) = \sum_l h_l(k) \tilde{W_l} '(g),
\end{equation*}
and similarly for $W_2$, with $g_j$ in place of $h_l$.

Now we set $f_i = h_l g j$, and form $\tilde{W}(g k) = \sum_i f_i(k) \tilde{W}_i(g)$.  Since $\pi$ is distingiushed, its central character is distinguished.

Using the Iwasawa decomposition, we get
\begin{equation*}
  I(1 - s, \tilde{W}, \hat{\Phi})
  = \dotsb,
\end{equation*}
then divide to get:
\begin{align*}
  \frac{I(1 - s, \tilde{W}, \hat{\Phi})}{L(1 - s, \tilde{\pi}, \mathrm{RS})}
  &= \int_{K_n}
    \frac{\Psi(1 - s, \tilde{\pi}(k) \tilde{W})}{ L(1 - s, \tilde{\pi}, \mathrm{RS})}
    \int_{F^\times} \hat{\Phi}(e_n a k) \lvert a \rvert^{n(1 - s)} \, d ^\times  a \, d k \\
  &=
    \dotsb.
\end{align*}
This leads to
\begin{equation*}
  \frac{I(1, \tilde{W}, \hat{\Phi})}{L(1, \tilde{\pi}, \mathrm{RS})}
  = \int_{K_n}
  \frac{\Psi(1, \tilde{\pi}(k), \tilde{W})}{L(1, \tilde{\pi}, \mathrm{RS})}
  \int_{F^\times} \hat{\Phi}(e_n a k) \lvert a \rvert^n \,d^\times a \, d k.
\end{equation*}
As $\pi$ is distinguished, so is $\tilde{\pi}$.  Now pass from the $P_n(F)$-invariance to $\GL_n(F)$-invariance to eventually get the equality
\begin{equation*}
  \frac{I(1, \tilde{W}, \hat{\Phi})}{L(1, \tilde{\pi}, \mathrm{RS})}
  =
  \frac{\Psi(1, \tilde{W})}{ L(1, \tilde{\pi}, \mathrm{RS})}
  \Phi(0),
\end{equation*}
which we  now know miust hold for zll $W$ and $\Phi$.  Using the functional equation at $s = 0$, we get
\begin{equation*}
  \frac{I(1, \tilde{W}, \hat{\Phi})}{L(1, \tilde{\pi}, \mathrm{RS})}
  =
  \alpha
  \frac{I(0, {W}, {\Phi})}{L(0, {\pi}, \mathrm{RS})}
\end{equation*}
for some $\alpha \in \mathbb{C}^\times$.  From the earlier calculation, the left hand side of the above equation vanishes for any $\Phi \in S^1$, thus the ratio on the right hand side of the above vanishes.  But then $s = 0$ is a pole for $L(s,, \pi, \mathrm{RS})$, say of order $d$.  Thus $s =0$ must be a pole for the family $\mathcal{J}(\pi)$ of maximal order $d$.  Observe that
\begin{equation*}
  B_0(W, \Phi) = \left( \lim_{s \rightarrow 0} s^d L(s, \pi, \mathrm{RS}) \right)
  \frac{I(0, W, \Phi)}{ L(0, \pi, \mathrm{RS})}
\end{equation*}
for all $W$ and $\Phi$.  By previous discussion, the bilinear form $B_0$ is nontrivial and vanishes on $S^1$.  Therefore, $s =0$ must be an exceptional pole of level $0$ for the family $\mathcal{J}(\pi)$.  This completes the proof of the theorem.  Thank you.

\bibliography{refs}{} \bibliographystyle{plain}
\end{document}
