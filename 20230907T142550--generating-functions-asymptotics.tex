\documentclass[reqno]{amsart} \usepackage{graphicx, amsmath, amssymb, amsfonts, amsthm, stmaryrd, amscd}
\usepackage[usenames, dvipsnames]{xcolor}
\usepackage{tikz}
% \usepackage{tikzcd}
% \usepackage{comment}

% \let\counterwithout\relax
% \let\counterwithin\relax
% \usepackage{chngcntr}

\usepackage{enumerate}
% \usepackage{enumitem}
% \usepackage{times}
\usepackage[normalem]{ulem}
% \usepackage{minted}
% \usepackage{xypic}
% \usepackage{color}


% \usepackage{silence}
% \WarningFilter{latex}{Label `tocindent-1' multiply defined}
% \WarningFilter{latex}{Label `tocindent0' multiply defined}
% \WarningFilter{latex}{Label `tocindent1' multiply defined}
% \WarningFilter{latex}{Label `tocindent2' multiply defined}
% \WarningFilter{latex}{Label `tocindent3' multiply defined}
\usepackage{hyperref}
% \usepackage{navigator}


% \usepackage{pdfsync}
\usepackage{xparse}


\usepackage[all]{xy}
\usepackage{enumerate}
\usetikzlibrary{matrix,arrows,decorations.pathmorphing}



\makeatletter
\newcommand*{\transpose}{%
  {\mathpalette\@transpose{}}%
}
\newcommand*{\@transpose}[2]{%
  % #1: math style
  % #2: unused
  \raisebox{\depth}{$\m@th#1\intercal$}%
}
\makeatother


\makeatletter
\newcommand*{\da@rightarrow}{\mathchar"0\hexnumber@\symAMSa 4B }
\newcommand*{\da@leftarrow}{\mathchar"0\hexnumber@\symAMSa 4C }
\newcommand*{\xdashrightarrow}[2][]{%
  \mathrel{%
    \mathpalette{\da@xarrow{#1}{#2}{}\da@rightarrow{\,}{}}{}%
  }%
}
\newcommand{\xdashleftarrow}[2][]{%
  \mathrel{%
    \mathpalette{\da@xarrow{#1}{#2}\da@leftarrow{}{}{\,}}{}%
  }%
}
\newcommand*{\da@xarrow}[7]{%
  % #1: below
  % #2: above
  % #3: arrow left
  % #4: arrow right
  % #5: space left 
  % #6: space right
  % #7: math style 
  \sbox0{$\ifx#7\scriptstyle\scriptscriptstyle\else\scriptstyle\fi#5#1#6\m@th$}%
  \sbox2{$\ifx#7\scriptstyle\scriptscriptstyle\else\scriptstyle\fi#5#2#6\m@th$}%
  \sbox4{$#7\dabar@\m@th$}%
  \dimen@=\wd0 %
  \ifdim\wd2 >\dimen@
    \dimen@=\wd2 %   
  \fi
  \count@=2 %
  \def\da@bars{\dabar@\dabar@}%
  \@whiledim\count@\wd4<\dimen@\do{%
    \advance\count@\@ne
    \expandafter\def\expandafter\da@bars\expandafter{%
      \da@bars
      \dabar@ 
    }%
  }%  
  \mathrel{#3}%
  \mathrel{%   
    \mathop{\da@bars}\limits
    \ifx\\#1\\%
    \else
      _{\copy0}%
    \fi
    \ifx\\#2\\%
    \else
      ^{\copy2}%
    \fi
  }%   
  \mathrel{#4}%
}
\makeatother
% \DeclareMathOperator{\rg}{rg}

\usepackage{mathtools}
\DeclarePairedDelimiter{\paren}{(}{)}
\DeclarePairedDelimiter{\abs}{\lvert}{\rvert}
\DeclarePairedDelimiter{\norm}{\lVert}{\rVert}
\DeclarePairedDelimiter{\innerproduct}{\langle}{\rangle}
\newcommand{\Of}[2]{{\operatorname{#1}} {\paren*{#2}}}
\newcommand{\of}[2]{{{{#1}} {\paren*{#2}}}}

\DeclareMathOperator{\Shim}{Shim}
\DeclareMathOperator{\sgn}{sgn}
\DeclareMathOperator{\fdeg}{fdeg}
\DeclareMathOperator{\SL}{SL}
\DeclareMathOperator{\slLie}{\mathfrak{s}\mathfrak{l}}
\DeclareMathOperator{\soLie}{\mathfrak{s}\mathfrak{o}}
\DeclareMathOperator{\spLie}{\mathfrak{s}\mathfrak{p}}
\DeclareMathOperator{\glLie}{\mathfrak{g}\mathfrak{l}}
\newcommand{\pn}[1]{{\color{ForestGreen} \sf PN: [#1]}}
\DeclareMathOperator{\Mp}{Mp}
\DeclareMathOperator{\Mat}{Mat}
\DeclareMathOperator{\GL}{GL}
\DeclareMathOperator{\Gr}{Gr}
\DeclareMathOperator{\GU}{GU}
\def\gl{\mathfrak{g}\mathfrak{l}}
\DeclareMathOperator{\odd}{odd}
\DeclareMathOperator{\even}{even}
\DeclareMathOperator{\GO}{GO}
\DeclareMathOperator{\good}{good}
\DeclareMathOperator{\bad}{bad}
\DeclareMathOperator{\PGO}{PGO}
\DeclareMathOperator{\htt}{ht}
\DeclareMathOperator{\height}{height}
\DeclareMathOperator{\Ass}{Ass}
\DeclareMathOperator{\coheight}{coheight}
\DeclareMathOperator{\GSO}{GSO}
\DeclareMathOperator{\SO}{SO}
\DeclareMathOperator{\so}{\mathfrak{s}\mathfrak{o}}
\DeclareMathOperator{\su}{\mathfrak{s}\mathfrak{u}}
\DeclareMathOperator{\ad}{ad}
% \DeclareMathOperator{\sc}{sc}
\DeclareMathOperator{\Ad}{Ad}
\DeclareMathOperator{\disc}{disc}
\DeclareMathOperator{\inv}{inv}
\DeclareMathOperator{\Pic}{Pic}
\DeclareMathOperator{\uc}{uc}
\DeclareMathOperator{\Cl}{Cl}
\DeclareMathOperator{\Clf}{Clf}
\DeclareMathOperator{\Hom}{Hom}
\DeclareMathOperator{\hol}{hol}
\DeclareMathOperator{\Heis}{Heis}
\DeclareMathOperator{\Haar}{Haar}
\DeclareMathOperator{\h}{h}
\def\sp{\mathfrak{s}\mathfrak{p}}
\DeclareMathOperator{\heis}{\mathfrak{h}\mathfrak{e}\mathfrak{i}\mathfrak{s}}
\DeclareMathOperator{\End}{End}
\DeclareMathOperator{\JL}{JL}
\DeclareMathOperator{\image}{image}
\DeclareMathOperator{\red}{red}
\def\div{\operatorname{div}}
\def\eps{\varepsilon}
\def\cHom{\mathcal{H}\operatorname{om}}
\DeclareMathOperator{\Ops}{Ops}
\DeclareMathOperator{\Symb}{Symb}
\def\boldGL{\mathbf{G}\mathbf{L}}
\def\boldSO{\mathbf{S}\mathbf{O}}
\def\boldU{\mathbf{U}}
\DeclareMathOperator{\hull}{hull}
\DeclareMathOperator{\LL}{LL}
\DeclareMathOperator{\PGL}{PGL}
\DeclareMathOperator{\class}{class}
\DeclareMathOperator{\lcm}{lcm}
\DeclareMathOperator{\spann}{span}
\DeclareMathOperator{\Exp}{Exp}
\DeclareMathOperator{\ext}{ext}
\DeclareMathOperator{\Ext}{Ext}
\DeclareMathOperator{\Tor}{Tor}
\DeclareMathOperator{\et}{et}
\DeclareMathOperator{\tor}{tor}
\DeclareMathOperator{\loc}{loc}
\DeclareMathOperator{\tors}{tors}
\DeclareMathOperator{\pf}{pf}
\DeclareMathOperator{\smooth}{smooth}
\DeclareMathOperator{\prin}{prin}
\DeclareMathOperator{\Kl}{Kl}
\newcommand{\kbar}{\mathchar'26\mkern-9mu k}
\DeclareMathOperator{\der}{der}
% \DeclareMathOperator{\abs}{abs}
\DeclareMathOperator{\Sub}{Sub}
\DeclareMathOperator{\Comp}{Comp}
\DeclareMathOperator{\Err}{Err}
\DeclareMathOperator{\dom}{dom}
\DeclareMathOperator{\radius}{radius}
\DeclareMathOperator{\Fitt}{Fitt}
\DeclareMathOperator{\Sel}{Sel}
\DeclareMathOperator{\rad}{rad}
\DeclareMathOperator{\id}{id}
\DeclareMathOperator{\Center}{Center}
\DeclareMathOperator{\Der}{Der}
\DeclareMathOperator{\U}{U}
% \DeclareMathOperator{\norm}{norm}
\DeclareMathOperator{\trace}{trace}
\DeclareMathOperator{\Equid}{Equid}
\DeclareMathOperator{\Feas}{Feas}
\DeclareMathOperator{\bulk}{bulk}
\DeclareMathOperator{\tail}{tail}
\DeclareMathOperator{\sys}{sys}
\DeclareMathOperator{\atan}{atan}
\DeclareMathOperator{\temp}{temp}
\DeclareMathOperator{\Asai}{Asai}
\DeclareMathOperator{\glob}{glob}
\DeclareMathOperator{\Kuz}{Kuz}
\DeclareMathOperator{\Irr}{Irr}
\newcommand{\rsL}{ \frac{ L^{(R)}(\Pi \times \Sigma, \std, \frac{1}{2})}{L^{(R)}(\Pi \times \Sigma, \Ad, 1)}  }
\DeclareMathOperator{\GSp}{GSp}
\DeclareMathOperator{\PGSp}{PGSp}
\DeclareMathOperator{\BC}{BC}
\DeclareMathOperator{\Ann}{Ann}
\DeclareMathOperator{\Gen}{Gen}
\DeclareMathOperator{\SU}{SU}
\DeclareMathOperator{\PGSU}{PGSU}
% \DeclareMathOperator{\gen}{gen}
\DeclareMathOperator{\PMp}{PMp}
\DeclareMathOperator{\PGMp}{PGMp}
\DeclareMathOperator{\PB}{PB}
\DeclareMathOperator{\ind}{ind}
\DeclareMathOperator{\Jac}{Jac}
\DeclareMathOperator{\jac}{jac}
\DeclareMathOperator{\im}{im}
\DeclareMathOperator{\Aut}{Aut}
\DeclareMathOperator{\Int}{Int}
\DeclareMathOperator{\PSL}{PSL}
\DeclareMathOperator{\co}{co}
\DeclareMathOperator{\irr}{irr}
\DeclareMathOperator{\prim}{prim}
\DeclareMathOperator{\bal}{bal}
\DeclareMathOperator{\baln}{bal}
\DeclareMathOperator{\dist}{dist}
\DeclareMathOperator{\RS}{RS}
\DeclareMathOperator{\Ram}{Ram}
\DeclareMathOperator{\Sob}{Sob}
\DeclareMathOperator{\Sol}{Sol}
\DeclareMathOperator{\soc}{soc}
\DeclareMathOperator{\nt}{nt}
\DeclareMathOperator{\mic}{mic}
\DeclareMathOperator{\Gal}{Gal}
\DeclareMathOperator{\st}{st}
\DeclareMathOperator{\std}{std}
\DeclareMathOperator{\diag}{diag}
\DeclareMathOperator{\Sym}{Sym}
\DeclareMathOperator{\gr}{gr}
\DeclareMathOperator{\aff}{aff}
\DeclareMathOperator{\Dil}{Dil}
\DeclareMathOperator{\Lie}{Lie}
\DeclareMathOperator{\Symp}{Symp}
\DeclareMathOperator{\Stab}{Stab}
\DeclareMathOperator{\St}{St}
\DeclareMathOperator{\stab}{stab}
\DeclareMathOperator{\codim}{codim}
\DeclareMathOperator{\linear}{linear}
\newcommand{\git}{/\!\!/}
\DeclareMathOperator{\geom}{geom}
\DeclareMathOperator{\spec}{spec}
\def\O{\operatorname{O}}
\DeclareMathOperator{\Au}{Aut}
\DeclareMathOperator{\Fix}{Fix}
\DeclareMathOperator{\Opp}{Op}
\DeclareMathOperator{\opp}{op}
\DeclareMathOperator{\Size}{Size}
\DeclareMathOperator{\Save}{Save}
% \DeclareMathOperator{\ker}{ker}
\DeclareMathOperator{\coker}{coker}
\DeclareMathOperator{\sym}{sym}
\DeclareMathOperator{\mean}{mean}
\DeclareMathOperator{\elliptic}{ell}
\DeclareMathOperator{\nilpotent}{nil}
\DeclareMathOperator{\hyperbolic}{hyp}
\DeclareMathOperator{\newvector}{new}
\DeclareMathOperator{\new}{new}
\DeclareMathOperator{\full}{full}
\newcommand{\qr}[2]{\left( \frac{#1}{#2} \right)}
\DeclareMathOperator{\unr}{u}
\DeclareMathOperator{\ram}{ram}
% \DeclareMathOperator{\len}{len}
\DeclareMathOperator{\fin}{fin}
\DeclareMathOperator{\cusp}{cusp}
\DeclareMathOperator{\curv}{curv}
\DeclareMathOperator{\rank}{rank}
\DeclareMathOperator{\rk}{rk}
\DeclareMathOperator{\pr}{pr}
\DeclareMathOperator{\Transform}{Transform}
\DeclareMathOperator{\mult}{mult}
\DeclareMathOperator{\Eis}{Eis}
\DeclareMathOperator{\reg}{reg}
\DeclareMathOperator{\sing}{sing}
\DeclareMathOperator{\alt}{alt}
\DeclareMathOperator{\irreg}{irreg}
\DeclareMathOperator{\sreg}{sreg}
\DeclareMathOperator{\Wd}{Wd}
\DeclareMathOperator{\Weil}{Weil}
\DeclareMathOperator{\Th}{Th}
\DeclareMathOperator{\Sp}{Sp}
\DeclareMathOperator{\Ind}{Ind}
\DeclareMathOperator{\Res}{Res}
\DeclareMathOperator{\ini}{in}
\DeclareMathOperator{\ord}{ord}
\DeclareMathOperator{\osc}{osc}
\DeclareMathOperator{\fluc}{fluc}
\DeclareMathOperator{\size}{size}
\DeclareMathOperator{\ann}{ann}
\DeclareMathOperator{\equ}{eq}
\DeclareMathOperator{\res}{res}
\DeclareMathOperator{\pt}{pt}
\DeclareMathOperator{\src}{source}
\DeclareMathOperator{\Zcl}{Zcl}
\DeclareMathOperator{\Func}{Func}
\DeclareMathOperator{\Map}{Map}
\DeclareMathOperator{\Frac}{Frac}
\DeclareMathOperator{\Frob}{Frob}
\DeclareMathOperator{\ev}{eval}
\DeclareMathOperator{\pv}{pv}
\DeclareMathOperator{\eval}{eval}
\DeclareMathOperator{\Spec}{Spec}
\DeclareMathOperator{\Speh}{Speh}
\DeclareMathOperator{\Spin}{Spin}
\DeclareMathOperator{\GSpin}{GSpin}
\DeclareMathOperator{\Specm}{Specm}
\DeclareMathOperator{\Sphere}{Sphere}
\DeclareMathOperator{\Sqq}{Sq}
\DeclareMathOperator{\Ball}{Ball}
\DeclareMathOperator\Cond{\operatorname{Cond}}
\DeclareMathOperator\proj{\operatorname{proj}}
\DeclareMathOperator\Swan{\operatorname{Swan}}
\DeclareMathOperator{\Proj}{Proj}
\DeclareMathOperator{\bPB}{{\mathbf P}{\mathbf B}}
\DeclareMathOperator{\Projm}{Projm}
\DeclareMathOperator{\Tr}{Tr}
\DeclareMathOperator{\Type}{Type}
\DeclareMathOperator{\Prop}{Prop}
\DeclareMathOperator{\vol}{vol}
\DeclareMathOperator{\covol}{covol}
\DeclareMathOperator{\Rep}{Rep}
\DeclareMathOperator{\Cent}{Cent}
\DeclareMathOperator{\val}{val}
\DeclareMathOperator{\area}{area}
\DeclareMathOperator{\nr}{nr}
\DeclareMathOperator{\CM}{CM}
\DeclareMathOperator{\CH}{CH}
\DeclareMathOperator{\tr}{tr}
\DeclareMathOperator{\characteristic}{char}
\DeclareMathOperator{\supp}{supp}


\theoremstyle{plain} \newtheorem{theorem} {Theorem} \newtheorem{conjecture} [theorem] {Conjecture} \newtheorem{corollary} [theorem] {Corollary} \newtheorem{proposition} [theorem] {Proposition} \newtheorem{fact} [theorem] {Fact}
\theoremstyle{definition} \newtheorem{definition} [theorem] {Definition} \newtheorem{hypothesis} [theorem] {Hypothesis} \newtheorem{assumptions} [theorem] {Assumptions}
\newtheorem{example} [theorem] {Example}
\newtheorem{assertion}[theorem] {Assertion}
\newtheorem{note}[theorem] {Note}
\newtheorem{conclusion}[theorem] {Conclusion}
\newtheorem{claim}            {Claim}
\newtheorem{homework} {Homework}
\newtheorem{exercise} {Exercise}  \newtheorem{question}[theorem] {Question}    \newtheorem{answer} {Answer}  \newtheorem{problem} {Problem}    \newtheorem{remark} [theorem] {Remark}
\newtheorem{notation} [theorem]           {Notation}
\newtheorem{terminology}[theorem]            {Terminology}
\newtheorem{convention}[theorem]            {Convention}
\newtheorem{motivation}[theorem]            {Motivation}


\newtheoremstyle{itplain} % name
{6pt}                    % Space above
{5pt\topsep}                    % Space below
{\itshape}                   % Body font
{}                           % Indent amount
{\itshape}                   % Theorem head font
{.}                          % Punctuation after theorem head
{5pt plus 1pt minus 1pt}                       % Space after theorem head
% {.5em}                       % Space after theorem head
{}  % Theorem head spec (can be left empty, meaning ‘normal’)

% \theoremstyle{mytheoremstyle}


\theoremstyle{itplain} %--default
% \theoremheaderfont{\itshape}
% \newtheorem{lemma}{Lemma}
\newtheorem{lemma}[theorem]{Lemma}
% \newtheorem{lemma}{Lemma}[subsubsection]

\newtheorem*{lemma*}{Lemma}
\newtheorem*{proposition*}{Proposition}
\newtheorem*{definition*}{Definition}
\newtheorem*{example*}{Example}

\newtheorem*{results*}{Results}
\newtheorem{results} [theorem] {Results}


\usepackage[displaymath,textmath,sections,graphics]{preview}
\PreviewEnvironment{align*}
\PreviewEnvironment{multline*}
\PreviewEnvironment{tabular}
\PreviewEnvironment{verbatim}
\PreviewEnvironment{lstlisting}
\PreviewEnvironment*{frame}
\PreviewEnvironment*{alert}
\PreviewEnvironment*{emph}
\PreviewEnvironment*{textbf}



\usepackage{xr-hyper}
\externaldocument{2023-introduction-to-zeta-and-l-functions}

\begin{document}
 
\title{Generating functions and asymptotics}
\begin{abstract}
  Part of the course notes for \href{2023-introduction-to-zeta-and-l-functions.pdf}{this course}.  Here we study the relationship between asymptotics of sequences $(c_n)_{n \in \mathbb{Z}}$ and meromorphic continuation of the associated generating series.  Reference:~\cite[\S5.2]{MR2172781}.
\end{abstract}


\section{Setup}
We consider a Laurent series
\begin{equation*}
  f (z) = \sum_{n \in \mathbb{Z} } c_n z^n.
\end{equation*}
Here the $c_n$ are complex coefficients, while $z$ is a nonzero complex argument.  We assume that this series converges absolutely for at least one value of $z$.

\begin{lemma}\label{lemma:cj3vqafpa6}
  There is a unique maximal open subinterval $(a,b)$ of $\mathbb{R}^+$ on which $f$ converges absolutely.  Its endpoints are given explicitly by
  \begin{equation*}
    a = \inf \left\{ r \in \mathbb{R}^+ : \sum_n \lvert c_n \rvert r^n < \infty  \right\},
  \end{equation*}

\begin{equation*}
  b = \sup \left\{ r \in \mathbb{R}^+ : \sum_n \lvert c_n \rvert r^n < \infty  \right\}.
\end{equation*}
\end{lemma}
We refer to the interval $(a,b)$ as the \emph{fundamental interval} for $f$ (or for the $c_n$).

The fundamental interval controls the growth of the coefficients $c_n$ as $n \rightarrow \pm \infty$:
\begin{lemma}
  Let $b^- < b$ and $a^+ > a$.  Then
  \begin{equation*}
    c_n \ll {(b^-)}^{-n} \quad \text{ as } n \rightarrow \infty
  \end{equation*}
  and
  \begin{equation*}
    c_n \ll {(a^+)}^{-n} \quad \text{ as } n \rightarrow -\infty.
  \end{equation*}
\end{lemma}

Set
\begin{equation*}
  \mathcal{C} (a, b) := \left\{ z \in \mathbb{C} : \lvert z  \rvert \in (a,b) \right\}.
\end{equation*}

\begin{lemma}
  $f(z)$ defines a holomorphic function on $\mathcal{C}(a,b)$.
\end{lemma}
\begin{proof}
  Follows from Theorem~\ref{theorem:cj3vqa91ti}.
\end{proof}

\section{Basic study of meromorphic continuation}

\begin{lemma}\label{lemma:cj3vqbs30d}
  $f$ does not extend to a holomorphic function on $\mathcal{C}(A,B)$ for any strictly larger interval $(A,B) \supsetneq (a,b)$.
\end{lemma}
\begin{proof}
  Suppose otherwise.  Let $r \in (A,B) - (a,b)$.  Then by Cauchy's integral formula (specifically, the estimate~\eqref{eq:cj3vqbiupy} of Theorem~\ref{theorem:cj3vqbjd26}), we see that $\sum_{n \in \mathbb{Z}} \lvert c_n \rvert r^n < \infty$.  This contradicts the formula for $a$ and $b$ given in Lemma~\ref{lemma:cj3vqafpa6}.
\end{proof}

\begin{note}
  It can happen that $f$ extends to a \emph{meromorphic} function on some strictly larger annulus (unique, in view of Corollary~\ref{corollary:cj3vqbthht}).  By Lemma~\ref{lemma:cj3vqbs30d}, this can only happen if $f$ has a pole at some point on the boundary of the fundamental annulus.
\end{note}

\begin{example}
  Take
  \begin{equation*}
    c_n =
    \begin{cases}
      2^n & \text{ if } n \geq 0, \\
      0 & \text{ if } n < 0.
    \end{cases}
  \end{equation*}
  Then the fundamental interval is $(a, b) = (0, 1/2)$.  However, the function $f(z)$, defined initially for $\lvert z \rvert < 1/2$, evaluates to a rational function:
  \begin{equation*}
    f(z) = \sum_{n \geq 0} 2^n z^n
    = \frac{1}{1 - 2 z}.
  \end{equation*}
  This is meromorphic on the entire complex plane; the only pole is a simple one at $z = 1/2$, with residue $-1/2$.
\end{example}

The possibility of meromorphically extending $f$ corresponds to the coefficients $c_n$ having asymptotic expansions as $n \rightarrow \pm \infty$.  For example:
\begin{lemma}[Meromorphic continuation vs.\ asymptotic expansion, special case]\label{lemma:cj3vqfrerl}
  Let $f$ and $(a,b)$ be as above.  Let $\beta \in \mathbb{C}$ with $\lvert \beta \rvert = b$.  Let $B > b$ and $\gamma \in \mathbb{C} $.  Then the following are equivalent:
  \begin{enumerate}[(i)]
  \item\label{enumerate:cj3vqef009} $f$ extends to a meromorphic function on $\mathcal{C}(a, B)$ with a unique simple pole at $z = \beta$ with residue $\gamma$.
  \item\label{enumerate:cj3vqef2m9} For each $B^- < B$, we have as $n \rightarrow \infty$ that
    \begin{equation}\label{eq:cj3wnbmxaw}
      c_n = - \gamma \beta ^{-n-1} + \O \left( {(B^-)}^{-n} \right),
    \end{equation}
  \end{enumerate}
\end{lemma}
\begin{proof}
  To see that~\eqref{enumerate:cj3vqef009} implies~\eqref{enumerate:cj3vqef2m9}, we start with Cauchy's integral formula on the disc of radius $b^-$ for some $b^- \in (a,b)$, then shift the contour, picking up the contribution of the unique pole:
  \begin{align}
    c_n
    &= \oint_{\lvert z \rvert = b^-} \frac{f(z)}{z^{n}} \, \frac{d z}{2 \pi i z} \nonumber
    \\
    &= \oint_{\lvert z \rvert = B^-} \frac{f(z)}{z^{n}} \, \frac{d z}{2 \pi i z} \label{align:cj3vqg2jvd}
      - \frac{\gamma}{\beta^{n+1}}.
  \end{align}
  We then estimate this last integral using that $f$ is bounded on compact sets.

  Conversely, to verify that~\eqref{enumerate:cj3vqef2m9} implies~\eqref{enumerate:cj3vqef009}, we define the coefficients
  \begin{equation*}
    b_n :=
    \begin{cases}
      - \gamma \beta^{- n - 1 } &  \text{ if } n \geq 0, \\
      0 & \text{ if } n < 0,
    \end{cases}
  \end{equation*}
  The corresponding series
  \begin{equation*}
    f_+(z) := \sum_{n \in \mathbb{Z} } b_n z^n
  \end{equation*}
  may be evaluated explicitly: a simple geometric series calculation, left to the reader, gives
  \begin{equation*}
    f_+(z) = \frac{\gamma}{z - \beta }.
  \end{equation*}
  Our hypothesis concerning the $c_n$ reads
  \begin{equation}\label{eq:cj3vqey6o9}
    b_n - c_n = \O \left( {(B^-)}^{-n} \right) \quad \text{ as } n \rightarrow \infty.
  \end{equation}
  On the other hand, because $f$ has fundamental interval $(a,b)$ and $b_n$ vanishes as $n \rightarrow -\infty$, we have for each $a^+ > a$ that
  \begin{equation}\label{eq:cj3vqey9fd}
    b_n - c_n = \O \left( {(a^+)}^{-n} \right) \quad \text{ as } n \rightarrow -\infty.
  \end{equation}
  From~\eqref{eq:cj3vqey6o9} and~\eqref{eq:cj3vqey9fd}, we deduce that the series $f - f_+$ with coefficients $c_n - b_n$ has fundamental interval containing $(a,B)$.  This implies that the function
  \begin{equation*}
    f(z) - \frac{\gamma }{z - \beta },
  \end{equation*}
  defined initially as a holomorphic function on $\mathcal{C}(a,b)$, extends to a holomorphic function on $\mathcal{C}(a,B)$.  Equivalently, $f$ extends to a meromorphic function on $\mathcal{C}(a,B)$ with polar behavior as described in~\eqref{enumerate:cj3vqef2m9}.
\end{proof}

\begin{exercise}
  Generalize the above lemma to describe in terms of the coefficients $c_n$ what it means for $f$ to extend to a meromorphic function on $\mathcal{C}(A,B)$ for some $A < a$ and $B > b$, allowing the possibility of multiple poles of arbitrary order.
\end{exercise}

\section{Further examples}\label{sec:cj4unj3scf}


\begin{example}
  Suppose that
  \begin{equation*}
    c_n =
    \begin{cases}
      \beta^{- n} & \text{ if } n \geq 0, \\
      0 & \text{ if } n < 0,
    \end{cases}
  \end{equation*}
  so that, initially for $\lvert z \rvert < |\beta|$,
  \begin{equation*}
    f (z) = \sum_{n \geq 0} \beta^{-n} z^n = \frac{1}{1 - z / \beta }.
  \end{equation*}
  The function $f$ extends meromorphically, having a simple pole at $z = \beta$ with residue $-\beta$.  The sequence $c_n$ has the asymptotic behavior indicated in~\eqref{eq:cj3wnbmxaw}, in a very strong sense: the sequence is \emph{equal} to the asymptotic.
\end{example}

\begin{exercise}
  Let $c_n$ denote the Fibonacci sequence, thus $c_n = 0$ for $n < 0$ and
  \begin{equation*}
    c_0 = 1, \quad c_1 = 1, \quad
    c_{n+2} - c_{n+1} - c_n = 0.
  \end{equation*}
  This exercise rederives a standard formula for this sequence in a way that is intended to illustrate the technique of Lemma~\ref{lemma:cj3vqfrerl}.
  \begin{enumerate}
  \item Verify by crude estimation that the fundamental interval for the series $f(z) = \sum_n c_n z^n$ contains $(0,1/2)$.
  \item Show that
    \begin{equation*}
      f(z) = \frac{1}{1 - z - z^2} =
      \frac{1}{(1 - z/\varphi) ( 1 - z / \varphi ')},
    \end{equation*}
    where
    \begin{equation*}
      \varphi = \frac{1 + \sqrt{5}}{2} = 1.618 \dotsb, \quad
      \varphi ' = \frac{1 - \sqrt{5}}{2} = -0.618 \dotsb.
    \end{equation*}
  \item Following the proof of Lemma~\ref{lemma:cj3vqfrerl}, show that
    \begin{equation*}
      c_n = \frac{\varphi^n - {(\varphi ')}^n }{\varphi - \varphi ' }.
    \end{equation*}
    (Use that $f(z) \ll |z|^2$ for $|z| \geq 2$ to show that the ``remainder term'', namely the integral in~\eqref{align:cj3vqg2jvd}, tends to zero as $B^- \rightarrow \infty $.)
  \end{enumerate}
\end{exercise}

\begin{example}\label{example:cj4ss5hv85}
  Let $\beta \in \mathbb{C} - \{0\}$ and $a \in \mathbb{Z}_{\geq 0}$.  Then one verifies by induction on $a$, using differentiation, that
  \begin{equation*}
    \frac{1}{(z - \beta)^{a+1}} =
    (-\beta)^{-a-1}
    \sum_{n \geq 0}
    \binom{n + a}{n} \beta^{-n} z^n,
  \end{equation*}
  where the binomial coefficient expands to a polynomial of degree $a$ in $n$:
  \begin{equation*}
    \binom{n+a}{a} = \frac{(n+a)!}{a! n!} =
    \frac{(n+1) (n+2) \dotsb (n+a)}{a!}.
  \end{equation*}
  More generally, given any coefficients $c_0, c_1\dotsc, c_k$, we have
  \begin{equation*}
    \sum_{k=0}^{a}
    \frac{c_k}{(z - \beta)^{k+1}} = \sum_{n \geq 0} P(n) \beta^{-n} z^n
  \end{equation*}
  for some polynomial $P(n)$ of degree at most $a$.  Conversely, given such a polynomial, we may find coefficients so that the above identity holds.
\end{example}


\begin{example}\label{example:cj4ungymno}
  Take
  \begin{equation*}
    c_n:= e^{- 2^n }.
  \end{equation*}
  Observe that
  \begin{equation*}
    c_n \rightarrow
    \begin{cases}
      0 &  \text{ if } n \rightarrow \infty, \\
      1 & \text{ if } n \rightarrow - \infty.
    \end{cases}
  \end{equation*}
  Moreover, as $n \rightarrow \infty$, the convergence of the $c_n$ to zero is rapid in the sense that for each $B < \infty$, we have
  \begin{equation*}
    c_n \ll B^{-n}.
  \end{equation*}
  The fundamental interval is thus $(1,\infty)$: the series $f(z) = \sum_{n} c_n z^n$ converges absolutely for $|z| > 1$ and defines a holomorphic function there.  We will show that $f$ \emph{extends to a meromorphic function on} $\mathbb{C} - \{0\}$\emph{, which is holomorphic away from simple poles at} $1 / 2^k$ \emph{(for} $k \in \mathbb{Z}_{\geq 0}$\emph{) with residue} $(-1/4)^k / k!$\emph{.}  To that end, observe first that the contribution to $f$ from $n \geq 0$, namely
  \begin{equation*}
    f_+(z) := \sum_{n > 0} c_n z^n,
  \end{equation*}
  converges absolutely and is thus holomorphic on the entire complex plane.  The meromorphic continuation of $f$ thereby reduces to that of the complementary sum
  \begin{equation*}
    f_-(z) := \sum_{n \leq  0} c_n z^n.
  \end{equation*}
  Inspired by Lemma~\ref{lemma:cj3vqfrerl}, we study the asymptotics of the coefficients $c_n$ as $n \rightarrow - \infty$.  These are described by the Taylor series of the exponential functions:
  \begin{equation*}
    e^{x} = \sum_{k \geq 0} \frac{x^k }{k!}.
  \end{equation*}
  By estimating the tail of this series, we sees that for $x = \O(1)$ and $M = \O(1)$, we have
  \begin{equation*}
    e^x = \sum_{k = 0}^{N-1} \frac{x^k}{k!} + \O (x^M).
  \end{equation*}
  It follows that for $n \leq 0$,
  \begin{equation}\label{eq:cj3wnu8lqf}
    c_n = \sum_{k = 0}^{N-1} \frac{{(-2^n)}^k}{k!} + \O (2^{nM}).
  \end{equation}
  By the method of proof of Lemma~\ref{lemma:cj3vqfrerl}, we deduce from this estimate that $f_-$ the required assertions concerning the meromorphic continuation of $f$.  Let us spell this deduction out for the sake of practice.  Set
  \begin{equation*}
    g_k(z) := 
    \sum_{n \leq 0}
    \frac{{(-2^n)}^k}{k!} z^n.
  \end{equation*}
  The estimate~\eqref{eq:cj3wnu8lqf} implies that the modified series
  \begin{equation}\label{eq:cj3wnu7518}
    f_-(z)
    -
    \sum_{k = 0}^{N-1}
    g_k(z)
    = 
    \sum_{n \leq 0}
    \left( c_n -
      \sum_{k = 0}^{N-1} \frac{{(-2^n)}^k}{k!}
    \right) z^n
  \end{equation}
  converges absolutely for $\lvert z \rvert > 1 / 2^M$, hence defines a holomorphic function there.  On the other hand, for $\lvert z \rvert > 1$, we see by summing the geometric series that
  \begin{equation*}
    g_k(z)
    =
    \frac{{(-1)}^k}{k!}
    \frac{1}{1 - 1 / 2^k z}
    =
    \frac{{(-1/2)}^k}{k!}
    \frac{z}{z - 1 / 2^k}.
  \end{equation*}
  Thus $g_k$ extends to a meromorphic function whose only pole is a simple one at $z = 1/2^k$ with residue ${(-1/4)}^k / k!$.  It follows that $f$ has the claimed meromorphic properties.
\end{example}

\section{Regularization}\label{sec:cj4unj4gyo}

\begin{remark}
  We can cases view $f(1)$ as the ``regularized sum'' of the (possibly divergent) series $\sum_n c_n$:
  \begin{equation*}
    \sum_{n}^{\reg} c_n := \left( \sum_n c_n z^n \right)|_{z=1}, 
  \end{equation*}
  keeping in mind here that the series may initially convergent away from the point $z=1$, so that the specialization is understood as the result of analytic continuation.  We make this definition whenever the series is holomorphic at $z=1$.

  For example,
  \begin{equation*}
    \sum_{n \geq 0}^{\reg} (-1)^n
    =
    \left( \sum_{n \geq 0} (-1)^n z^n \right)|_{z=1}
    = \frac{1}{1 + z} \vert_{z=1} = \frac{1}{2}.
  \end{equation*}
  In this example, we may understand $f(1)$ as the limit of the quantities $f(z)$ for $z < 1$ as $z \rightarrow 1$, and also as the Cesaro mean of the partial sums of the series $\sum_{n \geq 0} (-1)^n$, so the interpretation of $f(1)$ as a regularized sum makes intuitive sense.

  In other examples, the interpretation may be less clear.  For example,
  \begin{equation*}
    \sum_{n \geq 0}^{\reg} 10^n =
    \left( \sum_{n \geq 0} 10^n z^n \right)|_{z=1}
    =
    \frac{1}{1 - 10 z} \vert_{z=1} =
    \frac{-1}{9},
  \end{equation*}
  the intuitive meaning of which may be less clear.  One way to understand the regularization is as follows: the value $f(1)$ is insensitive to replacing the sequence $(c_n)_n$ by any of its shifts $(c_{n+k})_n$.  Setting
  \begin{equation*}
    S := \sum_{n \geq 0}^{\reg} 10^n,
  \end{equation*}
  we should thus have
  \begin{equation*}
    10 S = \sum_{n \geq 0}^{\reg} 10^{n+1}
    = 1 + S,
  \end{equation*}
  from which it follows that $S = -1/9$.
\end{remark}


In fact, we can define regularized sums even in cases where the series does not converge at any point.  The idea is to split the sum into two pieces, one near $+\infty$ and the other near $-\infty$, then to meromorphically continue each part from some initial domain and add together the resulting meromorphic continuations.  This is the content of the following definitions and results.
\begin{definition}
  Let us say that a function $\mathbb{Z} \rightarrow \mathbb{C}$ is \emph{finite} if it is a finite linear combination of functions of the form
  \begin{equation*}
    n \mapsto n^a \beta^{- n},
  \end{equation*}
  where $a \in \mathbb{Z}_{\geq 0}$ and $\beta \in \mathbb{C}^\times$.
\end{definition}
\begin{remark}
The intrinsic interpretation of this definition is that the finite functions are those whose translates span a finite-dimensional subspace of the space of all functions $\mathbb{Z} \rightarrow \mathbb{C}$.
\end{remark}
\begin{definition}
  Let us say that a function $c : \mathbb{Z} \rightarrow \mathbb{C}$ is \emph{regularizable} if for each $0 < A < B < \infty$, there exist finite functions $c_{\pm}$ so that
  \begin{equation*}
    c(n) = c_+(n) + \O(B^{-n}) \text{ as } n \rightarrow \infty,
  \end{equation*}
  \begin{equation*}
    c(n) = c_-(n) + \O(A^{-n}) \text{ as } n \rightarrow -\infty.
  \end{equation*}
\end{definition}
\begin{example}
Any finite function is regularizable.
\end{example}
\begin{definition}\label{definition:cj4ss5ec2g}
  Given a regularizable function $c$ as above, we may define its \emph{regularized generating function}
  \begin{equation}\label{eq:cj4ss5ffxd}
    f(z) := \sum_{n \in \mathbb{Z} }^{\reg} c(n) z^n,
  \end{equation}
  which will be a meromorphic function of $z \in \mathbb{C}^\times$, as follows.  Choose $N \in \mathbb{Z}$.  Define
  \begin{equation*}
    f_+(z) := \sum_{n \geq N} c(n) z^n,
  \end{equation*}
  initially for $\lvert z \rvert$ sufficiently small, then in general by meromorphic continuation (with poles described by the asymptotics of $c$ as $n \rightarrow \infty$, corresponding to terms of $c_+$).  Similarly, we define
  \begin{equation*}
    f_-(z) := \sum_{n < N} c(n) z^n,
  \end{equation*}
  initially for $\lvert z \rvert$ sufficiently large, then in general by meromorphic continuation (with poles described by the asymptotics of $c$ as $n \rightarrow -\infty$, corresponding to terms of $c_-$).  We then set
  \begin{equation*}
    f(z) := f_+(z) + f_-(z).
  \end{equation*}
\end{definition}
\begin{lemma}\label{lemma:cj4ss2kx0v}
  The above definition is independent of the choice of $N$.
\end{lemma}
\begin{proof}
  Modifying $N$ has the effect of adding a polynomial to $f_{\pm}$ and subtracting the same polynomial from $f_{\mp}$.
\end{proof}
\begin{lemma}\label{lemma:cj4ss2nscv}
  Suppose that $c(n)$ is regularizable, and let $k \in \mathbb{Z}$.  Then the shifted sequence $d(n) := c(n+k)$ is also regularizable.  The regularized generating function $g$ for $d$ is related to the generating function $f$ for $c$ via
  \begin{equation*}
    g(z) = z^{-k} f(z).
  \end{equation*}
  In other words,
  \begin{equation*}
    \sum_{n \in \mathbb{Z}}^{\reg} c(n) z^n
    = 
    z^k \sum_{n \in \mathbb{Z}}^{\reg} c(n+k) z^n.
  \end{equation*}
\end{lemma}
\begin{proof}
  This may be deduced from Lemma \ref{lemma:cj4ss2kx0v}.
\end{proof}
\begin{exercise}\label{exercise:cj4ss5ia8l}
  Show that if $c$ is a finite function, then its regularized generating function vanishes.  [This can be seen via explicit calculation using the definition, or, as in lecture, from Lemma \ref{lemma:cj4ss2nscv}.]
\end{exercise}

\begin{example}
  Take
  \begin{equation*}
    c_n = n^3 e^{- 2^{- n^2 }}.
  \end{equation*}
  Then the series $f$ converges absolutely nowhere: the fundamental interval is empty.  But we can still define its regularized generating function, which is actually an entire function of $n$: it may be written explicitly as the everywhere convergent series
  \begin{equation}\label{eq:cj4ss5ikdb}
\sum_{n \in \mathbb{Z} } (c_n - n^3) z^n.
\end{equation}
Note that the individual pieces $f_{\pm}$ as in Definition~\ref{definition:cj4ss5ec2g} are not entire functions of $z$: they have quadruple poles at $z=1$, in view of Example~\ref{example:cj4ss5hv85}.  However, the poles cancel thanks to Exercise~\ref{exercise:cj4ss5ia8l}, so their sum is the entire function~\eqref{eq:cj4ss5ikdb}.
\end{example}


\section{Permutations without small cycles}\label{sec:cj4unj04kx}
Reference:~\cite[p176]{MR2172781}.

Let $S(n)$ denote the symmetric group, consisting of permutations $\sigma$ of the set $ \{1, \dotsc, n\}$.  We have $\# S(n) = n!$.

Each permutation may be written uniquely as a product of disjoint cyclic permutations of some lengths $n_1, \dotsc, n_k \in \mathbb{Z}_{\geq 1}$, where $n_1 + \dotsb + n_k = n$.

For each subset $S$ of $\mathbb{N} := \mathbb{Z}_{\geq 1}$ and each $n \geq 0$, let $c_n^S$ denote the number of permutations $\sigma \in S(n)$ each of whose cycle lengths $n_j$ lies in $S$.  We denote by
\begin{equation*}
  f^S(z) := \sum_{n \geq 0} \frac{c_n^S}{n!} z^n
\end{equation*}
the ``exponential generating function'' of this sequence.  Since $c_n^S \leq n!$, we see that the series converges absolutely for $|z| < 1$.
\begin{example}
  We have
  \begin{equation*}
    f^{\mathbb{N} } (z) = \sum_{n \geq 0} z^n = \frac{1}{1-z}
    = \exp \log \frac{1}{1 - z}
    =
    \exp \sum _{n \geq 1} \frac{z^n}{n}.
  \end{equation*}
\end{example}
\begin{example}
  We have
  \begin{equation*}
    f^{\{1\}}(z) = \sum_{n \geq 0} \frac{z^n}{n!} = \exp z
    = \exp \sum_{n = 1} \frac{z^n }{n}.
  \end{equation*}
\end{example}
\begin{lemma}
  We have
  \begin{equation*}
    f^S(z) = \exp \sum_{n \in S} \frac{z^n}{n}.
  \end{equation*}
\end{lemma}
\begin{proof}
  Using the series definition $\exp x = \sum_{k \geq 0} x^k / k!$, we see that
  \begin{equation*}
    \exp \sum_{n \in S} \frac{z^n}{n}
    = \sum_{k \geq 0}
    \frac{1}{k!}
    \sum_{n_1, \dotsc, n_k \in S}
    \frac{z^{n _1 + \dotsb + n_k}}{n_1 \dotsb n_k}.
  \end{equation*}
  Our task is thus to verify that
  \begin{equation}\label{eq:cj4sszpnpp}
    c_n^S = \sum_{k \geq 0}
    \sum_{
      \substack{
        n_1, \dotsc, n_k \in S : \\
        n_1 + \dotsb + n_k = n
      }
    }
    \frac{n!}{k ! n_1 \dotsb n_k}.
  \end{equation}
  The set of permutations attached to a given multiset of lengths $\{n_1, \dotsc, n_k\}$ is a conjugacy class in $S(n)$.  The size of that conjugacy class is described by the orbit-centralizer formula.  The centralizer of a permutation is the group generated by each of the cycles in the decomposition; this group has order $n_1 \dotsb n_k$.  It follows that the conjugacy class has size
  \begin{equation*}
    \frac{n!}{n_1 \dotsb n_k}.
  \end{equation*}
  Since the order of the $n_j$ doesn't matter, we deduce the claimed formula \eqref{eq:cj4sszpnpp}.
\end{proof}

As an application, take $S$ to consist of all integers greater than some given integer $\ell$.  Then
\begin{equation*}
  f^S(z) = \exp \left(  \sum_{n \geq 1} \frac{z^n}{ n } - \sum_{n = 1}^{\ell} \frac{z^n }{n}\right)
  =
  \frac{1}{1 - z} \exp \left( - \sum_{n = 1}^{\ell} \frac{z^n }{n} \right).
\end{equation*}
This defines a meromorphic function on the entire complex plane, whose only pole is a simple one at $z = 1$.  By the usual analysis, we deduce the following asymptotic formula for the coefficients of $f^S$:
\begin{equation}\label{eq:cj4ssz06pg}
  \frac{c_n^S}{n!} = \exp \left( - \sum_{n = 1 }^{\ell} \frac{z^n }{n} \right) + \O(B^{-n}),
\end{equation}
for fixed $\ell$ and fixed $B < \infty$.

We may interpret the left hand side of \eqref{eq:cj4ssz06pg} as the probability that a random permutation of length $n$ has no cycles of length $\leq \ell$.

\section{Series that do not admit meromorphic continuations}\label{sec:cj4unj06mw}
\begin{example}\label{example:cj4ss1hy2q}
  The series
  \begin{equation*}
    \sum_{n \geq 1} \log(n) z^n
  \end{equation*}
  converges absolutely for $\lvert z \rvert < 1$, but does not continue meromorphically to $|z| < r$ for any $r > 1$, because $\log(n)$ cannot be approximated up to error $\O(r^{-n})$ by a finite linear combination of the functions $n^a \beta^{-n}$ ($a \in \mathbb{Z}_{\geq 0}$, $\beta \in \mathbb{C}^\times$).  Indeed, we would need to have each $\beta = 1$, so the main point is to check that $\log(n)$ cannot be approximated exponentially well by a polynomial.  This is because $\log(n)$ grows faster than any constant polynomial but slower than any non-constant polynomials.
\end{example}
\begin{question}
  Does the series in Example \ref{example:cj4ss1hy2q} continue to any open set strictly containing the unit disc?
\end{question}
\begin{exercise}
  Similarly analyze the series
  \begin{equation*}
    \sum_{n \geq 1} n^{1 / n} z^n.
  \end{equation*}
\end{exercise}
\begin{example}
  For $k \geq 0$ and $n \in \mathbb{Z}_{\geq 1}$, define
  \begin{equation*}
    \sigma_k(n) := \sum_{d | n} d^k,
  \end{equation*}
  where $d$ runs over the positive divisors of $n$.  For example, $\sigma_0(n) = \tau(n)$ is the number of positive divisors of $n$.  The series
  \begin{equation*}
    \sum_{n \geq 0} \sigma_k (n) z^n
  \end{equation*}
  converges absolutely on the disc $\lvert z \rvert < 1$.  It does not extend to a meromorphic function on any strictly larger open set.  This can be seen most efficiently using the basic theory of modular forms, which we discuss later in the course.
\end{example}


\bibliography{refs}{} \bibliographystyle{plain}
\end{document}
