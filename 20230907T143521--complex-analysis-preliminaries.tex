\documentclass[reqno]{amsart} \usepackage{graphicx, amsmath, amssymb, amsfonts, amsthm, stmaryrd, amscd}
\usepackage[usenames, dvipsnames]{xcolor}
\usepackage{tikz}
% \usepackage{tikzcd}
% \usepackage{comment}

% \let\counterwithout\relax
% \let\counterwithin\relax
% \usepackage{chngcntr}

\usepackage{enumerate}
% \usepackage{enumitem}
% \usepackage{times}
\usepackage[normalem]{ulem}
% \usepackage{minted}
% \usepackage{xypic}
% \usepackage{color}


% \usepackage{silence}
% \WarningFilter{latex}{Label `tocindent-1' multiply defined}
% \WarningFilter{latex}{Label `tocindent0' multiply defined}
% \WarningFilter{latex}{Label `tocindent1' multiply defined}
% \WarningFilter{latex}{Label `tocindent2' multiply defined}
% \WarningFilter{latex}{Label `tocindent3' multiply defined}
\usepackage{hyperref}
% \usepackage{navigator}


% \usepackage{pdfsync}
\usepackage{xparse}


\usepackage[all]{xy}
\usepackage{enumerate}
\usetikzlibrary{matrix,arrows,decorations.pathmorphing}



\makeatletter
\newcommand*{\transpose}{%
  {\mathpalette\@transpose{}}%
}
\newcommand*{\@transpose}[2]{%
  % #1: math style
  % #2: unused
  \raisebox{\depth}{$\m@th#1\intercal$}%
}
\makeatother


\makeatletter
\newcommand*{\da@rightarrow}{\mathchar"0\hexnumber@\symAMSa 4B }
\newcommand*{\da@leftarrow}{\mathchar"0\hexnumber@\symAMSa 4C }
\newcommand*{\xdashrightarrow}[2][]{%
  \mathrel{%
    \mathpalette{\da@xarrow{#1}{#2}{}\da@rightarrow{\,}{}}{}%
  }%
}
\newcommand{\xdashleftarrow}[2][]{%
  \mathrel{%
    \mathpalette{\da@xarrow{#1}{#2}\da@leftarrow{}{}{\,}}{}%
  }%
}
\newcommand*{\da@xarrow}[7]{%
  % #1: below
  % #2: above
  % #3: arrow left
  % #4: arrow right
  % #5: space left 
  % #6: space right
  % #7: math style 
  \sbox0{$\ifx#7\scriptstyle\scriptscriptstyle\else\scriptstyle\fi#5#1#6\m@th$}%
  \sbox2{$\ifx#7\scriptstyle\scriptscriptstyle\else\scriptstyle\fi#5#2#6\m@th$}%
  \sbox4{$#7\dabar@\m@th$}%
  \dimen@=\wd0 %
  \ifdim\wd2 >\dimen@
    \dimen@=\wd2 %   
  \fi
  \count@=2 %
  \def\da@bars{\dabar@\dabar@}%
  \@whiledim\count@\wd4<\dimen@\do{%
    \advance\count@\@ne
    \expandafter\def\expandafter\da@bars\expandafter{%
      \da@bars
      \dabar@ 
    }%
  }%  
  \mathrel{#3}%
  \mathrel{%   
    \mathop{\da@bars}\limits
    \ifx\\#1\\%
    \else
      _{\copy0}%
    \fi
    \ifx\\#2\\%
    \else
      ^{\copy2}%
    \fi
  }%   
  \mathrel{#4}%
}
\makeatother
% \DeclareMathOperator{\rg}{rg}

\usepackage{mathtools}
\DeclarePairedDelimiter{\paren}{(}{)}
\DeclarePairedDelimiter{\abs}{\lvert}{\rvert}
\DeclarePairedDelimiter{\norm}{\lVert}{\rVert}
\DeclarePairedDelimiter{\innerproduct}{\langle}{\rangle}
\newcommand{\Of}[2]{{\operatorname{#1}} {\paren*{#2}}}
\newcommand{\of}[2]{{{{#1}} {\paren*{#2}}}}

\DeclareMathOperator{\Shim}{Shim}
\DeclareMathOperator{\sgn}{sgn}
\DeclareMathOperator{\fdeg}{fdeg}
\DeclareMathOperator{\SL}{SL}
\DeclareMathOperator{\slLie}{\mathfrak{s}\mathfrak{l}}
\DeclareMathOperator{\soLie}{\mathfrak{s}\mathfrak{o}}
\DeclareMathOperator{\spLie}{\mathfrak{s}\mathfrak{p}}
\DeclareMathOperator{\glLie}{\mathfrak{g}\mathfrak{l}}
\newcommand{\pn}[1]{{\color{ForestGreen} \sf PN: [#1]}}
\DeclareMathOperator{\Mp}{Mp}
\DeclareMathOperator{\Mat}{Mat}
\DeclareMathOperator{\GL}{GL}
\DeclareMathOperator{\Gr}{Gr}
\DeclareMathOperator{\GU}{GU}
\def\gl{\mathfrak{g}\mathfrak{l}}
\DeclareMathOperator{\odd}{odd}
\DeclareMathOperator{\even}{even}
\DeclareMathOperator{\GO}{GO}
\DeclareMathOperator{\good}{good}
\DeclareMathOperator{\bad}{bad}
\DeclareMathOperator{\PGO}{PGO}
\DeclareMathOperator{\htt}{ht}
\DeclareMathOperator{\height}{height}
\DeclareMathOperator{\Ass}{Ass}
\DeclareMathOperator{\coheight}{coheight}
\DeclareMathOperator{\GSO}{GSO}
\DeclareMathOperator{\SO}{SO}
\DeclareMathOperator{\so}{\mathfrak{s}\mathfrak{o}}
\DeclareMathOperator{\su}{\mathfrak{s}\mathfrak{u}}
\DeclareMathOperator{\ad}{ad}
% \DeclareMathOperator{\sc}{sc}
\DeclareMathOperator{\Ad}{Ad}
\DeclareMathOperator{\disc}{disc}
\DeclareMathOperator{\inv}{inv}
\DeclareMathOperator{\Pic}{Pic}
\DeclareMathOperator{\uc}{uc}
\DeclareMathOperator{\Cl}{Cl}
\DeclareMathOperator{\Clf}{Clf}
\DeclareMathOperator{\Hom}{Hom}
\DeclareMathOperator{\hol}{hol}
\DeclareMathOperator{\Heis}{Heis}
\DeclareMathOperator{\Haar}{Haar}
\DeclareMathOperator{\h}{h}
\def\sp{\mathfrak{s}\mathfrak{p}}
\DeclareMathOperator{\heis}{\mathfrak{h}\mathfrak{e}\mathfrak{i}\mathfrak{s}}
\DeclareMathOperator{\End}{End}
\DeclareMathOperator{\JL}{JL}
\DeclareMathOperator{\image}{image}
\DeclareMathOperator{\red}{red}
\def\div{\operatorname{div}}
\def\eps{\varepsilon}
\def\cHom{\mathcal{H}\operatorname{om}}
\DeclareMathOperator{\Ops}{Ops}
\DeclareMathOperator{\Symb}{Symb}
\def\boldGL{\mathbf{G}\mathbf{L}}
\def\boldSO{\mathbf{S}\mathbf{O}}
\def\boldU{\mathbf{U}}
\DeclareMathOperator{\hull}{hull}
\DeclareMathOperator{\LL}{LL}
\DeclareMathOperator{\PGL}{PGL}
\DeclareMathOperator{\class}{class}
\DeclareMathOperator{\lcm}{lcm}
\DeclareMathOperator{\spann}{span}
\DeclareMathOperator{\Exp}{Exp}
\DeclareMathOperator{\ext}{ext}
\DeclareMathOperator{\Ext}{Ext}
\DeclareMathOperator{\Tor}{Tor}
\DeclareMathOperator{\et}{et}
\DeclareMathOperator{\tor}{tor}
\DeclareMathOperator{\loc}{loc}
\DeclareMathOperator{\tors}{tors}
\DeclareMathOperator{\pf}{pf}
\DeclareMathOperator{\smooth}{smooth}
\DeclareMathOperator{\prin}{prin}
\DeclareMathOperator{\Kl}{Kl}
\newcommand{\kbar}{\mathchar'26\mkern-9mu k}
\DeclareMathOperator{\der}{der}
% \DeclareMathOperator{\abs}{abs}
\DeclareMathOperator{\Sub}{Sub}
\DeclareMathOperator{\Comp}{Comp}
\DeclareMathOperator{\Err}{Err}
\DeclareMathOperator{\dom}{dom}
\DeclareMathOperator{\radius}{radius}
\DeclareMathOperator{\Fitt}{Fitt}
\DeclareMathOperator{\Sel}{Sel}
\DeclareMathOperator{\rad}{rad}
\DeclareMathOperator{\id}{id}
\DeclareMathOperator{\Center}{Center}
\DeclareMathOperator{\Der}{Der}
\DeclareMathOperator{\U}{U}
% \DeclareMathOperator{\norm}{norm}
\DeclareMathOperator{\trace}{trace}
\DeclareMathOperator{\Equid}{Equid}
\DeclareMathOperator{\Feas}{Feas}
\DeclareMathOperator{\bulk}{bulk}
\DeclareMathOperator{\tail}{tail}
\DeclareMathOperator{\sys}{sys}
\DeclareMathOperator{\atan}{atan}
\DeclareMathOperator{\temp}{temp}
\DeclareMathOperator{\Asai}{Asai}
\DeclareMathOperator{\glob}{glob}
\DeclareMathOperator{\Kuz}{Kuz}
\DeclareMathOperator{\Irr}{Irr}
\newcommand{\rsL}{ \frac{ L^{(R)}(\Pi \times \Sigma, \std, \frac{1}{2})}{L^{(R)}(\Pi \times \Sigma, \Ad, 1)}  }
\DeclareMathOperator{\GSp}{GSp}
\DeclareMathOperator{\PGSp}{PGSp}
\DeclareMathOperator{\BC}{BC}
\DeclareMathOperator{\Ann}{Ann}
\DeclareMathOperator{\Gen}{Gen}
\DeclareMathOperator{\SU}{SU}
\DeclareMathOperator{\PGSU}{PGSU}
% \DeclareMathOperator{\gen}{gen}
\DeclareMathOperator{\PMp}{PMp}
\DeclareMathOperator{\PGMp}{PGMp}
\DeclareMathOperator{\PB}{PB}
\DeclareMathOperator{\ind}{ind}
\DeclareMathOperator{\Jac}{Jac}
\DeclareMathOperator{\jac}{jac}
\DeclareMathOperator{\im}{im}
\DeclareMathOperator{\Aut}{Aut}
\DeclareMathOperator{\Int}{Int}
\DeclareMathOperator{\PSL}{PSL}
\DeclareMathOperator{\co}{co}
\DeclareMathOperator{\irr}{irr}
\DeclareMathOperator{\prim}{prim}
\DeclareMathOperator{\bal}{bal}
\DeclareMathOperator{\baln}{bal}
\DeclareMathOperator{\dist}{dist}
\DeclareMathOperator{\RS}{RS}
\DeclareMathOperator{\Ram}{Ram}
\DeclareMathOperator{\Sob}{Sob}
\DeclareMathOperator{\Sol}{Sol}
\DeclareMathOperator{\soc}{soc}
\DeclareMathOperator{\nt}{nt}
\DeclareMathOperator{\mic}{mic}
\DeclareMathOperator{\Gal}{Gal}
\DeclareMathOperator{\st}{st}
\DeclareMathOperator{\std}{std}
\DeclareMathOperator{\diag}{diag}
\DeclareMathOperator{\Sym}{Sym}
\DeclareMathOperator{\gr}{gr}
\DeclareMathOperator{\aff}{aff}
\DeclareMathOperator{\Dil}{Dil}
\DeclareMathOperator{\Lie}{Lie}
\DeclareMathOperator{\Symp}{Symp}
\DeclareMathOperator{\Stab}{Stab}
\DeclareMathOperator{\St}{St}
\DeclareMathOperator{\stab}{stab}
\DeclareMathOperator{\codim}{codim}
\DeclareMathOperator{\linear}{linear}
\newcommand{\git}{/\!\!/}
\DeclareMathOperator{\geom}{geom}
\DeclareMathOperator{\spec}{spec}
\def\O{\operatorname{O}}
\DeclareMathOperator{\Au}{Aut}
\DeclareMathOperator{\Fix}{Fix}
\DeclareMathOperator{\Opp}{Op}
\DeclareMathOperator{\opp}{op}
\DeclareMathOperator{\Size}{Size}
\DeclareMathOperator{\Save}{Save}
% \DeclareMathOperator{\ker}{ker}
\DeclareMathOperator{\coker}{coker}
\DeclareMathOperator{\sym}{sym}
\DeclareMathOperator{\mean}{mean}
\DeclareMathOperator{\elliptic}{ell}
\DeclareMathOperator{\nilpotent}{nil}
\DeclareMathOperator{\hyperbolic}{hyp}
\DeclareMathOperator{\newvector}{new}
\DeclareMathOperator{\new}{new}
\DeclareMathOperator{\full}{full}
\newcommand{\qr}[2]{\left( \frac{#1}{#2} \right)}
\DeclareMathOperator{\unr}{u}
\DeclareMathOperator{\ram}{ram}
% \DeclareMathOperator{\len}{len}
\DeclareMathOperator{\fin}{fin}
\DeclareMathOperator{\cusp}{cusp}
\DeclareMathOperator{\curv}{curv}
\DeclareMathOperator{\rank}{rank}
\DeclareMathOperator{\rk}{rk}
\DeclareMathOperator{\pr}{pr}
\DeclareMathOperator{\Transform}{Transform}
\DeclareMathOperator{\mult}{mult}
\DeclareMathOperator{\Eis}{Eis}
\DeclareMathOperator{\reg}{reg}
\DeclareMathOperator{\sing}{sing}
\DeclareMathOperator{\alt}{alt}
\DeclareMathOperator{\irreg}{irreg}
\DeclareMathOperator{\sreg}{sreg}
\DeclareMathOperator{\Wd}{Wd}
\DeclareMathOperator{\Weil}{Weil}
\DeclareMathOperator{\Th}{Th}
\DeclareMathOperator{\Sp}{Sp}
\DeclareMathOperator{\Ind}{Ind}
\DeclareMathOperator{\Res}{Res}
\DeclareMathOperator{\ini}{in}
\DeclareMathOperator{\ord}{ord}
\DeclareMathOperator{\osc}{osc}
\DeclareMathOperator{\fluc}{fluc}
\DeclareMathOperator{\size}{size}
\DeclareMathOperator{\ann}{ann}
\DeclareMathOperator{\equ}{eq}
\DeclareMathOperator{\res}{res}
\DeclareMathOperator{\pt}{pt}
\DeclareMathOperator{\src}{source}
\DeclareMathOperator{\Zcl}{Zcl}
\DeclareMathOperator{\Func}{Func}
\DeclareMathOperator{\Map}{Map}
\DeclareMathOperator{\Frac}{Frac}
\DeclareMathOperator{\Frob}{Frob}
\DeclareMathOperator{\ev}{eval}
\DeclareMathOperator{\pv}{pv}
\DeclareMathOperator{\eval}{eval}
\DeclareMathOperator{\Spec}{Spec}
\DeclareMathOperator{\Speh}{Speh}
\DeclareMathOperator{\Spin}{Spin}
\DeclareMathOperator{\GSpin}{GSpin}
\DeclareMathOperator{\Specm}{Specm}
\DeclareMathOperator{\Sphere}{Sphere}
\DeclareMathOperator{\Sqq}{Sq}
\DeclareMathOperator{\Ball}{Ball}
\DeclareMathOperator\Cond{\operatorname{Cond}}
\DeclareMathOperator\proj{\operatorname{proj}}
\DeclareMathOperator\Swan{\operatorname{Swan}}
\DeclareMathOperator{\Proj}{Proj}
\DeclareMathOperator{\bPB}{{\mathbf P}{\mathbf B}}
\DeclareMathOperator{\Projm}{Projm}
\DeclareMathOperator{\Tr}{Tr}
\DeclareMathOperator{\Type}{Type}
\DeclareMathOperator{\Prop}{Prop}
\DeclareMathOperator{\vol}{vol}
\DeclareMathOperator{\covol}{covol}
\DeclareMathOperator{\Rep}{Rep}
\DeclareMathOperator{\Cent}{Cent}
\DeclareMathOperator{\val}{val}
\DeclareMathOperator{\area}{area}
\DeclareMathOperator{\nr}{nr}
\DeclareMathOperator{\CM}{CM}
\DeclareMathOperator{\CH}{CH}
\DeclareMathOperator{\tr}{tr}
\DeclareMathOperator{\characteristic}{char}
\DeclareMathOperator{\supp}{supp}


\theoremstyle{plain} \newtheorem{theorem} {Theorem} \newtheorem{conjecture} [theorem] {Conjecture} \newtheorem{corollary} [theorem] {Corollary} \newtheorem{proposition} [theorem] {Proposition} \newtheorem{fact} [theorem] {Fact}
\theoremstyle{definition} \newtheorem{definition} [theorem] {Definition} \newtheorem{hypothesis} [theorem] {Hypothesis} \newtheorem{assumptions} [theorem] {Assumptions}
\newtheorem{example} [theorem] {Example}
\newtheorem{assertion}[theorem] {Assertion}
\newtheorem{note}[theorem] {Note}
\newtheorem{conclusion}[theorem] {Conclusion}
\newtheorem{claim}            {Claim}
\newtheorem{homework} {Homework}
\newtheorem{exercise} {Exercise}  \newtheorem{question}[theorem] {Question}    \newtheorem{answer} {Answer}  \newtheorem{problem} {Problem}    \newtheorem{remark} [theorem] {Remark}
\newtheorem{notation} [theorem]           {Notation}
\newtheorem{terminology}[theorem]            {Terminology}
\newtheorem{convention}[theorem]            {Convention}
\newtheorem{motivation}[theorem]            {Motivation}


\newtheoremstyle{itplain} % name
{6pt}                    % Space above
{5pt\topsep}                    % Space below
{\itshape}                   % Body font
{}                           % Indent amount
{\itshape}                   % Theorem head font
{.}                          % Punctuation after theorem head
{5pt plus 1pt minus 1pt}                       % Space after theorem head
% {.5em}                       % Space after theorem head
{}  % Theorem head spec (can be left empty, meaning ‘normal’)

% \theoremstyle{mytheoremstyle}


\theoremstyle{itplain} %--default
% \theoremheaderfont{\itshape}
% \newtheorem{lemma}{Lemma}
\newtheorem{lemma}[theorem]{Lemma}
% \newtheorem{lemma}{Lemma}[subsubsection]

\newtheorem*{lemma*}{Lemma}
\newtheorem*{proposition*}{Proposition}
\newtheorem*{definition*}{Definition}
\newtheorem*{example*}{Example}

\newtheorem*{results*}{Results}
\newtheorem{results} [theorem] {Results}


\usepackage[displaymath,textmath,sections,graphics]{preview}
\PreviewEnvironment{align*}
\PreviewEnvironment{multline*}
\PreviewEnvironment{tabular}
\PreviewEnvironment{verbatim}
\PreviewEnvironment{lstlisting}
\PreviewEnvironment*{frame}
\PreviewEnvironment*{alert}
\PreviewEnvironment*{emph}
\PreviewEnvironment*{textbf}



\usepackage{xr-hyper}
\externaldocument{2023-introduction-to-zeta-and-l-functions}

\begin{document}
 

\title{Complex analytic preliminaries}
\begin{abstract}
  Part of the course notes for \href{2023-introduction-to-zeta-and-l-functions.pdf}{this course}.
\end{abstract}

\section{Holomorphic continuation}\label{sec:cj41z47j43}

\begin{theorem}[Identity principle for holomorphic functions]\label{theorem:cj41z47je7}
  Let $U \subset \mathbb{C} $ be a connected open set.  Let $f, g : U \rightarrow \mathbb{C} $ be holomorphic functions.  If $f = g$ on a set with a limit point in $U$, then $f = g$ on all of $U$.
\end{theorem}
\begin{corollary}\label{corollary:cj3vqbthht}
  Let $U \subset \Omega \subseteq \mathbb{C} $ be open subsets, with $U$ nonempty and $\Omega$ connected.  Let $f : U \rightarrow \mathbb{C}$ be a holomorphic function.  Then there is at most one extension of $f$ to a holomorphic function $\Omega \rightarrow \mathbb{C}$.
\end{corollary}

\section{Cauchy's integral formula}\label{sec:cj41z47ie0}
\begin{theorem}\label{theorem:cj41z47hph}
  Let $f : U \rightarrow \mathbb{C} $ be a holomorphic function defined on an open subset $U$.  Let $\gamma$ be a closed rectifiable curve in $U$.  Then $\int_\gamma f(z) \, d z = 0$.
\end{theorem}

\begin{theorem}\label{theorem:cj3vqbjd26}
  Let $0 \leq a < b \leq \infty$.  Let $f(z)$ be a holomorphic function on the annulus $\{z \in \mathbb{C} : a < \lvert z \rvert < b\}$ given by a convergent Laurent series
  \begin{equation*}
    f(z) = \sum_{n \in \mathbb{Z} } c_n z^n.
  \end{equation*}
  \begin{enumerate}
  \item For any $r \in (a,b)$ and $n \in \mathbb{Z}$, we have
    \begin{align*}
      c_n &=  \oint_{\lvert z \rvert = r} \frac{f(z)}{z^{n}} \, \frac{d z}{2 \pi i z} \\
          &= \frac{1}{2 \pi r^n } \int_{\theta = 0 }^{2 \pi } f (r e^{i \theta }) e^{- i n \theta } \,d \theta.
    \end{align*}
  \item For each compact subset $E$ of $(a,b)$, there exists $M \geq 0$ so that for all $r \in E$, we have
    \begin{equation}\label{eq:cj3vqbiupy}
      \sum_{n \in \mathbb{Z}} \lvert c_n \rvert r^n \leq M.
    \end{equation}
  \end{enumerate}
\end{theorem}

\begin{theorem}\label{theorem:cj3wnb89dd}
  Let $U$ be an open subset of $\mathbb{C}$, let $f : U \rightarrow \mathbb{C} $ be meromorphic.  Let $\gamma$ be a smooth closed curve in $U$, oriented counterclockwise, that does not pass through any pole of $f$.  Then
  \begin{equation*}
    \int_\gamma f(z) \, d z = 2 \pi i \sum_{\substack{z \in \operatorname{interior}(\gamma) \\ \text{pole of $f$}}} \operatorname{res}_z(f).
  \end{equation*}
\end{theorem}
\begin{remark}\label{remark:cj41z47fjv}
  Let $0 < r < R$.  Let $f$ be a meromorphic function on a neighborhood of the annulus $\{z : r < |z| < R\}$ that has no poles on either of the circles $|z| = r, R$.  Then
  \begin{equation*}
    \oint_{|z| = R} f(z) \, d z
    =
    \oint_{|z| = r} f(z) \, d z
    + 2 \pi i \sum_{
      \substack{
        r < |z| < R 
        \\
        \text{pole of $f$}
      }
    }
    \res_z(f).
  \end{equation*}
\end{remark}

\section{Holomorphy of limits and series}\label{sec:cj41z47df7}
\begin{theorem}\label{theorem:cj3vqa91ti}
  Let $U$ be an open subset of the complex plane.  Let $f_n$ be a sequence of holomorphic functions on $U$.
  \begin{enumerate}
  \item Suppose that the sequence $f_n$ converges pointwise to some function $f$, uniformly on compact subsets of $U$.  Then $f$ is holomorphic.
  \item Suppose that the partial sums $\sum_{n \leq N} f_n$ converge pointwise to some function $f$, uniformly on compact subsets of $U$.  Then the sum $\sum_n f_n$ is holomorphic.
  \end{enumerate}
\end{theorem}

\section{Blashke factors}\label{sec:cj41z47br7}
\begin{definition}\label{definition:cj41z47aw9}
  Let $\alpha \in \mathbb{C}$ with $\lvert \alpha  \rvert < 1$.  The \emph{Blashke factor} $B_\alpha$ is the function
  \begin{equation*}
B_\alpha(z) := \frac{\alpha - z }{ 1- \bar{\alpha } z}.
  \end{equation*}
\end{definition}
\begin{lemma}\label{lemma:cj41z469wv}
The Blashke factors enjoy the following properties:
\begin{enumerate}
\item $B_\alpha(0) = \alpha$ and $B_\alpha(\alpha) = 0$.
\item $B_\alpha$ defines a holomorphic automorphism of the unit disc $\{z : \lvert z \rvert < 1\}$, with inverse $B_\alpha$ (i.e., $B_\alpha(B_\alpha(z)) = z$).
\item $\lvert B_\alpha(z) \rvert = 1$ for all $z$ with $\lvert z \rvert = 1$.
\end{enumerate}
\end{lemma}

\section{Harmonic functions}\label{sec:cj41z468dg}
Let $U$ be an open subset of the plane $\mathbb{C}$.

\subsection{Definition}\label{sec:cj41z467iy}

\begin{definition}\label{definition:cj41z466s8}
We say that a smooth function $u : U \rightarrow \mathbb{R}$ is \emph{harmonic} if $u_{x x} + u_{y y} = 0$.
\end{definition}

\subsection{Relation with holomorphic functions}\label{sec:cj41z46427}

\begin{lemma}
  Let $f : U \rightarrow \mathbb{C}$ be holomorphic.  Then, writing $f = u + i v$, with $u, v : U \rightarrow \mathbb{R}$, we have that $u$ is harmonic.  Conversely, if $U$ is simply-connected, then every harmonic $u$ arises in this way for some holomorphic $f$.
\end{lemma}
\begin{proof}
The first claim follows from the Cauchy-Riemann equations.  For the second claim, we solve the Cauchy-Riemann equations to find $v$ such that $f = u + i v$ is holomorphic on each connected component of $U$.
\end{proof}

\subsection{Mean value theorems}\label{sec:cj41z4625i}
For the following results, let $u : U \rightarrow \mathbb{R}$ be harmonic, and suppose that $U$ contains the disc $\{z : \lvert z - z_0 \rvert \leq r\}$.
\begin{lemma}\label{lemma:cj41z461iw}
  We have
  \begin{equation*}
u(z_0) = \int_{0}^{2 \pi } u (z_0 + r e^{i \theta }) \, \frac{d \theta }{ 2 \pi }.
\end{equation*}
\end{lemma}
\begin{proof}
We may assume that $U$ is simply-connected.  The claim then follows from the Cauchy integral formula applied to a holomorphic $f$ with $\Re(f) = u$.
\end{proof}

\begin{lemma}\label{lemma:cj41z46r5i}
  For $\lvert z - z_0 \rvert \leq r$, we have
  \begin{equation*}
    u(z) = \int_0^{2 \pi } u (z_0 + r e^{i \theta }) \Re \frac{r e^{i \theta } + (z - z_0 )}{r e^{i \theta } - (z - z_0 )}
    \, \frac{d \theta }{2 \pi }.
  \end{equation*}
\end{lemma}
\begin{proof}
  We can reduce to the previous lemma using Blashke factors.  Alternatively, we can reduce first to the case $z_0 = 0$ and $r = 1$ and $0 < z < 1$.  Then Cauchy's integral formula reads
  \begin{equation*}
    f(z) =
    \int_{w : \lvert w \rvert = r}
    f(w)
    \underbrace
{
\frac{1}{2} \left(
      \frac{w + z}{w - z}
      + 
      \frac{w^{-1}  + z}{w^{-1}  - z}
    \right)
}_{
\Re \left( \frac{w + z}{w - z} \right)
}
\, \frac{d w}{2 \pi i},
  \end{equation*}
  which yields the required formula upon taking real parts.
\end{proof}

\begin{lemma}\label{lemma:cj41z46zau}
  For $\lvert z - z_0 \rvert \leq r_0 < r$, we have
  \begin{equation*}
    \lvert u(z) \rvert \leq \frac{r + r_0}{r - r_0}
\int_{0 }^{2 \pi } \lvert u (z_0 + r e^{i \theta }) \rvert \, \frac{d \theta }{2 \pi}.    
  \end{equation*}
\end{lemma}
\begin{proof}
We apply Lemma \ref{lemma:cj41z46r5i} and majorize the integrand in absolute value.
\end{proof}

\section{Approximate factorizations of holomorphic functions}\label{sec:cj41z5fwla}
The next lemma closely follows the presentation of Theorem 21 of \href{these notes}{https://terrytao.wordpress.com/2014/12/05/245a-supplement-2-a-little-bit-of-complex-and-fourier-analysis/}.

\begin{lemma}\label{lemma:cj41z5hd4a}
  Fix $0 < c_2 < c_1 < 1$.  Let $f$ be a holomorphic function, on a neighborhood of $\bar{D}$, where $D := \{\lvert z - z_0 \rvert < r\}$.  Assume that $f(z_0) \neq 0$.  Assume given $M \geq 1$ so that whenever $|z - z_0| = r$, we have
  \begin{equation*}
    \lvert f(z) \rvert \leq M \lvert f(z_0) \rvert.
\end{equation*}
  Let $\rho$ run over the zeros of $f$, counted with multiplicity.  Then
  \begin{equation*}
    \# \left\{ \rho : \lvert \rho - z_0 \rvert \leq c_1 r \right\} \ll_{c_1} \log M
  \end{equation*}
  and
  \begin{equation*}
    \frac{f'}{f}(z) = \sum_{
      \lvert \rho - z_0 \rvert \leq c_1 r
    }
    \frac{1}{ z - \rho } + \O_{c_1, c_2} \left( \frac{\log M}{r} \right).
  \end{equation*}
\end{lemma}
\begin{proof}
  By replacing $f(z)$ with $f(z_0 + z)$, we may assume that $z_0 = 0$.  By writing $f(z) = g( r z)$ (so that $f'(z) = r g'(r z)$, hence $\frac{g'}{g}(r z) = \frac{1}{r} \frac{f'}{f}(z)$), we reduce to the case $r = 1$. Our task is then to show that
  \begin{equation}\label{eq:cj41zznej1}
    \# \left\{ \rho : \lvert \rho  \rvert \leq c_1 \right\} \ll_{c_1} \log M
  \end{equation}
  and that for $\lvert z \rvert \leq c_2$, we have
  \begin{equation}\label{eq:cj41zzn7hn}
    \frac{f ' }{f}(z) = \sum_{\lvert \rho \rvert \leq c_1}
    \frac{1}{z - \rho }
    + \O_{c_1, c_2} \left( \log M \right).
  \end{equation}
  By Jensen's formula, we have
  \begin{equation}\label{eq:cj41zzjhfa}
    0 \leq \sum_{\lvert \rho \rvert \leq 1} \log \frac{1}{\lvert \rho \rvert} = \int_0^{2 \pi} \frac{\log \lvert f (e^{i \theta }) \rvert}{\log \lvert f(0) \rvert} \, \frac{d \theta }{ 2 \pi } \leq \log M.
  \end{equation}
  On the other hand, for $\lvert \rho \rvert \leq c_1$, we have $\log 1 / \lvert \rho \rvert \geq \log 1/c_1$.  It follows that
  \begin{equation*}
    \# \left\{ \rho : \lvert \rho  \rvert \leq c_1 \right\}
    \leq \frac{\log M}{ \log 1/c_1} \ll_{c_1} \log M,
  \end{equation*}
  whence \eqref{eq:cj41zznej1}.
  
  Turning to the proof of \eqref{eq:cj41zzn7hn}, we may write
  \begin{equation*}
    f = g \prod_{|\rho| \leq 1} B_\rho,
  \end{equation*}
  where $g$ is a holomorphic function on a neighborhood of $\bar{D}$ that does not vanish on $\bar{D}$.  We then have
  \begin{equation*}
    \frac{f'}{f} = \frac{g'}{g} + \sum_{\lvert \rho \rvert \leq 1} \frac{B_\rho '}{B_\rho}.
  \end{equation*}
  Below, we estimate separately the contributions to \eqref{eq:cj41zzn7hn} from $g$, from $\lvert \rho \rvert \leq c_1$, and from $c_1 < \lvert \rho \rvert \leq 1$.  The result follows by combining these estimates.

  We first estimate the contribution of $g$.  We will show that
  \begin{equation}\label{eq:cj41zxfz74}
    \frac{g'}{g}(z) = \O_{c_1, c_2} \left( \log M \right),
  \end{equation}
  We may normalize $f$ so that $g(0) = 1$.  Then, since $g$ does not vanish, Jensen's formula reads
  \begin{equation}\label{eq:cj41zze0vv}
    \int _{0}^{2 \pi} \log \lvert g(e^{i \theta}) \rvert \, \frac{d \theta }{2 \pi } = 0.
  \end{equation}
  On the other hand, our hypothesis reads
  \begin{equation*}
    \lvert g(e^{i \theta}) \rvert = \lvert f(e^{i \theta}) \rvert \leq M \lvert f(0) \rvert = M \prod_{\rho} |\rho| \leq M,
  \end{equation*}
  hence
  \begin{equation*}
    \log \lvert g(e^{i \theta}) \rvert \leq \log M.
  \end{equation*}
  By combining this pointwise upper bound with the mean zero property \eqref{eq:cj41zze0vv}, we deduce that
  \begin{equation*}
    \int_0^{2 \pi } \left\lvert
      \log \lvert g (e^{i \theta }) \rvert
    \right\rvert
    \leq \frac{1}{2} \log M.
  \end{equation*}
  Since $g$ does not vanish, we may find a holomorphic primitive $G$ for $g'/g$ (i.e., $G$ is a logarithm of $g$).  Lemma \ref{lemma:cj41z46zau} gives, for $\lvert z \rvert \leq c_2$, the estimate
  \begin{equation*}
    \lvert \Re(G(z)) \rvert \ll_{c_2} \log  M,
  \end{equation*}
  together with a similar bound for the derivatives of the real part of $G$.  By the Cauchy--Riemann equations, the same bound holds for the first derivatives of the imaginary part of $G$, hence for those of $G$ itself.  The estimate \eqref{eq:cj41zxfz74} follows.

  We next estimate the contribution from $\lvert \rho \rvert \leq c_1$.  After the calculation
  \begin{equation}\label{eq:cj41z21ka5}
    \frac{B_\rho '}{B_\rho }(z) = \frac{1}{z - \rho } - \frac{1}{z - 1 / \bar{\rho }},
  \end{equation}
  one term of which matches up with a term on the right hand side of \eqref{eq:cj41zzn7hn}, we see that it suffices to show that
  \begin{equation}\label{eq:cj41zzwalu}
    \sum_{\lvert \rho \rvert \leq c_1} \frac{1}{z - 1 / \bar{\rho }} \ll_{c_1, c_2} \log M.
  \end{equation}
  Since $\lvert z \rvert \leq c_2$, we have
  \begin{equation*}
    \left\lvert \frac{1}{z - 1 / \bar{\rho }} \right\rvert \leq \frac{1}{1/c_1 - c_2} \ll_{c_1,c_2} 1.
  \end{equation*}
  Therefore the required bound \eqref{eq:cj41zzwalu} follows from \eqref{eq:cj41zznej1}.

  We turn finally to the contribution from $c_1 < \lvert \rho \rvert \leq 1$.  For such an element $\rho$, and $z$ with $\lvert z \rvert \leq c_2$, we have
  \begin{equation*}
    z - \rho \asymp _{c_1,c_2} 1, \qquad z - 1 / \bar{\rho } \asymp_{c_1,c_2} 1,
  \end{equation*}
  where we recall that $A \asymp B$ means that $A \ll B \ll A$.  By cross-multiplying in \eqref{eq:cj41z21ka5}, it follows that
  \begin{equation*}
    \frac{B_\rho '}{B_\rho }(z) \asymp_{c_1,c_2} \rho - 1 / \bar{\rho }.
  \end{equation*}
  On the other hand, using the Taylor expansion of the logarithm, we have
  \begin{equation*}
    \rho - 1 / \bar{\rho } \asymp_{c_1} 1 - |\rho|^2 \asymp_{c_2} \log \frac{1}{\lvert \rho \rvert}.
  \end{equation*}
  We thereby obtain
  \begin{equation}\label{eq:cj41z3ecu9}
    \sum_{c_1 < \lvert \rho  \rvert \leq 1} \frac{B_\rho '}{B_\rho }(z)
    \ll_{c_1, c_2}
    \sum_{c_1 < \lvert \rho  \rvert \leq 1}
    \log
    \frac{1}{\lvert \rho  \rvert}
    \leq
    \log M,
  \end{equation}
  where in the final step we appealed to \eqref{eq:cj41zzjhfa}.
\end{proof}




\bibliography{refs}{} \bibliographystyle{plain}
\end{document}
