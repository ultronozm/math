\documentclass[11pt]{amsart}
\usepackage[utf8]{inputenc}

\usepackage{amsfonts}
\usepackage{amsmath}
\usepackage{amsthm}
\usepackage{amssymb}
\usepackage{mathrsfs}

\usepackage{ytableau}
\usepackage{xcolor}
 
\usepackage[all]{xy}
\usepackage{appendix}

\theoremstyle{definition}
\newtheorem{defi}{Definition}[section]
\newtheorem{eg}[defi]{Example}
\newtheorem{rem}[defi]{Remark}
\newtheorem{prop}[defi]{Proposition}
\newtheorem{lem}[defi]{Lemma}
\newtheorem{theo}[defi]{Theorem}
\newtheorem{q}{Question}
\newtheorem{con}{Construction}
\newtheorem{ex}[defi]{Exercise}
\newtheorem{cor}[defi]{Corrollary}

\newcommand{\C}{\mathbb{C}}
\newcommand{\Z}{\mathbb{Z}}
\newcommand{\Hom}{\operatorname{Hom}}
\title{Discrete Fourier transform}
\begin{abstract}
  Part of the course notes for \href{2023-introduction-to-zeta-and-l-functions.pdf}{this course}.  Sergey's notes for his lectures on the discrete Fourier transform.
\end{abstract}
\author{S. A.}



\begin{document}

\maketitle
\section{The group of characters for $A$}
Consider a finite Abelian group $A$. Recall that a group map $\chi:\ A\to \C^*$ is called a character of $A$. Notice that since for every $a\in A$ there exists $s:\ a^s=1$, the absolute value $|\chi(a)|=1$, i.e. the image of $\chi$ belongs to the circle subgroup $S=\{\alpha\in\C| \ |\alpha|=1\}$. 

We define the product of characters $\chi_1\chi_2$ pointwise. The trivial chracter is the unit function, the inverse character $\chi^{-1}(a)=\overline{\chi(a)}$.  
\begin{defi}
    The group of characters of $A$ is called the Pontryagin dual group and is denoted by $\hat{A}$.
\end{defi}
Denote the vector space of complex valued functions on $A$ (resp. on $\hat{A}$) by $C(A)$
(resp. by $C(A)$).
\begin{eg}
\begin{enumerate}
    \item Any number $c\in\C$ defines a constant function, in particular $1$ is the unit function in $C(A)$. Notice that pointwise multiplication makes $C(A)$ into an associative algebra with the unit.
    \item Any element $a\in A$ gives rise to a delta function $\delta_a$ whose value equals $1$ at $a$ and $0$ at all other elements of $A$.
    \item Certainly any character $\chi\in \hat{A}$ provides an element in $C(A)$.
\end{enumerate}
\end{eg}
\begin{rem} It is known from the algebra course that as elements of $C(A)$ all characters are linearly independent. In particular, the cardinality of $\hat{A}$ is no greater than the number of elements in $A$.
\end{rem}
\begin{eg} Take the finite cyclic group $A=\Z/n=\{1,s,\ldots,s^{n-1}\}$. Then any $k\in\{0,\ldots,n-1\}$
 provides a character $\chi_k:\ \Z/n\to \C^*, s^m\mapsto \xi^{mk}$. Here $\xi$ denotes a chosen primitive root of unity of degree $n$. In particular, $\hat{A}$ is a cyclic group of the same order. It is isomorphic to $A$ non-canonically. 

 Classification Theorem for finite Abelian groups implies that for any such $A$ the group $\hat{A}$ is non-canonically isomorphic to $A$.
\end{eg}
The vector space $C(A)$ (resp. $C(\hat{A})$) is equipped with a Hermitian form: 
$$
\langle f,g\rangle=\sum_{a\in A}f(a)\overline g(a).$$
\begin{rem}
    Notice that the functions $\delta_a, \ a\in A,$ form an orthonormal basis of $C(A)$.
\end{rem}
\begin{lem} Let $\chi_1$ and $\chi_2$ be two different characters of $A$. Then as elements of $C(A)$ they are orthogonal to each other. 
\end{lem}
\begin{proof}
    Let $b\in A$ be an element such that $\chi_1(b)\ne\chi_2(b)$. We have
    $$\langle \chi_1,\chi_2\rangle=\sum_{a\in A}\chi_1(a)\overline \chi_2(a)=
    \sum_{a\in A}\chi_1(ab)\overline \chi_2(ab)=(\sum_{a\in A}\chi_1(a)\overline \chi_2(a))\chi_1(b)\chi_2^{-1}(b). 
    $$
    By our choice of $b\in A$, the first factor in the last expression equals to $0$.
\end{proof}
\begin{lem}
The map $A\to \hat{\hat{A}},\ a\to \hat{a},\ \hat{a}(\chi)=\chi(a)$, provides a canonical isomorphism between $A$ and the double Pontryagin dual group of $A$.
\end{lem}
\begin{proof}
    We know that the source and the target of the canonical map have the same cardinality. It is clear that the map respects multiplication in the group. Moreover, for any non-unit $a$, the character $\hat{a}$ is non-trivial. It follows that the map is an injective. Thus it is an isoporphism.
\end{proof}
\section{Discrete Fourier transform}
\begin{defi}
    The vector space map $$F:\ C(A)\to C(\hat{A}), \ F(f)(\chi)=\sum_{a\in A}f(a)\overline{\chi(a)},$$ is called the Fourier transform for the group $A$.
\end{defi}
\begin{eg}
\begin{enumerate} Let us calculate some examples:
\item
$F(\delta_a)(\chi)=\overline{\chi(a)}=\hat{a}^{-1}(\chi).$ 
\item 
$F(1)(\chi)=\sum_{a\in A}\overline{\chi(a)}$. It is known that for a non-trivial character such sum equals to $0$. The value at $\chi_0$ equals to $\#(A)$. Thus we have $F(1)=\delta_{\chi_0}$.
\item 
More generally, $F(\chi_1)(\chi_2)=\langle\chi_1,\chi_2\rangle=0$ for $\chi_1\ne\chi_2$. We have $F(\chi)(\chi)=\langle\chi,\chi\rangle=\#(A)$. It follows that $F(\chi)=\#(A)\delta_\chi$.
\end{enumerate}
\end{eg}
\begin{cor}
    It follows that the Fourier transform $F=F_A$ is an isomorphism of the vector spaces $C(A)$ and $C(\hat{A})$ with its inverse equal to $\#(A)F_{\hat{A}}$.
\end{cor}
\begin{rem}
    We have checked that an orthogonal basis in $C(A)$ is mapped to an orthogonal basis in $C(\hat{A})$. Thus up to a scalar multiple $F$ is an isometry. 
\end{rem}
\begin{defi} Given a map of finite sets $\alpha:\ X\to Y)$, consider the pull back morphism 
$$\alpha^*:\ C(Y)\to C(X),\ \alpha^*(f)(x)=f(\alpha(x))$$
and the push forward morphism
$$\alpha_*:\ C(X)\to C(Y),\ \alpha_*(f)(y)=\sum_{\alpha(x)=y}f(x).$$
\end{defi}
\begin{rem}
    \begin{enumerate}
    \item The pull-back map respects the pointwise product of functions.
    \item For two maps $\alpha:\ X\to Y$ and $\beta:\ Y\to Z$ we have 
    $$(\beta\circ\alpha)^*=\alpha^*\circ\beta^*,\ (\beta\circ\alpha)_*=\beta_*\circ\alpha_*$$. This is a direct calculation. 
        \end{enumerate}
\end{rem}
Let $\alpha: B\to A$ be a morphism of finite Abelian groups. Then any character of $A$ defines a character of $B$. We obtain the Pontryagin dual map $\hat{\alpha}:\ \hat{A}\to \hat{B}$.
\begin{lem} We have $(\hat{\alpha})^*\circ F=F\circ \alpha_*$.
\end{lem}
We introduce the second multiplication on $C(A)$. Denote the multiplication map $A\times A\to A$ by $m$.
\begin{defi} For $f,g \in C(A)$ denote the function on $A\times A$ given by $(a,b)\mapsto f(a)g(b)$ by $f\times g$. The convolution $f\star g=m_*(f\times g)$.
\end{defi}
\begin{rem} 
\begin{enumerate}
\item The element $\delta_1$ is the unit for the operation. 
\item One checks directly that convolution is associative.
\item Compare the definition of convolution with the following observation. Denote the diagonal embedding $A\to A\times A$ by $\Delta$. The pointwise product of functions is given by $f\cdot g=\Delta^*(f\times g)$.
\end{enumerate}
\end{rem}
\begin{lem} We have $F(f\star g)=F(f)\cdot F(g)$.
\end{lem}
\begin{proof}
Consider the map $\hat{m}:\ \hat{A}\to \hat{A}\times \hat{A}$. We have
$$\hat{m}(\chi)(a,b)=\chi(ab)=\chi(a)\chi(b)=(\chi\times \chi)(a,b).$$ Thus the Pontryagin dual for the product map is the diagonal embedding. The proof follows.
\end{proof}
Consider the argument shift operation $L_a:\ C(A)\to C(A), (L_af)(b)=f(a+b).$
\begin{lem}
 We have $(F\circ L_a)(f)=\hat{a}\cdot F(f)$.   
\end{lem}
\begin{proof}
    The key observation in the proof is that $L_a(f)=\delta_a\star f$. Then use the previous Lemma.
\end{proof}

\section{Generalized Poisson summation}
Below we describe the setting for Poisson summation. Let $A$ be a finite Abelian group with a subgroup $B$. 
Consider the embedding map $i:\ B\to A$ and the projection map $p:\ A\to A/B$. The map $\hat{i}$ is surjective, its kernel $B^\perp$ consists of characters $\chi$ whose restriction to $B$ is trivial. We identify the kernel with $\widehat{(A/B)}$.
\begin{theo} For  $f\in C(A)$, for any $a\in A$, we have 
$$
\sum_{b\in B}f(a+b)=\sum_{\chi\in B^\perp}F(f)(\chi)\overline{\chi}(a).$$
\end{theo}
    
\begin{proof}
    First notice that LHS in the formula in question equals $p_*(f)(aB)$. From Section 2  we know that the latter equals $F_{A/B}^{-1}\circ (\hat{p})^*\circ F_A$. Notice that
    $$F_{A/B}^{-1}\circ (\hat{p})^*\circ F_A(f)(aB)=
    \sum_{\chi\in B^\perp }F(f)(\chi)\overline{\chi}(a).$$
\end{proof}
\end{document}
